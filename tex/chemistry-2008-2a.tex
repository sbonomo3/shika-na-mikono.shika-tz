\section{2008 - CHEMISTRY 2A ALTERNATIVE A PRACTICAL}
%2008 requires students to answer two (2) of the following questions, including Question 1.

\begin{enumerate}

\item[1.] You are provided with the following:\\
\vspace{4pt}
Solution M containing 9.0 g of H$_2$X per dm$^3$ of the solution.\\
\vspace{4pt}
Solution N containing 4.91 g of sodium hydroxide per dm$^3$ of the solution.\\
\vspace{4pt}
Solution P is phenolphthalein indicator.\\
\vspace{10pt}
\textbf{Procedure}\\
\vspace{4pt}
Put solution M into the burette. Pipette 25 cm$^3$ (or 20 cm$^3$) of solution N into the titration flask. Put two to three drops of P into the titration flask. Titrate solution M from the burette against solution N in the titration flask until a colour change is observed. Note the burette reading. Repeat the procedure to obtain three more readings. Record your results as shown in Table 1.\\
\vspace{10pt}
\textbf{Results}\\
\vspace{6pt}
\textbf{Table 1: Burette readings}\\

\begin{center}
\begin{tabular}{|l|p{2cm}|p{2cm}|p{2cm}|p{2cm}|} \hline
\textbf{Titration}&\multicolumn{1}{|c|}{\textbf{Pilot}}&\multicolumn{1}{|c|}{\textbf{1}}&\multicolumn{1}{|c|}{\textbf{2}}&\multicolumn{1}{|c|}{\textbf{3}}\\ \hline
Final reading (cm$^3$)&&&&\\ \hline
Initial reading (cm$^3$)&&&&\\ \hline
Volume used (cm$^3$)&&&&\\ \hline
\end{tabular}\\
\end{center}

\begin{enumerate}
\item[(a)] Give the volume of the pipette used.
\item[(b)] Give the volume of solution M needed for complete neutralization of solution N.
\item[(c)] Tell the colour change of the indicator at the end point of the titration.
\item[(d)] Write the balanced chemical equation for the reaction between solution M and N.
\item[(e)] Calculate the
\begin{enumerate}
\item[(i)] molarity of solution M
\item[(ii)] molar mass of H$_2$X
\item[(iii)] mass of X in H$_2$X.
\end{enumerate}
\end{enumerate}

\raggedleft \textbf{(25 marks)}\newpage

\raggedright

\item[2.] Sample D is a simple salt containing one cation and one anion. Carry out carefully the experiments described below recording all your observations and appropriate inferences as shown in Table 2 to identify the cation and anion present in D.\\
\vspace{10pt}
\textbf{Table 2}\\

\begin{center}
\begin{tabular}{|l|p{8cm}|l|l|}
\hline
\multicolumn{2}{|c|}{\textbf{Experiment}}&\textbf{Observation}&\textbf{Inference}\\ \hline
(a)&Observe the appearnce of salt D.&&\\ \hline
(b)&Put a little solid sample D in a clean and dry test tube and heat.&&\\ \hline
(c)&Put a spatulaful of sample D in a test tube, add distilled water, stir and divide the obtained solution into four portions in different test tubes. To the&&\\ \hline
&\begin{enumerate}
\item[(i)] first portion of the solution of sample D in a state tube add aqueous ammonia slowly till excess.
\end{enumerate}&&\\ \hline
&\begin{enumerate}
\item[(ii)] second portion of the solution of sample D in a test tube add aqueous ammonia slowly till excess.
\end{enumerate}&&\\ \hline
&\begin{enumerate}
\item[(iii)] third portion of the solution of sample D in a test tube add potassium hexacyanoferrate (II).
\end{enumerate}&&\\ \hline
&\begin{enumerate}
\item[(iv)] fourth portion of the solution of sample D in a test tube add dilute HCl followed by BaCl$_2$ solution.
\end{enumerate}&&\\ \hline
\end{tabular}\\

\end{center}


\textbf{Conclusion:}\\
\vspace{6pt}
The cation in sample D is \_\_\_\_ and the anion is \_\_\_\_.\\
The molecular formula of salt D is \_\_\_\_.\\

\raggedleft \textbf{(25 marks)}

\raggedright

\item[3.] Sample Y is a simple salt containing one anion and one cation. Using systematic qualitative analysis procedures carry out tests on sample Y and make appropriate observations and inferences to identify the cation and anion present in sample Y.\\

\begin{center}
\begin{tabular}{|p{4cm}|p{4cm}|p{4cm}|}
\hline
\textbf{Experiment}&\textbf{Observation}&\textbf{Inference}\\ \hline
&&\\
&&\\
&&\\
&&\\
\hline
\end{tabular}\\
\end{center}

\textbf{Conclusion:}\\
\vspace{6pt}
The cation in sample Y is \_\_\_\_ and the anion is \_\_\_\_.\\

\end{enumerate}

\raggedleft \textbf{(25 marks)}\\

\raggedright
