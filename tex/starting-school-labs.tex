\chapter{Starting School Laboratories}

A science laboratory is any place 
where students learn science with their hands. 
It might be a room, 
or just a box. 
The goal is to develop a space that facilitates hands-on learning.

\section{Benefits of a School Laboratory}
There are many benefits of having a laboratory:
\begin{itemize}
\item{Students learn more and better science}
\item{Students get more excited about science class}
\item{Students have to go to the lab for class, 
thus eliminating those too lazy to walk over}
\item{Practical exams are easier than the alternative-to-practical exams}
\item{Everyone thinks practicals are important, 
and that science without practicals is silly.}
\end{itemize}

\section{Challenges of a School Laboratory}
There are some challenges with having a laboratory:
\begin{itemize}
\item{They are places where people can get hurt\\
\textit{This is true. 
Please see the sections on \nameref{cha:classmanagement} and \nameref{cha:labsafety} 
to mitigate this risk.}}
\item{Many teachers do not know how to use a laboratory\\
\textit{Then use the lab to teach them how to use it, thus spreading skills.}}
\item{Laboratories are far too expensive for poor schools to build and stock\\
\textit{This is simply incorrect. 
Any room will work for a lab, 
and any school can afford the materials required to stock it. 
The rest of this book is dedicated to this point.}}
\end{itemize}

So you want to build a laboratory?

\section{Step one: Location}
A permanent location is obviously preferable. 
If your school has an extra classroom, 
great. 
The only requirements of a potential room are 
that it be well ventilated 
(have windows that either open or lack glass altogether) 
and be secure: bars in the windows, 
a sturdy door, 
and a lock. 
If you plan to put fancy equipment in your lab, 
remember that hack saw blades are cheap 
and that the latch through which many pad locks pass 
can be cut quickly regardless of the lock it holds. 
But if you are just starting, 
there will probably not be any fancy equipment; 
a simple lock is enough to keep overly excited students 
from conducting unsupervised experiments.

If there is no extra space at all, 
the lab can live in a few buckets and be deployed in a classroom 
during class time. 
``There is no lab room,'' is no excuse for not having a lab.

\section{Step two: Funding}
Yes, 
some is required. 
But the amount is surprisingly little -- 
in most countries a single month of a teacher's salary 
is enough to furnish a basic laboratory. 
Almost every school can find the amount required to get started, 
and if not the community certainly can. 
A single cow in most countries would pay 
for a basic laboratory many times over. 
A cow is valuable. 
So is science education.

We encourage you to resist the temptation 
to ask people outside of the school community or school system 
to pay for the lab. 
There is simply no need to encourage that sort of dependence; 
this can be done locally, and it should be.
