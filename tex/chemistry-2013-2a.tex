\section{2013 - CHEMISTRY 2A ACTUAL PRACTICAL A}

\begin{enumerate}
\item[1.] You are provided with the following solutions:
\begin{enumerate}
\item[ ] \textbf{JJ}: Containing 3.0 g of acetic acid in 0.50 dm$^3$ of solution;
\item[ ] \textbf{KK}: Containing 1.5 g of impure potassium hydroxide in 250 dm$^3$ of solution;
\item[ ] Phenolphthalein indicator.\\
\end{enumerate}

\textbf{Questions}:
\begin{enumerate}
\item[(a)] Is the use of methyl orange indicator in this experiment as suitable as phenolphthalein?\\
Give a reason for your answer.
\item[(b)] Titrate the acid (in a burette) against the base (in a conical flask) using two drops of your indicator and obtain three titre values.
\item[(c)] 
\begin{enumerate}
\item[(i)] \_\_\_\_ cm$^3$ of \textbf{JJ} required \_\_\_\_ cm$^3$ of \textbf{KK} for complete reaction.
\item[(ii)] Write a balanced chemical equation for the reaction between \textbf{JJ} and \textbf{KK}.
\end{enumerate}
\item[(d)] Showing your procedures clearly, calculate the percentage purity of potassium hydroxide.\\
\end{enumerate}

\item[2.] Your are provided with the following:
\begin{enumerate}
\item[ ] \textbf{L$_1$}:  0.50 M sodium thiosulphate;
\item[ ] \textbf{L$_2$}:  0.10 M hydrochloric acid;
\item[ ] Distilled water;
\item[ ] Stop watch;
\item[ ] Plain paper.
\end{enumerate}

\textbf{Theory}\\
When a solution of sodium thiosulphate is mixed with hydrochloric acid, they react quantitatively and gradually the solution becomes opaque.\\

\textbf{Procedure}\\
\vspace{-6pt}
\begin{enumerate}
\item[(i)] Write a clear letter X on a white piece of paper.
\item[(ii)] Place a 100 cm$^3$ beaker on top of letter X, such that the letter X is visible when viewed from above.
\item[(iii)] Using a measuring cylinder, measure 25 cm$^3$ of \textbf{L$_1$} and pour into the 100 cm$^3$ beaker in (ii) above.
\item[(iv)] Measure 25 cm$^3$ of \textbf{L$_2$} and pour it into the beaker containing solution \textbf{L$_1$} in (iii) above and immediately start the stop watch/clock.
\item[(v)] Shake the reaction mixture only once and record the time taken for the letter X to disappear completely.
\item[(vi)] Repeat steps (ii) to (v) by varying the volume of \textbf{L$_1$} and distilled water as indicated in Table 1.
\end{enumerate}

\newpage

\begin{center}
\begin{tabular}{|p{2.5cm}|p{2.5cm}|p{2.5cm}|p{2cm}|p{4cm}|}
\multicolumn{1}{l}{Table 1}&\multicolumn{1}{l}{ }&\multicolumn{1}{l}{ }\\ \hline
\textbf{Volume of \textbf{L$_1$} in cm$^3$}&\textbf{Volume of\newline water in cm$^3$}&\textbf{Volume of \textbf{L$_2$} in cm$^3$}&\textbf{Time (t)/s}&\textbf{Rate of reaction $\frac{1}{t}$(s$^-1$)}\\ \hline
25&0&25&&\\ \hline
20&5&25&&\\ \hline
15&10&25&&\\ \hline
10&15&25&&\\ \hline
5&20&25&&\\ \hline
\end{tabular}
\end{center}

\textbf{Questions:}
\begin{enumerate}
\item[(a)] What is the aim of this experiment?
\item[(b)] Complete Table 1.
\item[(c)] Write the electronic configuration of the product which causes the solution to cloud letter X.
\item[(d)] With state symbols, write the ionic equation for the reaction between \textbf{L$_1$} and \textbf{L$_2$}.
\item[(e)] Plot a graph of volume of \textbf{L$_1$} against rate of reaction.
\item[(f)] What can you conclude from the graph?\\
\end{enumerate}

\item[3.] Sample \textbf{U} contains one cation and one anion. Using systematic qualitative analysis procedures, record carefully your experiments, observations, inferences and finally identify the anion and cation present in sample \textbf{U}. Record your work in a tabular form as Table 2 shows.\\

\begin{center}
\begin{tabular}{|p{1cm}|p{5cm}|p{3cm}|p{3cm}|}
\hline
\textbf{S/n}&\textbf{Experiment}&\textbf{Observation}&\textbf{Inference}\\ \hline
&&&\\
&&&\\
&&&\\
&&&\\
&&&\\
&&&\\
\hline
\end{tabular}\\
\end{center}

\textbf{Conclusion}
\begin{enumerate}
\item[(i)] The cation in sample \textbf{U} is \_\_\_\_.
\item[(ii)] The anion in sample \textbf{U} is \_\_\_\_.
\end{enumerate}

\end{enumerate}