\section{2009 - BIOLOGY 2A (ALTERNATIVE A PRACTICAL)}

\begin{enumerate}
\item[1.] 
\begin{enumerate}
\item[(a)] You are provided with solution \textbf{S$_1$}. Carry out experiments to identify the food substances present in it. Record your procedure, observation and inferences as shown in the table below.

\begin{center}
\begin{tabular}{|p{3cm}|p{3cm}|p{3cm}|p{3cm}|} \hline
\multicolumn{1}{|c|}{\textbf{Test for}}&\multicolumn{1}{c|}{\textbf{Procedure}}&\multicolumn{1}{c|}{\textbf{Observation}}&\multicolumn{1}{c|}{\textbf{Inference}} \\ \hline
&&& \\
&&& \\
&&& \\
&&& \\ \hline
\end{tabular} \\[10pt]
\end{center}

\item[(b)]
\begin{enumerate}
\item[(i)] Name the food substances you have identified.
\item[(ii)] State \textbf{two (2)} sources of each food substance named in 1(b) (i) above.
\item[(iii)] Mention \textbf{one (1)} role of each food substance you have identified.
\end{enumerate}
\item[(c)] In which parts of the digestive system are the above mentioned food substances digested? In each case mention the enzyme and the products.
\end{enumerate}

\item[2.]
\begin{enumerate}
\item[(a)] Using a hand lens examine specimen A$_1$.
\begin{enumerate}
\item[(i)] Identify specimen A$_1$ by its common name.
\item[(ii)] Name the phylum and class to which specimen A$_1$ belongs.
\item[(iii)] Give an example of another organism which belongs to the same phylum as specimen A$_1$.
\end{enumerate}
\item[(b)] Draw a well labelled diagram of specimen A$_1$.
\item[(c)] How is specimen A$_1$ adapted to its mode of nutrition?
\item[(d)] What is the economic importance of specimen A$_1$?
\item[(e)] Where can specimen A$_1$ be found?
\end{enumerate}
\end{enumerate}
