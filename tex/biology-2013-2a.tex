\section{2013 - BIOLOGY 2A (ACTUAL PRACTICAL A)}

\begin{enumerate}
\item[1.] You have been provided with solution \textbf{B}.
\begin{enumerate}
\item[(a)] Identify the food substances present in solution \textbf{B} by using the reagents provided. Tabulate your work as shown in the following Table:

\begin{center}
\begin{tabular}{|p{3cm}|p{3cm}|p{3cm}|p{3cm}|} \hline
\multicolumn{1}{|c|}{\textbf{Food Tested}}&\multicolumn{1}{c|}{\textbf{Procedure}}&\multicolumn{1}{c|}{\textbf{Observation}}&\multicolumn{1}{c|}{\textbf{Inference}} \\ \hline
&&& \\
&&& \\
&&& \\
&&& \\ \hline
\end{tabular} \\[10pt]
\end{center}

\item[(b)] For each food substance identified in 1(a);
\begin{enumerate}
\item[(i)] Name two common sources.
\item[(ii)] State their role in the body of human being.
\end{enumerate}
\item[(c)] The digestion of one of the identified food substance in 1(a) starts in the mouth.
\begin{enumerate}
\item[(i)] Name this food substance.
\item[(ii)] Identify the enzyme responsible for its digestion in the mouth.
\end{enumerate}
\item[(d)] The digestive system of human being has several parts.
\begin{enumerate}
\item[(i)] Name the part of digestive system in which most of digestion and absorption of food takes place.
\item[(ii)] Explain how the named part in (d) (i) is adapted for absorption of digested food substances.
\end{enumerate}
\end{enumerate}

\item[2.] You have been provided with specimens \textbf{S$_1$}, \textbf{S$_2$}, \textbf{S$_3$} and \textbf{S$_4$}.
\begin{enumerate}
\item[(a)] Use the hand lens to observe these specimens then:
\begin{enumerate} 
\item[(i)] Identify specimen \textbf{S$_1$}, \textbf{S$_2$}, \textbf{S$_3$} and \textbf{S$_4$} by their common names.
\item[(ii)] Classify specimen \textbf{S$_1$}, \textbf{S$_2$} and \textbf{S$_3$} to Class level.
\end{enumerate}
\item[(b)] Study specimen \textbf{S$_1$} carefully then answer the following questions:
\begin{enumerate}
\item[(i)] Draw a neat, large and well labeled diagram of specimen \textbf{S$_3$}.
\item[(ii)] State the habitat of specimen \textbf{S$_3$}.
\item[(iii)] In what ways is specimen \textbf{S$_3$} important to a farmer?
\end{enumerate}
\item[(c)] State two advantages of specimen \textbf{S$_1$}.
\item[(d)] State four advantages of specimen \textbf{S$_4$}.
\item[(e)] Give reason why specimen \textbf{S$_4$} was formally placed in the Kingdom Plantae?
\end{enumerate}
\end{enumerate}