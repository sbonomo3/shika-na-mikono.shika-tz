\chapter{Science Competitions} \index{Science! competitions}
\label{cha:sci-comp}
\setcounter{secnumdepth}{0}

Students, just like nearly all other people, enjoy competing against one another. Likening math and science-related activities to the competition of a football league can be a wonderful motivator for students. Given below are some suggestions for utilizing competitions while teaching about science.

\begin{itemize}
\item Combine students of various abilities together on a team. This will allow the bright students to develop leadership skills and help to bring up the slow learners.
\item Limit teams to 3-5 students. Balance the number of boys and girls on a team, or choose to have all-boys teams compete against all-girls teams.
\item Allow students to pick their own team name, and possibly draw a team flag if time allows.
\item Create a standings board for the competition. For example:\\

\begin{table}[h!]
	\centering
		\begin{tabular}{|l|l|l|l|l|l||l|}\hline
		& \rotatebox{90}{
		\begin{tabular}{l}
		Egg Drop
		\end{tabular}
		}
		& \rotatebox{90}{
		\begin{tabular}{l}
		Jenga Jengo
		\end{tabular}
		}
		& \rotatebox{90}{
		\begin{tabular}{l}
		Raft Rally
		\end{tabular}
		}
		& \rotatebox{90}{
		\begin{tabular}{l}
		Drop Zone
		\end{tabular}
		}
		& \rotatebox{90}{
		\begin{tabular}{l}
		Bridge Challenge
		\end{tabular}
		}
		& \rotatebox{90}{
		\begin{tabular}{l}
		\textbf{Total}
		\end{tabular}
		} \\ \hline
		Big Stars &&&&&&\\ \hline
		Chelsea &&&&&&\\ \hline
		Simba &&&&&&\\ \hline
		Arsenal &&&&&&\\ \hline
		Kings &&&&&&\\ \hline
		Manchester United &&&&&&\\ \hline
		\end{tabular}
\end{table}

\item Have teams present and explain their designs to the audience where applicable. Allow other students to ask questions/provide criticisms.
\item Follow up each activity with a short lesson about a concept illustrated by the competition (e.g. Archimedes' Principle for \emph{Raft Rally}; see competition write-ups for more).
\item Ask students how they would revise their designs or make improvements if they could do the activity again.
\item Explain how the activities can be applied to solve real-life problems (e.g. \emph{Jenga Jengo} for Civil Engineers).
\end{itemize}

\vfill
\pagebreak

%=================================================================================================%

\section{Egg Drop} \index{Egg drop}
\label{sec:egg-drop}


\textbf{Time:} 1 hour

\subsection{How It Works}
Students must build a device to transport an egg through a given drop distance without cracking.

\subsection{What You Need (per team)}
\begin{tabular}{p{0.33\textwidth} p{0.33\textwidth} p{0.33\textwidth}}
$\Box$ Plain paper – 4 sheets	&	$\Box$ Straws – 10		&	$\Box$ Paper clips - 10 \\
$\Box$ Plastic bag - 2		&	$\Box$ Toothpicks – 10		&	$\Box$ Toilet paper – 1 roll \\
$\Box$ String – 1 metre	&		$\Box$ Tongue depressors – 4	&	$\Box$ Bottle (500 ml) - 1 \\
$\Box$ Masking tape – 1 roll	&	$\Box$ Rubber bands – 4	&		$\Box$ Newspaper – 1 sheet \\
$\Box$ Balloons – 4		&	$\Box$ Index cards – 4		&	$\Box$ Egg – 1
\end{tabular}

\subsection{Rules}
\begin{itemize*}
\item 45 minute time limit for construction.
\item Devices dropped from a height of 3-5 metres.
\item Teams can use only the materials given, but do not need to use everything.
\item Egg is placed at time of testing.  It must be possible to place and remove egg freely without altering the device.
\item Once egg is placed, no further adjustments may be made. This means the egg cannot have any kind of ``seat belt'' or strap fastened after placing the egg.
\end{itemize*}

\subsection{Points}
Egg Survives - 50 pts \quad Egg Cracks - 0 pts

\subsection{Additional Materials}
\begin{itemize*}
\item Ladder / chair
\item Scissors for community use
\end{itemize*}

\subsection{Notes}
\begin{itemize*}
\item Do not give eggs to teams until time of testing.
\item If possible, increase drop height for surviving eggs and give bonus 25 pts for each additional successful drop.
\end{itemize*}

\subsection{Science Applications}
\begin{description}
\item[Air Resistance (Physics Form I):]{Air resistance provides a frictional force which opposes the object's motion as gravity attracts it towards the centre of the earth. This upward force reduces the speed of the object as it falls, allowing it to land more softly and protect the egg. Thus, we want to maximize the air resistance on the object (e.g. by using a large parachute).}
\item[Pressure (Physics Form I):]{The force of impact on the device when it hits the ground can be reduced by increasing the surface area which contacts the ground. Constructing a wide base (e.g. using balloons) reduces the impact on the egg and thus helps to protect it.}
\end{description}

\subsection{Taking It Further}
Did students utilize the parachute concept? If not, show a brief example. How does this help to protect the egg? Is it better to have a large parachute or a small one?

\pagebreak

%=================================================================================================%

\section{Jenga Jengo} \index{Jenga Jengo}
\label{sec:jenga-jengo}


\textbf{Time:} 30-45 minutes

\subsection{How It Works}
Students must build the tallest structure possible, using only paper and tape, as quickly as possible and while ensuring good stability.

\subsection{What You Need (per team)}
$\Box$ Plain paper – 25 sheets\\
$\Box$ Masking tape – 1 roll

\subsection{Rules}
\begin{itemize*}
\item 20-minute time limit for construction.
\item Cannot use tape roll as weight inside structure.
\item Stability tested by waving book at structure (Wind Test).
\end{itemize*}

\subsection{Points}
1$^{st}$ to finish – 50 pts\\
Tallest structure – 50 pts\\
Passes Wind Test – 50 pts

\subsection{Additional Materials}
\begin{itemize*}
\item Tape Measure
\item Stopwatch
\item Book / waving device
\end{itemize*}

\subsection{Notes}
\begin{itemize*}
\item All structures that pass the Wind Test are awarded 50 pts.
\item Alternate Wind Test: place structures outside on a windy day. Those standing after 1 minute pass.
\end{itemize*}

\subsection{Science Applications}
\begin{description}
\item[Centre of Gravity (Physics Form I):]{Civil engineers construct buildings with a low centre of gravity, making them less likely to fall over due to wind forces. To maintain \emph{stable equilibrium}, a building should have a wide base with a large mass, while the top of the building should have a small area and less mass.}
\end{description}

\subsection{Taking It Further}
\begin{itemize*}
\item Show students pictures of buildings and structures from around the world after the competition. Did the students’ structures resemble any of them?
\item Try variations, giving students index cards, straws or matches instead of plain paper.
\end{itemize*}

\pagebreak

%=================================================================================================%

\section{Raft Rally} \index{Raft Rally}
\label{sec:raft-rally}


\textbf{Time:} 30-45 minutes

\subsection{How It Works}
Students must build a raft using only aluminum foil that can support the heaviest load before sinking.

\subsection{What You Need (per team)}
$\Box$ Aluminum foil – 20 cm $\times$ 20 cm sheet\\
$\Box$ Straws - 4 (optional)

\subsection{Rules}
\begin{itemize*}
\item 10-minute time limit for construction.
\item Replacement sheet may be given in case of rips/tears, at a 20 pt deduction.
\end{itemize*}

\subsection{Points}
1$^{st}$ Place – 100 pts \\
2$^{nd}$ Place – 75 pts \\
3$^{rd}$ Place – 50 pts \\
4$^{th}$ Place – 25 pts \\
Others – 0 pts

\subsection{Additional Materials}
\begin{itemize*}
\item Large container or bucket (clear if possible) filled with water
\item Nails ($\times$ 200) / Bottle caps ($\times$ 200) / Other small weights for testing
\end{itemize*}

\subsection{Notes}
\begin{itemize*}
\item As raft approaches the point of sinking, add weights more slowly.
\item Raft is finished when water begins to enter, and total number of weights is recorded.
\end{itemize*}

\subsection{Science Applications}
\begin{description}
\item[Archimedes' Principle (Physics Form I):]{\emph{Archimedes' Principle} states that $$\mathrm{Upthrust} = \text{Weight of displaced fluid}$$
Here, we want to maximize the force of upthrust to avoid sinking. So that means maximize the Weight of the displaced water: $\mathrm{Weight} = \mathrm{mass} \times \text{acceleration due to gravity}$, or					 $$W=mg$$
Gravity is a constant , but mass depends on 2 things: density ($\rho$) and volume ($V$). We know that
$\rho = \frac{m}{V}$, so that means				           $$m = \rho V$$
The density of the water is constant, so the only thing we can change is the \emph{Volume} of water displaced. Thus to get the most upthrust and prevent sinking, we need to displace a large volume of water, i.e. build a raft with a large base.}
\end{description}

\subsection{Taking It Further}
\begin{itemize*}
\item Ask students how they would revise their designs if they could do it again.
\item Try variations, giving students straws, toothpicks, tongue depressors or index cards.
\end{itemize*}

\pagebreak

%=================================================================================================%

\section{Drop Zone} \index{Drop Zone}
\label{sec:drop-zone}


\textbf{Time:} 30-45 minutes

\subsection{How It Works}
Students must build a parachute using limited materials to carry a paper clip passenger as close as possible to a target, while maximizing hang time.

\subsection{What You Need (per team)}
\begin{tabular}{p{0.33\textwidth} p{0.33\textwidth} }
$\Box$ Paper clip – 1	&		$\Box$ Plastic bag - 2 \\ 
$\Box$ String – 1 metre		&	$\Box$ Plain paper – 2 sheets \\ 
$\Box$ Masking tape – 15 cm	&	$\Box$ Scorecard (see example below) \\ 
\end{tabular}

\subsection{Rules}
\begin{itemize*}
\item 10-15 minute time limit for construction.
\item Parachutes dropped from a height of 3-5 metres.
\item Average hang time and distance from target taken over 3 trials for each team.
\end{itemize*}

\subsection{Points}
\begin{tabular}{llllll}
Hang Time (Longest):& 1$^{st}$ – 50 pts,&	2$^{nd}$ – 35 pts,&	3$^{rd}$ – 20 pts,& 	4$^{th}$ – 5 pts,& 	Others – 0 pts\\
Distance (Shortest):& 1$^{st}$ – 50 pts,&	2$^{nd}$ – 35 pts,&	3$^{rd}$ – 20 pts,& 	4$^{th}$ – 5 pts,& 	Others – 0 pts
\end{tabular}

\subsection{Additional Materials}
\begin{tabular}{p{0.3\textwidth} p{0.3\textwidth}} \\[-30pt]
\begin{itemize*}
\item Tape measure		
\end{itemize*}	& \begin{itemize*}
\item Flipchart target
\end{itemize*}\\[-30pt]
\begin{itemize*}
\item Stopwatch	
\end{itemize*}	& \begin{itemize*}
\item Ladder / chair
\end{itemize*}\\[-10pt]
\end{tabular}

\subsection{Notes}
\begin{itemize*}
\item Scorecard:
\end{itemize*}
\begin{tabular}{|l|c|c|c|c|c|} \hline
\textbf{Team:} & \textbf{Trial 1} & \textbf{Trial 2} & \textbf{Trial 3} & \textbf{Average} & \textbf{Points} \\ \hline
\textbf{Hang Time (s)} & & & & & \\ \hline
\textbf{Distance from Target (cm)} & & & & & \\ \hline
\end{tabular}
\begin{itemize*}
\item Measure distance from paper clip to centre of target.
\end{itemize*}

\subsection{Science Applications}
\begin{description}
\item[Air Resistance (Physics Form I):]{Air resistance provides an upward force on the parachute, which acts against the force of gravity and causes the object to fall more slowly. The larger the surface area of the parachute, the more slowly it will fall.}
\end{description}

\subsection{Taking It Further}
\begin{itemize*}
\item Students may not be familiar with parachutes. Prepare a simple example to explain the concept and function.
\item Ask students questions: Why does the parachute slow the object down? To maximize hang time, do we want a very large or very small parachute? Would a parachute work on the moon?
\item Drop parachute side-by-side with a paper clip having no parachute. Which one made it safely?
\end{itemize*}

\pagebreak

%=================================================================================================%

\section{Bridge Challenge} \index{Bridge Challenge}
\label{sec:bridge-challenge}


\textbf{Time:} 1 hour 30 minutes

\subsection{How It Works}
Students must build a bridge that can support the most weight, while using a limited budget of \emph{\nameref{cha:sci-shillings}} to purchase construction materials.

\subsection{What You Need (per team)}
\begin{tabular}{p{0.3\textwidth} p{0.3\textwidth} p{0.4\textwidth}}
$\Box$ Straws - 20		&	$\Box$ String – 3 m	&	$\Box$ Masking tape – 1 roll\\
$\Box$ Bamboo skewers - 20	&	$\Box$ Office glue – 1 tube	& $\Box$ Duct tape – 1 roll (total)\\
$\Box$ Bamboo stick (fimbo) – 1	&	$\Box$ Rubber bands - 10	&	$\Box$ Ruler - 1\\
$\Box$ Tongue depressors – 10	&	$\Box$ Pencils - 2		&	$\Box$ Scissors - 1\\
$\Box$ Toothpicks – 2 small cans	&	$\Box$ Index cards – 10	&	$\Box$ \nameref{cha:sci-shillings} – 20 (see p. \pageref{cha:sci-shillings})
\end{tabular}

\subsection{Rules}
\begin{itemize*}
\item Approximately 45 minute time limit for construction.
\item Teams begin with only a ruler, scissors and 20 \emph{Science Shillings}. These items may NOT be used in construction of bridge.
\item All building materials must be purchased from a science shop. Suggested prices are as follows:\\

\begin{tabular}{p{0.25\textwidth} p{0.1\textwidth} p{0.25\textwidth} p{0.1\textwidth}}
Straws (bundle of 10)	& 1 /=	&	Office glue (1 tube)	&	3 /=\\
Skewers (bundle of 10) &	2 /=	&	Rubber bands ($\times$ 5)	&	1 /=\\
Fimbo		&		3 /=	&	Pencil 		&		1 /=\\
Tongue depressors ($\times$ 5)	 & 2 /=	&	Index cards ($\times$ 5)	&	1 /=\\
Toothpicks (2 small cans) &	1 /=	&	Masking tape (1 roll)	&	3 /=\\
String (1 metre)	&	1 /=	&	Duct tape (30 cm)	&	1 /=\\[10pt]
\end{tabular}

\item Bridges will be loaded by placing rocks or other weights into a small bucket that must rest on top of the bridge.
\item Bridge must span a 30 cm gap between two chairs / tables.
\item 1 student from each team must be designated as team accountant. Only this student may purchase items from the shop.
\end{itemize*}

\subsection{Points}
(Based on number of rocks / weights placed before bridge fails)\\
1$^{st}$ – 100 pts, \quad	2$^{nd}$ – 75 pts, \quad	3$^{rd}$ – 50 pts,\quad 	4$^{th}$ – 25 pts,\quad 	Others – 0 pts\\
\textbf{BONUS}: 5 pts per \emph{Science Shilling} remaining after construction

\subsection{Additional Materials}
\begin{tabular}{p{0.5\textwidth} p{0.5\textwidth}} \\[-30pt]
\begin{itemize}
\item Small bucket 		
\end{itemize}	& \begin{itemize}
\item Science shop table
\end{itemize}\\[-30pt]
\begin{itemize*}
\item Several large rocks (20) / other large weights 		
\end{itemize*}	& \begin{itemize*}
\item Extra bamboo sticks (fimbos) – 2-4
\end{itemize*}\\[-30pt]
\begin{itemize*}
\item 2 chairs / tables 30 cm apart		
\end{itemize*}	& \begin{itemize*}
\item Index Cards for price signs – 12
\end{itemize*}\\[-10pt]
\end{tabular}

\subsection{Notes}
\begin{itemize*}
\item Student accountants may only purchase 1 of each item at a time. They must allow other students to make purchases before buying another of that item.
\item At some point 15-20 minutes into construction time, shopkeeper may announce a newly received shipment of bamboo sticks (fimbos). However, due to demand, the price has increased to 5 /=.
\item Bridge has failed when it either collapses / breaks or when the bucket can no longer be balanced on top of it. Record largest number of weights successfully added.
\end{itemize*}

%\subsection{Science Applications}
%\begin{description}
%\item[]{}
%\end{description}

\subsection{Taking It Further}
\begin{itemize*}
\item Ask students to present their bridges and describe how they decided to manage their money. What materials did they purchase and why?
\item What would they do differently if they could start again?
\item Which team finished with the most total points after the \textbf{BONUS}? Did the most money spent result in the strongest bridge? Who was the most \emph{efficient} with their money?
\item Show students pictures of bridges from around the world after the competition. Did the students’ bridges resemble any of them?
\end{itemize*}

%\pagebreak

%=================================================================================================%

%\section{Paper Airplanes}

%=================================================================================================%

%\section{Card Tower}

%=================================================================================================%

%\section{Write It, Do It}

%=================================================================================================%

