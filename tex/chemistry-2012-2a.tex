\section{2012 - CHEMISTRY 2A ACTUAL PRACTICAL A}

\begin{enumerate}
\item[1.] You are provided with the following solution:\\

\textbf{TZ}: Containing 3.5 g of impure sulphuric acid in 500 cm$^3$ of solution;\\
\textbf{LO}: Containing 4 g of sodium hydroxide in 1000 cm$^3$ of solution;\\
Phenolphthalein and Methyl indicators.\\[10pt]

\textbf{Questions}:\\
\begin{enumerate}
\item[(a)] 
\begin{enumerate}
\item[(i)] What is the suitable indicator for the titration of the given solutions?\\
Give a reason for your answer.
\item[(ii)] Write a balanced chemical equation for the reaction between \textbf{TZ} and \textbf{LO}.
\item[(iii)] Why is it important to swirl or shake the contents of the flask during the addition of the acid?\\
\end{enumerate}

\item[(b)] Titrate the acid (in a burette) against the base (in a conical flask) using two drops of your indicator and obtain three titre values.\\

\item[(c)] 
\begin{enumerate}
\item[(i)] \_\_\_\_ cm$^3$ of acid required \_\_\_\_ cm$^3$ of base for complete reaction.
\item[(ii)] Showing your procedures clearly, calculate the percentage purity of \textbf{TZ}.
\end{enumerate}

\end{enumerate}
\raggedleft \textbf{(20 marks)}

\raggedright

\item[2.] Your are provided with the following materials:\\
\begin{enumerate}
\item[ ] \textbf{ZO}:  A solution of 0.13 M Na$_2$S$_2$O$_3$ (sodium thiosulphate);
\item[ ] \textbf{UU}:  A solution of 2 M HCl;
\item[ ] Thermometer;
\item[ ] Heat source/burner;
\item[ ] Stopwatch.\\
\end{enumerate}

Procedure:\\
\begin{enumerate}
\item[(i)] Place 500 cm$^3$ beaker, which is half-filled with water, on the heat source as a water bath.
\item[(ii)] Measure 10 cm$^3$ of \textbf{ZO} and 10 cm$^3$ of \textbf{UU} into two separate test tubes.
\item[(iii)] Put the two test tubes containing \textbf{ZO} and \textbf{UU} solutions into a water bath.
\item[(iv)] When the solutions attain a temperature of 60$^o$C, remove the test tubes from the water bath and pour both solutions into 100 cm$^3$ empty beaker and immediately start the stop watch.
\item[(v)] Place the beaker with the contents on top of a piece of paper marked \textbf{X}.
\item[(vi)] Note the time taken for the mark \textbf{X} to disappear.
\item[(vii)] Repeat step (i) to (vi) at temperature 70$^o$C, 80$^o$C and 90$^o$C.
\item[(viii)] Record your results as in Table 1.
\end{enumerate}

\newpage

\begin{center}
\begin{tabular}{|p{5cm}|p{5cm}|p{3cm}|}
\multicolumn{1}{l}{Table 1}&\multicolumn{1}{l}{ }&\multicolumn{1}{l}{ }\\ \hline
\textbf{Experiment}&\textbf{Temperature}&\textbf{Time (s)}\\ \hline
1&60$^o$C&\\ \hline
2&70$^o$C&\\ \hline
3&80$^o$C&\\ \hline
4&90$^o$C&\\ \hline
\end{tabular}
\end{center}

\textbf{Questions:}\\
\begin{enumerate}
\item[(a)] Write a balanced chemical equation for reaction between \textbf{UU} and \textbf{ZO}.
\item[(b)] What is the product which causes the solution to cloud the letter \textbf{X}?
\item[(c)] Plot a graph of temperature against time (s).
\item[(d)] What conclusion can you draw from you graph?\\
\end{enumerate}

\raggedleft \textbf{(15 marks)}

\raggedright


\item[3.] Substance \textbf{V} is a simple salt which contains one cation and one anion. Carry our the experiments described below. Record carefully your observations and make appropriate inferences and hence identify the anion and cation present in sample \textbf{V}.\\

\begin{center}
\begin{tabular}{|l|p{8cm}|l|l|}
\hline
\textbf{S/n}&\textbf{Experiment}&\textbf{Observation}&\textbf{Inference}\\ \hline
1&Observe the appearance of sample \textbf{V}.&&\\ \hline
2&Put a little amount of sample \textbf{V} in a test tube then add water and shake.&&\\ \hline
3&Heat a little amount of \textbf{V} in a dry test tube.&&\\ \hline
{\multirow{4}{*}{4}}&To a little sample \textbf{V} in a test tube add dilute Hydrochloric acid. Add more of the acid until the test tube is half full. Divide the resulting solution into three portions and add the following:&&\\ \cline{2-4}
&\begin{enumerate}
\item[a)] To the one portion add NaOH solution drop wise then excess.
\end{enumerate}&&\\ \cline{2-4}
&\begin{enumerate}
\item[b)] To the second portion add ammonia solution drop wise then in excess.
\end{enumerate}&&\\ \cline{2-4}
&\begin{enumerate}
\item[c)] To the third portion add ammonium oxalate solution.
\end{enumerate}&&\\ \hline
5&Perform flame test.&&\\ \hline
\end{tabular}\\

\end{center}

Conclusion\\

\begin{enumerate}
\item[(i)] The cation in sample \textbf{V} is \_\_\_\_.\\
\item[(ii)] The anion in sample \textbf{V} is \_\_\_\_.\\
\item[(iii)] The chemical formula of \textbf{V} is \_\_\_\_.\\
\item[(iv)] The name of compound \textbf{V} is \_\_\_\_.\\
\end{enumerate}


\raggedleft \textbf{(15 marks)} \pagebreak

\raggedright

\end{enumerate}