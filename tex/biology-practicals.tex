\chapter{Biology Practicals}

\section{Introduction to Biology Practicals}

\subsection{Format}

Until 2008, NECTA biology practicals contained three questions. Question 1 was required, and was a food test. Students then chose to answer either question 2 or question 3. One of these questions was usually classification.

The format changed in 2008. Now, the practical contains two questions, and both are required. Food test and classification remain the most common questions, but sometimes only one of these two topics is on a given exam. The second question may cover one of a variety of topics, including respiration, transport, coordination, photosynthesis, and movement.

Each question is worth 25 marks.

\subsection{Common Practicals}
\begin{itemize}
\item{Food test: students must test a solution for starch, sugars, fats, and protein}
\item{Classification: students must name and classify specimens, then answer questions about their characteristics}
\item{Respiration: students use lime water to test air from the lungs for carbon dioxide}
\item{Transport: students investigate osmosis by placing leaf petioles or pieces of raw potato in solutions of different solute concentrations}
\item{Photosynthesis: students test a variegated leaf for starch to prove that chlorophyll is necessary for photosynthesis}
\item{Coordination: students look at themselves in the mirror and answer questions about the sense organs they see}
\item{Movement: students name bones and answer questions about their structure and position in the body}
\end{itemize}

Note: These are the most common practicals, but they are not necessarily the only practicals that can occur on the national exam. Biology practicals frequently change, and it is possible that a given exam will contain a new kind of question. Look through past NECTA practicals yourself to get an idea of the kind of questions that can occur

%==============================================================================
\section{Food Tests}

In this practical, students test a solution of unknown food substances for starch, protein, reducing sugars, non-reducing sugars, and fats/oils. They record their procedure, observation, and conclusions, then answer questions about nutrition and the digestive system.

This section contains the following:
\begin{itemize}
\item{Preparation of test solutions}
\item{Preparation of food solutions}
\item{How to carry out food tests}
\item{How to write a report}
\item{Sample practical with solutions}
\end{itemize}

\subsection{Test for Starch}

\subsubsection{Materials}
\begin{itemize}
\item{iodine tincture from the pharmacy (any brand)}
\item{tap or clean river water}
\end{itemize}

\subsubsection{Procedure to make 400 mL}
\begin{enumerate}
\item{40 mL of iodine tincture to a 500 mL plastic water bottle.}
\item{Add about 400 mL of water.}
\item{Cap the bottle and shake.}
\item{Use a permanent pen to label the bottle:\\
IODINE SOLUTION FOR FOOD TESTS}

The solution may be stored in any plastic or glass bottle and will keep indefinitely.
\end{enumerate}

\subsection{Test for protein}

The best test for protein is the Biuret test. This requires two solutions: 1\% CuSO4 and 1 M NaOH.

\subsubsection{Materials}
\begin{itemize}
\item{copper sulfate}
\item{sodium hydroxide}
\item{tap or clean river water}
\end{itemize}

\subsubsection{Procedure to make 1 liter copper sulfate solution}
\begin{enumerate}
\item{Use a small metal or plastic spoon (tea size) to transfer one level spoon of copper sulfate into a 1 or 1.5 liter water bottle.}
\item{Add about one liter of water. The amount does not have to be exact.}
\item{Cap the bottle and shake until the copper sulfate has completely dissolved.}
\item{Use a permanent pen to label the bottle:\\
1\% Copper (II) Sulfate Solution\\
For food tests}
\end{enumerate}

The solution may be stored in any plastic or glass bottle and will keep indefinitely.

\subsubsection{Procedure to make 250 mL of sodium hydroxide solution}
\begin{enumerate}
\item{Use a small PLASTIC spoon (tea size) to transfer one level spoon of sodium hydroxide into a 500~mL water bottle. Caustic soda (sodium hydroxide) reacts with metal. DO NOT TOUCH.

\textit{SAFETY NOTE: prepare about 100~mL of citric acid or ethanoic acid solution to have available to neutralize sodium hydroxide spills on skin or lab tables. One spoon of citric acid in about 100~mL of water is a suitable concentration. Ethanoic acid solutions of the proper concentration are sold in food shops as vinegar.}

}%item

\item{Add about 250 mL of water to the bottle. In the new 500~mL Maji Africa bottles, this is the first straight line above the curving lines. The addition of water to sodium hydroxide gets HOT.}
\item{Cap the bottle very well and shake to mix.}
\item{Use a permanent pen to label the bottle:\\
1M SODIUM HYDROXIDE SOLUTION FOR FOOD TESTS\\
CORROSIVE. Neutralize spills with weak acid soolution.}
\end{enumerate}

The hydroxide solution will react with carbon dioxide in the air if the container is not well sealed. The solution should not be stored in glass bottles with glass stoppers for overnight or longer as these will stick. The solution may be stored in plastic bottles indefinitely.

\subsection{Test for lipids}

\subsubsection{Materials}
\begin{itemize}
\item{Provodine iodine tincture from the pharmacy -- the tincture must be without ethanol / alcohol / “spiriti”}
\item{tap or clean river water}
\end{itemize}

\subsubsection{Procedure to make 400 mL}
See instructions for preparing iodine solution for Test for Starch above.

\subsubsection{A note on theory}

Many biology books call for a chemical called Sudan III to test for lipids. Sudan III is a bright red pigment that is much more soluble in oil than in water. For this reason, Sudan III solution is usually prepared using ethanol to bring the Sudan III pigment into the solution. In mixtures of oil and water, the oil separates and moves to the top. When shaken with Sudan III, this oil absorbs the Sudan III, turns red, and produces a "red ring" at the top of the test tube. However, the ethanol used to make Sudan III causes the water and oil to form an emulsion. In an emulsion, the oil is broken into very small particles and it takes a long time for this emulsion to break down and form an oil layer on the top. Hence testing with Sudan III takes a long time to show a clear result.

Iodine is another coloured molecule that is more soluble in oil than in water. When a mixture of oil and water is shaken with iodine solution, the iodine moves to the oil layer, colouring it orange or red. This also gives the result of a "red ring" at the top of the test tube. To prevent an emulsion forming -- as happens with Sudan III -- it is very important to make iodine solution from pharmacy tincture that is without ethanol. Another benefit of using iodine is that while Sudan III is always red, iodine is uniquely yellow in water and red in oil, making the difference between positive and negative results easier to see. Because there is no ethanol in iodine solution, the result also comes much faster, usually within 10-20 seconds.

Note that if the oil and water mixture settles before you transfer it to the test tube, there may be too little or too much oil in the test tube. Shake the food sample solution before taking each sample.

\subsection{Test for reducing sugars with Benedict's solution}

\subsubsection{Materials}
\begin{itemize}
\item{sodium carbonate (soda ash) if available, otherwise sodium hydrogen carbonate (bicarbonate of soda)}
\item{citric acid}
\item{copper sulfate}
\item{tap or clean river water}
\item{plastic water bottle with screw cap}
\end{itemize}

\subsubsection{Procedure to make 1 litre using soda ash}
Combine the following into the bottle in order, adding slowly:
\begin{enumerate}
\item{about 1 litre water}
\item{five spoons of sodium carbonate}
\item{three spoons of citric acid}
\item{one spoon of copper sulfate.}
\end{enumerate}

All measurements with the spoon should be level to ensure that the volume measured is consistent. The final solution should have a bright blue color.

\subsubsection{Procedure to make 500 mL using sodium hydrogen carbonate}
\begin{enumerate}
\item{Add 500 mL of water to a cooking pot.}
\item{Add a box (70 g) of bicarbonate of soda to the water.}
\item{Heat the pot on a stove until boiling. Let boil for ten minutes.}
\item{Remove the pot from the stove and let cool. When cool, transfer the liquid to a plastic water bottle.}
\item{Slowly add one and a half spoons of citric acid.}
\item{Add half a spoon of copper sulfate. Cap and shake to mix.}
\end{enumerate}

Label the final solution:

\begin{center}	
BENEDICT'S SOLUTION FOR FOOD TESTS.
\end{center}

The solution may be stored in any plastic or glass bottle and will keep indefinitely.

\subsection{Test for non-reducing sugar}

\subsubsection{Materials}
\begin{itemize}
\item{Benedict's solution, above}
\item{\~1M sodium hydroxide solution, above under protein test}
\item{citric acid}
\item{tap or clean river water}

To perform this test, students first test their sample with Benedict's solution to eliminate the possibility of a reducing sugar. Then, they must use an acid solution to hydrolyze any non-reducing sugars and then a base solution to neutralize the sample solution. Then the students perform again the test for non-reducing sugars.

\subsubsection{Procedure to make 500ml of acid solution}
\begin{enumerate}
\item{Put 500 mL of water into a plastic water bottle.}
\item{Add five spoons of citric acid.}
\item{Cap the bottle very well and shake to mix.}
\item{Use a permanent pen to label the bottle:\\
0.5 M CITRIC ACID FOR FOOD TESTS}
\end{enumerate}

The solution may be stored indefinitely in any plastic or glass container.

\subsection{Preparation of Food Sample Solutions}

Try to make food solutions colorless so that students cannot guess what is in them, and so that they can see the colors that form during food tests. You do not need to measure the ingredients of the solution, but make sure to test solutions before the practical.

\subsubsection{Starch solution}
The easiest solution is the water left from boiling pasta or potatoes. If you have been only ugali and rice lately, add some wheat or corn flour to boiling water. Let cool. Decant the solution or filter it with a tea filter to remove the largest particles. If you are in a hurry, you can also just mix flour with cold water, but then it will be obvious that flour is present.

\subsubsection{Lipids}
Add cooking oil to water. Shake immediately before use. Sunflour oil is best – avoid fats that solidify near room temperature.

\subsubsection{Protein solution}
Combine egg whites with water. If you do not have egg whites, you can use fresh or powder milk, although this will give the solution a white color and add a reducing sugar (lactose). The water used in boiling beans also contains some protein, but it may not be enough to see a color change.

\subsubsection{Reducing sugar}
The easiest is to buy glucose powder from a shop and dissolve it in water. You can also grind pieces of onion with water and filter the resulting solution.

\subsubsection{Non-reducing sugar}
Dissolve sugar in water. Table sugar is sucrose, a non-reducing sugar.

\subsection{How to Carry Out Food Tests}

\subsubsection{Starch}
Add a few drops of iodine to the solution and shake well. A blue-black color forms if starch is present

\subsubsection{Lipids}
Add a few drops of iodine to the solution and shake well. A red ring will form at the top of the test tube is lipids are present.

You can also have your students do the grease spot test -- rub a drop of solution onto a piece of paper, and let dry. A translucent spot forms if fat is present. This test is great for its simplicity, but is not used on national exams.

\subsubsection{Protein (Biuret test)}
Add a few drops of 1 M NaOH to the solution and shake well. Then add a few drops of 1\% CuSO4 solution and shake. A violet color forms if protein is present. Sometimes the color takes a minute or two to appear. 

Some textbooks may recommend using Millon's reagent to test for protein. This reagent contains mercury, which is extremely poisonous and should never be handled by students.

The purple colour from a positive test is the result of a complex between four nitrogen atoms and the copper (II) ion. Specifically, these nitrogen atoms are all part of peptide bonds. These peptide bonds are adjacent on a protein, either two from one protein and two from another, or two from one part of a protein and two from another part of the same protein.

\subsubsection{Reducing sugar}
Place some food solution in a test tube, and add an equal volume of Benedict’s solution. Heat to boiling, then let cool. A brick red or orange precipitate forms if a reducing sugar is present.

Benedict's solution contains aqueous copper (II) sulphate, sodium carbonate, and sodium citrate. The citrate ions in Benedict's solution complex the copper (II) ions to prevent the formation of insoluble copper (II) carbonate. In the presence of a reducing sugar, however, the copper (II) ions are reduced to copper (I) ions which form a brick red precipitate of copper (I) oxide. The oxygen in the copper (I) oxide come from hydroxide; the purpose of the sodium carbonate is to provide this hydroxide by creating an alkaline environment.

Normally, sugar molecules form five or six member rings and have no reducing properties. In water, however, the rings of some sugar molecules can open to form a linear structure, often with an aldehyde group at one end. These aldehyde groups react with copper (II) to reduce it to copper (I). Sugars that do not have an aldehyde group in the linear structure or that are not able to open are not able to reduce copper (II) ions and are thus called non-reducing sugars. Students do not need to understand this chemistry for their exam, but they may ask about what is happening in the reaction.

\subsubsection{Non-reducing sugar}
Do the test for a reducing sugar using Benedict's solution. Notice that no reaction occurs. Add a few drops of citric acid solution to the solution, then heat to boiling. Let solution cool. Add a few drops of 1~M NaOH, and shake well. Then, add some Benedict’s solution (equal in volume to the liquid in the test tube). Boil the solution, and let it cool. A brick red or orange precipitate forms if a non-reducing sugar is present.

This experiment will also test positive for all reducing sugars. Therefore it is important to first perform the test for reducing sugars before considering this test. If the test for reducing sugars is positive, there is no reason to perform the test for non-reducing sugars - the conclusion will be invalid.

Non-reducing sugars are a misnomer, that is, their name is incorrect. This test does not test for any sugar that is not reducing. Rather, this is a test for any molecule made of multiple reducing sugars bound together, such as sucrose or starch. When these polysaccharides are heated in the presence of acid, they hydrolyse and release monosaccharides. The presence of these monosaccharides is then identified with Benedict's solution.

The purpose of the sodium hydroxide is to neutralize the citric acid added for hydrolysis. If the citric acid is not hydrolysed, it will react with the sodium carbonate in Benedict's solution, possibly making the solution ineffective.

\subsection{How to Write a Report}
Food test data is reported in a table containing four columns: test for, procedure, observation, and inference. With the exception of the `test for’ column, data should be reported in full sentences written in past tense. The procedure should also be in passive voice. No, this is not the way professional scientists write. However, students here must use passive voice to get marks on the national exam.

Note that every column is worth marks on the exam. Even if students fail to do the food tests correctly, they can still get marks for writing what they are testing for and what the procedure should be.

See the sample practical below for an example of a report.

\subsection{Sample Food Test Practical}

You have been provided with Solution K. Carry out food test experiments to identify the food substances present in the solution.
\begin{enumerate}

\item{Record your experimental work as shown in the table below.}

\begin{center}
\begin{tabular}{ | c | c | c | c |}
\hline
Test for & Procedure & Observations & Inferences\\ \hline
Protein & & & \\ \hline
Starch & & & \\ \hline
Lipids & & & \\ \hline
Reducing sugars & & & \\ \hline
Non-reducing sugars & & & \\ \hline
\hline
\end{tabular}
\end{center}

\item{Suggest two natural food substances from which solution K might have been prepared.}
\item{What is the function of each of the food substances in solution K to human beings?}
\item{For each food substance identified, name the enzyme and end product of digestion taking place in the:}
\begin{enumerate}
\item{Stomach}
\item{Duodenum}
\end{enumerate}
\item{What deficiency diseases are caused by a lack of the identified food substances?}
\end{enumerate}

\subsection{Sample Practical Solutions}
(Assume Solution K contains protein and starch.)

\item{The results were as follows}

\begin{center}
\begin{tabular}{ | c | c | c | c |}
\hline
Test for & Procedure & Observations & Inferences\\ \hline
Protein & 
A few drops of NaOH solution & A violet color & Protein was present. \\
& were added to Solution K. & was observed. &\\
& The solution was shaken.& & \\
& Then a few drops of & &\\
& \ce{CuSO4} solution were & & \\
& added to Solution K, and the & &\\
& solution was shaken again. & &\\ \hline
Starch & A few drops of iodine & A blue-black color & Starch was present. \\
& solution were added & was observed. & \\
& to Solution K, and & & \\
& the solution was shaken. & & \\ \hline
Lipids & A few drops of Sudan III & A red ring did not & Fats/oils were absent. \\
& solution (or iodine solution) & form at the surface. & \\
& were added to Solution K. & & \\
& The solution was shaken & & \\ 
& and then allowed to stand. & & \\ \hline
Reducing & A small amount of Benedict’s & There was no & Reducing sugars  \\
sugars & solution was added & precipitate & were absent.\\
& to Solution K. & & \\
& The solution was boiled & & \\
& and allowed to cool. & & \\ \hline
Non- & A small amount of dilute & There was no & Non-reducing \\
reducing & acid was added to & precipitate.& sugars were absent.\\
sugars & Solution K. The solution & &\\
& was boiled and allowed to cool. & & \\
& Then a small amount of & & \\
& NaOH solution was added, & & \\
& and the solution was shaken. & & \\
& Finally, a small amount of & & \\
& Benedict’s solution was added. & & \\
& The solution was boiled & & \\
& and let cool. & &\\ \hline
\hline
\end{tabular}
\end{center}

\item{Solution K could have been prepared from egg and maize. \textit{(Note: Any non-processed food containing protein or starch is correct here.)}}
\item{Starch provides energy to the body. Proteins are used in growth and tissue repair.}

\begin{center}
\begin{tabular}{ | c | c | c | c |}
\hline
Food Substance & Location & Enzyme & End Product of Digestion\\ \hline
Protein & Stomach & \textit{Pepsin} & Polypeptides \\ \hline
Protein & Duodenum & \textit{Trypsin} & Amino acids \\ \hline
Starch & Duodenum & \textit{Pancreatic amylase} & Maltose \\ \hline
\hline
\end{tabular}
\end{center}

\item{A deficiency of protein causes kwashiorkor. A deficiency of starch causes marasmus.}
\end{itemize}

%==============================================================================
\section{Classification}

The classification practical requires students to identify specimens of animals, plants, and fungi. The students must write the common name, kingdom, phylum, and sometimes class of each specimen. They also answer questions about the characteristics and uses of the specimens.

This section contains the following:
\begin{itemize}
\item{Common specimens}
\item{Where to find specimens}
\item{Storage of specimens}
\item{Sample practical with solutions}
\item{Additional classification questions}
\end{itemize}

\subsection{Common Specimens}
\begin{description}
\item[Fungi]{Mushroom, yeast, bread mold}
\item[Plants]{Fern, moss, bean plant, bean seed, maize plant, maize seed, pine tree, cactus, sugar cane, Irish potato1, cypress tree, acacia tree, hibiscus leaf, cassava}
\item[Animals]{Millipede, centipede, grasshopper, lizard, tilapia (fish)3, scorpion, frog, tapeworm, liver fluke, cockroach, spider}
\end{description}

\subsection{Where to Find Specimens}
\begin{itemize}
\item{Start collecting specimens several months before the NECTA exams, as some specimens can be hard to find in the dry season.}
\item{Ask your students to bring specimens! Students are especially good at finding insects and other animals. You can even find primary school children to gather insects such as grasshopper and millipedes.}
\item{Ferns, hibiscus, pines, and cypresses are used in landscaping. Try looking near nice hotelis or guestis. Ferns should have sori (sporangia) on the underside of their leaves.}
\item{Moss often grows near water tanks and in shady corners of courtyards. It is hard to find in the dry season.}
\item{Sugarcane, Irish potato, cassava, tilapia, bean seeds, and maize seeds can be found at the market. Yeast is available at shops.}
\item{Mushrooms are hard to find in the dry season. However, they are available at grocery stores in large cities, and you may be able to find dried mushrooms at the market. You can also collect mushrooms in the rainy season and dry them yourself.}
\item{Tapeworms and liver flukes may be acquired from butchers. Find out where livestock is slaughtered and ask the butchers to look for worms (minyoo). Liver flukes are found in the bile ducts inside the liver, while tapeworms are found in the intestines. You can also try going to a livestock fair/market (mnada) or talking to the local meat inspector (mkaguzi wa nyama).}
\item{Grow your own bread mold. Just put some bread in a plastic bag and leave it in a warm place. But do it ahead of time -- it can take two weeks to obtain bread mold with visible sporangia.}
\end{itemize}

\subsection{Storage of Specimens}
\begin{itemize}
\item{Insects and mushrooms can be dried and stored in jars. However, they become brittle and break easily.}
\item{A 10\% solution of formaldehyde is the best way of storing specimens. Formaldehyde is often sold as a 40\% solution. It should be stored in glass jars and out of the sun. Check specimens periodically for evaporation. Formaldehyde works because it is toxic; handle carefully.}
\item{In a pinch, a 70\% solution of ethanol can also be used to store insects, lizards, and worms. However, specimens sometimes decay in ethanol.}
­­\end{itemize}

\subsection{Sample Classification Practical}

You have been provided with specimens L, M, N, O, and P.
\begin{enumerate}
\item{Identify the specimens by their common names.}
\item{Classify each specimen to the phylum level.}	
\item{Further classification:
\begin{enumerate}
\item{Write the classes of specimens L and M.}
\item{List two observable differences between specimens L and M.}
\end{enumerate}
}%item
\item{Explain why specimen P cannot grow taller.}
\item{Write down two distinctive characteristics of the phylum to which specimen O belongs.}
\item{Reproduction:
\begin{enumerate}
\item{List the modes of reproduction in specimens M and N.}
\item{What are two differences between these modes of reproduction?}
\end{enumerate}
}%item
\end{enumerate}

\subsection{Sample Practical Solutions}

\begin{enumerate}

\item{Common names of specimens:}
\begin{itemize}
\item{L: maize plant}
\item{M: bean plant}
\item{N: yeast}
\item{O: millipede}
\item{P: moss}
\end{itemize}


\item{Classifaction by kingdom and phylum:}

\begin{center}
\begin{tabular}{ | c | c | c |}
\hline
Specimen & Kingdom & Phylym \\ \hline
L (maize plant) & Plantae & Angiospermophyta \\ \hline
M (bean plant) & Plantae & Angiospermophyta \\ \hline
N (yeast) & Fungi & Ascomycota \\ \hline
O (millipede) & Animalia & Arthropoda \\ \hline
P (moss) & Plantae & Bryophyta \\ \hline
\hline
\end{tabular}
\end{center}

\item{Further classification:}
\begin{itemize}
\item{Specimen L (maize plant): Class Monocotyledonae}
\item{Specimen M (bean plant): Class Dicotyledonae}
\item{Observable differences:}

\begin{center}
\begin{tabular}{ | c | c | c |}
\hline
Specimen & Vein structure & Root structure \\ \hline
L (maize plant) & Parallel veins & Fibrous roots \\ \hline
M (bean plant) & Net veins & Tap roots \\ \hline
\hline
\end{tabular}
\end{center}

\textit{tThe answers to this question should be differences between monocots and dicots that the student can see by observing the plants with their naked eyes. Hence answers such as ``vascular bundles in a ring'' are not correct.}

\end{itemize}

\item{Specimen P (moss) cannot grow taller because it has no xylem and phloem. If it grew taller, it would not be able to transport food and water throughout the plant.}
\item{Characteristics of phylum Arthropoda:}
\begin{itemize}
\item{jointed legs}
\item{segmented body}
\item{exoskeleton made of chitin}
\end{itemize}
\item{Reproduction}
\begin{enumerate}
\item{Specimen M (bean plant) reproduces by sexual reproduction. Specimen N (yeast) reproduces by asexual reproduction.}

\begin{center}
\begin{tabular}{ | c | c | c | c |}
\hline
Method & Genetic variation & Parents & Gametes \\ \hline
Asexual & There is no genetic & Requires one & No gametes  \\
reproduction & variation between offspring. & parent only. & are involved. \\ \hline
Sexual & There is genetic & Usually requires & Involves fusion  \\
reproduction & variation between offspring. & two parents. & of two gametes.\\ \hline
\hline
\end{tabular}
\end{center}

\end{enumerate}
\end{enumerate}

\subsection{Additional Classification Questions}
\begin{itemize}
\item{Identify specimen X, Y, and Z by their common names.}
\item{Classify specimens X, Y, and Z to the class level. (This means write the kingdom, phylum, and class.)}
\item{Write the observable features of specimen X.}
\item{List three observable differences/similarities between specimens X and Y.}
\item{State the economic importance of specimen X.}
\item{What characteristics are common among specimens X and Y?}
\item{Why are specimens X and Y placed in different classes/phyla/kingdoms?}
\item{Why are specimens X and Y classified under the same class/phylum/kingdom?}
\item{What distinctive features place specimen X in its respective kingdom/phylum/class?}
\item{How is specimen X adapted to its way of life?}
\item{Suggest possible habitats for specimens X and Y.}
\item{Which specimen is a primary producer/parasite/decomposer?}
\item{For mushroom, yeast, bread mold, grasshopper, moss, tilapia, liver fluke, and tapeworm: Draw and label a diagram of specimen X.}
\item{For tilapia: Draw a big and well-labeled diagram showing a lateral view of specimen X.}
\item{For maize and bean:}
\begin{itemize}
\item{Mention the type of pollination in specimen X [wind pollinated or insect pollinated].}
\item{How is specimen X adapted to this type of pollination?}
\item{Mention the type of germination [hypogeal or epigeal] in specimen X.}
\end{itemize}
\item{For bean seed:}
\begin{itemize}
\item{List three observable features of specimen X and state their biological importance.}
\item{Split specimen X into two natural halves. Draw and label the half containing the embryo.}
\end{itemize}
\item{For fern:}
\begin{itemize}
\item{Observe the underside of the leaves of specimen X}
\item{What is the name of the structures you have observed?}
\item{Give the function of the structures named above.}
\item{Draw specimen X and show the structures named above.}
\end{itemize}
\end{itemize}
%==============================================================================
\section{Respiration}

The purpose of this practical is to investigate the properties of air exhaled from the lungs. This section contains the following:

\begin{itemize}
\item{Limewater (properties and preparation)}
\item{Apparatus}
\item{Cautions and advice when using traditional materials}
\item{Sample practical with solutions}
\end{itemize}

\subsection{Limewater}
Limewater is a saturated solution of calcium hydroxide. It is used to test for carbon dioxide. When carbon dioxide is bubbled through limewater, the solution becomes cloudy. This is due to the precipitation of calcium carbonate by the reaction:

\ce{CO2}$_{(g)}$ + \ce{Ca(OH)2}$_{(aq)}$ $\longrightarrow$ \ce{CaCO3}$_{(s)}$

Limewater can be prepared from either calcium hydroxide or calcium oxide. Calcium oxide reacts with water to form calcium hydroxide, so either way you end up with a calcium hydroxide solution. Calcium oxide is the primary component in cement. Calcium hydroxide is available from building supply shops as \textit{chokaa}.

To prepare lime water, add three spoons of fresh \textit{chokaa} or cement to a bottle of water. Shake vigorously and then let stand until the suspended solids precipitate. Decant the clear solution. \textit{Chokaa} produces a solution much faster than cement.

The exact mass of calcium hydroxide or calcium oxide used is not important. Just check whether some calcium hydroxide remains undissolved at the end--a sign that you have made a saturated solution.

Test limewater by blowing air into a sample with a straw. It should become cloudy. If it does not, then the concentration of \ce{Ca(OH)2} is too low.

\subsection{Apparatus}
Many books call for delivery tubes, test tubes, and stoppers. These are totally unnecessary. Add the limewater to any small clear container and blow into it with a straw.

\subsection{Cautions and Advice When Using Traditional Materials}
If you use a delivery tube and pass it through a rubber stopper, do not use a single-holed stopper. This is what the pictures on NECTA practicals suggest, but it is a terrible idea. A single-holed stopper has no space for air to escape. So when a student blows air into the solution, the pressure in the test tube increases. The high pressure air then pushes limewater up the straw into the student's mouth. Alternatively, the student blows the stopper out of the test tube. If you use a stopper, use a double-holed stopper so that the extra air has a place to escape.

Is a glass delivery tube stuck in a rubber stopper? Do not pull hard on it. Just soak the stopper in warm water for a few minutes. The rubber will soften and the tube will come out.

Are your test tubes and delivery tubes cloudy after the practical? Clean them with dilute acid. This will dissolve any calcium carbonate that has been deposited on the glass.

\subsection{Sample Respiration Practical}

You have been provided with Solution B in a test tube. Use a delivery tube to breathe (exhale) into the solution until its color changes. (See diagram below.) 

\begin{enumerate}
\item{What is the aim of this experiment?}
\item{What is Solution B?}
\begin{enumerate}
\item{What changes did you observe after breathing into Solution B?}	
\item{What can you conclude from these changes?}
\end{enumerate}
\item{Breathe out over the palm of your hand. What do you observe?}
\item{Breathe out over a mirror. What do you observe?}
\item{Using your observations in the three experiments above, list three properties of exhaled air.}
\item{Explain why exhaled air is different from inhaled air. Where do the substances you identified in exhaled air come from?}
\end{enumerate}

\subsection{Sample Practical Solutions}

\begin{enumerate}
\item{The aim of this experiment is to test exhaled air for carbon dioxide.}
\item{Solution B is limewater.}
\begin{enumerate}
\item{Solution B became cloudy (or milky).}
\item{Conclusion: exhaled air contains carbon dioxide.}
\end{enumerate}
\item{Air breathed out over the palm of the hand is warm.}
\item{Droplets of water condense on the mirror.}
\item{Conclusions:}
\begin{itemize}
\item{exhaled air contains carbon dioxide}
\item{exhaled air contains water}
\item{exhaled air is warm}
\end{itemize}
\item{Exhaled air contains the waste products of aerobic respiration. The carbon dioxide and water in exhaled air are products of respiration.}
\end{enumerate}

%==============================================================================
\section{Transport}

The purpose of this practical is to investigate osmosis by observing the changes in a leaf petiole placed in a hypotonic solution (water) and a hypertonic solution (water containing salt or sugar). 

This section contains the following:
\begin{itemize}
\item{Materials}
\item{Sample practical with solutions}
\item{Additional questions}
\end{itemize}

\subsection{Materials}
The petiole is the stalk which attaches a leaf to a branch. The papaya leaf petioles in this practical should be soft petioles from young leaves, not stiff petioles from older leaves. Cut the petioles into pieces, and give each student two pieces of about 6 cm in length. Cylinders cut from a raw potato  may be used instead of petioles.

The hypertonic solution may be made with by mixing either salt or sugar with water. The hypotonic solution is tap water.

\subsection{Sample Transport Practical}

\subsubsection{Instructions}

You have been provided with two pieces of a papaya leaf petiole, Solution A, and Solution B.
 
Use a razor blade to split the pieces of petiole longitudinally, up to a half of their length. You should have four strips at one end of each petiole, while the other end remains intact. 

Place one petiole in solution A, and place the other petiole in solution B. Let the petiole  sit for about ten minutes, then touch them to feel their hardness or softness.

Draw a sketch of each petiole after sitting in its respective solution for ten minutes.

Record your observations and explanations about the petioles in the table below.

\begin{center}
\begin{tabular}{ | c | c | c | }
\hline
Solution & Observation & Explanation \\ \hline
A & & \\ \hline
B & & \\ \hline
\hline
\end{tabular}
\end{center}

\subsubsection{Questions}
\begin{enumerate}
\item{What was the aim of this experiment?}
\item{What was the biological process demonstrated by this experiment?}
\item{What is the importance of this process to plants?}
\item{Which solution contained:}
\begin{enumerate}
\item{pure water}
\item{a high concentration of solutes}
\end{enumerate}
\item{What happened to the cells of the petioles in each solution? Illustrate your answer.}
\item{What would happen to the cells of the petioles in solution A if their cell walls were removed?}
\end{enumerate}

\subsection{Sample Practical Solutions}

(Assume Solution A is pure water, and Solution B is a concentrated solution of water and salt.)

\begin{center}
\begin{tabular}{ | c | c | c | }
\hline
Solution & Observation & Explanation \\ \hline
A & The petiole became hard (turgid) & Water diffused into the petiole cells \\ \hline
B & The petiole became soft (flaccid) & Water diffused out of the petiole cells \\ \hline
\hline
\end{tabular}
\end{center}

\subsubsection{Answers to the questions}
\begin{enumerate}
\item{The aim of the experiment was to investigate the effect of osmosis on plant cells.}
\item{The experiment demonstrated osmosis.}
\item{Importance of osmosis in plants:}
\begin{enumerate}
\item{Water enters plant cells by osmosis so that they become turgid. Turgor helps support the plant and hold it upright.}
\item{Water diffuses into the xylem from the soil via osmosis.}
\end{enumerate}
\item{Solution identification}
\begin{enumerate}
\item{Pure water: Solution A.}
\item{High concentration of solutes: Solution B}
\end{enumerate}
\item{[Illustrations]}
\item{The petiole cells would burst in Solution A if their cell walls were removed.}
\end{enumerate}

\subsection{Additional Questions}

You can extend this experiment by giving students two pieces of meat in addition to the petioles. The piece of meat placed in pure water should expand and become soft due to the cells bursting. The piece placed in salt water should shrink and become hard due to water diffusing out of the cells. This experiment helps to teach the different effects of osmosis on plant and animal cells.

If your school has a good microscope, try observing plant cells under the microscope after letting them sit in hypotonic and hypertonic solutions. 

You can add critical thinking questions to the practical that require the student to use their knowledge of osmosis. For example:
\begin{itemize}
\item{Why does a freshwater fish die if it is placed in salt water?}
\item{Why do merchants spray vegetables with water in the market?}
\item{You can die if a doctor injects pure water into your bloodstream. Why?}
\end{itemize}

%==============================================================================
\section{Photosynthesis}

The purpose of this practical is to prove that chlorophyll is required for photosynthesis. This is done by using iodine to test a variegated leaf for starch. The parts of the leaf containing chlorophyll are expected to contain starch, while the parts lacking chlorophyll are expected to lack starch.

This section contains the following:
\begin{itemize}
\item{Procedure}
\item{Cautions}
\item{Materials and where to find them}
\item{Sample practical with solutions}
\item{Additional practicals}
\end{itemize}

\subsection{Procedure}
\begin{enumerate}
\item{Use iodine tincture from the pharmacy without dilution.}
\item{Prepare hot water bathes. The water should be boiling.}
\item{While the water gets hot, send the students to gather small leaves. The best have no waxy coating and are varigated (have sections without green).}
\item{The leaves should be boiled in the hot water bath for one minute.}
\item{Each group should then move its leaf into their test tube and cover it with methylated spirit.}
\item{Each group should then heat their test tube in a water bath. Over time, the leaf should decolorize and the methylated spirit will turn bight green. The chlorophyll has been extracted and moved to the spirit. A well chosen leaf should turn completely white, although this does not always happen.}
\item{After decolorization, dips the leaves briefly in the hot water.}
\item{For leaves that turn white, students should test them for starch with drops of iodine solution.}
\end{enumerate}

\subsection{Cautions}
Ethanol is flammable! It should never be heated directly on a flame. Use a hot water bath -- place a test tube or beaker of ethanol in a beaker or bowl of hot water and let it heat slowly. The boiling point of ethanol is lower than the boiling point of water, so it will start boiling before the water. If the ethanol does catch fire, cover the burning test tube with a petri dish or other non-flammable container to extinguish the flame.

\subsection{Materials and Where to Find Them}
\begin{itemize}
\item{Variegated leaf: this is a leaf that contains chlorophyll in some parts, but not in others. Often variegated leaves are green and white or green and red. Look at the flower beds around the school and at the teachers’ houses -- they often contain variegated leaves. Test the leaves before the practical, as some kinds are too waxy to be decolorized by ethanol. Also, check for chlorophyll by looking at the underside of the leaves; the leaves you use have at least a small section of white on their undersides, signifying a lack of chlorophyll.}
\item{Source of heat: anything that boils water – Motopoa is best, followed by kerosene and charcoal}
\item{Ethanol: use the least expensive strong ethanol available; this is probably methylated spirits unless your village specializes in high proof gongo.}
\end{itemize}

\subsection{Sample Photosynthsis Practical}

You have been provided with specimen G. 
\begin{enumerate}
\item{Identify specimen G.}
\item{Make a sketch showing the color pattern of specimen G.}
Carry out the following experiment:
\begin{enumerate}
\item{Place specimen G in boiling water for one minute.} 
\item{Boil specimen G in ethanol using a hot water bath. Do not heat the ethanol directly on a flame.}
\item{Remove specimen G from the ethanol. Dip it in hot water.}
\item{Spread specimen G on a white tile and drip iodine solution onto it. Use enough iodine to cover the entire specimen.}
\item{Make a sketch showing the color pattern of specimen G at the end of the experiment.}
\end{enumerate}
\item{What was the aim of this experiment?}
\item{Why was specimen G}
\begin{enumerate}
\item{Boiled in water for one minute}
\item{Boiled in ethanol}
\item{Dipped in hot water at the end of the experiment}
\end{enumerate}
\item{What was the purpose of the iodine solution?}
\item{Why was the ethanol heated using a hot water bath?}
\item{What can you conclude from this experiment? Why?}
\end{enumerate}
	 
\subsection{Sample Practical Solutions}
\begin{enumerate}
\item{Specimen G is a variegated leaf.}
\item{Drawing: See diagram above.}
\item{The aim of this experiment was to investigate whether chlorophyll is required for photosynthesis.}
\item{Specimen G was:}
\begin{enumerate}
\item{boiled in water to kill the cells and stop all metabolic processes.}
\item{boiled in ethanol to decolorize it (to remove the chlorophyll).}
\item{dipped in hot water to remove the ethanol. (If ethanol is left on the leaf it will become hard and brittle.)}
\end{enumerate}
\item{The purpose of the iodine solution was to test for starch.}
\item{The ethanol was heated using a hot water bath because ethanol is flammable.}
\item{The experiment shows that chlorophyll is required for photosynthesis. We know this because the parts of the leaf containing chlorophyll also contained starch, which is a product of photosynthesis. Thus, the parts of the leaf containing chlorophyll performed photosynthesis. The parts of the leaf lacking chlorophyll lacked starch. Hence, these parts of the leaf did not perform photosynthesis.}
\end{enumerate}

\subsection{Additional Practicals}
\subsubsection{To test if light is required for photosynthesis}

Take a live plant, and leave it in the dark for 24 hours to destarch all leaves. Then, cover some of its leaves with cardboard or aluminum foil, while leaving others uncovered. Let the plant sit in bright light for several hours. Give each group of students one leaf that was covered in cardboard, and one leaf that was uncovered. Have them use the procedure above to test for starch. They should find that the covered leaf contains no starch, while the uncovered leaf contains starch.

A cool variation on this experiment is to cover leaves with pieces of cardboard that have letters or pictures cut out of them. The area where the cardboard is cut out will perform photosynthesis and produce starch. When the students do a starch test, a blue-black letter or picture will appear on the leaf.

\subsubsection{To prove that oxygen is a product of photosynthesis}
This experiment requires a water plant. Basically, place a live water plant under water*, then cover it with an inverted funnel. Place an upside-down test tube filled with water on top of the funnel. Let the plant sit in bright light until the water in the test tube is displaced and the test tube fills with gas. Use a glowing splint to test the gas--if it is oxygen, it will relight the splint.

*Note: some books suggest putting sodium bicarbonate (baking soda) in the water.
