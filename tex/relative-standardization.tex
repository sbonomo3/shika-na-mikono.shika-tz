\chapter{Relative Standardization} \index{Relative standardization}
\label{cha:rel-stan}

Preparing large volumes of solution is difficult with great accuracy. Relative standardization is a technique to correct the concentration of solutions so that they give the correct results for practical exercises. Note that this technique is only useful in educational situations where the purpose is to prepare a pair of solutions for titration that give an answer known by the teacher. In scientific research, the aforementioned technique -- absolute standardization -- is used because the concentration of one of the solutions is truly unknown.

All schools should use relative standardization to check the concentration of the solutions they prepare for the national examinations. This ensures that the tests measure the ability of the students to perform the practical, and not the quality of the school's balance, water supply, glassware, etc. While useful for all schools, relative standardization is particularly helpful for schools with few resources, as it allows the preparation of high quality solutions with extremely low cost apparatus and chemicals.

\section{General Theory}

The principle of a titration is that the chemical in the burette is added until it exactly neutralizes the chemical in the flask. If the two chemicals react 1:1, e.g. 

\[ \mathrm{HCl}_{(aq)} + \mathrm{NaOH}_{(aq)} \longleftarrow \mathrm{NaCl}_{(aq)} + \mathrm{H}_{2}\mathrm{O}_{(l)} \]

then exactly one mole of the burette chemical is required to neutralize one mole of the chemical in the flask. If the two chemicals react 2:1, e.g. 

\[ 2\mathrm{HCl}_{(aq)} + \mathrm{Na}_{2}\mathrm{CO}_{3(aq)} \longleftarrow 2\mathrm{NaCl}_{(aq)} + \mathrm{H}_{2}\mathrm{O}_{(l)} + \mathrm{CO}_{2(g)} \]

then exactly two moles of the burette chemical is required to neutralize one mole of the chemical in the flask. Let us think of this reaction as a mole ratio.

\[ \frac{\mbox{moles of }A}{\mbox{moles of }B} = \frac{n_{A}}{n_{B}} \]

Where $ n_{A} $ and $ n_{B} $ are the stoichiometric coefficients of A and B respectively.

\[ \mathrm{moles} = \mathrm{molarity} \times \mathrm{volume} = M \times V \mbox{ (so long as V is measured in liters)} \]

By substitution,

\[ \frac{(M_{A})(V_{A})}{(M_{B})(V_{B})} = \frac{n_{A}}{n_{B}} \]

A student performing a titration might rearrange this equation to get

\[ M_{A} = \frac{(n_{a})(M_{B})(V_{B})}{(n_{B})(V_{A})} \]

or

\[ M_{B} = \frac{(n_{B})(M_{A})(V_{A})}{(n_{A})(V_{B})} \]

As teachers, however, we care with something else: making sure that our students find the required volume in the burette. Solving the equation for $ V_{A} $ we find that

\[ V_{A} = \frac{(n_{a})(M_{B})(V_{B})}{(n_{B})(M_{A})} \]

As $ n_{A} $ and $ n_{B} $ are both set by the reaction, as long as we use the correct chemicals there is no problem here.

$ V_{B} $ is measured by the students -- it is the volume they transfer into the flask. As long as the students know how to use plastic syringes accurately, they should get this value almost perfectly correct.

The remaining term, $ \frac{M_{B}}{M_{A}} $ is for the teacher, not the student, to make correct. If we prepare the solutions poorly, our students can do everything right but still get the wrong value for $ V_{A} $. It is very important that we ensure that our solutions have the correct ratio of $ \frac{M_{B}}{M_{A}} $ so that the exercise properly assesses the ability of our students.

Many people look at this ratio and decide that they therefore need to prepare both solutions perfectly, so that $ M_{B} $ and $ M_{A} $ are exactly what is required. This not true. The actual values for $ M_{B} $ and $ M_{A} $ are not important; what matters is the ratio $ M_{B} $ to $ M_{A} $!

For example, if the titration requires 0.10~M HCl and 0.10~M NaOH, our expected mole ratio is:

\[ \frac{M_{HCl}}{M_{NaOH}} = \frac{0.10}{0.10} = 1 \]

Preparing 0.11 M HCl and 0.09 M NaOH will cause the students to get the wrong answer:

\[ \frac{M_{HCl}}{M_{NaOH}} = \frac{0.11}{0.09} = 1.22 \]

However, preparing exactly 0.05 M HCl and 0.05 M NaOH results in the same molar ratio:

\[ \frac{M_{HCl}}{M_{NaOH}} = \frac{0.05}{0.05} = 1 \]

Thus the students can get exactly the right answer if they use the right technique even though neither solution was actually the correct concentration.

How can we ensure that we have the correct molar ratio between our solutions? Titrate your solutions against each other. If the volume is not the expected value, one of your solutions is too concentrated relative to the other. You can calculate exactly how much too concentrated and add the exact amount of water necessary to perfect the ratio. This process is called relative standardization, because you are standardizing one solution relative to the other.

\section{Procedure for Relative Standardization}

In some titrations the acid is in the burette and in some it is the base is in the burette. So let us not use ``acid'' and ``base'' to refer to the solutions, but rather ``solution 1'' and ``solution 2'' where solution 1 is the solution measured in the burette and solution 2 is measured by pipette (syringe).

You should have prepared a bucket or so of each. The volume you have prepared is $ V_{1} $ liters of solution 1 and $ V_{2} $ liters of solution 2.

Titrate the solutions against each other. Call the volume you measure in the burette ``actual titration volume'' You know the desired molarity of each solution, so from the above student equations you can calculate the burette volume you expect, which you might call ``theoretical titration volume.''

After the titration, there are three possibilities. If the actual titration volume equals the theoretical titration volume, your solutions are perfect. Well done.

If the actual titration volume is smaller than the theoretical titration volume, solution 1 is too concentrated and must be diluted. Use the ratio:

\[ \frac{V_{1} \mbox{ (before dilution)}}{V_{1} \mbox{ (after dilution)}} = \frac{\mbox{actual titration volume}}{\mbox{theoretical titration volume}} \]

If the actual titration volume is larger than the theoretical titration volume, solution 2 is too concentrated and must be diluted. Use the ratio:

\[ \frac{V_{2} \mbox{ (before dilution)}}{V_{2} \mbox{ (after dilution)}} = \frac{\mbox{theoretical titration volume}}{\mbox{actual titration volume}} \]

After diluting one of your solutions, repeat the process. After a few cycles, the solutions should be perfect. Remember that the volume ``before dilution'' is the volume actually in the bucket, so the amount you made less the amount used for these test titrations.