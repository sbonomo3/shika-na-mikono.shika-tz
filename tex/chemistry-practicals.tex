\chapter{Chemistry Practicals}

\section{Introduction to Chemistry Practicals}

\subsection{Format}
The format of the Chemistry practical exam was revised in 2011 to keep up with the 2007 updated syllabus. As such, there will be no further Alternative to Practical exams, pending approval from the Ministry of Education.

The Chemistry practical has 3 questions and students must answer all of them. Question 1 is on Volumetric Analysis and Laboratory Techniques and Safety. Question 2 is taken from Ionic Theory and Electrolysis\slash Chemical Kinetics, Equilibrium and Energy. Question 3 is on Qualitative Analysis. Question 1 is worth 20 marks, while Questions 2 and 3 carry 15 marks each. Students have 2$\frac{1}{2}$ hours to complete the exam.

Students are allowed to use Qualitative Analysis guidesheet pamphlets in the examination room.

\subsubsection{Chemistry 1 Theory Format}
The theory portion of the Chemistry exam comprises 100 marks, while the practical carries 50 marks. A student's final grade for Chemistry is thus found by taking her total marks from both exams out of 150.

The theory exam for Chemistry contains 3 sections. Section A has 2 questions and is worth 20 marks - Question 1 is 10 multiple choice and Question 2 is 10 matching. Section B has 9 short answer questions, each having two items, for a total of 54 marks. Section C has 2 essay questions without items for a total of 26 marks. Students are required to answer all questions.

\paragraph{Note} This information is current as of the time of publication of this manual. Updated information may be obtained by contacting the Ministry of Education.

\subsection{Notes for Teachers}

\subsubsection{NECTA Advance Instructions}
Advance instructions are given to teachers at least one month before the date of the exam, as well as 24 hour advance instructions, to enable them to prepare apparatus, equipment and materials required for the examination.

%[tips for identifying practicals from advance instructions, example advance instructions?]

%\subsubsection{Words of Advice}

%\subsection{NECTA Marking Information}
%[how NECTA marks the practicals, highlight most important parts]

\subsection{Common Practicals}
\begin{description}
\item[Volumetric Analysis]{determine the concentration of a solution of a known chemical by reacting it with a known concentration of another solution}
\item[Qualitative Analysis]{systematically identify an unknown salt through a series of chemical tests}
\item[Chemical Kinetics and Equilibrium]{observe changes in chemical reaction rates by varying conditions such as temperature and concentration}
\end{description}

\paragraph{Note} These are the most common practicals, but they are not necessarily the only practicals that can occur on a NECTA exam. Although the updated exam format lists Questions 1 and 3 as Volumetric Analysis and Qualitative Analysis respectively, Question 2 can come from a variety of topics which may not yet have been used in older past papers. Be sure to regularly check the most recent past NECTA papers to get a good idea of the types of questions to expect. 

%==============================================================================

\section{Volumetric Analysis}
%[brief 1 paragraph explanation of practical]\\

This section contains the following:
\begin{itemize}
\item Volumetric Analysis Theory
\item Preparation of Solutions
\item Relative Standardization
\item Preparation of Solutions without a Balance
\item Substituting Chemicals in Volumetric Analysis
\item Properties of Indicators
\item Traditional Volumetric Analysis Technique
\item Volumetric Analysis without Burettes
%\item Sample Practical with Solutions
\end{itemize}

\subsection{Volumetric Analysis Theory}

Most examples of volumetric analysis involve acid-base reactions, so first is a bit of acid-base theory.

\subsubsection{Acids, Bases, and pH}

The Bronsted-Lowery definition of an acid is a substance that provides $\mathrm{H}^{+}$ to a solution while a base is a substance that removes $\mathrm{H}^{+}$ from a solution.

It is important to remember that in a water solution, $\mathrm{H}^{+}$ does not exist. Rather, $\mathrm{H}^{+}$ binds with water to form the hydronium ion, $ \mathrm{H}_3 \mathrm{O}^{+} $ .

\[ \mathrm{H}^{+} + \mathrm{H}_2 \mathrm{O} \longrightarrow \mathrm{H}_3 \mathrm{O}^{+} \]

pH is defined as the power of the hydronium ion concentration. To find the pH of a solution:

\[ \mathrm{pH} = \log{[\mathrm{H}^{+}]_{aq}} \]

Pure water has $ 10^{7} $ moles of $\mathrm{H}_3 \mathrm{O}^{+}$ per liter, or $ \mathrm{pH} = 7 $. This is because some water molecules are always reversibly reacting with each other to form hydronium and hydroxide:

\[ 2\mathrm{H}_2\mathrm{O} \longleftrightarrow \mathrm{H}_3 \mathrm{O}^{+} + \mathrm{OH}^{-} \]

Acids increase the amount of $\mathrm{H}_3 \mathrm{O}^{+}$. By increasing the concentration of hydronium ion, the power of the concentration increases to a less negative number, and thus the solution will have a smaller pH. Bases decrease the amount of H3O+ and thus basic (alkaline) solutions have pH greater than 7.

\subsubsection{Types of Acids and Bases}

\paragraph{Strong Acids}

Strong acids are acids that dissociate completely to provide $\mathrm{H}^{+}$. One can approximate the molarity of $\mathrm{H}^{+}$ (or $\mathrm{H}_3 \mathrm{O}^{+}$) as the molarity of the acid. For example, a solution of 1~M HCl has one mole of $\mathrm{H}_3 \mathrm{O}^{+}$ per liter of solution (pH 0); most of the molecules of HCl have dissociated and the $\mathrm{H}^{+}$ has reacted with water to form $\mathrm{H}_3\mathrm{O}^{+}$.

\[ \mathrm{HCl} + \mathrm{H}_2\mathrm{O} \longrightarrow \mathrm{H}_3 \mathrm{O}^{+} + \mathrm{Cl}^{-} \]

The most common strong acids are sulfuric acid ($\mathrm{H}_2\mathrm{SO}_4$), hydrochloric acid (HCl), and nitic acid ($\mathrm{HNO}_3$).

\paragraph{Weak Acids}

Weak acids, however, are reticent to contribute $\mathrm{H}^{+}$ to solution. For example, in a solution of ethanoic acid, an equilibrium forms where only one in 250 ethanoic acid molecules dissociates to form $\mathrm{H}_3 \mathrm{O}^{+}$.

\[ \mathrm{CH}_3\mathrm{COOH} + \mathrm{H}_2\mathrm{O} \longleftrightarrow \mathrm{H}_3 \mathrm{O}^{+} + \mathrm{CH}_3\mathrm{COO}^{-} \]

The most common weak acids are ethanoic acid or acetic acid ($\mathrm{CH}_3\mathrm{COOH}$), ethandioic acid or oxalic acid ($\mathrm{C}_2\mathrm{H}_2\mathrm{O}_4$), and citric acid ($\mathrm{COOHCH}_2\mathrm{COH(COOH)CH}_2\mathrm{COOH}$).

One mole of hydrochloric acid and one mole of ethanoic acid both require the same amount of base for neutralization. The difference is how the pH of the solution changes during the titration. When hydrochloric acid is titrated, the pH remains very low until right before the endpoint when it jumps to alkaline. When ethanoic acid is titrated, the pH gradually rises through a range of acidic pH's and then jumps at the endpoint. This is why methyl orange cannot be used for titrations with weak acids – see Properties and Preparation of Indicators.

\paragraph{Strong Bases}

Strong bases are bases that either dissociate completely in solution to form $\mathrm{OH}^{-}$ which reacts to remove $\mathrm{H}_3\mathrm{O}^{+}$. The common strong are sodium hydroxide, NaOH, and potassium hydroxide, KOH.

\paragraph{Weak Bases}

Weak bases form an equilibrium with water where only a few of the molecules react to remove $\mathrm{H}_3\mathrm{O}^{+}$. Common weak bases include ammonia (ammonium hydroxide), soluble carbonates, $\mathrm{CO}_3^{2-}$ and all hydrogen carbonates, $\mathrm{HCO}_3^{-}$.

Much like strong and weak acids, both strong and weak bases readily react with acids to neutralize them. As with acids, weak bases will form a buffered solution that changes pH gradually whereas strong bases will change pH abruptly when the base is neutralized fully.

\subsubsection{Volumetric Analysis}

Volumetric Analysis is a method to find the concentration (molarity) of a solution of a known chemical by comparing it with the known concentration of a solution of another chemical known to react with the first.

For example, to find the concentration of a solution of citric acid, one might use a 0.1~M solution of sodium hydroxide because sodium hydroxide is known to react with citric acid.

The most common kinds of volumetric analysis are for acid-base reactions and oxidation-reduction reactions. Acid-base reactions require use of an indicator, a chemical that changes color at a known pH. Some oxidation-reduction reactions require an indicator, often starch solution, although many are self-indicating, that is one of the chemicals itself has a color. For more about indicators, read Properties and Preparation of Indicators. For more on the specific technique of volumetric analysis, read Traditional Volumetric Analysis Technique if you have burettes and Volumetric Analysis Without Burettes if you do not.

The process of volumetric analysis is often called \textit{titration.}


\subsection{Preparation of Solutions}

For many exercises, solutions do not need to be prepared accurately. Even a 50\% error in the preparation will still allow an effective experiment. For other activities, the solutions should be prepared with a great deal of accuracy. This is especially true for volumetric analysis and conductivity experiments. This section deals with the preparation of solutions when accuracy counts.

%==============================================================================
\subsubsection{Measure the Water}

\begin{itemize}

\item{Calculate the total volume of solution you need to prepare. For example, if you are doing a practical with 100 students and each requires 150~mL of solution you should make at least 15~L of solution. Making 20~L is probably wise, to have some extra.}

\item{Find a container large enough for the total volume. Plan ahead to ensure you have a large enough container.}

\item{Add the required volume of ordinary water.}

\item{If your syllabus encourages you to often practice acid-base titrations, designate a pair of suitably large buckets as your permanent ACID and BASE buckets and label them as such with a permanent pen. Then, use a 1 liter container to add water to these buckets, one liter at a time. Use the permanent pen to mark the water height after each liter. Use these marks when adding water to make solutions. Round up the volume you need to the nearest liter (e.g. 71 students * 200 mL per student = 14.2 L, so make 15 L). As long as you use relative standardization when you finish preparing the solutions, any errors you make when measuring the volume will not affect your students' results.}

\item{Distilled water is rarely necessary. If you are preparing solutions for volumetric analysis, read the section on Relative Standardization to learn how to correct small errors caused by the composition of the tap or river water. If the water forms a precipitate when making solutions of hydroxide or carbonate, allow the precipitate to settle and decant the solution. If you are making a dilute solution, you might add hydroxide or carbonate gradually with mixing until precipitation stops and then add the amount you actually need to the liquid after decantation. If the only water supply if muddy, let the dirt settle and decant or use a cloth filter. If the particles are very fine, add a chemical like potassium aluminum sulfate (alum) or iron sulfate to precipitate the dirt. If you think that you do need distilled water, rain water is almost always sufficient.}
\end{itemize}

What comes next depends on the nature of your stock chemical. In general, there are two kinds of solutions:
\begin{itemize}
\item{Solutions prepared from solid stock chemicals, e.g. sodium hydroxide, citric acid}
\item{Solutions prepared from liquid stock chemicals, e.g. sulfuric acid}
\end{itemize}

%==============================================================================
\subsubsection{Preparing solutions from solid stock chemicals}

\begin{itemize}

\item{Calculate the amount of solid chemical required. If the instructions give the required concentration in grams per liter (e.g. $ 4 ^g/_L $ NaOH solution), multiple the total volume by the required concentration (e.g. $ 4 ^g/_L \times 10 L = 40 g $). If the instructions give the required concentration in molarity or moles per liter (e.g. 0.1 M~NaOH solution), multiple the required molarity by the molecular mass of the compound to find the required concentration in grams per liter (e.g. $ 0.1 ^{mol}/_L \times 40 ^g/_{mol} = 4 ^g/_L $). Then, multiple the required concentration by the total volume ($ 4 ^g/_L \times 10 L = 40g $).}

\item{Use a balance to weigh the solid chemical. Remember to weigh the chemical in a plastic container or on a sheet of paper and not on the scale pan directly. Some chemicals (e.g. sodium hydroxide) react with the metal pan. If you are unfamiliar with how to use a balance, read How to Use a Beam Balance. If you do not have a balance, read the section on Preparation Solutions without a Balance.}

\item{Carefully add the solid chemical to the water and stir with something unreactive (e.g. glass rod, broken burette, thick copper wire) until it has completely dissolved.}

\end{itemize}

%==============================================================================
\subsubsection{Preparing solutions from liquid stock solutions}

\begin{itemize}

\item{Calculate the amount of liquid chemical required. To do this, you need to know the molarity of your stock chemical. See the section on Calculating the Molarity of Bottled Liquids. If the instructions give the required concentration in molarity or moles per liter, use the dilution equation to calculate the amount of concentrated required:

\[ (M_{concentrated})(V_{concentrated}) = (M_{dilute})(V_{dilute}) \]

rearranging

\[ V_{dilute} = \frac{(M_{concentrated})(V_{concentrated})}{M_{dilute}} \]

For example, if you need 10~L of 0.1~M HCl and you have 12~M stock solution, the required volume of concentrated acid is

\[ V_{dilute} = \frac{(12 M)(10 L)}{0.1 M} \]

}

\item{If the instructions give the required concentration in grams per liter, divide this concentration by the molecular mass to get molarity (e.g. $ \frac {3.65 ^g/_L}{36.5 ^g/_{mol}} = 0.1 ^{mol}/_L $) and then use the dilution equation as above.}

\item{Use a DRY measuring cylinder the measure the required amount of liquid chemical. Concentrated acids may be measured in standard lab grade plastic measuring cylinders – there is no need for glass. If you do not have a measuring cylinder, you can use a plastic syringe. Be sure to use the Air Cushion Method for measuring volumes with syringes (see the section on How to Use a Plastic Syringe) – concentrated acids will rapidly corrode the rubber in the syringe on contact, causing the syringe to jam and become dangerous. Also, please read the description of Concentrated Acids in Dangerous Chemicals.}

\item{Carefully pour the liquid chemical into the container of water. Stir with something non reactive (glass rod, broken burette, thick copper wire) for about one minute.}

\end{itemize}

Then, for all volumetric analysis solutions, use the instructions in the Relative Standardization section to perfect the mole ratio of your solutions.

\subsection{Relative Standardization}

Preparing large volumes of solution is difficult with great accuracy. Relative standardization is a technique to correct the concentration of solutions so that they give the correct results for practical exercises. Note that this technique is only useful in educational situations where the purpose is to prepare a pair of solutions for titration that give an answer known by the teacher. In scientific research, the aforementioned technique -- absolute standardization -- is used because the concentration of one of the solutions is truly unknown.

All schools should use relative standardization to check the concentration of the solutions they prepare for the national examinations. This ensures that the tests measure the ability of the students to perform the practical, and not the quality of the school's balance, water supply, glassware, etc. While useful for all schools, relative standardization is particularly helpful for schools with few resources, as it allows the preparation of high quality solutions with extremely low cost apparatus and chemicals.

\subsubsection{General Theory}

The principle of a titration is that the chemical in the burette is added until it exactly neutralizes the chemical in the flask. If the two chemicals react 1:1, e.g. 

\[ \mathrm{HCl}_{(aq)} + \mathrm{NaOH}_{(aq)} \longleftarrow \mathrm{NaCl}_{(aq)} + \mathrm{H}_{2}\mathrm{O}_{(l)} \]

then exactly one mole of the burette chemical is required to neutralize one mole of the chemical in the flask. If the two chemicals react 2:1, e.g. 

\[ 2\mathrm{HCl}_{(aq)} + \mathrm{Na}_{2}\mathrm{CO}_{3(aq)} \longleftarrow 2\mathrm{NaCl}_{(aq)} + \mathrm{H}_{2}\mathrm{O}_{(l)} + \mathrm{CO}_{2(g)} \]

then exactly two moles of the burette chemical is required to neutralize one mole of the chemical in the flask. Let us think of this reaction as a mole ratio.

\[ \frac{\mbox{moles of }A}{\mbox{moles of }B} = \frac{n_{A}}{n_{B}} \]

Where $ n_{A} $ and $ n_{B} $ are the stoichiometric coefficients of A and B respectively.

\[ \mathrm{moles} = \mathrm{molarity} \times \mathrm{volume} = M \times V \mbox{ (so long as V is measured in liters)} \]

By substitution,

\[ \frac{(M_{A})(V_{A})}{(M_{B})(V_{B})} = \frac{n_{A}}{n_{B}} \]

A student performing a titration might rearrange this equation to get

\[ M_{A} = \frac{(n_{a})(M_{B})(V_{B})}{(n_{B})(V_{A})} \]

or

\[ M_{B} = \frac{(n_{B})(M_{A})(V_{A})}{(n_{A})(V_{B})} \]

As teachers, however, we care with something else: making sure that our students find the required volume in the burette. Solving the equation for $ V_{A} $ we find that

\[ V_{A} = \frac{(n_{a})(M_{B})(V_{B})}{(n_{B})(M_{A})} \]

As $ n_{A} $ and $ n_{B} $ are both set by the reaction, as long as we use the correct chemicals there is no problem here.

$ V_{B} $ is measured by the students -- it is the volume they transfer into the flask. As long as the students know how to use plastic syringes accurately, they should get this value almost perfectly correct.

The remaining term, $ \frac{M_{B}}{M_{A}} $ is for the teacher, not the student, to make correct. If we prepare the solutions poorly, our students can do everything right but still get the wrong value for $ V_{A} $. It is very important that we ensure that our solutions have the correct ratio of $ \frac{M_{B}}{M_{A}} $ so that the exercise properly assesses the ability of our students.

Many people look at this ratio and decide that they therefore need to prepare both solutions perfectly, so that $ M_{B} $ and $ M_{A} $ are exactly what is required. This not true. The actual values for $ M_{B} $ and $ M_{A} $ are not important; what matters is the ratio $ M_{B} $ to $ M_{A} $!

For example, if the titration requires 0.10~M HCl and 0.10~M NaOH, our expected mole ratio is:

\[ \frac{M_{HCl}}{M_{NaOH}} = \frac{0.10}{0.10} = 1 \]

Preparing 0.11 M HCl and 0.09 M NaOH will cause the students to get the wrong answer:

\[ \frac{M_{HCl}}{M_{NaOH}} = \frac{0.11}{0.09} = 1.22 \]

However, preparing exactly 0.05 M HCl and 0.05 M NaOH results in the same molar ratio:

\[ \frac{M_{HCl}}{M_{NaOH}} = \frac{0.05}{0.05} = 1 \]

Thus the students can get exactly the right answer if they use the right technique even though neither solution was actually the correct concentration.

How can we ensure that we have the correct molar ratio between our solutions? Titrate your solutions against each other. If the volume is not the expected value, one of your solutions is too concentrated relative to the other. You can calculate exactly how much too concentrated and add the exact amount of water necessary to perfect the ratio. This process is called relative standardization, because you are standardizing one solution relative to the other.

\subsubsection{Procedure for Relative Standardization}

In some titrations the acid is in the burette and in some it is the base is in the burette. So let us not use ``acid'' and ``base'' to refer to the solutions, but rather ``solution 1'' and ``solution 2'' where solution 1 is the solution measured in the burette and solution 2 is measured by pipette (syringe).

You should have prepared a bucket or so of each. The volume you have prepared is $ V_{1} $ liters of solution 1 and $ V_{2} $ liters of solution 2.

Titrate the solutions against each other. Call the volume you measure in the burette ``actual titration volume'' You know the desired molarity of each solution, so from the above student equations you can calculate the burette volume you expect, which you might call ``theoretical titration volume.''

After the titration, there are three possibilities. If the actual titration volume equals the theoretical titration volume, your solutions are perfect. Well done.

If the actual titration volume is smaller than the theoretical titration volume, solution 1 is too concentrated and must be diluted. Use the ratio:

\[ \frac{V_{1} \mbox{ (before dilution)}}{V_{1} \mbox{ (after dilution)}} = \frac{\mbox{actual titration volume}}{\mbox{theoretical titration volume}} \]

If the actual titration volume is larger than the theoretical titration volume, solution 2 is too concentrated and must be diluted. Use the ratio:

\[ \frac{V_{2} \mbox{ (before dilution)}}{V_{2} \mbox{ (after dilution)}} = \frac{\mbox{theoretical titration volume}}{\mbox{actual titration volume}} \]

After diluting one of your solutions, repeat the process. After a few cycles, the solutions should be perfect. Remember that the volume ``before dilution'' is the volume actually in the bucket, so the amount you made less the amount used for these test titrations.

\subsection{Preparation of Solutions without a Balance}

The procedure in the section on Relative Standardization allows us to do something seemingly impossible – prepare solutions for volumetric analysis that allow students to get perfect results without using either a balance or volumetric glassware in the preparation. All that you have to do is make two solutions that are close, and then use several cycles of relative standardization to prefect the molarity ratio. 

To measure volume, we can use marks on plastic water bottles as described in the entry for volumetric glassware in the Sources of Equipment section. What follows is an example of how rough solutions can be prepared in Tanzania based on the water bottles available in that country. We encourage people in other countries to calibrate their water bottles and then to customize these instructions for the resources available to them.

\subsubsection{To make 0.05~M sulfuric acid (equivalent to 0.1~M HCl) for fifty students}
\begin{enumerate}
\item{Put 9.9 liters of water into a bucket. On the new 1.5~L Kilimanjaro water bottle, the bottom points of the crown embossed on the side correspond to 300~ml and the top of the mountain corresponds to 1.5~L. Therefore one can measure 9.9 liters by filling the bottle to the mountain top six times and then to the bottom points of the crown three times.}
\item{Add 110~mL of battery acid. This may be accomplished easily by filling a 10~mL plastic syringe eleven times. Please read the safety note in Dangerous Chemicals.}
\end{enumerate}

\subsubsection{To make 0.033~M citric acid (equivalent to 0.1~M HCl) for fifty students}
\begin{enumerate}
\item{Put 10 liters of water into a bucket. One the new 500~mL Kilimanjaro water bottle, the second straight line corresponds to 300~mL and the highest straight line corresponds to 400~mL. Therefore one can use the 1.5~L bottle six times to add nine liters and then use the 500~mL bottle to add one more liter, 400~mL + 300~mL + 300~mL.}
\item{Add 64~g of citric acid. In the absence of a balance, one can often have $^1/_8$ of a kilogram (125~g) measured in the market. Dissolved this in 20~L of water to produce a 0.033~M solution. Alternately, use a plastic syringe to find the volume of a plastic spoon. Fill the spoon with citric acid and push off any extra acid until there is a flat surface (like the water). Then use that spoon to add a total $38 \mathrm{cm}^3$ or mL of citric acid soda knowing the volume of each spoonful.}
\end{enumerate}

\subsubsection{To make 0.1~M sodium hydroxide for fifty students}
\begin{enumerate}
\item{Put 10 liters of water into a bucket. See the instructions above.}
\item{Add 40~g of caustic soda. In the absence of a balance, measure the volume of a spoon as above and add $19 \mathrm{cm}^3 \mathrm{or mL}$ of caustic soda. Please read the safety note in Dangerous Chemicals.}
\end{enumerate}

\subsubsection{To make 0.1~M sodium hydrogen carbonate for fifty students}
\begin{enumerate}
\item{Put 10 liters of water into a bucket. See the instructions above.}
\item{Add 84~g of bicarbonate of soda. In the absence of a balance, find the volume of a spoon as above and add $39 \mathrm{cm}^3 \mathrm{or mL}$ of bicarbonate of soda. Alternately, if 8.33 liters of solution is sufficient, measure this volume of water and then add one whole box of bicarbonate of soda. A box is 70~g.}
\end{enumerate}

\subsection{Substituting Chemicals in Volumetric Analysis}
\label{cha:subchemvolana}
\subsubsection{Theory}

The volumetric analysis practical exercises sometimes call for expensive chemicals, for example potassium hydroxide or oxalic acid. As the purpose of exercises and exams is to train or test the ability of the students and not the resources of the school, it is possible to use different chemicals as long as the solutions are calibrated to give equivalent results. For example, if the instructions call for a potassium hydroxide solution, you can use sodium hydroxide to prepare this solution. It will not affect the results of the practical -- if you make the correct calibration. How to calibrate solutions when substituting chemicals is the subject of this section.

Technically, only two chemicals are required to perform any volumetric analysis practical: one strong acid and one strong base. The least expensive options are sulfuric acid, as battery acid, and sodium hydroxide, as caustic soda. To substitute one chemical for another in volumetric analysis, the resulting solution must have the same normality (N).

\begin{itemize}

\item{For all monoprotic acids (HCl, ethanoic acid), the normality is the molarity.\\
\textit{Example: 0.1~M ethanoic acid = 0.1~N ethanoic acid}}
\item{For diprotic acids (sulfuric acid, ethandiotic acid), the normality is twice the molarity, because each molecule of diprotic acid brings two molecules of $\mathrm{H}^{+}$.\\
\textit{Example: 0.5~M sulfuric acid = 1.0~N sulfuric acid}}
\item{For the hydroxides and hydrogen carbonates used in ordinary level (NaOH, KOH, NaHCO$_{3}$), the normality is the molarity.\\
\textit{Example: 0.08~M KOH = 0.08~N KOH}}
\item{For the carbonates most commonly used ($\mathrm{Na}_2\mathrm{CO}_3$, $\mathrm{Na}_2\mathrm{CO}_3 /dot 10\mathrm{H}_2\mathrm{O}$, $\mathrm{K}_2\mathrm{CO}_3$), the normality is twice the molarity.\\
\textit{Example: $0.4 M \mathrm{Na}_2\mathrm{CO}_3 = 0.8 N \mathrm{Na}_2\mathrm{CO}_3$}}

\end{itemize}

\subsubsection{Substitution Calculations}

When instructions describe solutions in terms of molarity, calculating the molarity of the substitution is relatively simple. For example, suppose we want to use sulfuric acid to make a 0.2~M solution of ethanoic acid. 0.2~M ethanoic acid is 0.2~N ethanoic acid which will titrate the same as 0.2~N sulfuric acid. 0.2 N sulfuric acid is 0.1~M sulfuric acid, and thus we need to prepare 0.1~M sulfuric acid.

When instructions describe solutions in terms of concentration ($^g/_L$), we just need to add an extra conversion step. For example, suppose we want to use sodium hydroxide to make a $14.3 ^g/_L$ solution of sodium carbonate decahydrate. $14.3 ^g/_L$ sodium carboante decahydrate is 0.05~M sodium carbonate decahydrate which is 0.1~N sodium carbonate decahydrate. This will titrate the same as 0.1~N sodium hydroxide, which is 0.1~M sodium hydroxide or $4 ^g/_L$ sodium hydroxide, and thus we need to prepare $4 ^g/_L$ sodium hydroxide to have a solution that will titrate identically to $14.3 ^g/_L$ sodium carbonate decahydrate.

\subsubsection{Common Substitutions}
\label{sec:commonsubs}
To simplify future calculations, we have prepared general conversions for the most common chemicals used in volumetric analysis. Remember to check all final solutions with relative standardization to ensure that they indeed give the correct results.

\newgeometry{margin=1cm}
\begin{landscape}
\thispagestyle{empty}

\begin{table}
\centering

%\begin{center}
\begin{tabular}{| p{3.5cm} | p{4cm} | p{8cm} | p{4cm} | p{4cm} |}
\hline

\multicolumn{1}{|c}{\textbf{Required Chemical}} & 
\multicolumn{1}{|c}{\textbf{Low Cost Alternative}} & 
\multicolumn{1}{|c}{\textbf{Substiution Method}} & 
\multicolumn{1}{|c}{\textbf{Molarity Example}} & 
\multicolumn{1}{|c|}{\textbf{Concentration Example}} \\ \hline

Hydrochloric Acid & 
Sulfuric Acid (Battery Acid) & 
If you are required to prepare an X molarity solution of HCl, prepane a X$\times 0.5$ molarity solution of battery acid & 
The instructions call for 0.12~M HCl. Instead, prepare 0.06~M sulfuric acid & 
 \\ \hline

Ethanoic (Acetic) Acid & 
Sulfuric Acid (Battery Acid) & 
If you are required to prepare an M molarity solution of ethanoic acid, prepare a M$\times 0.5$ molarity solution of sulfuric acid & 
The instructions call for 0.10~M ethanoic acid. Prepare 0.05~M sulfuric acid. & 
 \\ \hline

Ethandioic (Oxalic) Acid dihydrate (C$_{2}$H$_{2}$O$_{4} \cdot$2H$_{2}$O) & 
Sulfuric Acid (Battery Acid) & 
If you are required to prepare an M molarity solution of ethandioic acid, prepare an M molarity solution of sulfuric acid. If you are required to prepare a C concentration solution of ethandioic acid, prepare a $^\text{C}/_{126}$ molarity solution of sulfuric acid. & 
The instructions call for 0.075~M ethandioic acid. Prepare 0.075~M sulfuric acid. & 
The instructions call for 6.3 $^\text{g}/_\text{L}$ ethandioic acid. Prepare 0.05~M sulfuric acid. \\ \hline

Potassium Hydroxide & 
Sodium Hydroxide (Caustic Soda) & 
For M molarity potassium hydroxide, make M molarity sodium hydroxide. For C concentration potassium hydroxide, make C$\times ^{40}/_{56}$ concentration sodium hydroxide. & 
The instructions call for 0.1~M potassium hydroxide. Just prepare 0.1~M sodium hydroxide. &
The instructions call for 2.8 $^\text{g}/_\text{L}$ potassium hydroxide. Prepare 2 $^\text{g}/_\text{L}$ sodium hydroxide. \\ \hline

Anhydrous Sodium Carbonate & 
Sodium Carbonate Decahydrate (Soda Ash) &  
For M molarity anhydrous sodium carbonate, make M molarity sodium carbonate decahydrate. For C concentration anhydrous sodium carbonate, make C$\times ^{286}/_{106}$ sodium carbonate decahydrate. & 
The instructions call for 0.09~M anhydrous sodium carbonate. Make 0.09~M sodium carbonate decahyrate. & 
The instructions call for 5.3 $^\text{g}/_\text{L}$ anhydrous sodium carbonate. Make 14.3 $^\text{g}/_\text{L}$ sodium carbonate decahydrate. \\ \hline

Anhydrous Sodium Carbonate & 
Sodium Hydroxide (caustic soda) & 
For M molarity anhydrous sodium carbonate, make M$\times 2$ molarity sodium hydroxide. For C concentration anhydrous sodium carbonate, make C$\times 2 \times ^{40}/_{106}$ sodium hydroxide. & 
The instructions call for 0.09~M anhydrous sodium carbonate. Make 0.18~M sodium hydroxide. & 
The instructions call for 5.3 $^\text{g}/_\text{L}$ anhydrous sodium carbonate. 4.0 $^\text{g}/_\text{L}$ sodium hydroxide. \\ \hline

Sodium Carbonate Decahydrate (Na$_2$CO$_3\cdot$10H$_2$O) &
sodium hydroxide (caustic soda) &
For M molarity sodium carbonate ecahydrate, make M$\times 2$ molarity sodium hydroxide. For C concentration sodium carbonate decahydrate, make C$\times 2 \times ^{40}/_{286}$ sodium hydroxide. &
The instructions call for 0.09~M sodium carbonate decahydrate. Make 0.18~M sodium hydroxide. &
The instructions call for 14.3 $^\text{g}/_\text{L}$ sodium carbonate decahydrate. Make 4.0 $^\text{g}/_\text{L}$ sodium hydroxide. \\ \hline

Anhydrous Potassium Carbonate & 
Sodium Carbonate decahydrate (Soda Ash) & 
For M molarity potassium carbonate, make M molarity sodium carbonate decahydrate. For C concentration potassium carbonate, make C$\times ^{286}/_{122}$ concentration sodium carbonate. & 
The instructions call for 0.08~M anhydrous potassium carbonate. Prepare 0.08~M sodium carbonate decahydrate. & 
The instructions call for 6.1 $^\text{g}/_\text{L}$ anhydrous potassium carbonate. Prepare 14.3 $^\text{g}/_\text{L}$ sodium carbonate decahydrate. \\ \hline

Anhydrous Potassium Carbonate & 
Sodium Hydroxide (caustic soda) & 
For M molarity potassium carbonate, make M$\times 2$ molarity sodium hydroxide. For C concentration potassium carbonate, make C$\times 2 \times ^{40}/_{122}$ concentration sodium hydroxide. & 
The instructions call for 0.08~M anhydrous potassium carbonate. Prepare 0.16~M sodium hydroxide. & 
The instructions call for 6.1 $^\text{g}/_\text{L}$ anhydrous potassium carbonate. Prepare 4.0 $^\text{g}/_\text{L}$ sodium hydroxide. \\ \hline

\end{tabular}
%\end{center}

\end{table}
\end{landscape}
\restoregeometry

\subsubsection{Additional Notes}

\begin{itemize}

\item{In volumetric analysis experiments with two indicators, it is not possible to substitute one chemical for another as the acid/base dissociation constant is critical and specific for each chemical. It is still possible to substitute sodium carbonate decahydrate for anhydrous sodium carbonate with the above conversion.}

\item{These substitutions only work for volumetric analysis. In qualitative analysis, the nature of the chemical matters. If the instructions call for sodium carbonate, you cannot provide sodium hydroxide and expect the students to get the right answer!}

\end{itemize}

\subsection{Properties of Indicators}

\subsubsection{Acid-base Indicators}\label{sss:acid-baseind}
These indicators are chemicals that change colors in a specific pH range, which makes them suited to use in acid-base reactions. When the pH of changes from low pH to high pH or from high to low, the color of the solution changes. 

Four common acid-base indicators are methyl orange (MO), phenolphthalein (POP), bromothymol blue (BB), and universal indicator (U)

\begin{itemize}

\item{Methyl Orange, MO, is always used when titrating a strong acid against a weak base. The pH range of MO is 4.0-6.0 and thus no color change is observed until the base is completely neutralized. If you use MO with a weak acid, the color might start to change before completely neutralizing the acid.}

\item{Phenolphthalein, POP, is always used when titrating a weak acid against a strong base. The pH range of POP is 8.3-10.0, and thus no color change is observed until the weak acid is completely neutralized. If you use POP with a weak base, the color might start to change before completely neutralizing the base.}

\item{Bromothymol Blue, BB, is used in the same manner as methyl orange.}

\item{Universal indicator, U, is not suitable for volumetric analysis involving either weak acids or bases as it changes color continuously rather than in a limited pH range. It is very useful for tracking the pH continuously over a titration, perhaps by performing two titrations side by side, one with a standard indicator and another with universal indicator.}

\end{itemize}

Any indicator can be used when titrating a strong acid against a strong base. Universal indicator, however, will not produce very accurate results.

No indicator is suitable for titrating a weak acid against a weak base.

In some experiments, more than one indicator may be used in the same flask, for example when titrating a mixture of strong and weak acids or bases.

\paragraph{Colors of Indicators}
The colors of the above indicators in acid and base are:

\begin{center}
\begin{tabular}{| c | c | c | c |}
\hline
Indicator & Acid & Neutral & Base \\ \hline
Methyl Orange & Red & Orange & Yellow \\ \hline
Phenolphthalein & Colorless & Colorless & Pink \\ \hline
Bromothymol Blue & Yellow & Blue & Blue \\ \hline
Universal Indicator & Red, Orange, Yellow & Yellow/Green & Green, Blue, Indigo  \\
\hline
\end{tabular}
\end{center}

Titration is finished when the indicator starts a permanent color change. For example, when methyl orange turns orange, the titration is finished. If students wait until methyl orange turns pink (or yellow) they have overshot the endpoint of the titration, and their volume will be incorrect. Likewise, POP indicates that the titration is finished when it turns light pink. If students wait until they have an intensely pink solution, they will use too much base and get the wrong answer. 

Note that light pink POP solutions may turn colorless if left for a few minutes. This is due to carbon dioxide in the air reacting to neutralize bases in solution.

\paragraph{Note on technique}
Students should use as little acid-base indicator as possible. This is because some acid or base is required to react with the indicator so that it changes color. If a lot of indicator is used, students will add more acid or base than they need.

\subsubsection{Other Indicators}
Starch indicator is used in oxidation-reduction titrations involving iodine. This is because iodine forms an intense blue to black colored complex in the presence of starch. Thus starch allows a very sensitive assessment of the presence of iodine in a solution.

It is important to add the starch indicator close to the end point when there is an acid present. The acid will cleave the starch and that will prevent the starch from working properly. Students using starch should use a pilot run to get an idea when to add the starch indicator.

\subsubsection{Preparation of Indicators}
\begin{itemize}

\item{Methyl orange (MO): if you have a balance, weigh out about 1~g of methyl orange powder and dissolve it in about 1~L of water. Store the solution in a 
plastic water bottle with a screw on cap and it will keep for years. If it gets thick and cloudy, add a bit more water and shake. If you do not have a balance, add half of a small tea spoon to a liter of water.}

\item{Phenolphthalein (POP): Dissolve about 0.2~g of phenolphthalein powder in 100~mL of pure ethanol; then add 100~mL water with constant stirring. If you use 
much more water than ethanol, solid phenolphthalein will precipitate. Store POP in a plastic water bottle with a screw on cap. We recommend making POP in smaller quantities than MO as it does not keep as well, mostly due to the evaporation of ethanol. If the solution develops a precipitate, add a bit of ethanol and shake. We do not recommend using purple methylated spirits as a source of ethanol for making POP. You can distill purple spirits to make clear spirits. For clear methylated spirits, use 140ml of spirit and 60ml of water, as spirits generally are already 30\% water.}

\item{Starch: place about 1~g of starch in 10~mL of water in a test tube. Mix well. Pour this suspension into 100~mL of boiling water and continue to boil for 
one minute or so. Alternatively, use the water leftover after boiling pasta or potatoes. If this is too concentrated, dilute it with regular water.}

\item{The authors have never prepared bromothymol blue or universal indicator from powder, but suspect their preparation is similar to methyl orange.}

\end{itemize}

Note that the exact mass of indicator used is not very important. You just need to use enough so that the color is clearly visible. Students use very little indicator in each titration, and a liter of indicator solution should last you a long time.

\subsection{Traditional Volumetric Analysis Technique}
\label{cha:volanatech}

\subsubsection{Burettes}

In most acid-base titrations, the acid comes from the burette, although sometimes the burette holds the base. Prior to use, the student should thoroughly wash the burette to remove any residue from previous use. Then, the student should close the stopcock and add about 5ml of the solution that they will use in the burette. With their thumb over the open end of the burette they should make sure the solution covers every surface of the burette. They should then run this solution out into a waste container. This step is to replace the residue of water from the first washing with a layer of the titration solution. If students do not perform this step, the water reside will dilute their titration solution.

Most burettes have a volume of 50 mL. The 0 mL mark is at the top, and the 50 mL mark is at the bottom. This is because the burette tells you the volume of solution used, not the volume of solution present. If you start at 0 mL, and finish at 20 mL, then you have used 20 mL of acid.

Many burettes do not have stopcocks. Instead, they have a piece of rubber tubing at the bottom, which has a glass tip inserted into it. Either a metal clip is used to hold the rubber tubing closed or there is a small bead in the tubing around which fluid may pass when the tube is squeezed at that point. Broken burettes can often be repaired; see the section on Repairing Burettes.

\subsubsection{Reading Measurements}

\begin{itemize}

\item{Always read burettes at eye-level. If the burette is clamped to a stand, remove it from the stand so you can hold it at eye-level. Or move the stand.}

\item{Always read from the bottom of the meniscus. Students often forget this; it helps to remind them at the beginning of a practical. In plastic apparatus, there is often no meniscus.}

\item{Burettes are accurate to 2 decimal places. Many times, students are taught that the last number should be either 5 or 0, like 15.55 or 15.50. This is incorrect – students should estimate the fluid level in the burette to the nearest 0.01 mL.}

\end{itemize}

\subsubsection{Titration Procedure}

\begin{itemize}

\item{Clean the burette with water. Then rinse it with the solution you will be using for titration.}

\item{Fill the burette with the solution. Allow a little solution to run out of the tip until the top of the fluid is at either 0.00 mL exactly or any value below. An initial volume of 1.32 mL is completely acceptable, at least from a scientific point of view. Your country may have specific expectations for marking exams.}

\item{Record the initial burette reading.}

\item{Use a syringe to transfer the other solution into a conical flask. Record the volume moved by the syringe.}

\item{If you are using indicator, add a few drops to the conical flask. For acid-base indicators, the less indicator used the better. In order to change color the indicator itself must react with some of the fluid from the burette. This consumes more chemical than is technically needed for neutralization; the additional chemicals required for titrating the indicator is called indicator error. One or two drops is really all you need. For starch indicator, use about 1 mL. The starch is not titrated, unlike acid-base indicators, so you can use more and often must to get a good color.}

\item{Slowly add solution from the burette to the conical flask. As you titrate, swirl the flask to mix. Do not shake it back and forth, because the solution in the flask will splatter onto the sides of the flask and thus will not be part of the neutralization reaction. Much the same, be careful to add the drops from the burette so they fall into the solution and are not stuck on the side of the flask. Stop titration when the indicator starts a permanent, slight color change. This is the endpoint. Again, the slightest change in color to the appropriate color indicates the endpoint, as long as the color remains after a few swirls.}

\item{Record the final burette reading.}

\end{itemize}

Titration is often done four times: a pilot followed by three trials. The purpose of the pilot is to find the approximate volume from the burette. The pilot is done quickly, and often overshoots the endpoint. In subsequent titration, use the results of the pilot to avoid overshooting while speeding up the work. For example, if the pilot gave an endpoint of 26 mL, add your volume rapidly from the burette until about 20 mL. Then add drop by drop until you find the endpoint.

The result from the pilot is not considered in calculations, as it is not expected to be accurate. Do not include it when finding the average volume or the variance.

\subsection{Volumetric Analysis without Burettes}

\subsubsection{Theory}

Burettes are not necessary to perform volumetric analysis with reasonable precision. Students may use plastic syringes in place of burettes. These should be the most precise syringes available, which as of late 2010 were the 10~mL NeoJect brand plastic syringes. These syringes are more accurate than the low cost glass pipettes that many schools purchase. As the accuracy of the titration is no better than its least accurate instrument, a titration with two plastic syringes is more accurate than a titration with a burette and a cheap glass pipette.

If use of these syringes is new to you, please read Use of Plastic Syringes before proceeding.

To get maximum precision from plastic syringes, students should learn how to estimate values between the lines on the syringe body. The NeoJect syringes are marked with lines every 0.2~mL. Students should observe the top of the fluid and decide if it is on the line exactly, half way in between, or in between half way and one of the lines. This allows them to divide the space between lines into four parts, giving them a precision of 0.05~mL. Estimation between gradations is standard practice with scientific instruments; even students using burettes should estimate the fluid height between the lines to at least 0.05~mL. Syringes have the capacity to deliver the precision required by most if not all  national exams.

If students are using syringes in place of burettes, they require two syringes for the practical, one as a burette and a different one as the pipette. We recommend that you label the syringes, for example, on one syringe writing ‘Burette’ with a permanent pen to help students remember which is which.

\subsubsection{Titration Procedure without Burettes}

\begin{enumerate}

\item{Clean the ‘pipette’ syringe with water. Then rinse it with the acid or base solution you will be putting in the flask.}

\item{Use a syringe to transfer the required amount of acid or base to the flask. To do this transfer accurately, add first 1 mL of air to the syringe and then suck up the fluid to beyond the desired amount. Push back the plunger until the top of the fluid is exactly the volume required. Delivering the required volume to the flask may take multiple transfers with the single syringe. Record the total volume transferred to the flask as the ‘volume of pipette used’}

\item{Add one or two drops of indicator to the flask.}

\item{Clean the ‘burette’ syringe with water. Then rinse it with the acid or base solution you will be using to titrate.}

\item{Add 1~mL of air to the syringe and then suck up the acid or base solution to beyond the 10~mL mark. Slowly push back the plunger until the top of the fluid is exactly at the 10 mL line.}

\item{Slowly add the solution from the syringe to the flask. As you titrate, swirl the flask to mix. As described above, swirl instead of shaking to keep all of the liquid together. Make sure that each drop from the syringe hits the liquid rather than getting suck on the edge of the container. Stop titration when the indicator starts a permanent color change. Just as with a burette, this is the endpoint.}

\item{Often the volume required from the ‘burette’ is greater than 10~mL. This is no problem – after finishing the syringe students should simply fill it again as they did the first time and continue. On their rough paper (scratch paper), they should note that they have already consumed 10~mL.}

\end{enumerate}

\subsubsection{Table of Results when using syringes in place of burettes}

At present, many national exam marking boards expect students to use burettes. The obvious problem is that while the top line on a burette is 0~mL, the top of the syringe reads 10~mL. For students to get the marks their careful technique deserves, they must record their results in a manner consistent with traditional reporting. On rough paper, students should calculate the volume of solution used in their titration. This is easy -- if the syringe started at 10.00 mL and ended at 2.55~mL, the student used $10.00 \mathrm{mL} - 2.55 \mathrm{mL} = 7.45 \mathrm{mL}$ of solution. If the student used two full syringes and the third finished at 4.65~mL, then the student used $10.00 \mathrm{mL} - 4.65 \mathrm{mL} = 5.35 \mathrm{mL}$ in the last syringe plus 10~mL in each of the first two syringes, so $5.35 \mathrm{mL} + 10 \mathrm{mL} + 10 \mathrm{mL} = 25.35 \mathrm{mL}$ total.

In the Table of Results, the student should then write 25.35~mL for the Volume Used. If this volume had been used in a burette, the student would have found an initial volume of 0.00~mL and a final volume of 25.35~mL. The rest of the table should be filled in this manner. When using a syringe as a burette, the student should always write 0.00~mL for the Initial Volume and then for Final Volume they should write the total number they calculated for Volume Used. This method will ensure that the students gets the marks he or she deserves for careful titration – and likewise ensure that he or she loses the appropriate marks for mistakes.

\subsection{Sample Practical Question}
The following is a sample practical question from 2012.\\


\noindent You are provided with the following solution:\\

\noindent \textbf{TZ}: Containing 3.5 g of impure sulphuric acid in 500 cm$^3$ of solution;\\
\textbf{LO}: Containing 4 g of sodium hydroxide in 1000 cm$^3$ of solution;\\
Phenolphthalein and Methyl indicators.\\

\textbf{Questions}:\\
\begin{enumerate}
\item[(a)]
\begin{enumerate}
\item[(i)] What is the suitable indicator for the titration of the given solutions?\\
Give a reason for your answer.
\item[(ii)] Write a balanced chemical equation for the reaction between \textbf{TZ} and \textbf{LO}.
\item[(iii)] Why is it important to swirl or shake the contents of the flask during the addition of the acid?\\
\end{enumerate}

\item[(b)] Titrate the acid (in a burette) against the base (in a conical flask) using two drops of your indicator and obtain three titre values.\\

\item[(c)] 
\begin{enumerate}
\item[(i)] \_\_\_\_ cm$^3$ of acid required \_\_\_\_ cm$^3$ of base for complete reaction.
\item[(ii)] Showing your procedures clearly, calculate the percentage purity of \textbf{TZ}.
\end{enumerate}
\end{enumerate}


\raggedleft \textbf{(20 marks)}

\raggedright

\subsubsection{Discussion}
This practical question requires students to know and understand how to use volumetric analysis apparatus and technique. Since this question involves the titration of sulphuric acid (strong acid) and sodium hydroxide (strong base), either phenolphthalein or methyl orange are acceptable indicators to use. An explanation of suitable indicators can be found in \nameref{sss:acid-baseind}.

Make sure that students create a table for the first pilot titration and three titre values, for a total of four titrations. Only the titre values (not the pilot) that are within $\pm$0.02 ml of each other will be used to calculate the average titrated volume. Students should also be swirling the contents of the volumetric flask in order to thoroughly mix the acid and base together. The titration is complete only when there is permanent color change in the indicator.

Note that although this procedure states the number of drops of indicator and how many number of titre values, it does not indicate what volume to use in the flask. The typical volume is 25 ml, but students can use any volume as long as they are consistent for each trial.

The practical question for volumetric analysis will always ask students to either determine percentage purity, molecules of crystallization of water, unknown concentration of one of the solutions, or molar mass of one of the solutions. %See ----- for more explanation on volumetric analysis calculations.


%\subsection{Sample Practical Solutions}


%==============================================================================
%==============================================================================

\section{Qualitative Analysis}
\label{cha:qualana}
%[brief 1 paragraph explanation of practical]\\

This section contains the following:
\begin{itemize}
\item Overview of Qualitative Analysis
\item Teaching Qualitative Analysis with Local and Low Cost Materials
\item The Steps of Qualitative Analysis
\item Hazards and Cleanliness
\item Sample Practical: Preparation of Copper Carbonate for Qualitative Analysis
\end{itemize}

%==============================================================================

\subsection{Overview of Qualitative Analysis}

The salts requiring identification have one cation and one anion. Generally, these are identified separately although often knowing one helps interpret the results of tests for the other. For ordinary level in Tanzania, students are confronted with binary salts made from the following ions:

\begin{itemize}
\item{Cations: NH$_{4}^{+}$, 
Ca$^{2+}$, 
Fe$^{2+}$, 
Fe$^{3+}$, 
Cu$^{2+}$, 
Zn$^{2+}$, 
Pb$^{2+}$, 
Na$^{+}$}
\item{Anions: CO$_{3}^{2-}$, 
HCO$_{3}^{-}$, 
NO$_{3}^{-}$, 
SO$_{4}^{2-}$, 
Cl$^{-}$}
\end{itemize}
At present, 
ordinary level students receive only one salt at a time. The teacher may also make use of qualitative analysis to identify unlabeled salts.

The ions are identified by following a series of ten steps, divided into three stages. These are:
\begin{itemize}
\item{Preliminary tests:
These tests use the solid salt. They are: appearance, action of heat, action of dilute H$_{2}$SO$_{4}$, action of concentrated H$_{2}$SO$_{4}$, flame test, and solubility.}
\item{Tests in solution: The compound should be dissolved in water before carrying out these tests. If it is not soluble in water, use dilute acid (ideally \ce{HNO3}) to dissolve the compound. The tests in solution involve addition of NaOH and NH$_{3}$.}
\item{Confirmatory tests: These tests confirm the conclusions students draw from the previous steps. By the time your students start the confirmatory tests, they should have a good idea which cation and which anion are present. Have students do one confirmatory test for the cation they believe is present, and one for the anion you believe is present. Even if several confirmatory tests are listed, students only need to do one. When identifying an unlabelled container, however, you might be moved to try several, especially if you are new to this process.}
\end{itemize}

%==============================================================================

\subsection{Teaching Qualitative Analysis with Local and Low Cost Materials}

\subsubsection{General Suggestions}
\begin{itemize}
\item{Heat sources: Motopoa burners cost nothing to make (soda bottle caps) and consume only a small amount of fuel. They give a non-luminous flame ideal for flame tests and still produce enough heat for the other tests.}
\item{Test tubes: Most of the tests do not involve heating, so students may perform these experiments in plastic tubes made from disposable plastic syringes. Many of the tests requiring heating use the salt in solution and thus can be performed by holding the plastic test tube in a hot water bath. For the Action of Heat test, salts may be heating in metal spoons to observe residue products, although it is difficult to test the gases produced. Do not wait for test tubes to start teaching qualitative analysis. Do try to find at least one borosilicate (Pyrex, 
Borosil) test tube for each student before the national exams.}
\item{Litmus paper: Make your own. See the instructions in chapter on Acids and Bases. Rosella flowers give very good results.}
\item{Low cost sources of chemicals. Many chemicals have low cost alternatives, for example table salt (sodium chloride), gypsum powder (calcium sulphate), ammonium sulphate (sulphate of ammonia fertilizer), cautic soda (sodium hydroxide), soda ash (sodium carbonate), battery acid (sulphuric acid), and copper (II) sulphate (the local medicine mlutulutu). Other chemicals can be manufactured locally in small quantities, for example iron (II) sulphate, iron (III) sulphate, calcium carbonate, copper carbonate, zinc sulphate, and zinc carbonate. Indeed the preparations several of these compounds are described in the chapter on Compounds of Metals. For more information about any specific chemical, read its entry in the Sources of Chemicals section.}
\item{Share expensive chemicals among many schools. A single container of potassium ferrocyanide, for example, can supply ten or even twenty schools for several years. Schools should consider bartering 10~g of one chemical for 10~g of another. Schools without any expensive chemicals could produce Benedict's solution from local materials, for example, and exchange this for 10~g samples of expensive salts. Another alternative is for all of the schools in a district or town to pool money to buy one container of each required imported reagent, and then divide the chemicals evenly. Remember to keep chemicals in air-tight containers and out of sunlight. Also remember to label containers very clearly.}
\end{itemize} 

%===================
\subsubsection{Timeline of Lessons}

To teach qualitative analysis, first make sure the students know how to use the apparatus: test tubes, droppers, and motopoa burners. All of these apparatus are used in other experiments throughout this book.

Once the students are familiar with the apparatus, teach them each step separately, using the activities outlined below. Give adequate time to each step; each one can be used to review chemistry learned in previous topics.

Once the students are proficient at the individual steps, practice the whole process with example unknown salts. Sodium carbonate and locally manufactured copper (II) carbonate are good options for practice.

Finally, as the exam approaches, get some lead nitrate, a small amount of fully concentrated sulphuric acid, and some borosilicate test tubes. Use these materials to teach:
\begin{description}
\item[Conformation of sulphates by addition of lead nitrate solution]{Prepare this solution by dissolving about one level tea spoon of lead nitrate in about 100~mL of distilled (rain) water. Use only 2-3 \textit{drops} to confirm sulphates.}
\item[Thermal decomposition of nitrates to form nitrogen dioxide]{Nitrogen dioxide is a poisonous brown gas. Add a very small amount of lead nitrate to a borosilicate test tube and head strongly over a motopoa burner.}
\item[Flame test for lead]{See the instructions for flame tests below. Only a very small amount of lead nitrate is required for the test.}
\item[Conformation of nitrates by brown ring test]{Add a very small amount of lead nitrate to a test tube and dissolve in 2~mL of distilled (rain) water. In a separate test tube prepare about 1~mL of iron (II) sulphate solution from locally manufactured iron (II) sulphate (make sure it is still light green and not yellow!) in distilled water. Mix the solutions and note that lead sulphate will precipitate. Use this to teach the conformation of lead by precipitation with sulphate. Note that the nitrate remains in solution. Decant the liquid into a borosilicate test tube. Hold the test tube at an angle and carefully add about 1~mL of fully concentrated sulphuric acid down the inside. The acid will form a separate layer at the bottom. If nitrates are present, in a few minutes a brown ring should form where the two layers meet. Remember to neutralize this waste before disposal.}
\end{description}

Note that lead nitrate is poisonous. Add some dilute sulphuric acid* to all waste containing lead nitrate to precipitate any soluble lead. Note also that fully concentrated sulphuric acid is very dangerous. Only use it for the brown ring test. Dissolve one box of bicarbonate of soda in 500~mL of water and have this solution available wherever concentrated acid is being used. In the advent of acid spills, use this solution to neutralize the acid to stop burns.

%==============================================================================

\subsection{The Steps of Qualitative Analysis}
\setcounter{secnumdepth}{3}
\subsubsection{Appearance}

Three properties of the salt may be observed directly: colour, texture, and smell.
\begin{description}

\item[Colour]{While most salts are white, salts of transition metals are often colored. Thus colour is an easy way to identify iron and copper cations in salts.}

\item[Texture]{Carbonates and hydrogen carbonates generally form powders although sometimes they can form crystals. Sulphate, nitrates, and chlorides are almost always founds as crystals.}

\item[Smell]{Some ammonium salts smell distinctly like ammonia. Some, however, have no smell. Therefore the smell of ammonia can confirm the presence of ammonium cations, but its absent can not be used to prove the absence of ammonium.}

\end{description}

\paragraph{Materials}
soda bottle caps, table salt, bicarbonate of soda, soda ash (sodium carbonate), copper (II) sulphate*, ammonium sulphate*, locally manufactured iron (II) sulphate*, locally manufactured iron (III) sulphate*, locally manufactured copper (II) carbonate

\paragraph{Preparation}
\begin{enumerate}
\item{Place a small amount of each sample in a different soda bottle cap for observation.}
\end{enumerate}

\paragraph{Activity Steps}
\begin{enumerate}
\item{Look at the samples. Describe their colour, texture, and smell. Do not touch or inhale the salts.}
\end{enumerate}

\paragraph{Results and Conclusion}
\begin{itemize}
\item{Colour}
\begin{description}
\item[White]{Copper and iron absent}
\item[Blue]{Copper cation present}
\item[Green]{Iron (II) or copper present}
\item[Light green]{Iron (II) present}
\item[Yellow or red-brown]{Iron (III) present}
\end{description}
\item{Texture}
\begin{description}
\item[Powder]{Carbonate or hydrogen carbonate anion present}
\item[Crystals]{Sulphate, chloride, or nitrate anion probably present}
\item[Wet crystals]{Chloride or nitrate anion present}
\end{description}
\item{Smell}
\begin{description}
\item[Smell of ammonia]{Ammonium cation present}
\item[No smell of ammonia]{Inconclusive -- some ammonium compounds have no smell}
\end{description}
\end{itemize}

\paragraph{Clean Up}
\begin{enumerate}
\item{Collect salts for use another day. Do not mix.}
\item{Wash and return soda bottle caps.}
\end{enumerate}

\paragraph{Notes}
%Note that ions by themselves have no colour. The colour in all colored salts are transition metal complex compounds. For example, copper (II) salts are blue because

Wet crystals are the result of the salt absorbing water from the atmosphere. Qualitative analysis salts with this property are not locally available. However, caustic soda (sodium hydroxide) has this property, so samples of caustic soda can be used to show the absorption of water from the air and how this changes the appearance of the salt. Note that caustic soda burns skin, blinds in eyes and corrodes metal, so care is required.
%======================================
\subsubsection{Action of heat}

Many salts thermally decompose when heated. When these salts decompose, they produce gases that may be identified to identify the anion of the salt. After decomposition, many salts also leave a residue that may identify the cation.

\paragraph{Materials}
soda bottle caps, motopoa, matches, long handled metal spoons, steel wool, sand, beaker*, water, table salt, copper (II) sulphate*, bicarbonate of soda, locally prepared copper (II) carbonate*, soda ash (sodium carbonate), locally prepared zinc carbonate*

%locally prepared iron (II) sulphate* OR locally prepared iron (III) sulphate* (a mixture will also suffice), -- ?? use carbonate?? does iron (II) carbonate exist??

\paragraph{Hazards and Safety}
\begin{itemize}
\item{Ammonium nitrate explodes when heated. For this reason, ammonium nitrate should never be used in qualitative analysis when the Action of Heat test is used.}
\end{itemize}

\paragraph{Preparation}
\begin{enumerate}
\item{Fill a beaker with water.}
\item{Make a small pile of sand on the table for resting the hot spoon.}
\item{Place a small amount of each sample in a different soda bottle cap.}
\item{Add motopoa to another soda bottle cap to use as a burner.}
\end{enumerate}

\paragraph{Activity Steps}
\begin{enumerate}
\item{Light the motopoa. Note that the flame will be invisible.}
\item{Place a very small amount of a sample on the spoon. Generally, the smallest amounts of sample give the best results because they are easier to heat to a hotter temperature.}
\item{Heat the sample strongly, observing all changes.}
\item{Place the hot spoon on the sand to cool.}
\item{Once the spoon has mostly cooled, dip it in the beaker of water to remove the rest of the heat.}
\item{Use the steel wool to remove all residue from the spoon.}
\item{Repeat these steps with each sample.}
\end{enumerate}

\paragraph{Results and Conclusion}
\begin{itemize}
\item{Gas released}
\begin{description}
\item[Brown gas]{Nitrogen dioxide, nitrates present, confirmed}
\item[Colourless gas with smell of ammonia]{Ammonia, ammonium present, confirmed}
\item[Colourless gas with no smell]{Very likely carbon dioxide, especially if the compound decomposes near the start of heating, carbonate or hydrogen carbonate present}
\item[No change]{Salt probably a chloride, sulphate (very high temperatures are required to decompose many sulphates), or sodium carbonate}
\end{description}
\item{Residue}
\begin{description}
\item[No residue]{Ammonium cation present}
\item[Black residue]{Copper cation probably present}
\item[Red residue when hot, dark when cool]{Iron cation present}
\item[Yellow residue when hot, white when cool]{Zinc cation present}
\item[Red residue when hot, yellow when cool]{Lead cation present}
\end{description}
\item{Sound}
\begin{description}
\item[Cracking sound]{Sodium chloride or lead nitrate present}
\end{description}
\end{itemize}

\paragraph{Clean Up}
\begin{enumerate}
\item{Thoroughly remove all residues from the spoons.}
\end{enumerate}

\paragraph{Notes}
Sodium carbonate is the only carbonate used in qualitative analysis that does not thermally decompose. Therefore a white powder that does not decompose when heated is probably sodium carbonate.

%======================================
\subsubsection{Action of dilute \texorpdfstring{\ce{H2SO4}}{H2SO4}}

Carbonates and hydrogen carbonates react with dilute acid. Sulphates, chlorides and nitrates do not. Therefore reaction with dilute acid is useful test to help identify the anion. Sulphuric acid is used because it is the least expensive.

\paragraph{Materials}
dilute sulphuric acid*, droppers*, bicarbonate of soda, table salt

\paragraph{Hazards and Safety}
\begin{itemize}
\item{Use only a few drops of acid. These are all that are necessary and using more can be dangerous.}
\end{itemize}

\paragraph{Preparation}
\begin{enumerate}
\item{Place a small amount of each sample in a different soda bottle cap.}
\item{Fill droppers with 1-2~mL dilute acid.}
\end{enumerate}

\paragraph{Activity Steps}
\begin{enumerate}
\item{Add a few drops of acid to each sample. Observe the results.}
\end{enumerate}

\paragraph{Results and Conclusion}
\begin{description}
\item[Bubbles of gas]{Carbon dioxide produces; carbonate or hydrogen carbonate anion present}
\item[No bubbles of gas]{Carbonate and hydrogen carbonate absent}
\end{description}

\paragraph{Clean Up}
\begin{enumerate}
\item{Neutralize spills of dilute sulphuric acid with bicarbonate of soda.}
\item{Mix the remains from the reactions together so the extra bicarbonate of soda can neutralize the acid used to test table salt. Dilute the resulting mixture with a large amount of water and dispose down a sink, into a waste storage tank, or into a pit latrine.}
\end{enumerate}

\paragraph{Notes}
You can confirm that the gas produced is carbon dioxide by testing to see if it extinguishes a glowing splint. To do this, light a match, use about 0.5~mL of acid (rather than a few drops), and see if the gas released will extinguish the match.

%======================================
\subsubsection{Action of concentrated \texorpdfstring{\ce{H2SO4}}{H2SO4}}

Concentrated sulphuric acid can convert chloride anions to hydrogen chloride gas and some nitrates to nitrogen dioxide. Because both of these gases are easy to detect, the addition of concentrated acid is used to distinguish between nitrates, chlorides, and sulphates. The concentrated acid used in this experiment should be about 5~M, similar to battery acid.

\paragraph{Materials}
battery acid, droppers*, spoons, test tubes*, test tube rack*, test tube holder*, heat source*, hot water bath*, table salt (sodium chloride), gypsum (calcium sulphate)*, ammonium sulphate*, blue litmus paper*, beaker*, water

\paragraph{Hazards and Safety}
\begin{itemize}
\item{Use battery acid or another source of 5~M sulphuric acid for this experiment. Do not use fully concentrated 18~M sulphuric acid directly from either industry or laboratory supply. 18~M is too concentrated and very dangerous to use.}
\item{Concentrate acid reacts violently with carbonates and hydrogen carbonates. The previous test -- the addition of dilute acid -- will detect carbonates and hydrogen carbonates. If that test is positive, do not test the sample with concentrated sulphuric acid.}
\end{itemize}

\paragraph{Preparation}
\begin{enumerate}
\item{Place a small amount of each sample in a different soda bottle cap.}
\item{Add about 1~mL of air to each dropper syringe (no needle!) and then 2~mL of battery acid. Distribute the dropper syringes in the test tube racks so they stand with the outlet pointing down. The goal is to prevent the battery acid from reacting with the rubber plunger.}
\end{enumerate}

\paragraph{Activity Steps}
\begin{enumerate}
\item{Light the heat source and start heating the hot water bath. The water in the hot water bath should boil.}
\item{Use the spoon to add a small amount of a sample to a test tube.}
\item{Add two drops of battery acid to the sample to make sure there is no violent reaction.}
\item{Add just enough battery acid to cover the sample. Avoid spilling drops of acid on the inside walls of the test tube.}
\item{If a brown gas is released, stop at this step.}
\item{Moisten the blue litmus paper by quickly dipping it in the water of the hot water bath.}
\item{Place the litmus paper over the mouth of the test tube to receive any gases produces. If the litmus paper changes colour, stop at this step.}
\item{Hold the test tube in the hot water bath and heat for a while. Stop heating before the acid in the test tube boils. If the litmus paper changes colour before the acid boils, this is a useful result. If the acid boils, fumes from the acid itself will change the colour of the litmus paper -- this result is not useful, and acid fumes are dangerous.}
\end{enumerate}

\paragraph{Results and Conclusion}
\begin{description}
\item[Bubbles with a few drops of acid]{Carbonate or hydrogen carbonate anion  present}
\item[Brown gas produced]{Nitrate anion present}
\item[Litmus changes to red]{Hydrogen chloride gas produced; chloride anion present}
\item[No effect observed]{Sulphate anion probably present}
\end{description}

\paragraph{Clean Up}
\begin{enumerate}
\item{Fill a large beaker half way with room temperature water. This will be the waste beaker.}
\item{Pour waste from the test tubes into the waste beaker.}
\item{Fill each test tube half way with water and add this water to the waste beaker.}
\item{Return unused battery acid from the droppers to a well-labelled storage container for future use. Immediately fill each dropper (syringe) with water and transfer this water to the waste beaker.}
\item{Slowly add bicarbonate of soda to the waste beaker until addition no longer causes bubbling. This is to neutralize the acid in the waste.}
\item{Dilute the resulting mixture with a large amount of water and dispose down a sink, into a waste storage tank, or into a pit latrine.}
\item{Thoroughly wash all apparatus, including the test tubes and droppers, and return them to the proper places.}
\end{enumerate}

%======================================
\subsubsection{Flame test}

Some metal ions produce a characteristically coloured flame when added to fire.

\paragraph{Materials}
soda bottle caps, motopoa, metal spoons, beaker*, steel wool, water, table salt (sodium chloride), gypsum (calcium sulphate)*, copper (II) sulphate*, ammonium sulphate*

\paragraph{Preparation}
\begin{enumerate}
\item{Fill a beaker with water.}
\item{Place a small amount of each sample in a different soda bottle cap.}
\item{Add motopoa to another soda bottle cap to use as a burner.}
\end{enumerate}

\paragraph{Activity Steps}
\begin{enumerate}
\item{Light the motopoa. Note that the flame will be invisible.}
\item{Place a small amount of sample on the edge of the spoon. For some spoons, it is better to hold the spoon by the wide part and to place the sample on the end of the handle.}
\item{Hold the sample into the hottest part of the flame, 1-2~cm above the motopoa. If necessary, tilt the spoon so that the sample touches the flame directly. Do not spill the sample into the flame.}
\item{Dip the hot end of the spoon into the beaker of water to cool it and remove the sample. If necessary, clean the spoon with steel wool.}
\item{Repeat these steps with each sample.}
\end{enumerate}

\paragraph{Results and Conclusion}
\begin{description}
\item[Blue or green flame]{Copper present, confirmed}
\item[Golden yellow flame]{Sodium present, confirmed}
\item[Brick red flame]{Calcium present}
\item[Bluish white flame]{Lead present}
\item[No flame colour]{Copper and sodium absent; calcium and lead probably absent; cation is probably ammonia, iron, or zinc}
\end{description}

\paragraph{Clean Up}
\begin{enumerate}
\item{Collect unused samples for use another day.}
\item{Wash and return all apparatus.}
\end{enumerate}

%======================================
\subsubsection{Solubility}

\paragraph{Materials}
soda bottle caps, two spoons, test tubes*, test tube rack*, hot water bath*, heat source*, distilled (rain) water*, table salt (sodium chloride), soda ash (sodium carbonate)*, gypsum (calcium sulphate)*, powdered coral rock (calcium carbonate)* or locally manufactured calcium carbonate* or locally manufactured copper (II) carbonate*

\paragraph{Preparation}
\begin{enumerate}
\item{Fill a beaker with water.}
\item{Place a small amount of each sample in a different soda bottle cap.}
\end{enumerate}

\paragraph{Activity Steps}
\begin{enumerate}
\item{Light the heat source and start heating the hot water bath. The water in the hot water bath should boil.}
\item{Decide which spoon will be used for transferring samples and which will be used for stirring.}
\item{Use the transfer spoon to transfer a very small amount of a sample to a test tube.}
\item{Add 3-5~mL of distilled water to the test tube.}
\item{Use the handle of the stirring spoon to thoroughly mix the contents of the test tube.}
\item{If the sample does not dissolve, heat the test tube in the water bath until the contents of the test tube are almost boiling (small bubbles rise from the bottom). Mix.}
\item{Repeat these steps with each sample.}

\end{enumerate}

\paragraph{Results and Conclusion}
\begin{description}
\item[Sample dissolves in room temperature water]{Soluble salt present}
\item[Sample dissolves only in hot water]{Calcium sulphate or lead chloride present}
\item[Sample does not dissolve in even hot water]{Insoluble salt present}
\end{description}

Solubility Rules
\begin{itemize}
\item{All Group I (sodium, potassium, etc) and ammonium salts are soluble (sodium borate is an exception but not relevant to qualitative analysis)}
\item{All nitrates and hydrogen carbonates are soluble}
\item{Most chlorides are soluble (silver and lead chlorides are exceptions, 
although the latter is soluble in hot water)}
\item{Carbonates of metals outside of Group I are generally insoluble (note that aluminum and iron (III) carbonate do not exist)}
\item{Lead sulphate is insoluble and calcium sulphate is soluble only in hot water. Magnesium sulphate is completely soluble while sulphates of the Group II metals heavier than calcium 
(strontium and barium) are insoluble. All other sulphates used in qualitative analysis are soluble}]
\end{itemize}

Table of Solubility for Qualitative Analysis
\begin{center}
\begin{tabular}{ | c | c | c | c | c | c | c | c | } \hline
& ammonium & sodium & copper & iron & zinc & calcium & lead \\ \hline
nitrate & O & O & O & O & O & O & O \\ \hline
chloride & O & O & O & O & O & O & $\Delta$ \\ \hline
sulphate & O & O & O & O & O & $\Delta$ & X \\ \hline
carbonate & O & O & X & X & X & X & X \\ \hline
hydrogen carbonate & O & O & -- & -- & -- & -- & -- \\ \hline
\end{tabular}
\end{center}

KEY:
\begin{itemize}
\item{O = soluble at room temperature}
\item{$\Delta$ = soluble only when heated}
\item{X = insoluble in water}
\item{-- = salt does not exist}
\end{itemize}

\paragraph{Clean Up}
\begin{enumerate}
\item{Collect all unused (dry) samples for use another day.}
\item{Unless copper carbonate is used, none of the salts listed in the materials section of this activity are harmful to the environment.}
\item{Dispose of solutions in a sink, waste tank, or pit latrine.}
\item{Dispose of solids and liquid wastes with precipitates in a waste tank or pit latrine -- never dispose of solids in sinks.}
\item{If using copper carbonate, collect all waste containing copper carbonate and filter to recover the copper carbonate. Save for use another day.}
\item{If you do this activity with a lead nitrate or lead chloride, collect these wastes in a separate container. Add dilute sulphuric acid dropwise until no further precipitation is observed. Neutralize with bicarbonate of soda. Dispose this mixture in a waste tank or a pit latrine. The lead sulphate precipitate is highly insoluble will not enter the environment.}
\item{Wash and return all apparatus.}
\end{enumerate}

\paragraph{Notes}
Calcium carbonate or copper carbonate are recommended qualitative analysis salts to use as examples of insoluble salts. If these are difficult to get, other insoluble compounds may be used for teaching this specific step (but not for other parts of qualitative analysis). Examples of other insoluble compounds include sulphur power, manganese (IV) oxide from batteries, and chokaa (calcium hydroxide, which is only slightly soluble so a significant precipitate will remain).
%========================================
\subsubsection{Addition of NaOH solution}

\paragraph{Materials}
soda bottle caps, two spoons, test tubes*, test tube rack*, beakers*, medium droppers (5~mL syringes without needles)*, large droppers (10~mL syringes without needles), caustic soda (sodium hydroxide)*, table salt (sodium chloride), ammonium sulphate*, copper (II) sulphate*, locally manufactured iron (II) sulphate*, locally manufactured iron (III) sulphate*, locally manufactured zinc sulphate*, distilled (rain) water

\paragraph{Preparation}
\begin{enumerate}
\item{Fill a 500~mL water bottle about half way with distilled (rain) water.}
\item{Add one level tea spoon of caustic soda and then wash the spoon.}
\item{Label the bottle ``1~M sodium hydroxide -- corrosive''}
\item{Place a small amount of each sample in a different soda bottle cap.}
\item{Pour some of the sodium hydroxide solution into a clean beaker.}
\item{For each small dropper syringe, suck in about 1~mL of air and then add about 4~mL of sodium hydroxide solution. Distribute the dropper syringes in the test tube racks so they stand with the outlet pointing down. The goal is to prevent the sodium hydroxide from reacting with the rubber plunger.}
\end{enumerate}

\paragraph{Activity Steps}
\begin{enumerate}
\item{Decide which spoon will be used for transferring samples and which will be used for stirring.}
\item{Use the transfer spoon to transfer a very small amount of a sample to a test tube.}
\item{Use the large dropper syringe to add 3-5~mL of distilled water to the test tube.}
\item{Use the handle of the stirring spoon to thoroughly mix the contents of the test tube.}
\item{Use the small dropper to add a few drops of sodium hydroxide solution to the test tube.}
\item{Observe the colour of any precipitate formed. Also waft the air from the top of the test tube towards your nose to test for smell.}
\item{If a white precipitate forms, use the stir spoon to transfer a very small quantity of the precipitate to a clean test tube. Add 1-2~mL of sodium hydroxide directly to this sample to see if the precipitate is soluble in excess sodium hydroxide solution.}
\end{enumerate}

\paragraph{Results and Conclusion}
\begin{description}
\item[No precipitate and smell of ammonia]{Ammonium cation present, confirmed}
\item[No precipitate and no smell]{Sodium cation probably present}
\item[Blue precipitate]{Copper (II) cation present}
\item[Green precipitate]{Iron (II) cation present}
\item[Red-brown precipitate]{Iron (III) cation present}
\item[White precipitate not soluble in excess NaOH]{Calcium cation present}
\item[White precipitate soluble in excess NaOH]{Lead or zinc cation present}
\end{description}

\paragraph{Clean Up}
\begin{enumerate}
\item{Save all waste from this experiment, labeling it ``basic qualitative analysis waste, no heavy metals'' and leave it in an open container. Over time atmospheric carbon dioxide will react with the sodium hydroxide to make less harmful carbonates. After 2-3 days, dispose of the waste in a waste tank or a pit latrine.}
\end{enumerate}

%======================================
\subsubsection{Addition of \texorpdfstring{\ce{NH3}}{NH3} solution}

This test is very similar to the addition of sodium hydroxide solution. The useful difference is that zinc forms a precipitate in ammonia that is soluble in excess ammonia whereas lead forms a precipitate in ammonia that is not soluble in excess ammonia. Therefore, this test is mainly used to separate lead and zinc. Neither lead salts nor ammonia are locally available in Tanzania. Because the process of this test is the same as the addition of NaOH and the results so similar, students can adequately learn about the Addition of \ce{NH3} test by practicing the Addition of NaOH. For the national exam, a small amount of ammonia solution can be obtained.

Note also that the addition of ammonia to a solution of copper (II) will produce a blue precipitate that dissolves in excess ammonia to form a deep blue solution. This is a useful conformation of the presence of copper, but such conformation is generally unnecessary because the flame test for copper is so reliable.

If you have ammonia solution, store it in a well-sealed container to prevent the ammonia from escaping. A good container for this is a well labeled plastic water bottle with a screw on cap.

%======================================
\subsubsection{Confirmatory Tests}
\setcounter{secnumdepth}{4}

Every cation and anion has at least one specific test that can be used to prove its presence. Not all of these tests are possible with local materials, but many of them are. The following list shows how to confirm each possible cation and anion.

\paragraph{Confirmatory Tests for the Cation}

\subparagraph{Ammonium}
\begin{itemize}
\item{Example salt: ammonium sulphate*}
\item{Procedure: add sodium hydroxide solution and heat in a water bath}
\item{Confirming result: smell of ammonia} 
\item{Reagents: NaOH solution as used above}
\end{itemize}

\subparagraph{Calcium}
\begin{itemize}
\item{Example salt: calcium sulphate}

\item{Procedure: Two options
\begin{enumerate}
\item{flame test}
\item{addition of NaOH solution}
\end{enumerate}
} % Procedure

\item{Confirming results:
\begin{enumerate}
\item{flame test: brick red flame}
\item{addition of NaOH: white precipitate insoluble in excess}
\end{enumerate}
} % Confirming results

\item{Reagents:
\begin{enumerate}
\item{none}
\item{NaOH solution}
\end{enumerate}
} % Reagents

\end{itemize} % Calcium

\subparagraph{Copper}
\begin{itemize}
\item{Example salt: copper sulphate}
\item{Procedure: flame test}
\item{Confirming result: blue/green flame}
\item{Reagents: none}
\end{itemize}

\subparagraph{Iron (II)}
\begin{itemize}
\item{Example salt: locally manufactured iron sulphate 
(keep away from water and air)}
\item{Procedure: addition of sodium hydroxide solution 
and then transfer of precipitate to the table surface}
\item{Confirming result: green precipitate 
that oxidizes to brown when exposed to air}
\item{Reagent: sodium hydroxide solution from above}
\end{itemize}

\subparagraph{Iron (III)}
\begin{itemize}
\item{Example salt: locally manufactured iron sulphate 
(oxidized by water and air)}
\item{Procedure: addition of sodium ethanoate solution}
\item{Confirming result: yellow to red solution}
\item{Reagent: slowly add bicarbonate of soda to vinegar; stop adding when further addition does not cause bubbles; label the solution ``sodium ethanoate for detection of iron (III)''}
\end{itemize}

\subparagraph{Lead}
\begin{itemize}
\item{Example salt: no local sources for safe manufacture, consider purchasing lead nitrate}

\item{Procedure: Three options
\begin{enumerate}
\item{flame test} 
\item{addition of dilute sulphuric acid}
\item{addition of potassium iodide solution}
\end{enumerate}
} % Procedure

\item{Confirming results:
\begin{enumerate}
\item{flame test: blue/white flame}
\item{addition of dilute sulphuric acid: white precipitate}
\item{addition of KI solution: yellow precipitate that dissolves when heated and reforms when cold}
\end{enumerate}
} % Confirming results

\item{Reagents:
\begin{enumerate}
\item{none but a very hot flame, e.g. Bunsen burner, is required} 
\item{dilute sulphuric acid used in Step 5 above}
\item{obtain pure potassium iodide by evaporating iodine tincture until only white crystals remain; do this outside and do not breathe the fumes; it might also be possible to use the KI solution prepared for electrolysis in the chapter on ionic theory}
\end{enumerate}
} % Reagents

\end{itemize} % Lead

\subparagraph{Sodium}
\begin{itemize}
\item{Example salts: sodium chloride, sodium carbonate, sodium hydrogen carbonate}
\item{Procedure: flame test}
\item{Confirming result: golden yellow flame}
\item{Reagents: none}
\end{itemize}

\subparagraph{Zinc}
\begin{itemize}
\item{Example salt: locally manufactured zinc carbonate 
or zinc sulphate}
\item{Procedure: addition of 0.1~M potassium ferrocyanide solution}
\item{Confirming result: gelatinous gray precipitate}
\item{Reagents: no local source of potassium ferrocyanide -- consider collaborating with many schools to share a container; only a very small quantity is required}
\end{itemize}

\paragraph{Confirmatory Tests for the Anion} 

\subparagraph{Hydrogen carbonate}
\begin{itemize}
\item{Example salt: sodium hydrogen carbonate}
\item{Procedure: add magnesium sulphate solution and then boil in a water bath}
\item{Confirming result: white precipitate forms only after boiling}
\item{Reagent: dissolve Epsom salts (magnesium sulphate)* in distilled (rain) water*}
\end{itemize}

\subparagraph{Carbonate}
\begin{itemize}
\item{Example salt: sodium carbonate}
\item{Procedure for soluble salts: addition of magnesium sulphate solution}
\item{Confirming result: white precipitate forming in cold solution}
\item{Reagent: dissolve Epsom salts (magnesium sulphate)* in distilled (rain) water*}

\item{Note that insoluble salts that effervesce with dilute acid are likely carbonates. 
None of the other anions described here produce gas with dilute acid. Note also that all hydrogen carbonates are soluble.}

\end{itemize}

\subparagraph{Chloride}
\begin{itemize}
\item{Example salt: sodium chloride}
\item{Procedure: Three Options
\begin{enumerate}
\item{addition of silver nitrate solution}
\item{addition of manganese (IV) oxide and concentrated sulphuric acid followed by heating in a water bath}
\item{addition of weak acidified potassium permanganate solution followed by heating in a water bath}
\end{enumerate}
} % Procedure
\item{Confirming results:
\begin{enumerate}
\item{silver nitrate: white precipitate of silver chloride} 
\item{manganese (IV) oxide: production of chlorine gas that bleaches litmus} 
\item{acidified permanganate: decolourization of permanganate}
\end{enumerate}
} % Confirming results
\item{Reagents:
\begin{enumerate}
\item{Silver nitrate has no local source but may be shared among many schools as only a very small amount is required.}
\item{Manganese dioxide may be purified from used batteries and battery acid is concentrated sulphuric acid. Note that careful purification is required to remove all chlorides from the battery powder. This method is useful because of its low cost, but remember that chlorine gas is poisonous! Students should use very little sample salt in this test.}
\item{Prepare a solution of potassium permanganate, dilute with distilled water until the colour is light pink, and then add about 1 percent of the solution's volume in battery acid. Note that this solution will cause lead to precipitate, and will also be decolourized by iron II, so it is not a perfect substitute for silver nitrate. This final option is also not yet recognized by examination boards, i.e. NECTA}
\end{enumerate}
} % Reagents
\end{itemize}

\subparagraph{Sulphate}
\begin{itemize}
\item{Example salt: copper sulphate, calcium sulphate, iron sulphate}
\item{Procedure: addition of a few drops of a solution of lead nitrate, 
barium nitrate, or barium chloride}
\item{Confirming result: white precipitate}
\item{Reagents: none of these chemicals have local sources. Because lead nitrate is also an example salt, it is the most useful and the best to buy. The ideal strategy is to share one of these chemicals among many schools. Remember that all are quite toxic.}
\end{itemize}

\setcounter{secnumdepth}{2}
\paragraph{Notes}

Emphasize to students that they need to carry out only one confirmatory test for the cation, 
and one for the anion. If the test gives the expected result, then they can be sure that the ion they have identified is present. If the test does not give the expected result, they have probably made a mistake, and they should revisit the results of their previous tests 
and choose a different possibility to confirm.

%==============================================================================

\subsection{Hazards and Cleanliness}

Qualitative analysis practicals are full of hazards, 
from open flames to concentrated acids. 
To reduce the risk of accidents, 
teach students how to use their flame source 
before the day of the practical, 
especially if you are using Bunsen burners. 
Most students have never used gas before, 
and do not know the basic safety precautions involved in using gas. 
If you have choice about what salts are offered, 
do away with those requiring concentrated acid, and poisonous reagents like lead and barium.

Teach students to hold their test tubes at an angle when they heat them or perform reactions in them. Test tubes should be pointed away from the student holding them and from other students. 
This will prevent injuries due to splashing chemicals, and will also minimize inhalation of any gases produced.

Teach students never to fill test tubes or any other container more than half. That way, 
they minimize spills and boiling over of chemicals during heating. In addition, this also prevents bumping in the test tubes (when a gas bubble forms suddenly), which can cause dangerous spray.

Teach students that if they get chemicals on their hands, they should wash them off immediately, without asking for permission first. Some students have been taught to wait for a teacher's permission before doing anything in the lab, even if concentrated acid is burning their hands. On the first day, give them permission to wash their hands if they ever spill chemicals on them. Also, teach students to tell you immediately when chemicals are spilled. 
Sometimes they hide chemical spills for fear of punishment. Do not punish them for spills -- legitimate accidents happen. Do punish them for unsafe behavior of any kind, even if it does not result in an accident. 

Practicals involving nitrates, chlorides, ammonium compounds, and some sulphates produce harmful gases. Open the lab windows to maximize airflow. Kerosene stoves also produce noxious fumes -- it is much better to use motopoa. If students feel dizzy or sick from the fumes, let them go outside to recover.

Make absolutely sure that students clean their tables and glassware before they leave. Leaving chemicals lying around is dangerous, especially when they are not labelled. Qualitative analysis experiments can leave residues in glass test tubes that are difficult to clean with brushes alone. If possible, heat samples in metal spoons. To remove stubborn residues in glass test tubes, pour a little dilute nitric acid into the test tube. The acid should dissolve the precipitate and leave a clean test tube behind. Remember that the utility of nitric acid -- 
that it will dissolve almost anything -- is also a serious hazard.

%==============================================================================

%\subsection{Sample Practical: Preparation of Copper Carbonate for Qualitative Analysis}
%
%After teaching students all of the individual steps of qualitative analysis, it is good to allow them to practice all of the step on one practical session, as they will have to do for NECTA. The following activity describes how to prepare copper carbonate and how to perform qualitative analysis on it. The teacher should try this activity alone first to review qualitative analysis and then students should perform this activity in groups.
%
%\paragraph{Objective}
%\begin{itemize}
%\item{To perform all steps of qualitative analysis to identify an unknown salt}
%\end{itemize}
%
%\subsubsection{Materials}
%plastic water bottles, filter funnel*, heat source* and take-away tray (optional), plastic spoon, copper (II) sulphate*, soda ash (sodium carbonate)*
%
%\subsubsection{Preparation}
%\begin{enumerate}
%\item{Combine 5 spoons of copper (II) sulphate and 200 mL water in a plastic bottle. Cap and shake until the copper (II) sulphate has fully dissolved.}
%\item{Combine 10 spoons of sodium carbonte and 200 mL water in a separate plastic bottle. Cap and shake until the sodium carbonate has fully dissolved.}
%\item{Combine the two solutions. A green/blue precipitate should form.}
%\item{Alternatively, if another form is practising precipitation reactions using copper (II) sulphte and sodium carbonate, save their solid waste.}
%\item{Pour the mixture into a filter funnel. Let sit until all liquid has passed through.}
%\item{Transfer the solid to a plastic bottle and add about half a litre of water. Shake thoroughly. This is to remove any sodium or sulphate from the original chemicals.}
%\item{Pour the mixture into a filter funnel. Let sit until all liquid as passed through.}
%\item{Dry the precipitate, either in the sun or by heating very gentle in a take-away container over a heat source. If heating, stir often, and remove from the heat before all the water has evaporated or else the copper carbonate will start to thermally decompose.}
%\item{Transfer the dry blue powder to a clean container and label it ``copper (II) carbonate.''}
%\end{enumerate}
%
%\subsubsection{Activity Procedure}
%\begin{enumerate}
%\item{Perform the qualitative analysis steps described above on the sample.}
%\end{enumerate}
%
%\subsubsection{Notes}
%Because copper carbonate is insoluble in water, add a little dilute sulphuric acid solution to bring the ion into solution for the NaOH test. Add the acid drop by drop and avoid adding excess.

\subsection{Sample Practical Question}
The following is a sample practical question from 2012.\\[6pt]


Substance \textbf{V} is a simple salt which contains one cation and one anion. Carry our the experiments described below. Record carefully your observations and make appropriate inferences and hence identify the anion and cation present in sample \textbf{V}.\\

\begin{center}
\begin{tabular}{|l|p{8cm}|l|l|}
\hline
\textbf{S/n}&\textbf{Experiment}&\textbf{Observation}&\textbf{Inference}\\ \hline
1&Observe the appearance of sample \textbf{V}.&&\\ \hline
2&Put a little amount of sample \textbf{V} in a test tube then add water and shake.&&\\ \hline
3&Heat a little amount of \textbf{V} in a dry test tube.&&\\ \hline
{\multirow{4}{*}{4}}&To a little sample \textbf{V} in a test tube add dilute Hydrochloric acid. Add more of the acid until the test tube is half full. Divide the resulting solution into three portions and add the following:&&\\ \cline{2-4}
&\begin{enumerate}
\item[a)] To the one portion add NaOH solution drop wise then excess.
\end{enumerate}&&\\ \cline{2-4}
&\begin{enumerate}
\item[b)] To the second portion add ammonia solution drop wise then in excess.
\end{enumerate}&&\\ \cline{2-4}
&\begin{enumerate}
\item[c)] To the third portion add ammonium oxalate solution.
\end{enumerate}&&\\ \hline
5&Perform flame test.&&\\ \hline
\end{tabular}\\

\end{center}

Conclusion\\

\begin{enumerate}
\item[(i)] The cation in sample \textbf{V} is \_\_\_\_.\\
\item[(ii)] The anion in sample \textbf{V} is \_\_\_\_.\\
\item[(iii)] The chemical formula of \textbf{V} is \_\_\_\_.\\
\item[(iv)] The name of compound \textbf{V} is \_\_\_\_.\\
\end{enumerate}


\raggedleft \textbf{(15 marks)}

\raggedright

\subsubsection{Discussion}
This particular example was to identify calcium carbonate; however, the above procedure follows the same format commonly used for other unknown salts. The only differences may be the specific solutions used in some of the steps.

At times, the procedure may not be explicit or indicated whatsoever and the student is required to write the detailed procedure in addition to the observations and inferences. %The proper way to write the procedure, observations, and inferences can be found in ------ 

Emphasize to students that in addition to the qualitative analysis procedure they have to do only one confirmatory test for cations and one for anions.


%==============================================================================
%==============================================================================

\section{Chemical Kinetics and Equilibrium}
%[brief 1 paragraph explanation of practical]\\

%This section contains the following:
%\begin{itemize}
%\item 
%\end{itemize}

\subsection{Theory}
Compared to the other two NECTA chemistry practicals - Acid/Base Titration (i.e. Volumetric Analysis) and Qualitative Analysis - Chemical Kinetics has few alternative chemicals that can be used.\\

The chemical reaction in the NECTA exam is a precipitation of sulphur. 

\[ \mathrm{Na_2S_2O_3}_{(aq)} + \mathrm{2HCl}_{(aq)} \longrightarrow \mathrm{2NaCl}_{(aq)} + \mathrm{H}_{2}\mathrm{O}_{(l)} + \mathrm{SO_2}_{(g)} + \mathrm{S}_{(s)} \]

This reaction is consistent and easy to practice, but it requires sodium thiosulphate which can be expensive and hard to get a hold of. The hydrochloric acid can be replaced with sulphuric (battery) acid. An alternative reaction to demonstrate chemical kinetics (that can be performed for less than 5000 shillings) is the iodization of acetone, seen in the reaction below.

\[ \mathrm{CH_3COCH_3}_{(aq)} + \mathrm{I_2} \longrightarrow \mathrm{CH_3COCH_2I}_{(aq)} + \mathrm{H}^{+}_{(aq)} + \mathrm{I}^{-}_{(aq)} \]

This reaction starts as a dark opaque solution and eventually proceeds to a colorless, transparent solution. It requires an acidic environment to occur (either hydrochloric or sulphuric acid suffice). The amount of time it takes to reach the point of colorlessness varies depending on the concentration of the acetone. It is easy to demonstrate the relationship between concentration and rate of reaction. (By varying the temperature or amount of acid catalyst, the reaction visibly proceeds at differing rates.)

\subsection{Materials}
7 beakers, 3 syringes, stopwatch

\subsection{Chemicals}
Nail polish remover, iodine tincture, sulphuric (battery) acid, water

\subsection{Preparation}
\begin{enumerate}
\item Prepare an iodine solution using iodine tincture. Solutions purchased in the local drugstores are often 0.2 M Iodine (with several other chemicals as well). Create a 0.02 M solution by adding 9 parts water to one part tincture. Put this solution in the first beaker.
\item Prepare an acidic acetone solution by mixing one part nail polish remover to one part 1 M Sulphuric acid solution. Put this solution in the second beaker. This is roughly a 5 M solution of acetone.
\item In the third beaker place clean water.
\end{enumerate}

\subsection{Procedure}
\begin{enumerate}
\item Pour 8 mL of the acetone solution in a clean beaker. 
\item Clear and start a stopwatch and add 8 mL of iodine to the beaker. Swirl the solution and stop the watch when the solution becomes colorless. Record the time taken in a table as shown below.
\item  Repeat the experiment, but this time start with 6 mL of acetone solution and 2 mL of clean water in a clean beaker. Clear and start a stopwatch and add 8 mL of iodine to the beaker, swirl the solution, and stop the watch when the solution becomes colorless. Record the time taken.
\item Repeat the experiment for the remaining volumes of acetone solution and clean water as shown in the table below.
\end{enumerate}

\begin{center}
\begin{tabular}{|c|p{2cm}|p{2cm}|p{2cm}|p{2cm}|p{2cm}|p{1.5cm}|} \hline
Test&Volume of Iodine Solution (mL)&Volume of Acetone Solution (mL)&Volume of Water (mL)&Molarity of Acetone Solution ($^{\text{mol}}/_\text{L}$)&Time for color to disappear (s)&Reciprocal of time ($^1/_\text{s}$) \\ \hline
1&8&8&0&5 M&& \\ \hline
2&8&6&2&3.75 M&& \\ \hline
3&8&4&4&2.5 M&& \\ \hline
4&8&2&6&1.25 M&& \\ \hline
\end{tabular} \\[10pt]
\end{center}

This data can then be used to plot a graph of concentration of acetone against the rate of reaction.

\subsection{Notes}
\begin{itemize}
\item Patience is required for the tests using lower concentrations as they can take over 4 minute to complete.
\item Concentrations may vary depending on where the tincture and remover are purchased.
\item The endpoint of this reaction can be somewhat ambiguous depending on the color of the nail polish remover. Criteria for determining the endpoint may vary.
\item It should be noted that if the nail polish remover is already a specific color it will affect the final color of the solution. Some solutions may never become fully colorless.
\item The reaction used in the NECTA exams goes from transparent to opaque while this alternative goes from opaque to transparent. Make sure students understand this difference.
\item The reaction used in NECTA exams is a neutralization reaction so there is little that needs to be done to process the waste. This alternative reaction is very acidic when finished so be prepared to 	neutralize it before disposal.
\end{itemize}

\subsection{Sample Practical Question}
The following is a sample practical question from 2012.\\[6pt]



Your are provided with the following materials:\\
\begin{enumerate}
\item[ ] \textbf{ZO}:  A solution of 0.13 M Na$_2$S$_2$O$_3$ (sodium thiosulphate);
\item[ ] \textbf{UU}:  A solution of 2 M HCl;
\item[ ] Thermometer;
\item[ ] Heat source/burner;
\item[ ] Stopwatch.\\
\end{enumerate}

Procedure:\\
\begin{enumerate}
\item[(i)] Place 500 cm$^3$ beaker, which is half-filled with water, on the heat source as a water bath.
\item[(ii)] Measure 10 cm$^3$ of \textbf{ZO} and 10 cm$^3$ of \textbf{UU} into two separate test tubes.
\item[(iii)] Put the two test tubes containing \textbf{ZO} and \textbf{UU} solutions into a water bath.
\item[(iv)] When the solutions attain a temperature of 60$^o$C, remove the test tubes from the water bath and pour both solutions into 100 cm$^3$ empty beaker and immediately start the stop watch.
\item[(v)] Place the beaker with the contents on top of a piece of paper marked \textbf{X}.
\item[(vi)] Note the time taken for the mark \textbf{X} to disappear.
\item[(vii)] Repeat step (i) to (vi) at temperature 70$^o$C, 80$^o$C and 90$^o$C.
\item[(viii)] Record your results as in Table 1.
\end{enumerate}

\begin{center}
\begin{tabular}{|p{5cm}|p{5cm}|p{3cm}|}
\multicolumn{1}{l}{Table 1}&\multicolumn{1}{l}{ }&\multicolumn{1}{l}{ }\\ \hline
\textbf{Experiment}&\textbf{Temperature}&\textbf{Time (s)}\\ \hline
1&60$^o$C&\\ \hline
2&70$^o$C&\\ \hline
3&80$^o$C&\\ \hline
4&90$^o$C&\\ \hline
\end{tabular}
\end{center}

\textbf{Questions:}\\
\begin{enumerate}
\item Write a balanced chemical equation for reaction between \textbf{UU} and \textbf{ZO}.
\item What is the product which causes the solution to cloud the letter \textbf{X}?
\item Plot a graph of temperature against time (s).
\item What conclusion can you draw from you graph?\\
\end{enumerate}


\raggedleft \textbf{(15 marks)}

\raggedright

\subsubsection{Discussion}
