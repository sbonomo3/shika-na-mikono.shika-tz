\section{2010 - BIOLOGY 2A (ALTERNATIVE A PRACTICAL)}

\begin{enumerate}
\item[1.] You have been provided with solution \textbf{T$_1$}.
\begin{enumerate}
\item[(a)] Carry out food tests to identify the substances present in solution \textbf{T$_1$}. Record your work in a table as shown below.

\begin{center}
\begin{tabular}{|p{3cm}|p{3cm}|p{3cm}|p{3cm}|} \hline
\multicolumn{1}{|c|}{\textbf{Test for}}&\multicolumn{1}{c|}{\textbf{Procedure}}&\multicolumn{1}{c|}{\textbf{Observation}}&\multicolumn{1}{c|}{\textbf{Inference}} \\ \hline
&&& \\
&&& \\
&&& \\
&&& \\ \hline
\end{tabular} \\[10pt]
\end{center}

\item[(b)] What are the functions of the food substances identified in \textbf{T$_1$} in the human body?
\item[(c)] 
\begin{enumerate}
\item[(i)] State the favourable/suitable pH condition at which the enzymes which digest the food substances present in \textbf{T$_1$} work best.
\item[(ii)] Which of the food substances present in \textbf{T$_1$} is not stored in the human body?
\item[(iii)] What happens when the levels of this substance mentioned in (c) (ii) above, rises in the body?
\end{enumerate}
\end{enumerate}

\item[2.] You are provided with specimens \textbf{A}, \textbf{B}, \textbf{C}, \textbf{D} and \textbf{E}. Observe them carefully and answer the questions that follow:
\begin{enumerate}
\item[(a)]
\begin{enumerate}
\item[(i)] Write down the common names of specimens \textbf{A}, \textbf{B}, \textbf{C}, \textbf{D} and \textbf{E}.
\item[(ii)] To which kingdom do specimens \textbf{C} and \textbf{D} belong?
\item[(iii)] Name one (1) common epidemic disease transmitted by specimen \textbf{A}.
\end{enumerate}
\item[(b)]
\begin{enumerate}
\item[(i)] Draw a large well labelled diagram of specimen \textbf{C}.
\item[(ii)] State the economic importance of specimen \textbf{C}.
\end{enumerate}
\item[(c)]
\begin{enumerate}
\item[(i)] What are the distinguishing characteristics of the Phylum/Division to which specimen \textbf{E} belongs?
\item[(ii)] Where can specimen \textbf{E} be found?
\end{enumerate}

\end{enumerate}
\end{enumerate}