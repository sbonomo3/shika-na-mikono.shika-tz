\chapter{Qualitative Analysis}
\label{cha:qualana}

Qualitative analysis is the systematic identification of an unknown salt through a series of chemical tests.

\section{Overview of Qualitative Analysis}

The salts requiring identification have one cation and one anion. Generally, these are identified separately although often knowing one helps interpret the results of tests for the other. For ordinary level in Tanzania, students are confronted with binary salts made from the following ions:

\begin{itemize}
\item{Cations: NH$_{4}^{+}$, 
Ca$^{2+}$, 
Fe$^{2+}$, 
Fe$^{3+}$, 
Cu$^{2+}$, 
Zn$^{2+}$, 
Pb$^{2+}$, 
Na$^{+}$}
\item{Anions: CO$_{3}^{2-}$, 
HCO$_{3}^{-}$, 
NO$_{3}^{-}$, 
SO$_{4}^{2-}$, 
Cl$^{-}$}
\end{itemize}
At present, 
ordinary level students receive only one salt at a time. The teacher may also make use of qualitative analysis to identify unlabelled salts.

The ions are identified by following a series of ten steps, divided into three stages. These are:
\begin{itemize}
\item{Preliminary tests:
These tests use the solid salt. They are: appearance, action of heat, action of dilute H$_{2}$SO$_{4}$, action of concentrated H$_{2}$SO$_{4}$, flame test, and solubility.}
\item{Tests in solution: The compound should be dissolved in water before carrying out these tests. If it is not soluble in water, use dilute acid (ideally \ce{HNO3}) to dissolve the compound. The tests in solution involve addition of NaOH and NH$_{3}$.}
\item{Confirmatory tests: These tests confirm the conclusions students draw from the previous steps. By the time your students start the confirmatory tests, they should have a good idea which cation and which anion are present. Have students do one confirmatory test for the cation they believe is present, and one for the anion you believe is present. Even if several confirmatory tests are listed, students only need to do one. When identifying an unlabelled container, however, you might be moved to try several, especially if you are new to this process.}
\end{itemize}

%====================================
\section{Teaching Qualitative Analysis with Local and Low Cost Materials}

\subsection{General Suggestions}
\begin{itemize}
\item{Heat sources: Motopoa burners cost nothing to make (soda bottle caps) and consume only a small amount of fuel. They give a non-luminous flame ideal for flame tests and still produce enough heat for the other tests.}
\item{Test tubes: Most of the tests do not involve heating, so students may perform these experiments in plastic tubes made from disposable plastic syringes. Many of the tests requiring heating use the salt in solution and thus can be performed by holding the plastic test tube in a hot water bath. For the Action of Heat test, salts may be heating in metal spoons to observe residue products, although it is difficult to test the gases produced. Do not wait for test tubes to start teaching qualitative analysis. Do try to find at least one borosilicate (Pyrex, 
Borosil) test tube for each student before the national exams.}
\item{Litmus paper: Make your own. See the instructions in chapter on Acids and Bases. Rosella flowers give very good results.}
\item{Low cost sources of chemicals. Many chemicals have low cost alternatives, for example table salt (sodium chloride), gypsum powder (calcium sulphate), ammonium sulphate (sulphate of ammonia fertilizer), cautic soda (sodium hydroxide), soda ash (sodium carbonate), battery acid (sulphuric acid), and copper (II) sulphate (the local medicine mlutulutu). Other chemicals can be manufactured locally in small quantities, for example iron (II) sulphate, iron (III) sulphate, calcium carbonate, copper carbonate, zinc sulphate, and zinc carbonate. Indeed the preparations several of these compounds are described in the chapter on Compounds of Metals. For more information about any specific chemical, read its entry in the Sources of Chemicals section.}
\item{Share expensive chemicals among many schools. A single container of potassium ferrocyanide, for example, can supply ten or even twenty schools for several years. Schools should consider bartering 10~g of one chemical for 10~g of another. Schools without any expensive chemicals could produce Benedict's solution from local materials, for example, and exchange this for 10~g samples of expensive salts. Another alternative is for all of the schools in a district or town to pool money to buy one container of each required imported reagent, and then divide the chemicals evenly. Remember to keep chemicals in air-tight containers and out of sunlight. Also remember to label containers very clearly.}
\end{itemize} 

%===================
\subsection{Timeline of Lessons}

To teach qualitative analysis, first make sure the students know how to use the apparatus: test tubes, droppers, and motopoa burners. All of these apparatus are used in other experiments throughout this book.

Once the students are familiar with the apparatus, teach them each step separately, using the activities outlined below. Give adequate time to each step; each one can be used to review chemistry learned in previous topics.

Once the students are proficient at the individual steps, practice the whole process with example unknown salts. Sodium carbonate and locally manufactured copper (II) carbonate are good options for practice.

Finally, as the exam approaches, get some lead nitrate, a small amount of fully concentrated sulphuric acid, and some borosilicate test tubes. Use these materials to teach:
\begin{description}
\item[Conformation of sulphates by addition of lead nitrate solution]{Prepare this solution by dissolving about one level tea spoon of lead nitrate in about 100~mL of distilled (rain) water. Use only 2-3 \textit{drops} to confirm sulphates.}
\item[Thermal decomposition of nitrates to form nitrogen dioxide]{Nitrogen dioxide is a poisonous brown gas. Add a very small amount of lead nitrate to a borosilicate test tube and head strongly over a motopoa burner.}
\item[Flame test for lead]{See the instructions for flame tests below. Only a very small amount of lead nitrate is required for the test.}
\item[Conformation of nitrates by brown ring test]{Add a very small amount of lead nitrate to a test tube and dissolve in 2~mL of distilled (rain) water. In a separate test tube prepare about 1~mL of iron (II) sulphate solution from locally manufactured iron (II) sulphate (make sure it is still light green and not yellow!) in distilled water. Mix the solutions and note that lead sulphate will precipitate. Use this to teach the conformation of lead by precipitation with sulphate. Note that the nitrate remains in solution. Decant the liquid into a borosilicate test tube. Hold the test tube at an angle and carefully add about 1~mL of fully concentrated sulphuric acid down the inside. The acid will form a separate layer at the bottom. If nitrates are present, in a few minutes a brown ring should form where the two layers meet. Remember to neutralize this waste before disposal.}
\end{description}

Note that lead nitrate is poisonous. Add some dilute sulphuric acid* to all waste containing lead nitrate to precipitate any soluble lead. Note also that fully concentrated sulphuric acid is very dangerous. Only use it for the brown ring test. Dissolve one box of bicarbonate of soda in 500~mL of water and have this solution available wherever concentrated acid is being used. In the advent of acid spills, use this solution to neutralize the acid to stop burns.
%==============================================================================
\section{The Steps of Qualitative Analysis}

\subsection{Appearance}

Three properties of the salt may be observed directly: colour, texture, and smell.
\begin{description}

\item[Colour]{While most salts are white, salts of transition metals are often colored. Thus colour is an easy way to identify iron and copper cations in salts.}

\item[Texture]{Carbonates and hydrogen carbonates generally form powders although sometimes they can form crystals. Sulphate, nitrates, and chlorides are almost always founds as crystals.}

\item[Smell]{Some ammonium salts smell distinctly like ammonia. Some, however, have no smell. Therefore the smell of ammonia can confirm the presence of ammonium cations, but its absent can not be used to prove the absence of ammonium.}

\end{description}

\subsubsection{Materials}
soda bottle caps, table salt, bicarbonate of soda, soda ash (sodium carbonate), copper (II) sulphate*, ammonium sulphate*, locally manufactured iron (II) sulphate*, locally manufactured iron (III) sulphate*, locally manufactured copper (II) carbonate

\subsubsection{Preparation}
\begin{enumerate}
\item{Place a small amount of each sample in a different soda bottle cap for observation.}
\end{enumerate}

\subsubsection{Activity Steps}
\begin{enumerate}
\item{Look at the samples. Describe their colour, texture, and smell. Do not touch or inhale the salts.}
\end{enumerate}

\subsubsection{Results and Conclusion}
\begin{itemize}
\item{Colour}
\begin{description}
\item[White]{Copper and iron absent}
\item[Blue]{Copper cation present}
\item[Green]{Iron (II) or copper present}
\item[Light green]{Iron (II) present}
\item[Yellow or red-brown]{Iron (III) present}
\end{description}
\item{Texture}
\begin{description}
\item[Powder]{Carbonate or hydrogen carbonate anion present}
\item[Crystals]{Sulphate, chloride, or nitrate anion probably present}
\item[Wet crystals]{Chloride or nitrate anion present}
\end{description}
\item{Smell}
\begin{description}
\item[Smell of ammonia]{Ammonium cation present}
\item[No smell of ammonia]{Inconclusive -- some ammonium compounds have no smell}
\end{description}
\end{itemize}

\subsubsection{Clean Up}
\begin{enumerate}
\item{Collect salts for use another day. Do not mix.}
\item{Wash and return soda bottle caps.}
\end{enumerate}

\subsection{Notes}
%Note that ions by themselves have no colour. The colour in all colored salts are transition metal complex compounds. For example, copper (II) salts are blue because

Wet crystals are the result of the salt absorbing water from the atmosphere. Qualitative analysis salts with this property are not locally available. However, caustic soda (sodium hydroxide) has this property, so samples of caustic soda can be used to show the absorption of water from the air and how this changes the appearance of the salt. Note that caustic soda burns skin, blinds in eyes and corrodes metal, so care is required.
%======================================
\subsection{Action of heat}

Many salts thermally decompose when heated. When these salts decompose, they produce gases that may be identified to identify the anion of the salt. After decomposition, many salts also leave a residue that may identify the cation.

\subsubsection{Materials}
soda bottle caps, motopoa, matches, long handled metal spoons, steel wool, sand, beaker*, water, table salt, copper (II) sulphate*, bicarbonate of soda, locally prepared copper (II) carbonate*, soda ash (sodium carbonate), locally prepared zinc carbonate*

%locally prepared iron (II) sulphate* OR locally prepared iron (III) sulphate* (a mixture will also suffice), -- ?? use carbonate?? does iron (II) carbonate exist??

\subsubsection{Hazards and Safety}
\begin{itemize}
\item{Ammonium nitrate explodes when heated. For this reason, ammonium nitrate should never be used in qualitative analysis when the Action of Heat test is used.}
\end{itemize}

\subsubsection{Preparation}
\begin{enumerate}
\item{Fill a beaker with water.}
\item{Make a small pile of sand on the table for resting the hot spoon.}
\item{Place a small amount of each sample in a different soda bottle cap.}
\item{Add motopoa to another soda bottle cap to use as a burner.}
\end{enumerate}

\subsubsection{Activity Steps}
\begin{enumerate}
\item{Light the motopoa. Note that the flame will be invisible.}
\item{Place a very small amount of a sample on the spoon. Generally, the smallest amounts of sample give the best results because they are easier to heat to a hotter temperature.}
\item{Heat the sample strongly, observing all changes.}
\item{Place the hot spoon on the sand to cool.}
\item{Once the spoon has mostly cooled, dip it in the beaker of water to remove the rest of the heat.}
\item{Use the steel wool to remove all residue from the spoon.}
\item{Repeat these steps with each sample.}
\end{enumerate}

\subsubsection{Results and Conclusion}
\begin{itemize}
\item{Gas released}
\begin{description}
\item[Brown gas]{Nitrogen dioxide, nitrates present, confirmed}
\item[Colourless gas with smell of ammonia]{Ammonia, ammonium present, confirmed}
\item[Colourless gas with no smell]{Very likely carbon dioxide, especially if the compound decomposes near the start of heating, carbonate or hydrogen carbonate present}
\item[No change]{Salt probably a chloride, sulphate (very high temperatures are required to decompose many sulphates), or sodium carbonate}
\end{description}
\item{Residue}
\begin{description}
\item[No residue]{Ammonium cation present}
\item[Black residue]{Copper cation probably present}
\item[Red residue when hot, dark when cool]{Iron cation present}
\item[Yellow residue when hot, white when cool]{Zinc cation present}
\item[Red residue when hot, yellow when cool]{Lead cation present}
\end{description}
\item{Sound}
\begin{description}
\item[Cracking sound]{Sodium chloride or lead nitrate present}
\end{description}
\end{itemize}

\subsubsection{Clean Up}
\begin{enumerate}
\item{Thoroughly remove all residues from the spoons.}
\end{enumerate}

\subsubsection{Notes}
Sodium carbonate is the only carbonate used in qualitative analysis that does not thermally decompose. Therefore a white powder that does not decompose when heated is probably sodium carbonate.

%======================================
\subsection{Action of dilute \texorpdfstring{\ce{H2SO4}}{H2SO4}}

Carbonates and hydrogen carbonates react with dilute acid. Sulphates, chlorides and nitrates do not. Therefore reaction with dilute acid is useful test to help identify the anion. Sulphuric acid is used because it is the least expensive.

\subsubsection{Materials}
dilute sulphuric acid*, droppers*, bicarbonate of soda, table salt

\subsubsection{Hazards and Safety}
\begin{itemize}
\item{Use only a few drops of acid. These are all that are necessary and using more can be dangerous.}
\end{itemize}

\subsubsection{Preparation}
\begin{enumerate}
\item{Place a small amount of each sample in a different soda bottle cap.}
\item{Fill droppers with 1-2~mL dilute acid.}
\end{enumerate}

\subsubsection{Activity Steps}
\begin{enumerate}
\item{Add a few drops of acid to each sample. Observe the results.}
\end{enumerate}

\subsubsection{Results and Conclusion}
\begin{description}
\item[Bubbles of gas]{Carbon dioxide produces; carbonate or hydrogen carbonate anion present}
\item[No bubbles of gas]{Carbonate and hydrogen carbonate absent}
\end{description}

\subsubsection{Clean Up}
\begin{enumerate}
\item{Neutralize spills of dilute sulphuric acid with bicarbonate of soda.}
\item{Mix the remains from the reactions together so the extra bicarbonate of soda can neutralize the acid used to test table salt. Dilute the resulting mixture with a large amount of water and dispose down a sink, into a waste storage tank, or into a pit latrine.}
\end{enumerate}

\subsubsection{Notes}
You can confirm that the gas produced is carbon dioxide by testing to see if it extinguishes a glowing splint. To do this, light a match, use about 0.5~mL of acid (rather than a few drops), and see if the gas released will extinguish the match.

%======================================
\subsection{Action of concentrated \texorpdfstring{\ce{H2SO4}}{H2SO4}}

Concentrated sulphuric acid can convert chloride anions to hydrogen chloride gas and some nitrates to nitrogen dioxide. Because both of these gases are easy to detect, the addition of concentrated acid is used to distinguish between nitrates, chlorides, and sulphates. The concentrated acid used in this experiment should be about 5~M, similar to battery acid.

\subsubsection{Materials}
battery acid, droppers*, spoons, test tubes*, test tube rack*, test tube holder*, heat source*, hot water bath*, table salt (sodium chloride), gypsum (calcium sulphate)*, ammonium sulphate*, blue litmus paper*, beaker*, water

\subsubsection{Hazards and Safety}
\begin{itemize}
\item{Use battery acid or another source of 5~M sulphuric acid for this experiment. Do not use fully concentrated 18~M sulphuric acid directly from either industry or laboratory supply. 18~M is too concentrated and very dangerous to use.}
\item{Concentrate acid reacts violently with carbonates and hydrogen carbonates. The previous test -- the addition of dilute acid -- will detect carbonates and hydrogen carbonates. If that test is positive, do not test the sample with concentrated sulphuric acid.}
\end{itemize}

\subsubsection{Preparation}
\begin{enumerate}
\item{Place a small amount of each sample in a different soda bottle cap.}
\item{Add about 1~mL of air to each dropper syringe (no needle!) and then 2~mL of battery acid. Distribute the dropper syringes in the test tube racks so they stand with the outlet pointing down. The goal is to prevent the battery acid from reacting with the rubber plunger.}
\end{enumerate}

\subsubsection{Activity Steps}
\begin{enumerate}
\item{Light the heat source and start heating the hot water bath. The water in the hot water bath should boil.}
\item{Use the spoon to add a small amount of a sample to a test tube.}
\item{Add two drops of battery acid to the sample to make sure there is no violent reaction.}
\item{Add just enough battery acid to cover the sample. Avoid spilling drops of acid on the inside walls of the test tube.}
\item{If a brown gas is released, stop at this step.}
\item{Moisten the blue litmus paper by quickly dipping it in the water of the hot water bath.}
\item{Place the litmus paper over the mouth of the test tube to receive any gases produces. If the litmus paper changes colour, stop at this step.}
\item{Hold the test tube in the hot water bath and heat for a while. Stop heating before the acid in the test tube boils. If the litmus paper changes colour before the acid boils, this is a useful result. If the acid boils, fumes from the acid itself will change the colour of the litmus paper -- this result is not useful, and acid fumes are dangerous.}
\end{enumerate}

\subsubsection{Results and Conclusion}
\begin{description}
\item[Bubbles with a few drops of acid]{Carbonate or hydrogen carbonate anion  present}
\item[Brown gas produced]{Nitrate anion present}
\item[Litmus changes to red]{Hydrogen chloride gas produced; chloride anion present}
\item[No effect observed]{Sulphate anion probably present}
\end{description}

\subsubsection{Clean Up}
\begin{enumerate}
\item{Fill a large beaker half way with room temperature water. This will be the waste beaker.}
\item{Pour waste from the test tubes into the waste beaker.}
\item{Fill each test tube half way with water and add this water to the waste beaker.}
\item{Return unused battery acid from the droppers to a well-labelled storage container for future use. Immediately fill each dropper (syringe) with water and transfer this water to the waste beaker.}
\item{Slowly add bicarbonate of soda to the waste beaker until addition no longer causes bubbling. This is to neutralize the acid in the waste.}
\item{Dilute the resulting mixture with a large amount of water and dispose down a sink, into a waste storage tank, or into a pit latrine.}
\item{Thoroughly wash all apparatus, including the test tubes and droppers, and return them to the proper places.}
\end{enumerate}

%======================================
\subsection{Flame test}

Some metal ions produce a characteristically coloured flame when added to fire.

\subsubsection{Materials}
soda bottle caps, motopoa, metal spoons, beaker*, steel wool, water, table salt (sodium chloride), gypsum (calcium sulphate)*, copper (II) sulphate*, ammonium sulphate*

\subsubsection{Preparation}
\begin{enumerate}
\item{Fill a beaker with water.}
\item{Place a small amount of each sample in a different soda bottle cap.}
\item{Add motopoa to another soda bottle cap to use as a burner.}
\end{enumerate}

\subsubsection{Activity Steps}
\begin{enumerate}
\item{Light the motopoa. Note that the flame will be invisible.}
\item{Place a small amount of sample on the edge of the spoon. For some spoons, it is better to hold the spoon by the wide part and to place the sample on the end of the handle.}
\item{Hold the sample into the hottest part of the flame, 1-2~cm above the motopoa. If necessary, tilt the spoon so that the sample touches the flame directly. Do not spill the sample into the flame.}
\item{Dip the hot end of the spoon into the beaker of water to cool it and remove the sample. If necessary, clean the spoon with steel wool.}
\item{Repeat these steps with each sample.}
\end{enumerate}

\subsubsection{Results and Conclusion}
\begin{description}
\item[Blue or green flame]{Copper present, confirmed}
\item[Golden yellow flame]{Sodium present, confirmed}
\item[Brick red flame]{Calcium present}
\item[Bluish white flame]{Lead present}
\item[No flame colour]{Copper and sodium absent; calcium and lead probably absent; cation is probably ammonia, iron, or zinc}
\end{description}

\subsubsection{Clean Up}
\begin{enumerate}
\item{Collect unused samples for use another day.}
\item{Wash and return all apparatus.}
\end{enumerate}

%======================================
\subsection{Solubility}

\subsubsection{Materials}
soda bottle caps, two spoons, test tubes*, test tube rack*, hot water bath*, heat source*, distilled (rain) water*, table salt (sodium chloride), soda ash (sodium carbonate)*, gypsum (calcium sulphate)*, powdered coral rock (calcium carbonate)* or locally manufactured calcium carbonate* or locally manufactured copper (II) carbonate*

\subsubsection{Preparation}
\begin{enumerate}
\item{Fill a beaker with water.}
\item{Place a small amount of each sample in a different soda bottle cap.}
\end{enumerate}

\subsubsection{Activity Steps}
\begin{enumerate}
\item{Light the heat source and start heating the hot water bath. The water in the hot water bath should boil.}
\item{Decide which spoon will be used for transferring samples and which will be used for stirring.}
\item{Use the transfer spoon to transfer a very small amount of a sample to a test tube.}
\item{Add 3-5~mL of distilled water to the test tube.}
\item{Use the handle of the stirring spoon to thoroughly mix the contents of the test tube.}
\item{If the sample does not dissolve, heat the test tube in the water bath until the contents of the test tube are almost boiling (small bubbles rise from the bottom). Mix.}
\item{Repeat these steps with each sample.}

\end{enumerate}

\subsubsection{Results and Conclusion}
\begin{description}
\item[Sample dissolves in room temperature water]{Soluble salt present}
\item[Sample dissolves only in hot water]{Calcium sulphate or lead chloride present}
\item[Sample does not dissolve in even hot water]{Insoluble salt present}
\end{description}

Solubility Rules
\begin{itemize}
\item{All Group I (sodium, potassium, etc) and ammonium salts are soluble (sodium borate is an exception but not relevant to qualitative analysis)}
\item{All nitrates and hydrogen carbonates are soluble}
\item{Most chlorides are soluble (silver and lead chlorides are exceptions, 
although the latter is soluble in hot water)}
\item{Carbonates of metals outside of Group I are generally insoluble (note that aluminum and iron (III) carbonate do not exist)}
\item{Lead sulphate is insoluble and calcium sulphate is soluble only in hot water. Magnesium sulphate is completely soluble while sulphates of the Group II metals heavier than calcium 
(strontium and barium) are insoluble. All other sulphates used in qualitative analysis are soluble}]
\end{itemize}

Table of Solubility for Qualitative Analysis
\begin{center}
\begin{tabular}{ | c | c | c | c | c | c | c | c | } \hline
& ammonium & sodium & copper & iron & zinc & calcium & lead \\ \hline
nitrate & O & O & O & O & O & O & O \\ \hline
chloride & O & O & O & O & O & O & $\Delta$ \\ \hline
sulphate & O & O & O & O & O & $\Delta$ & X \\ \hline
carbonate & O & O & X & X & X & X & X \\ \hline
hydrogen carbonate & O & O & -- & -- & -- & -- & -- \\ \hline
\end{tabular}
\end{center}

KEY:
\begin{itemize}
\item{O = soluble at room temperature}
\item{$\Delta$ = soluble only when heated}
\item{X = insoluble in water}
\item{-- = salt does not exist}
\end{itemize}

\subsubsection{Clean Up}
\begin{enumerate}
\item{Collect all unused (dry) samples for use another day.}
\item{Unless copper carbonate is used, none of the salts listed in the materials section of this activity are harmful to the environment.}
\item{Dispose of solutions in a sink, waste tank, or pit latrine.}
\item{Dispose of solids and liquid wastes with precipitates in a waste tank or pit latrine -- never dispose of solids in sinks.}
\item{If using copper carbonate, collect all waste containing copper carbonate and filter to recover the copper carbonate. Save for use another day.}
\item{If you do this activity with a lead nitrate or lead chloride, collect these wastes in a separate container. Add dilute sulphuric acid dropwise until no further precipitation is observed. Neutralize with bicarbonate of soda. Dispose this mixture in a waste tank or a pit latrine. The lead sulphate precipitate is highly insoluble will not enter the environment.}
\item{Wash and return all apparatus.}
\end{enumerate}

\subsubsection{Notes}
Calcium carbonate or copper carbonate are recommended qualitative analysis salts to use as examples of insoluble salts. If these are difficult to get, other insoluble compounds may be used for teaching this specific step (but not for other parts of qualitative analysis). Examples of other insoluble compounds include sulphur power, manganese (IV) oxide from batteries, and chokaa (calcium hydroxide, which is only slightly soluble so a significant precipitate will remain).
%========================================
\subsection{Addition of NaOH solution}

\subsubsection{Materials}
soda bottle caps, two spoons, test tubes*, test tube rack*, beakers*, medium droppers (5~mL syringes without needles)*, large droppers (10~mL syringes without needles), caustic soda (sodium hydroxide)*, table salt (sodium chloride), ammonium sulphate*, copper (II) sulphate*, locally manufactured iron (II) sulphate*, locally manufactured iron (III) sulphate*, locally manufactured zinc sulphate*, distilled (rain) water

\subsubsection{Preparation}
\begin{enumerate}
\item{Fill a 500~mL water bottle about half way with distilled (rain) water.}
\item{Add one level tea spoon of caustic soda and then wash the spoon.}
\item{Label the bottle ``1~M sodium hydroxide -- corrosive''}
\item{Place a small amount of each sample in a different soda bottle cap.}
\item{Pour some of the sodium hydroxide solution into a clean beaker.}
\item{For each small dropper syringe, suck in about 1~mL of air and then add about 4~mL of sodium hydroxide solution. Distribute the dropper syringes in the test tube racks so they stand with the outlet pointing down. The goal is to prevent the sodium hydroxide from reacting with the rubber plunger.}
\end{enumerate}

\subsubsection{Activity Steps}
\begin{enumerate}
\item{Decide which spoon will be used for transferring samples and which will be used for stirring.}
\item{Use the transfer spoon to transfer a very small amount of a sample to a test tube.}
\item{Use the large dropper syringe to add 3-5~mL of distilled water to the test tube.}
\item{Use the handle of the stirring spoon to thoroughly mix the contents of the test tube.}
\item{Use the small dropper to add a few drops of sodium hydroxide solution to the test tube.}
\item{Observe the colour of any precipitate formed. Also waft the air from the top of the test tube towards your nose to test for smell.}
\item{If a white precipitate forms, use the stir spoon to transfer a very small quantity of the precipitate to a clean test tube. Add 1-2~mL of sodium hydroxide directly to this sample to see if the precipitate is soluble in excess sodium hydroxide solution.}
\end{enumerate}

\subsubsection{Results and Conclusion}
\begin{description}
\item[No precipitate and smell of ammonia]{Ammonium cation present, confirmed}
\item[No precipitate and no smell]{Sodium cation probably present}
\item[Blue precipitate]{Copper (II) cation present}
\item[Green precipitate]{Iron (II) cation present}
\item[Red-brown precipitate]{Iron (III) cation present}
\item[White precipitate not soluble in excess NaOH]{Calcium cation present}
\item[White precipitate soluble in excess NaOH]{Lead or zinc cation present}
\end{description}

\subsubsection{Clean Up}
\begin{enumerate}
\item{Save all waste from this experiment, labelling it ``basic qualitative analysis waste, no heavy metals'' and leave it in an open container. Over time atmospheric carbon dioxide will react with the sodium hydroxide to make less harmful carbonates. After 2-3 days, dispose of the waste in a waste tank or a pit latrine.}
\end{enumerate}

%======================================
\subsection{Addition of \texorpdfstring{\ce{NH3}}{NH3} solution}

This test is very similar to the addition of sodium hydroxide solution. The useful difference is that zinc forms a precipitate in ammonia that is soluble in excess ammonia whereas lead forms a precipitate in ammonia that is not soluble in excess ammonia. Therefore, this test is mainly used to separate lead and zinc. Neither lead salts nor ammonia are locally available in Tanzania. Because the process of this test is the same as the addition of NaOH and the results so similar, students can adequately learn about the Addition of \ce{NH3} test by practicing the Addition of NaOH. For the national exam, a small amount of ammonia solution can be obtained.

Note also that the addition of ammonia to a solution of copper (II) will produce a blue precipitate that dissolves in excess ammonia to form a deep blue solution. This is a useful conformation of the presence of copper, but such conformation is generally unnecessary because the flame test for copper is so reliable.

If you have ammonia solution, store it in a well-sealed container to prevent the ammonia from escaping. A good container for this is a well labelled plastic water bottle with a screw on cap.

%======================================
\subsection{Confirmatory tests}

Every cation and anion has at least one specific test that can be used to prove its presence. Not all of these tests are possible with local materials, but many of them are. The following list shows how to confirm each possible cation and anion.

\subsubsection{Ammonium}
\begin{itemize}
\item{Example salt: ammonium sulphate*}
\item{Procedure: add sodium hydroxide solution and heat in a water bath}
\item{Confirming result: smell of ammonia} 
\item{Reagents: NaOH solution as used above}
\end{itemize}

\subsubsection{Calcium}
\begin{itemize}
\item{Example salt: calcium sulphate}

\item{Procedure: Two options
\begin{enumerate}
\item{flame test}
\item{addition of NaOH solution}
\end{enumerate}
} % Procedure

\item{Confirming results:
\begin{enumerate}
\item{flame test: brick red flame}
\item{addition of NaOH: white precipitate insoluble in excess}
\end{enumerate}
} % Confirming results

\item{Reagents:
\begin{enumerate}
\item{none}
\item{NaOH solution}
\end{enumerate}
} % Reagents

\end{itemize} % Calcium

\subsubsection{Copper}
\begin{itemize}
\item{Example salt: copper sulphate}
\item{Procedure: flame test}
\item{Confirming result: blue/green flame}
\item{Reagents: none}
\end{itemize}

\subsubsection{Iron (II)}
\begin{itemize}
\item{Example salt: locally manufactured iron sulphate 
(keep away from water and air)}
\item{Procedure: addition of sodium hydroxide solution 
and then transfer of precipitate to the table surface}
\item{Confirming result: green precipitate 
that oxidizes to brown when exposed to air}
\item{Reagent: sodium hydroxide solution from above}
\end{itemize}

\subsubsection{Iron (III)}
\begin{itemize}
\item{Example salt: locally manufactured iron sulphate 
(oxidized by water and air)}
\item{Procedure: addition of sodium ethanoate solution}
\item{Confirming result: yellow to red solution}
\item{Reagent: slowly add bicarbonate of soda to vinegar; stop adding when further addition does not cause bubbles; label the solution ``sodium ethanoate for detection of iron (III)''}
\end{itemize}

\subsubsection{Lead}
\begin{itemize}
\item{Example salt: no local sources for safe manufacture, consider purchasing lead nitrate}

\item{Procedure: Three options
\begin{enumerate}
\item{flame test} 
\item{addition of dilute sulphuric acid}
\item{addition of potassium iodide solution}
\end{enumerate}
} % Procedure

\item{Confirming results:
\begin{enumerate}
\item{flame test: blue/white flame}
\item{addition of dilute sulphuric acid: white precipitate}
\item{addition of KI solution: yellow precipitate that dissolves when heated and reforms when cold}
\end{enumerate}
} % Confirming results

\item{Reagents:
\begin{enumerate}
\item{none but a very hot flame, e.g. Bunsen burner, is required} 
\item{dilute sulphuric acid used in Step 5 above}
\item{obtain pure potassium iodide by evaporating iodine tincture until only white crystals remain; do this outside and do not breathe the fumes; it might also be possible to use the KI solution prepared for electrolysis in the chapter on ionic theory}
\end{enumerate}
} % Reagents

\end{itemize} % Lead

\subsubsection{Sodium}
\begin{itemize}
\item{Example salts: sodium chloride, sodium carbonate, sodium hydrogen carbonate}
\item{Procedure: flame test}
\item{Confirming result: golden yellow flame}
\item{Reagents: none}
\end{itemize}

\subsubsection{Zinc}
\begin{itemize}
\item{Example salt: locally manufactured zinc carbonate 
or zinc sulphate}
\item{Procedure: addition of 0.1~M potassium ferrocyanide solution}
\item{Confirming result: gelatinous gray precipitate}
\item{Reagents: no local source of potassium ferrocyanide -- consider collaborating with many schools to share a container; only a very small quantity is required}
\end{itemize}

\subsection{Confirmatory Tests for the Anion} 

\subsubsection{Hydrogen carbonate}
\begin{itemize}
\item{Example salt: sodium hydrogen carbonate}
\item{Procedure: add magnesium sulphate solution and then boil in a water bath}
\item{Confirming result: white precipitate forms only after boiling}
\item{Reagent: dissolve Epsom salts (magnesium sulphate)* in distilled (rain) water*}
\end{itemize}

\subsubsection{Carbonate}
\begin{itemize}
\item{Example salt: sodium carbonate}
\item{Procedure for soluble salts: addition of magnesium sulphate solution}
\item{Confirming result: white precipitate forming in cold solution}
\item{Reagent: dissolve Epsom salts (magnesium sulphate)* in distilled (rain) water*}

\item{Note that insoluble salts that effervesce with dilute acid are likely carbonates. 
None of the other anions described here produce gas with dilute acid. Note also that all hydrogen carbonates are soluble.}

\end{itemize}

\subsubsection{Chloride}
\begin{itemize}
\item{Example salt: sodium chloride}
\item{Procedure: Three Options
\begin{enumerate}
\item{addition of silver nitrate solution}
\item{addition of manganese (IV) oxide and concentrated sulphuric acid followed by heating in a water bath}
\item{addition of weak acidified potassium permanganate solution followed by heating in a water bath}
\end{enumerate}
} % Procedure
\item{Confirming results:
\begin{enumerate}
\item{silver nitrate: white precipitate of silver chloride} 
\item{manganese (IV) oxide: production of chlorine gas that bleaches litmus} 
\item{acidified permanganate: decolourization of permanganate}
\end{enumerate}
} % Confirming results
\item{Reagents:
\begin{enumerate}
\item{silver nitrate has no local source but may be shared among many schools as only a very small amount is required.}
\item{Manganese dioxide may be purified from used batteries and battery acid is concentrated sulphuric acid. Note that careful purification is required to remove all chlorides from the battery powder. This method is useful because of its low cost, but remember that chlorine gas is poisonous! Students should use very little sample salt in this test.}
\item{Prepare a solution of potassium permanganate, dilute with distilled water until the colour is light pink, and then add about 1 percent of the solution's volume in battery acid. Note that this solution will cause lead to precipitate, and will also be decolourized by iron II, so it is not a perfect substitute for silver nitrate. This final option is also not yet recognized by examination boards, i.e. NECTA}
\end{enumerate}
} % Reagents
\end{itemize}

\subsubsection{sulphate}
\begin{itemize}
\item{Example salt: copper sulphate, calcium sulphate, iron sulphate}
\item{Procedure: addition of a few drops of a solution of lead nitrate, 
barium nitrate, or barium chloride}
\item{Confirming result: white precipitate}
\item{Reagents: none of these chemicals have local sources. Because lead nitrate is also an example salt, it is the most useful and the best to buy. The ideal strategy is to share one of these chemicals among many schools. Remember that all are quite toxic.}
\end{itemize}

\subsubsection{Notes}

Emphasize to students that they need to carry out only one confirmatory test for the cation, 
and one for the anion. If the test gives the expected result, then they can be sure that the ion they have identified is present. If the test does not give the expected result, they have probably made a mistake, and they should revisit the results of their previous tests 
and choose a different possibility to confirm.

%==============================================================================	
\section{Hazards and Cleanliness}
Qualitative analysis practicals are full of hazards, 
from open flames to concentrated acids. 
To reduce the risk of accidents, 
teach students how to use their flame source 
before the day of the practical, 
especially if you are using Bunsen burners. 
Most students have never used gas before, 
and do not know the basic safety precautions involved in using gas. 
If you have choice about what salts are offered, 
do away with those requiring concentrated acid, and poisonous reagents like lead and barium.

Teach students to hold their test tubes at an angle when they heat them or perform reactions in them. Test tubes should be pointed away from the student holding them and from other students. 
This will prevent injuries due to splashing chemicals, and will also minimize inhalation of any gases produced.

Teach students never to fill test tubes or any other container more than half. That way, 
they minimize spills and boiling over of chemicals during heating. In addition, this also prevents bumping in the test tubes (when a gas bubble forms suddenly), which can cause dangerous spray.

Teach students that if they get chemicals on their hands, they should wash them off immediately, without asking for permission first. Some students have been taught to wait for a teacher's permission before doing anything in the lab, even if concentrated acid is burning their hands. On the first day, give them permission to wash their hands if they ever spill chemicals on them. Also, teach students to tell you immediately when chemicals are spilled. 
Sometimes they hide chemical spills for fear of punishment. Do not punish them for spills -- legitimate accidents happen. Do punish them for unsafe behavior of any kind, even if it does not result in an accident. 

Practicals involving nitrates, chlorides, ammonium compounds, and some sulphates produce harmful gases. Open the lab windows to maximize airflow. Kerosene stoves also produce noxious fumes -- it is much better to use motopoa. If students feel dizzy or sick from the fumes, let them go outside to recover.

Make absolutely sure that students clean their tables and glassware before they leave. Leaving chemicals lying around is dangerous, especially when they are not labelled. Qualitative analysis experiments can leave residues in glass test tubes that are difficult to clean with brushes alone. If possible, heat samples in metal spoons. To remove stubborn residues in glass test tubes, pour a little dilute nitric acid into the test tube. The acid should dissolve the precipitate and leave a clean test tube behind. Remember that the utility of nitric acid -- 
that it will dissolve almost anything -- is also a serious hazard.

%==============================================================================
\section{Preparation of Copper Carbonate for Qualitative Analysis}

After teaching students all of the individual steps of qualitative analysis, it is good to allow them to practice all of the step on one practical session, as they will have to do for NECTA. The following activity describes how to prepare copper carbonate and how to perform qualitative analysis on it. The teacher should try this activity alone first to review qualitative analysis and then students should perform this activity in groups.

\subsubsection{Objective}
\begin{itemize}
\item{To perform all steps of qualitative analysis to identify an unknown salt}
\end{itemize}

\subsection{Materials}
plastic water bottles, filter funnel*, heat source* and take-away tray (optional), plastic spoon, copper (II) sulphate*, soda ash (sodium carbonate)*

\subsection{Preparation}
\begin{enumerate}
\item{Combine 5 spoons of copper (II) sulphate and 200 mL water in a plastic bottle. Cap and shake until the copper (II) sulphate has fully dissolved.}
\item{Combint 10 spoons of sodium carbonte and 200 mL water in a separate plastic bottle. Cap and shake until the sodium carbonate has fully dissolved.}
\item{Combine the two solutions. A green/blue precipitate should form.}
\item{Alternatively, if another form is practising precipitation reactions using copper (II) sulphte and sodium carbonate, save their solid waste.}
\item{Pour the mixture into a filter funnel. Let sit until all liquid has passed through.}
\item{Transfer the solid to a plastic bottle and add about half a litre of water. Shake thoroughly. This is to remove any sodium or sulphate from the original chemicals.}
\item{Pour the mixture into a filter funnel. Let sit until all liquid as passed through.}
\item{Dry the precipitate, either in the sun or by heating very gentle in a take-away container over a heat source. If heating, stir often, and remove from the heat before all the water has evaporated or else the copper carbonate will start to thermally decompose.}
\item{Transfer the dry blue powder to a clean container and label it ``copper (II) carbonate.''}
\end{enumerate}

\subsubsection{Activity Procedure}
\begin{enumerate}
\item{Perform the qualitative analysis steps described above on the sample.}
\end{enumerate}

\subsubsection{Notes}
Because copper carbonate is insoluble in water, add a little dilute sulphuric acid solution to bring the ion into solution for the NaOH test. Add the acid drop by drop and avoid adding excess.
