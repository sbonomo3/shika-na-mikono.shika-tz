\chapter{Improving an Existing School Laboratory} \index{Laboratory! improving}

If there is already a laboratory at your school, 
the immediate tasks are to see what it has, 
make it safe, 
get it organized, 
make repairs, 
and ensure smart use with sound management.

\section{Inventory}

\begin{itemize}

\item Making a list of what and how much of everything is in your lab is easy, 
if time consuming. 
Difficulties arise when you find apparatus you have never seen before, 
or containers of chemicals without labels.

\item There is no harm in unknown apparatus, 
they just are not useful until you know what they do. 
Ask around.

\item Unknown chemicals, 
however, 
pose a hazard, 
because it is unclear how to properly store them or how to clean up spills. 
If a chemical is unknown, 
there is no safe way to responsibly dispose of it. 
Therefore, 
it is best to attempt to identify unknown chemicals. 
For assistance in identifying unknown chemicals, 
please see \nameref{cha:unknownchemicals} (p.~\pageref{cha:unknownchemicals}).

\item Burettes and apparatus concerning electricity, 
for example voltmeters and ammeters, 
should be tested to ensure that they work. 
Please consult \nameref{cha:volanatech} (p.~\pageref{cha:volanatech})
to learn how to use burettes 
and \nameref{cha:voltamm} (p.~\pageref{cha:voltamm}) to do just that.

\end{itemize}

\section{Organize}

\subsection{Have enough space}
The key to organization is having enough space. 
Usually, 
this means building shelves. 
In the long term, 
find a carpenter to build good shelves. 
In the short term, 
boards and bricks, 
scrap materials, 
chairs, 
anything to provide sturdy and horizontal storage space. 
It should be possible to read the label of every chemical, 
and to see each piece of equipment

\subsection{Apparatus}
\begin{itemize}
\item{Arrange apparatus neatly so it is easy to find each piece.}
\item{Put similar things together.}
\item{Beakers can be nested like Russian dolls.}
\end{itemize}

\subsection{Chemicals}
\begin{itemize}
\item{Organize chemicals alphabetically. 
There are more complicated schemes involving the function 
or the properties of the chemical but what is most important 
is a scheme that everyone working in the lab can follow. 
ABC is the easiest, 
and has the best chance of being used. A good alternative is to organize by chemical makeup (e.g. sodium, etc.)}
\item{Glass bottles of liquid chemicals should be kept on the floor, 
unless the laboratory is prone to flooding, 
in which case they should be on a sufficiently elevated, 
broad and stable surface. 
What you do not want are these bottles falling and breaking open.}
\item{Million's Reagent, 
benzene, 
and other chemicals that should never be used should be kept in a special place, 
ideally locked away, 
and labelled to prevent use. 
See \nameref{cha:dangerchem} (p.~\pageref{cha:dangerchem}) for a list of chemicals that should never be used.}
\item{Label plastic containers directly with a permanent pen, 
especially if the printed label is starting to come off.} 
\item{Replace broken or cracked containers with new ones.}
\end{itemize}

\subsection{Make a map and ledger}

\begin{itemize}
\item Once you have labeled and organized everything in a lab, 
draw a map. 
\item Sketch the layout of your laboratory 
and label the benches and shelves. 
\item In a ledger or notebook, 
write down what you have and the quantity. 
For example, 
Bench 6 contains 20 test tubes, 
3 test tube holders, 
and 4 aluminum pots. 
\end{itemize}
This way, 
when you need something specific, 
you can find it easily. 
Further, 
this helps other teachers -- especially new ones -- better use the lab. 
Finally, 
having a continuously updated inventory will let you know what 
materials need to be replaced or are in short supply. 
Proper inventories are a critical part of maintaining a laboratory, 
and they really simplify things around exam time.


\section{Repair/Improve}

Once the lab is organized, 
it is easy to find small improvements. 
Here are some ideas:

\subsection{Build more shelves}
You really cannot have too many.

\subsection{Fix broken burettes}
Burettes are useful, 
expensive and -- if glass -- fragile. 
Broken burettes can often be made functional again. 
If you have broken burettes, 
see \nameref{cha:burettes} (p.~\pageref{cha:burettes}).

\subsection{Check voltage and current meters}
Voltmeters, ammeters and galvanometers often get discarded or unused despite still being able to function sufficiently for use during demonstrations and practicals. Before getting rid of these meters, see the section on \nameref{cha:voltamm} (p.~\pageref{cha:voltamm}).

\subsection{Identify key apparatus needs}
Sometimes a few pieces of apparatus can be very enabling, 
like enough measuring cylinders, 
for example. 
Buy plastic!

\section{What next?}

Once the lab is safe and organized, 
develop a system for keeping it that way. 
Consider the advice in \nameref{cha:routineup} (p.~\pageref{cha:routineup}). 
Make sure students and other teachers in involved.

Then, 
start using the lab! Every class can be a lab class. 
That is the whole point.
