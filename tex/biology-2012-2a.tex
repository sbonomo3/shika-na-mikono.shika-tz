\section{2012 - BIOLOGY 2A (ACTUAL PRACTICAL A)}

\begin{enumerate}
\item[1.] You have been provided with specimens \textbf{F} and \textbf{G}.
\begin{enumerate}
\item[(a)] Study specimens \textbf{F} and \textbf{G} carefully, then:
\begin{enumerate}
\item[(i)] Identify specimens \textbf{F} and \textbf{G} using their common names.
\item[(ii)] Compare specimens \textbf{F} and \textbf{G}, then state their observable differences.
\item[(iii)] Briefly explain the types of germination which occurs in specimens \textbf{F} and \textbf{G}.
\end{enumerate}
\item[(b)] Using a scalpel, remove the outer coat from specimen \textbf{F}. Split the two parts with the inner sides facing upwards. Then:
\begin{enumerate}
\item[(i)] Draw a well labelled diagram to show the structures of one part of the split specimen \textbf{F} as would be seen from above.
\item[(ii)] For each structure labelled in specimen \textbf{F}, state the role they play in seed germination.
\end{enumerate}
\item[(c)] Using a scalpel, prepare a longitudinal section of specimen \textbf{G}.
\begin{enumerate}
\item[(i)] Draw a well labelled diagram of the cut surface of specimen \textbf{G}.
\item[(ii)] Identify the part used by specimen \textbf{G} to absorb water during seed germination.
\end{enumerate}
\end{enumerate}

\item[2.] You have been provided with specimens \textbf{H}, \textbf{I}, \textbf{J} and\textbf{K}.
\begin{enumerate}
\item[(a)] Study carefully specimens \textbf{H} and \textbf{I} then:
\begin{enumerate}
\item[(i)] Identify specimens \textbf{H} and \textbf{I} using their common names.
\item[(ii)] Suggest the mode of locomotion of specimens \textbf{H} and \textbf{I}. Give reason to support your answer.
\item[(iii)] State the features used to place specimen \textbf{H} in the Kingdom Animalia.
\end{enumerate}
\item[(b)] Use the hand lens to observe specimens \textbf{J} and \textbf{K} then:
\begin{enumerate}
\item[(i)] Identify specimens \textbf{J} and \textbf{K} by their common names.
\item[(ii)] Name the habitats for each of specimens \textbf{J} and \textbf{K}.
\item[(iii)] Briefly explain the features which enable specimen \textbf{H} to survive in its habitat.
\item[(iv)] Classify specimens \textbf{J} and \textbf{K} to the phylum level.
\item[(v)] Write down one advantage and one disadvantage for each specimen \textbf{J} and \textbf{K}.
\end{enumerate}
\end{enumerate}

\end{enumerate}