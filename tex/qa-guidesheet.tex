\newgeometry{margin=1cm}
%\begin{landscape}
\thispagestyle{empty}

\chapter{Qualitative Analysis Guidesheet}

The goal of qualitative analysis is to identify an unknown salt. The ions used in the O-level qualitative analysis are:\\
Cations: NH$_4^+$, Ca$^{2+}$, Fe$^{2+}$, Fe$^{3+}$, Cu$^{2+}$, Zn$^{2+}$, Pb$^{2+}$, Na$^+$\\
Anions: CO$_3^{2-}$, HCO$_3^-$, NO$_3^-$, SO$_4^{2-}$, Cl$^-$\\
The ions are identified by following a series of ten steps, divided into three stages. These are:
\begin{enumerate*}
\item \textbf{Preliminary tests}: Preliminary tests use the solid salt. They are: appearance, flame test, action of heat, action of dilute H$_2$SO$_4$, action of concentrated H$_2$SO$_4$, and solubility. \hfill
\item \textbf{Tests in solution}: The compound should be dissolved in water before carrying out these tests. If it is not soluble in water, it should be dissolved in dilute HNO$_3$. The tests in solution are addition of NaOH and addition of NH$_3$. \hfill
\item \textbf{Confirmatory tests}: These tests confirm that the conclusions of the previous eight steps are correct. One confirmatory test should be done for the cation and one confirmatory test should be done for the anion.
\end{enumerate*}

\noindent \textbf{Description of Each Step}

\begin{enumerate}
\item \textbf{Appearance}: Observe three properties: the color of the salt, whether the salt is powdery or crystalline, and the salt’s smell.
\item \textbf{Flame Test}: A solid sample of the salt is placed in a flame using a wire, spatula, or glass rod. Some salts produce a characteristic flame color.
\item \textbf{Action of Heat}: The purpose of this test is to decompose the salt. Observe two things: whether a gas is formed, and whether a residue is formed. Damp pieces of blue and red litmus paper should be held over the test tube to test the acidity or alkalinity of any gas formed. Gases can be problematic. The only easily-identified gases are nitrogen dioxide (which is brown) and ammonia (which is alkaline). Most sulfates do not decompose to sulphur dioxide when heated, so do not expect to identify a gas if sulfate is present. Carbonates decompose, but carbon dioxide does not usually alter the color of litmus paper. Chlorides do not decompose when heated. So a test that produces no identifiable gas is inconclusive: a carbonate, sulfate, or chloride may be present.
\item \textbf{Action of dilute H$_2$SO$_4$}: This step tests for carbonates. If effervescence (bubbles) is observed, then carbonates or hydrogen carbonates are present. If there is no effervescence, then they are absent.
\item \textbf{Action of concentrated H$_2$SO$_4$}: If no identifiable gas was formed due to the action of heat or dilute H$_2$SO$_4$, the addition of concentrated H$_2$SO$_4$ will help determine whether a chloride or sulfate is present. Gently heat the solution and observe if any gas is present using litmus paper. If no gas is formed, then a sulfate is present. If a colorless, acidic gas (hydrogen chloride) forms, then a chloride is present. Do not heat the solution until boiling, as this will created sulphuric acid fumes.
\item \textbf{Solubility}: Determine if the salt is soluble in water, and observe the color of the solution. If the salt is not soluble in cold water, then test its solubility in warm water. A solubility chart can be used to help identify possible salts.
\item \textbf{Addition of NaOH solution}: NaOH solution is added to a solution of the salt. The purpose of this test is to identify the cation present. A small amount of NaOH solution should be added first (to identify the color of the precipitate), then a large amount should be added (to test if the precipitate is soluble in excess).
\item \textbf{Addition of NH$_3$ solution}: NH$_3$ solution is added to a solution of the salt. The purpose of this test is to identify the cation. As with NaOH, a small amount of solution should be added first, followed by a large amount.
\end{enumerate}




\begin{table}
\centering

%\begin{center}
\begin{tabular}{| p{3.5cm} | p{4cm} | p{8cm} | p{4cm} | p{4cm} |}
\hline

\multicolumn{1}{|c}{\textbf{Required Chemical}} & 
\multicolumn{1}{|c}{\textbf{Low Cost Alternative}} & 
\multicolumn{1}{|c}{\textbf{Substiution Method}} & 
\multicolumn{1}{|c}{\textbf{Molarity Example}} & 
\multicolumn{1}{|c|}{\textbf{Concentration Example}} \\ \hline

Hydrochloric Acid & 
Sulfuric Acid (Battery Acid) & 
If you are required to prepare an X molarity solution of HCl, prepane a X$\times 0.5$ molarity solution of battery acid & 
The instructions call for 0.12~M HCl. Instead, prepare 0.06~M sulfuric acid & 
 \\ \hline

Ethanoic (Acetic) Acid & 
Sulfuric Acid (Battery Acid) & 
If you are required to prepare an M molarity solution of ethanoic acid, prepare a M$\times 0.5$ molarity solution of sulfuric acid & 
The instructions call for 0.10~M ethanoic acid. Prepare 0.05~M sulfuric acid. & 
 \\ \hline

Ethandioic (Oxalic) Acid dihydrate (C$_{2}$H$_{2}$O$_{4} \cdot$2H$_{2}$O) & 
Sulfuric Acid (Battery Acid) & 
If you are required to prepare an M molarity solution of ethandioic acid, prepare an M molarity solution of sulfuric acid. If you are required to prepare a C concentration solution of ethandioic acid, prepare a $^\text{C}/_{126}$ molarity solution of sulfuric acid. & 
The instructions call for 0.075~M ethandioic acid. Prepare 0.075~M sulfuric acid. & 
The instructions call for 6.3 $^\text{g}/_\text{L}$ ethandioic acid. Prepare 0.05~M sulfuric acid. \\ \hline

Potassium Hydroxide & 
Sodium Hydroxide (Caustic Soda) & 
For M molarity potassium hydroxide, make M molarity sodium hydroxide. For C concentration potassium hydroxide, make C$\times ^{40}/_{56}$ concentration sodium hydroxide. & 
The instructions call for 0.1~M potassium hydroxide. Just prepare 0.1~M sodium hydroxide. &
The instructions call for 2.8 $^\text{g}/_\text{L}$ potassium hydroxide. Prepare 2 $^\text{g}/_\text{L}$ sodium hydroxide. \\ \hline

Anhydrous Sodium Carbonate & 
Sodium Carbonate Decahydrate (Soda Ash) &  
For M molarity anhydrous sodium carbonate, make M molarity sodium carbonate decahydrate. For C concentration anhydrous sodium carbonate, make C$\times ^{286}/_{106}$ sodium carbonate decahydrate. & 
The instructions call for 0.09~M anhydrous sodium carbonate. Make 0.09~M sodium carbonate decahyrate. & 
The instructions call for 5.3 $^\text{g}/_\text{L}$ anhydrous sodium carbonate. Make 14.3 $^\text{g}/_\text{L}$ sodium carbonate decahydrate. \\ \hline

Anhydrous Sodium Carbonate & 
Sodium Hydroxide (caustic soda) & 
For M molarity anhydrous sodium carbonate, make M$\times 2$ molarity sodium hydroxide. For C concentration anhydrous sodium carbonate, make C$\times 2 \times ^{40}/_{106}$ sodium hydroxide. & 
The instructions call for 0.09~M anhydrous sodium carbonate. Make 0.18~M sodium hydroxide. & 
The instructions call for 5.3 $^\text{g}/_\text{L}$ anhydrous sodium carbonate. 4.0 $^\text{g}/_\text{L}$ sodium hydroxide. \\ \hline

Sodium Carbonate Decahydrate (Na$_2$CO$_3\cdot$10H$_2$O) &
sodium hydroxide (caustic soda) &
For M molarity sodium carbonate ecahydrate, make M$\times 2$ molarity sodium hydroxide. For C concentration sodium carbonate decahydrate, make C$\times 2 \times ^{40}/_{286}$ sodium hydroxide. &
The instructions call for 0.09~M sodium carbonate decahydrate. Make 0.18~M sodium hydroxide. &
The instructions call for 14.3 $^\text{g}/_\text{L}$ sodium carbonate decahydrate. Make 4.0 $^\text{g}/_\text{L}$ sodium hydroxide. \\ \hline

Anhydrous Potassium Carbonate & 
Sodium Carbonate decahydrate (Soda Ash) & 
For M molarity potassium carbonate, make M molarity sodium carbonate decahydrate. For C concentration potassium carbonate, make C$\times ^{286}/_{122}$ concentration sodium carbonate. & 
The instructions call for 0.08~M anhydrous potassium carbonate. Prepare 0.08~M sodium carbonate decahydrate. & 
The instructions call for 6.1 $^\text{g}/_\text{L}$ anhydrous potassium carbonate. Prepare 14.3 $^\text{g}/_\text{L}$ sodium carbonate decahydrate. \\ \hline

Anhydrous Potassium Carbonate & 
Sodium Hydroxide (caustic soda) & 
For M molarity potassium carbonate, make M$\times 2$ molarity sodium hydroxide. For C concentration potassium carbonate, make C$\times 2 \times ^{40}/_{122}$ concentration sodium hydroxide. & 
The instructions call for 0.08~M anhydrous potassium carbonate. Prepare 0.16~M sodium hydroxide. & 
The instructions call for 6.1 $^\text{g}/_\text{L}$ anhydrous potassium carbonate. Prepare 4.0 $^\text{g}/_\text{L}$ sodium hydroxide. \\ \hline

\end{tabular}
%\end{center}

\end{table}
%\end{landscape}
\restoregeometry