\section{2008 - BIOLOGY 2A (ALTERNATIVE A PRACTICAL)}

\begin{enumerate}
\item[1.] You have been provided with specimens S$_1$, S$_2$, S$_3$ and S$_4$. Observe the specimens carefully and answer the following questions:
\begin{enumerate}
\item[(a)]
\begin{enumerate}
\item[(i)] What characteristics are common among specimens S$_1$, S$_2$, S$_3$ and S$_4$? \hfill \textbf{(3 marks)}
\item[(ii)] Name the kingdom and phylum/division to which specimens S$_1$, S$_2$, S$_3$ and S$_4$ belong.\flushright \textbf{(4 marks)}\\
\item[(iii)] Why are S$_3$ and S$_4$ placed in different classes? \hfill \textbf{(2 marks)}
\end{enumerate}
\item[(b)]
\begin{enumerate}
\item[(i)] What distinctive features place specimen S$_2$ in its respective kingdom? \hfill \textbf{(2 marks)}
\item[(ii)] Why are specimens S$_3$ and S$_4$ classified under the same phylum? \hfill \textbf{(4 marks)}
\end{enumerate}
\item[(c)]
\begin{enumerate}
\item[(i)] Suggest how the specimen labelled S$_1$ is adapted to its mode of life. \hfill \textbf{(4 marks)}
\item[(ii)] Give reasons why specimen S$_1$ can not grow taller? \hfill \textbf{(2 marks)}
\end{enumerate}
\item[(d)] Describe the advantages and disadvantages of the organisms which belong to the class into which S$_3$ is found. \hfill \textbf{(4 marks)} \\
\end{enumerate}

\item[2.] You have been provided with a variegated leaf and iodine solution. Carefully follow the instructions given below and answer the questions that follow.
\begin{enumerate}
\item[]
\begin{enumerate}
\item[(i)] Heat some water to boiling point in a beaker and then turn off the source of heat.
\item[(ii)] Use forceps to dip the leaf in the hot water for about 30 seconds.
\item[(iii)] Remove the leaf from the beaker.
\item[(iv)] Push the leaf into the bottom of the test-tube and cover it with alcohol (ethanol).
\item[(v)] Place the tube in hot water until the alcohol boils together with the leaf.
\item[(vi)] Remove the leaf from the test-tube containing ethanol and dip it into hot water.
\item[(vii)] Spread the decolourized leaf on a white tile and drop iodine solution on to it.
\end{enumerate}
\item[(a)] What was the aim of the experiment?
\item[(b)] Why was the leaf dipped in hot water for 30 seconds?
\item[(c)]
\begin{enumerate}
\item[(i)] Give reason, why the leaf was boiled in ethanol?
\item[(ii)] Why was the leaf dipped once again in hot water?
\end{enumerate}
\item[(d)] Give the interpretation of the results observed when a few drops of iodine solution were poured onto the decolourized leaf.
\flushright \textbf{(25 marks)}
\end{enumerate}

\end{enumerate}