\section{2009 - CHEMISTRY 2A ALTERNATIVE A PRACTICAL \hfill} \index{Past Papers!Chemistry! 2009}
%2009 requires students to answer two (2) of the following questions, including Question 1.

\begin{enumerate}

\item[1.] You are provided with the following solutions:\\
\vspace{2pt}
Solution WW containing 4.38 g of pure hydrochloric acid per dm$^3$ of solution.\\
\vspace{2pt}
Solution ZZ containing 14.30 g of hydrated sodium carbonate [Na$_2$CO$_3$ .x H$_2$O] per dm$^3$.\\
\vspace{2pt}
Methyl orange indicator.\\

\vspace{10pt}

\textbf{Procedure:}\\
Put solution WW in the burette. Pipette 20 cm$^3$ or (25 cm$^3$) of solution ZZ into a titration flask. Add about three to four drops of methyl orange indicator into the titration flask. Titrate solution WW against solution ZZ until the end point is reached. Note the burette reading. Repeat the procedure to obtain three more readings. Record your results as shown in the following Table.\\

\begin{enumerate}
\item[(a)] Table of results\\

\begin{enumerate}
\item[(i)] Burette readings\\

\begin{center}
\begin{tabular}{|l|p{2cm}|p{2cm}|p{2cm}|p{2cm}|} \hline
\textbf{Titration}&\multicolumn{1}{|c|}{\textbf{Pilot}}&\multicolumn{1}{|c|}{\textbf{1}}&\multicolumn{1}{|c|}{\textbf{2}}&\multicolumn{1}{|c|}{\textbf{3}}\\ \hline
Final reading (cm$^3$)&&&&\\ \hline
Initial reading (cm$^3$)&&&&\\ \hline
Volume used (cm$^3$)&&&&\\ \hline
\end{tabular}\\
\end{center}
\vspace{4pt}
\item[(ii)] The volume of pipette used was \_\_\_\_ cm$^3$.
\vspace{2pt}
\item[(iii)] The volume solution WW needed for complete neutralization was \_\_\_\_.
\vspace{2pt}
\item[(iv)] The colour change at the end point was from \_\_\_\_ to \_\_\_\_.\\
\end{enumerate}

\vspace{6pt}

\item[(b)] Write a balanced chemical equation for the reaction between solution ZZ and WW.\\
\vspace{4pt}
\item[(c)] Calculate the molarity of\\
\vspace{2pt}
\begin{enumerate}
\item[(i)] solution WW
\vspace{2pt}
\item[(ii)] solution ZZ.
\end{enumerate}
\vspace{4pt}
\item[(d)] Calculate the value of x in the formula (Na$_2$CO$_3$ .x H$_2$O).
\end{enumerate}

\raggedleft \textbf{(25 marks)}\newpage

\raggedright

\item[2.] Sample M is a simple salt containing \textbf{one} cation and \textbf{one} anion. Carry out carefully the experiments described in the following table. Record all your observations and appropriate inferences to identify the ions present in M.\\

\begin{center}
\begin{tabular}{|l|p{8cm}|l|l|}
\hline
\textbf{S/N}&\textbf{Experiment}&\textbf{Observation}&\textbf{Inference}\\ \hline
(a)&Appearance of sample M&&\\ \hline
(b)&Place a spatulaful of sample M in a test-tube and heat while rotating the tube. Test for any gas(es) evolved.&&\\
(c)&Place a spatulaful of sample M in a test tube and add dilute hydrochloric acid. Test for any gas(es) evolved. Add more of the acid until the test tube is half full. Divide the solution into three portions and then do the following;
\begin{enumerate}
\item[(i)] add sodium hydroxide solution dropwise and then in excess to the first portion.
\end{enumerate}&&\\ \cline{2-4}
&\begin{enumerate}
\item[(ii)] add a few drops of potassium iodide solution to the second portion.
\end{enumerate}&&\\ \cline{2-4}
&\begin{enumerate}
\item[(iii)] add ammonium hydroxide solution dropwise till excess to the third portion.
\end{enumerate}&&\\ \hline
\end{tabular}\\

\end{center}


\textbf{Conclusion}\\
The cation in sample M is \_\_\_\_ and the anion is \_\_\_\_.\\

\raggedleft \textbf{(25 marks)}

\raggedright

\item[3.] Substance V is a simple salt containing \textbf{one} cation and \textbf{one} anion. Using systematic qualitative analysis procedures carry out tests on sample V and make appropriate observations and inferences to identify the cation and anion in V. Record your experiments, observations and inferences as shown in the following table:\\

\begin{center}
\begin{tabular}{|p{4cm}|p{4cm}|p{4cm}|}
\hline
\textbf{Experiment}&\textbf{Observation}&\textbf{Inference}\\ \hline
&&\\
&&\\
&&\\
&&\\
\hline
\end{tabular}\\
\end{center}

\textbf{Conclusion}\\

The cation in sample V is \_\_\_\_ and the anion is \_\_\_\_.\\

\end{enumerate}

\raggedleft \textbf{(25 marks)}\\

\raggedright
