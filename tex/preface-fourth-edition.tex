\clearpage
\phantomsection
\addcontentsline{toc}{chapter}{Preface to the Fourth Edition}
\chapter*{Preface to the Fourth Edition}

\emph{Shika na Mikono} strives to provide all of the necessary tools and resources for educators to enable students to reach their full potential in math and science. Among these are the \emph{Shika na Mikono} teacher resource manuals.

Since its original release in 2009, this manual has served as a teaching resource and guide to incorporating interactive learning methods into Tanzanian laboratories and classrooms using locally available materials. Now in its fourth edition, the manual has broadened its scope in terms of its content and application to the Tanzanian secondary school curricula. The resources within the manual have been divided to better isolate and address the different challenges faced in laboratory and classroom settings. Despite the ability of both of these settings to foster an interactive science education, the two are distinguished only in their adherence to the educational requirements enforced by the Ministry of Education, such as NECTA practicals. Information regarding laboratory practicals (including a newly added compilation of NECTA past papers), as well as general guidance for starting, developing and maintaining school laboratories, is the focus of this manual. Version 4 now also contains a guide to incorporating and hosting various math- and science-related events for students and teachers, including science competitions, conferences and trainings. Hands-on activities for the classroom, as well as extensions to outside learning, science clubs, projects and community involvement are now located in the separate subject-specific \emph{Shika Express} companion manuals for Biology, Chemistry, Physics and Mathematics.

Continued development of the \emph{Shika na Mikono} resources is made possible by a dedicated team of individuals made up of Peace Corps Volunteers and Tanzanian teachers and facilitators.


