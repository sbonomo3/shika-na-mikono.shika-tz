\chapter{Hosting Science Events} \index{Science! events}

There are many great ways to promote math and science education through engaging activities for students and teachers alike. These can be done regularly through extracurricular clubs, but can also be organized together as part of a larger \emph{Science Day} event or multi-day \emph{Math and Science Conference}. What follow are some general tips and suggestions for hosting some of these various activities.

\section{Box of Fun} \index{Box of fun}
The \emph{Box of Fun} can be used as a teacher training exercise or as a student challenge. 
\begin{itemize}
\item Gather an assortment of everyday materials (see \nameref{cha:labequip}, p.~\pageref{cha:labequip}) and arrange them randomly across a table or in a large box. 
\item Ask participants to use the materials given to demonstrate some topic or principle in a subject of their choice (Biology, Chemistry, Physics or Math).
\item You may choose to put participants into groups and designate a specific subject for each group.
\item After at least 30 minutes, have groups come up to present their idea.
\item Additionally, you may ask groups to fill out an activity template (see \emph{Shika Express} companion manuals) to document their ideas.
\end{itemize}

The \emph{Box of Fun} is intended to foster participants' creativity and encourage them to see science in the world around them. Rather than thinking first of a topic and then deciding what materials are needed to show it, this activity encourages teachers to \emph{first look around and see what is available to them}, and then to think about how those things might be used to demonstrate some concept in science. \\

\noindent \textbf{CAUTION:} It may not be wise to assign specific topics to participants, as this can limit their creativity and may lead them to \emph{``destroy science''} by pretending a local substitute gives the same result as a traditional lab material when it really doesn't (e.g. pretending food colour is iodine because they have similar appearances). 

The purpose of using locally available materials is that they help to connect students to their everyday environment \emph{while still achieving the same results as expensive lab equipment}. However, if a local material does not give the same result, it should NOT be substituted merely for the sake of using local materials.

\section{Shika Express Gallery Walk} \index{Gallery walk}
This activity can be used to share ideas of science demonstrations among students and/or teachers.

\begin{itemize}
\item Choose 4-6 activities or demonstrations for each subject (see \emph{Shika Express} companion manuals for Biology, Chemistry, Physics and Math).
\item Prepare the demonstrations and arrange them across a set of tables, 1-2 tables per subject.
\item Spread the tables out evenly around a large empty room (e.g. dining hall).
\item Divide participants into equal groups based on the number of subjects being presented.
\item Have groups rotate among the different subject tables so that they are able to observe all demonstrations for each subject (approx. 15-20 mins per subject).
\item Following the rotations, give 20-30 mins to allow participants to return to a demonstration of their choice for further investigation or to construct it themselves.
\end{itemize}

\pagebreak

\textbf{Suggestions:}
\begin{itemize}
\item You may wish to have 1-2 student or teacher leaders for each subject to help explain the demonstrations during the rotations. 
\item Make copies of activity write-ups from the \emph{Shika Express} manuals for each demonstration that participants can read as they walk around. 
\item Allow participants to perform the demonstrations themselves as much as possible, and then ask them to explain what they see.
\end{itemize}

%\section{Scientific Method Stations}
%Set up a series of 4-6 stations with activities focused on the use of the scientific method. Divide students into groups and have them rotate among the various activities.
%
%\textbf{Suggestions:}
%\begin{itemize*}
%\item[] Biology
%	\begin{itemize*}
%	\item Lung Capacity
%	\end{itemize*}
%\end{itemize*}

\section{Science Fair} \index{Science! fair}
\emph{Science Fair} projects provide a great opportunity for students to apply their knowledge and investigate their interests in science.

\begin{itemize}
\item Have interested students form groups of 2-3.
\item Groups select a project idea based on a shared interest or question/problem to address. Encourage students to think about what problems or issues are faced in their own communities. 
%(For ideas, see suggestions below.)
%\nameref{cha:sci-fair-projects})
\item Review the steps of \nameref{cha:scientific-procedure} (see example activities on p.~\pageref{cha:scientific-procedure}). Have students identify the problem and form a hypothesis for their project.
\item Allow several weeks for groups to work on their projects (provide additional books or computer resources if available).
\item When completed, allow students to set up and explain the various projects around the school for all students to see.
\item Encourage students to apply to participate in the national \href{www.youngscientists.co.tz}{Young Scientists Tanzania (YST)} competition (\url{www.youngscientists.co.tz}) in Dar es Salaam.
\end{itemize}

%\subsection{Project Ideas}
%See the \emph{Shika Express} companion manuals for more information on these and many other ideas.
%\begin{itemize*}
%\item Biology:
%	\begin{itemize*}
%	\item Digestive system model (using local materials)
%	\item Human skeleton model
%	\item Plant and animal cell models / displays
%	\item Classification of specimens
%	\end{itemize*}
%\item Chemistry:
%	\begin{itemize*}
%	\item Production of carbon dioxide gas (vinegar and baking soda)
%	\item Effects of chemicals on soil pH
%	\item Requirements for rusting
%	\end{itemize*}
%\item Physics:
%	\begin{itemize*}
%	\item Sustainable energy sources (windmill, water wheel, etc.)
%	\item Simple electric circuit
%	\item Simple machines
%	\end{itemize*}	
%\end{itemize*}

%\subsection{Display Ideas}
%\begin{center}
%\includegraphics[width=10cm]{./img/}
%\end{center}

\section{Science Competitions} \index{Science! competitions}
Perform individual competitions or many strung together over the course of a day or weekend. 

For more, see the section on \nameref{cha:sci-comp} (p.~\pageref{cha:sci-comp}).

\section{Science Day} 
Engage the entire school (or multiple schools) by combining several activities into a \emph{Science Day} event.

\begin{itemize}
\item Invite a guest speaker to speak on career opportunities in math and science (e.g. accountant, engineer, doctor, nurse, carpenter, mechanic, store owner, etc.)
\item Explain applications of math and science in all walks of life (e.g. farming, buying/selling, health/disease, transport, weather, drinking water, football, etc.)
\item Incorporate \emph{\nameref{cha:sci-comp}} - elect 1 or 2 teams from each Form to compete, with the rest of the school as an audience.
\item Incorporate \emph{Box of Fun} and \emph{Gallery Walk} activities. It may be helpful to do the \emph{Gallery Walk} first to provide examples to participants.
\item Encourage girls' empowerment wherever possible.
\item Give out a survey to gauge students' perception of science.
\end{itemize}

A \emph{Science Day} event may not guarantee immediate improvements in test scores, but it shows students that \emph{the school and its teachers are not willing to give up on math and science, and neither should they!} Continued promotion of math and science will help to change students' perception of the subjects that they may initially write off as being too difficult. Excitement and interest is the first step in changing that perception.

\section{Math and Science Conferences} \index{Science! conferences}
Gather students from several nearby schools to hold a special week-long \emph{Math and Science Conference} in a nearby town or at a host school.

In addition to those ideas presented for a \emph{Science Day} event,
\begin{itemize}
\item Incorporate HIV/AIDS and malaria into science demonstrations/activities (see \emph{Shika Express} companion manuals for ideas).
\item Have students prepare \emph{Science Fair} projects over the course of the conference and present on the final day.
\item Give award certificates for participation and prizes for individual/team competitions.
\end{itemize}

\emph{Math and Science Conferences} encourage leadership among their participants. Students attending such events are likely to be good \emph{ambassadors of science}, sharing what they have done and learned with fellow students back at school. Those students may then try to improve their own performance in those subjects so that they can attend a similar event later on.

\section{Teacher Trainings} \index{Teacher trainings}

Conferences can also be directed towards improving teacher performance by conducting the suggested activities at a nearby Teacher's College. 