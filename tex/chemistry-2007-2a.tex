\section{2007 - CHEMISTRY 2A ALTERNATIVE A PRACTICAL \hfill} \index{Past Papers!Chemistry! 2007}
%2007 requires students to answer two (2) of the following questions, including number 1.

\begin{enumerate}
\item[1.] You are provided with the following:\\
\vspace{6pt}
Solution \textbf{G} containing 0.05 M sulphuric acid.\\
\vspace{2pt}
Solution \textbf{H} containing 2 g of \textbf{X}OH in 500 cm$^3$ of the solution.\\
\vspace{2pt}
Solution \textbf{F}, methyl orange indicator.\\
\vspace{10pt}
Determine the atomic mass of \textbf{X} in \textbf{X}OH.\\
\vspace{10pt}
\textbf{Procedure:}\\
\vspace{6pt}
Put solution \textbf{G} in the burette. Pipette 20 cm$^3$ or (25 cm$^3$) of solution \textbf{H} into the conical flask. Add two or three drops of methyl orange indicator. Titrate solution \textbf{H} against solution \textbf{G} from the burette until a colour change is observed. Note the burette reading. Repeat the procedure to obtain three more readings.
\begin{enumerate}
\item[(a)] Record your results in a table as shown below.\\
\begin{enumerate}
\item[(i)] Burette readings.\\

\begin{center}
\begin{tabular}{|l|p{2cm}|p{2cm}|p{2cm}|p{2cm}|} \hline
\multicolumn{1}{|c|}{\textbf{Titration}}&\textbf{Pilot}&\textbf{1}&\textbf{2}&\textbf{3}\\ \hline
Final reading (cm$^3$)&&&&\\ \hline
Initial reading (cm$^3$)&&&&\\ \hline
Volume used (cm$^3$)&&&&\\ \hline
\end{tabular}\\
\end{center}

\vspace{4pt}
\item[(ii)] The volume of pipette used was \_\_\_\_ cm$^3$.
\vspace{2pt}
\item[(iii)] Calculate the mean titre volume.
\vspace{2pt}
\item[(iv)] The volume of solution \textbf{H} needed for complete neutralization of \_\_\_\_ cm$^3$ of solution \textbf{G} was \_\_\_\_ cm$^3$.\\
\end{enumerate}

\vspace{4pt}
\item[(b)] Write a balanced chemical equation for the reaction between solution \textbf{G} and \textbf{H}.\\
\vspace{4pt}
\item[(c)] Calculate the\\
\vspace{2pt}
\begin{enumerate}
\item[(i)] molarity of \textbf{H}.
\vspace{2pt}
\item[(ii)] concentration of \textbf{H} in g/dm$^3$.
\vspace{2pt}
\item[(iii)] molar mass of \textbf{X}OH.
\vspace{2pt}
\item[(iv)] atomic mass of \textbf{X} in compound \textbf{X}OH.
\end{enumerate}

\end{enumerate}
\raggedleft \textbf{(25 marks)}\newpage

\raggedright

\item[2.] Sample \textbf{B} is a simple salt containing one cation and one anion. Carry out carefully the experiments described below and record all your observations and appropriate inferences. Identify the cation and anion present in sample \textbf{B}.\\
\begin{center}
\begin{tabular}{|p{8cm}|c|c|}
\hline
\multicolumn{1}{|c|}{\textbf{Experiment}}&\textbf{Observation}&\textbf{Inference}\\ \hline
\begin{enumerate}
\item[(a)] Appearance of sample \textbf{B}.
\end{enumerate}
&&\\ \hline
\begin{enumerate}
\item[(b)] To half a spatulaful of sample \textbf{B} in a test tube, add concentrated sulphuric acid and warm.
\end{enumerate}
&&\\ \hline
\begin{enumerate}
\item[(c)] To a spatulaful of sample \textbf{B} in a test tube, add 10 cm$^3$ of distilled water and stir to obtain a stock solution them divide into three portions.
\end{enumerate}
&&\\ \hline
\begin{enumerate}
\item[(d)] To the first portion of the stock solution, add sodium hydroxide till excess.
\end{enumerate}
&&\\ \hline
\begin{enumerate}
\item[(e)] To the second portion of the stock solution, add barium chloride solution.
\end{enumerate}
&&\\ \hline
\begin{enumerate}
\item[(f)] To the third portion of the stock solution, add freshly prepared acidified ferrous sulphate solution followed by concentrated sulphuric acid added slowly along the walls of the test tube.
\end{enumerate}
&&\\ \hline
\begin{enumerate}
\item[(g)] Perform a flame test on sample \textbf{B}.
\end{enumerate}
&&\\ \hline
\end{tabular}\\

\end{center}

Conclusion\\

The cation in sample \textbf{B} is \_\_\_\_ and the anion is \_\_\_\_.\\
\raggedleft \textbf{(25 marks)}

\raggedright

\item[3.] Sample \textbf{N} is a simple salt containing one cation and one anion. Using systematic qualitative analysis procedures, carry out tests on the sample and make appropriate observations and inferences to identify the cation and anion in sample \textbf{N}.\\

\begin{center}
\begin{tabular}{|p{4cm}|p{4cm}|p{4cm}|}
\hline
\textbf{Experiment}&\textbf{Observation}&\textbf{Inference}\\ \hline
&&\\
&&\\
&&\\
&&\\
\hline
\end{tabular}\\
\end{center}

Conclusion\\

The cation present in \textbf{N} is \_\_\_\_ and the anion is \_\_\_\_.\\

\raggedleft \textbf{(25 marks)}\\

\raggedright


\end{enumerate}