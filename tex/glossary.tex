\chapter{Kiswahili Laboratory Glossary}

While managing a lab filled with dangerous chemicals, breakable glassware, open flames, and inexperienced students, it is important to know how to communicate with students easily to keep the place safe and running smoothly. Below is a list of Kiswahili words and phrases that may be helpful in a lab setting.

\begin{center}
\begin{longtable}{|p{0.5\textwidth}|p{0.5\textwidth}|}\hline

\multicolumn{1}{|c|}{\textbf{English}}	&	\multicolumn{1}{c|}{\textbf{Kiswahili}}	\\	\hline

to absorb	&	-fyonza	\\	\hline
to be absorbed	&	-fyonzwa	\\	\hline
accident	&	ajali	\\	\hline
to affect, to influence	&	-athiri	\\	\hline
ashes	&	majivu	\\	\hline
battery acid	&	maji makali	\\	\hline
to boil 	&	-chemsha 	\\	
        Ex. I boil the water.	&	        Ninachemsha maji.	\\	\hline
To be boiling 	&	-chemka 	\\	
        Ex. The water is boiling.	&	        Maji yanachemka.	\\	\hline
to break 	&	-vunja 	\\	
        Ex. I broke the glass.	&	        Nimevunja kioo.	\\	\hline
to be broken 	&	-vunjika 	\\	
        Ex. The glass broke.	&	        Kioo kimevunjika.	\\	\hline
broken, bad, rotten	&	bovu	\\	\hline
to burn	&	-unguza 	\\	
        Ex. I am burning paper.  	&	        Ninaunguza karatasi.	\\	\hline
(to be) burnt	&	-ungua 	\\	
        Ex. The paper is burnt.	&	        Karatasi imeungua.	\\	\hline
calcium hydroxide solution (lime water)	&	maji chokaa	\\	\hline
carefully	&	taratibu	\\	\hline
to cause	&	-sababisha	\\	\hline
to be caused by	&	-sababishwa	\\	\hline
caution 	&	utaratibu 	\\	
        Ex. Heat with caution.	&	        Ex. Pasha na utaratibu.	\\	\hline
to change 	&	-badilisha 	\\	
        Ex. I am changing the color of this.	&	        Ninabadilisha rangi ya hii. 	\\	\hline
to be changed 	&	-badilika 	\\	
        Ex. The color has been changed. 	&	        Rangi imebadilika.	\\	\hline
changes	&	mabadiliko	\\	\hline
chemical	&	kemikali	\\	\hline
container, glassware	&	chombo	\\	\hline
to clean	&	-safisha	\\	\hline
to collide	&	-gongana	\\	\hline
color	&	rangi	\\	\hline
to cool	&	-poa	\\	\hline
danger	&	hatari	\\	\hline
to decrease (transitive)	&	-punguza	\\	
        Ex. Decrease the heat.	&	        Punguza moto.  	\\	\hline
to decrease in size, to grow smaller (intransitive)	&	-pungua	\\	
        Ex: The heat has decreased.	&	        Moto umepungua.	\\	\hline
to destroy, to damage, to contaminate	&	-haribu	\\	
        Ex. I contaminated/damaged the chemical.	&	        Nimeharibu kemikali.	\\	\hline
to be destroyed, damaged, contaminated, or expired.	&	-haribika	\\	
        Ex. The chemical has gone bad or expired.	&	        Kemikali imeharibika.	\\	\hline
distilled water (from a hardware or car repair shop) 	&	maji baridi	\\	\hline
to distribute	&	-gawa	\\	\hline
to draw	&	-chora	\\	\hline
to drink	&	-nywa	\\	
        Ex. Do not drink!	&	        Usinywe!	\\	\hline
drop (as in “a drop of water”)	&	tone 	\\	\hline
to dry	&	- kausha	\\	\hline
effect	&	athari	\\	\hline
to estimate	&	- kadiri	\\	\hline
to evaporate (causative)	&	-vukiza 	\\	
        Ex. I am evaporating water. 	&	        Ninavukiza maji.	\\	\hline
to explode	&	- lipuka	\\	\hline
to fill	&	-jaza	\\	
        Ex. I filled the container.	&	        Nimejaza chombo.	\\	\hline
to be full	&	-jaa	\\	
        Ex. The container is full.	&	        Chombo kimejaa. 	\\	\hline
to filter	&	-chuja	\\	\hline
fire	&	moto	\\	\hline
gas, air	&	hewa	\\	\hline
glass	&	kioo	\\	\hline
group	&	kikundi (vi-)	\\	\hline
to grow	&	kukua	\\	\hline
harm (harmful)	&	madhara (yenye madhara)	\\	\hline
to haul off and slap someone	&	-ezeka makofi	\\	\hline
to heat	&	-pasha	\\	\hline
height, length	&	urefu	\\	\hline
to hit, to knock	&	-gonga	\\	\hline
to increase in size or number, to swell (intransitive)	&	-ongezeka	\\	
        Ex. The amount of food present increased.	&	        Chakula kimeongezeka.	\\	\hline
to increase, to add (transitive)	&	-ongeza	\\	
        Ex. I added more food.	&	        Nimeongeza chakula.	\\	\hline
instrument/apparatus/	&	kifaa (vi-)	\\	\hline
tool	&		\\	\hline
to kick someone out     	&	-fukuza 	\\	
        Ex. If you break a rule, you will be kicked out     of the laboratory.	&	Kama unakiuka sheria, utafukuzwa maabara.	\\	\hline
lab activity/experiment	&	practical	\\	\hline
living	&	hai	\\	\hline
match / lighter	&	kiberiti (vi-) / kiberiti cha gasi	\\	\hline
to measure, to test	&	-pima	\\	\hline
to be melted, to be dissolved (Not exactly “to be melted.” It's just the intransitive form of to melt or dissolve.)	&	-yeyuka 	\\	
        Ex.The salt melted/dissolved.	&	        Chumvi imeyeyuka.	\\	\hline
mass	&	uzito	\\	\hline
methylated spirits	&	spiriti	\\	\hline
microscope	&	hadubini	\\	\hline
to mix	&	-changanya	\\	\hline
to be mixed	&	-changanyika	\\	\hline
mixture	&	mchanganyiko	\\	\hline
particle, atom, small piece	&	chembe ndogo ndogo	\\	\hline
permission	&	ruhusa	\\	\hline
poison	&	sumu	\\	\hline
to pour	&	-mimina	\\	\hline
to pour out	&	-mwaga	\\	\hline
to press on, to push on (used to described squeezing of eye droppers)	&	-bonyeza	\\	\hline
to put, to keep	&	-weka	\\	\hline
rubber, rubber tubing	&	mpira	\\	\hline
to rush, to hurry	&	-harakisha 	\\	
        Ex. Do not rush!	&	        usiharakishe	\\	\hline
scientist	&	mwanasayansi	\\	\hline
serious, attentive	&	makini	\\	\hline
to shake	&	-tikisa	\\	\hline
to share out, to divide	&	-gawana	\\	\hline
slowly	&	pole pole	\\	\hline
to solidify, to freeze	&	-ganda	\\	\hline
steam	&	mvuke	\\	\hline
to stir	&	-koroga	\\	\hline
to stop	&	-acha	\\	\hline
stove	&	jiko (ma-)	\\	\hline
strong, harsh, fierce, dangerous, concentrated (for an acid)	&	kali	\\	\hline
to suck, to pull	&	-vuta	\\	\hline
to throw away, to chuck	&	-tupa	\\	\hline
to touch 	&	-gusa	\\	
        Ex. Do not touch!	&	        Usiguse!	\\	\hline
to turn off	&	-zima	\\	\hline
to turn on, to light a flame	&	-washa 	\\	\hline
volume	&	ujazo	\\	\hline
to wash dishes or glassware	&	-osha	\\	\hline
to wash your hands	&	-nawa	\\	\hline
to watch	&	-angalia	\\	\hline
yeast	&	hamira	\\	\hline




\end{longtable}
\end{center}