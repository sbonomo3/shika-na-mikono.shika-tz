\chapter{Classroom Management in the Laboratory} \index{Laboratory! classroom management}
\label{cha:classmanagement}

In addition to the guidelines recommended 
in the \nameref{cha:labsafety} section, 
we recommend the following strategies to keep lab work safe, 
productive, 
and efficient.

\section{Set lab rules}
Before the first practical of the year, 
hold a short session to teach lab rules and lab first aid. 
Try to set a few clear, 
basic rules – 
like the four proposed in the Laboratory Safety section – 
instead of a long list of rules. 
Post these rules in the lab, 
and be consistent and strict 
in enforcing them with students and teachers.

\section{Train students in basic techniques}
For students just beginning laboratory-based education, 
you can probably teach each specific skill 
one at a time as they come up in experiments. 
For more advanced students, 
especially when they have different backgrounds 
in terms of laboratory experience, 
it is wise to spend several sessions practicing basic techniques (e.g. 
titrations for chemistry, 
using the galvanometer for physics, 
etc).

\section{Have students copy the lab instructions before entering the lab}
Do not let them into the lab unless 
they can show you their copy of the procedure, 
etc. 
Have a class dedicated to explaining 
the practical activity before the actual session. 
Bring a demo apparatus into the classroom.

\section{Demonstrate procedures at the beginning}
Do not assume that students know how to use a syringe 
or measure an object with calipers. 
If there are many new procedures, 
hold a special session before the practical to teach them the procedures. 
For titration, for example, 
hold a practice session in using burettes 
and syringes with water and food coloring. 
For food tests, 
explain and demonstrate each step to the students 
before holding a practical. 
It will save you a lot of trouble during the actual practical.

\section{Have enough materials available}
Always prepare 25-50 percent more reagent 
than you think you will need. 
Also have spare apparatus in case they fail in use. 
For example with physics, 
have extra springs, 
resistors, 
weights, 
etc. 
That said; do not make all of what you prepare 
immediately available to the students. 
As with sugar and salt, 
an obvious surplus increases consumption. 
If there is a definite scarcity of resources, 
it may be necessary to distribute the exact volumes necessary 
to each student. 
If you are doing this, 
make sure students understand that there is no more. 
In an exam, 
you might take unique objects, 
such as ID cards, 
to ensure each student receives her/his allotment only once.

\section{Have enough bottles of reagent available}
Even if only a small quantity of a reagent is needed, 
divide it into several bottles and put a bottle on each bench. 
If the volume is sufficiently small, 
distribute the chemical in plastic syringes. 
Do not use syringes for concentrated acids or bases – 
because these chemicals can degrade the rubber in the syringe, 
there is a risk of the syringe jamming 
and the student squirting chemicals into eyes. 
The waiting caused by shared bottles leads 
to frustration and quarrels between groups. 
The last thing you want are students wandering around the lab 
and crowding to get chemicals. 

\section{Designate fetchers}
If students must share a single material source, 
designate students to fetch materials
If a reagent needs to be shared among many students, 
explain this at the beginning, 
and have them come to the front of the room to get it 
rather than carrying it to their benches. 
This will help to avoid arguments 
and confusion over where the reagent is. 
If the students are in groups, 
have each group appoint one student 
to be in charge of fetching that chemical. 
However, 
it is much better to have the reagent available 
for each group at their workplace.

\section{Teach students to clean up before they leave}
This will save you a lot of time in preparing 
and cleaning the lab—and it is just a good habit. 
Do not let students leave the lab until their glassware is 
clean and the bench is free of mystery salts and scraps of paper. 
If they do, 
consider not letting them in for the next practical. 
This might take assigned seats if you have many students. 
When they perform this clean up, 
make sure they follow whatever guidelines 
you have set for proper waste disposal.

\section{Allow more time than you think you will need}
What seems like a half hour experiment to you 
may take an hour for your students. 
Add fifteen minutes to a half hour more 
than you think will be necessary. 
If you finish early, 
you can have them clean up and then do a bonus demonstration.

\section{Know the laboratory policies at the school}
What is the policy on replacing broken equipment at the school? 
As a teacher, 
you need to know what you are going to do 
when the student drops an expensive piece of glassware. 
It is no fun to make up procedure while a student is in tears. 
What criteria will you use to determine if the student is “at fault?” 
Of course, 
this is less of an issue if you do not use glass apparatus.

