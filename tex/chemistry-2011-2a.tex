\section{2011 - CHEMISTRY 2A ACTUAL PRACTICAL A} \index{Past Papers!Chemistry! 2011}

\begin{enumerate}
\item[1.] You are provided with the following:\\
\textbf{AA}  A solution of 0.2 M nitric acid (HNO$_3$);\\
\textbf{BB}  A solution of 4.2 g NaxCO$_3$ per 0.5 dm$^3$ of solution;\\
\textbf{MO} is methyl orange indicator.\\

\textbf{Procedure}\\
\vspace{-10pt}

Put solution \textbf{AA} into the burette. Pipette 20 cm$^3$ or 25 cm$^3$ of solution \textbf{BB} in a titration flask. Add two drops of methyl orange indicator into the titration flask. Titrate solution \textbf{BB} against \textbf{AA} until the end point is reached. Record the burette reading. Repeatthe procedure to obtain three more readings and record your results in a tabular form.\\[10pt]
\vspace{-10pt}
\textbf{Questions}:\\
\vspace{-6pt}
\begin{enumerate}
\item[(a)] 
\begin{enumerate}
\item[(i)] Calculate the average titre volume.
\item[(ii)] Summary:  \_\_\_\_ cm$^3$ of solution \textbf{BB} required \_\_\_\_ cm$^3$ of solution \textbf{AA} for complete reaction.
\end{enumerate}
\vspace{-2pt}
\item[(b)] If the mole ratio for the reaction is 1:1 find:\\
\vspace{-10pt}
\begin{enumerate}
\item[(i)] Concentration of NaxCO$_3$ in mol/dm$^3$ and g/dm$^3$.
\item[(ii)] Molecular mass of NaxCO$_3$.
\item[(iii)] Atomic mass of x and replace it in the formula NaxCO$_3$.
\end{enumerate}

\item[(c)] Write a balanced chemical equation for the reaction in this experiment.
\item[(d)] What is the significance of the indicator in this experiment?
\item[(e)] Why is there a colour change when enough acid has been added to the base?

\end{enumerate}
\raggedleft \textbf{(20 marks)}

\raggedright

\item[2.] Your are provided with the following materials:\\
\textbf{TT}:  A solution of 0.13 M Na$_2$S$_2$O$_3$ (sodium thiosulphate);
\textbf{HH}:  A solution of 2 M HCl;
Distilled water;
Stopwatch.\\
\vspace{6pt}
\textbf{Procedure:}\\
\vspace{-4pt}
\begin{enumerate}
\item[(i)] Using 10 cm$^3$ measuring cylinder, measure 20 cm$^3$ of solution \textbf{TT} and put into 100 cm$^3$ beaker.
\item[(ii)] Use different measuring cylinder to measure 10 cm$^3$ of \textbf{HH} and pour it into the beaker containing solution \textbf{TT}, immediately start the stop watch. Swirl the beaker twice.
\item[(iii)] Place the beaker with the contents on top of a piece of paper marked \textbf{X}.
\item[(iv)] Look down vertically through the mouth of the beaker so as to see the cross at the bottom of the beaker. Stop the clock when the letter \textbf{X} is invisible.
\item[(v)] Record the time taken for the letter \textbf{X} to disappear completely.
\item[(vi)] Repeat the experiment as shown in Table 1.
\item[(vii)] Record your results in tabular form as shown in Table 1.
\end{enumerate}

\indent Table 1: Table of results\\

\begin{center}
\begin{tabular}{|p{2cm}|p{2cm}|p{2cm}|p{2cm}|p{2.5cm}|p{2cm}|}
\hline
Exp. No.&Vol. of \textbf{HH} (cm$^3$)&Vol. of \textbf{TT} (cm$^3$)&Vol. of Distilled water (cm$^3$)&Time (sec)&$\frac{1}{t}$ (s$^-1$)\\ \hline
1&10&20&0&&\\ \hline
2&10&15&5&&\\ \hline
3&10&10&10&&\\ \hline
4&10&5&15&&\\ \hline
\end{tabular}
\end{center}

\newpage

\textbf{Questions:}\\
\begin{enumerate}
\item[(a)] Complete filling the table of results (Table 1).
\item[(b)] Write a balanced equation for reaction between \textbf{TT} and \textbf{HH}.
\item[(c)] What is the reaction product which causes the solution to cloud the letter \textbf{X}?
\item[(d)] How was the factor of concentration varied in this experiment?
\item[(e)] Plot a graph of 1/t against the volume of the thiosulphate.
\item[(f)] Use the graph to explain how variation of concentration affects the rate of chemical reaction.\\
\end{enumerate}

\raggedleft \textbf{(15 marks)}

\raggedright


\item[3.] Sample \textbf{S} is a simple salt containing one cation and one anion. Carry our the experiments described below. Record your observations and inferences as shown in Table 2.\\

Table 2: Experimental results\\

\begin{center}
\begin{tabular}{|l|p{8cm}|l|l|}
\hline
\textbf{S/n}&\textbf{Experiment}&\textbf{Observation}&\textbf{Inference}\\ \hline
(a)&Observe the appearance of sample \textbf{S}.&&\\ \hline
(b)&Place a spoonful of sample \textbf{S} in a test tube, add water and shake to dissolve.&&\\ \hline
(c)&Put a spatulaful of sample \textbf{S} in a test tube and heat.&&\\ \hline
(d)&Add three drops of sodium hydroxide solution to the solid sample in a test tube.&&\\ \hline
(e)&Put a spatulaful of sample \textbf{S} in a dry test tube and add concentrated sulphuric acid. Warm the mixture and test for any gas evolved.&&\\ \hline
(f)&Put a spatulaful of sample \textbf{S} in a dry test tube and add concentrated sulphuric acid and manganese dioxide. Warm the mixture and test for any gas evolved.&&\\ \hline
(g)&To a portion of the solution from (f) add aqueous silver nitrate followed by aqueous ammonia.&&\\ \hline
\end{tabular}\\
\end{center}

\textbf{Conclusion:}\\
\begin{enumerate}
\item[(a)] The cation present in \textbf{S} is \_\_\_\_ and the anion is \_\_\_\_.\\
\item[(b)] The name of sample \textbf{S} is \_\_\_\_.\\
\item[(c)] Write a balanced chemical equation for the reactions taking place in experiments (c) and (d).\\
\end{enumerate}


\raggedleft \textbf{(15 marks)} \pagebreak

\raggedright

\end{enumerate}