\chapter{Sources of Laboratory Equipment}
\label{cha:labequip}
Below are common apparatus you might order from a laboratory supply company, 
and comments about which are really necessary 
and which have good if not superior alternatives 
available in villages and towns. 
Given equal quality, 
it is generally better to use local materials, 
because these help connect classroom learning to students' lives.

\section{Alligator clips}
Are generally available. 
You can also glue aluminum foil to clothespins.

\section{Balance}
These are expensive. 
Look many places, 
especially port cities and capitals, 
to get a better price. 
A digital balance might be less expensive 
and is probably more accurate. 
If you know anyone going to the USA, 
digital balances for jewellers (and drug dealers) are cheap -- % <-- i'd toss the ref to drug dealers 
between \$20-30 on eBay. 
Search for 0.01~g precision.

\section{Beakers}
Beakers have many uses, 
so it is good to know which use you are trying to replace.

A jam jar, 
disposable plastic cup, 
or a cut off water bottle works well for holding solutions 
that you will transfer out via pipette or syringe.

Having the ``beak'' is nice when filling burettes or measuring cylinders, 
but the little plastic funnels that come with kerosene stoves work well too. 
You could also varnish a small metal funnel from the market. 
You can also fill measuring cylinders or burettes crudely 
from a jar or any other bottle 
and then use a syringe to add the final few milliliters.

A big borosilicate (e.g. 
Pyrex brand) beaker is useful for water baths, 
but an aluminum pot is superior if you have many things to heat. 
For warming a test tube or two only, 
consider using the bottom of a small metal can. 
You can cut the bottom from a beverage can 
by repeatedly scoring it with a razor blade, 
or scissors, 
and then use it to hold a water bat. 
If you use a cut can, 
fold down the cut edge to prevent cut fingers.

If you do purchase beakers, 
buy plastic ones. 
Plastic lab beakers withstand concentrated acids 
and most other forms of chemical attack 
and they do not break when dropped. 
The only exception to buying plastic is 
if you need beakers for heating on an open flame. 
Then they must be borosilicate (Pyrex) glass, 
but again, 
an aluminum pot will heat the water faster, 
be easier to handle, 
and again does not break if dropped.

\section{Bunsen Burner}
See \nameref{sec:heatsources}.

\section{Burettes}
Ideally, 
your school has enough of these for every student 
to use one if they are required for national exams, 
as well as extra for those that malfunction. 
For example, 
if you plan two sessions of forty students each, 
you want at least 45 burettes. 
First, 
note that broken burettes can often be repaired -- 
see \nameref{cha:burettes}. 
Second, 
if you buy burettes, 
buy plastic ones so they do not need to be repaired. 
They ARE available, 
especially if you order them in advance.

If burettes are not available, 
use 10~mL disposable plastic syringes with 0.2~mL gradations (e.g. 
NeoJect brand). 
Students can estimate between the lines to at least 0.05~mL. 
This is sufficiently precise. 
If more than 10~mL are required, 
the student can simply refill the syringe.

\section{Condenser}
Pass clear plastic tubing through a water bottle filled 
with cold water and prevent leaks with super glue. 
If your condenser is under-performing (i.e. 
steam comes out), 
coil the plastic tubing so a greater length is in the water. 
If your condenser is still under-performing 
or you plan to use it for longer period of time, 
devise a way to keep changing water inside the water bottle 
to keep it from getting too hot. 
Or, 
submerge this condenser in a trough of water. 
You could even run the plastic tubing through the sides of a bucket.

\section{Deflagrating Spoon}
\label{sec:deflagratingspoon}
For heating chemicals to observe melting, 
decomposition, 
or other changes on heating, 
metal spoons work well. 
They can usually be cleaned by scrubbing with steel wool, 
although for the national exam you might buy new spoons. 
Bigger spoons may be less expensive than the smallest ones.

\section{Delivery Tube}
You can buy clear plastic tubing at many hardware shops. 
Even better are intravenous `giving sets' available at pharmacies. 
If you are trying to prepare a gas that would corrode this tubing, 
think about what it would do to your students' lungs 
and consider a different experiment.

\section{Droppers}
There is no need to buy these. 
A 2~mL syringe works better and costs very little. 
They are available at almost any pharmacy.

\section{Electrodes}
See \nameref{sec:carbongraphite}, 
\nameref{sec:copper}, 
\nameref{sec:iron}, 
and \nameref{sec:zinc} in \nameref{cha:sourcesofchemicals}.

\section{Electrolytic cell}
Remove the plungers from two 10~mL syringes 
and bore out the needle port with something sharp (knife, 
thin pliers, 
etc.). 
Remove material gradually, 
rotating the piece to ensure a circular cut. 
When the hole is just big enough, 
force through a graphite battery electrode 
so 0.5-1~cm remains on the outside. 
Twist the electrode during insertion to prevent it from snapping. 
The gap should be air-tight, 
but if the cut was too big or irregular 
you can seal the holes with super glue. 
Attach wires to the exposed part of the electrodes.

To use the cell, 
fill the tubes with your electrolyte solution 
and place them wire end up in the cut off bottom of a large water bottle, 
also filled with electrolyte solution. 
Attach the wires to a power supply 
(three or more 1.5~V dry cell batteries in series, 
a 6~V motorcycle battery, 
or a 12~V car battery) to start electrolysis. 
The volume of gas produced at each electrode 
may be measured by the gradations on the syringes, 
and other products (copper metal plating, 
iodine in solution) may be clearly observed.

\section{Flasks}
Flasks can generally be replaced with clean, 
used glass liquor bottles, 
available in most markets. 
You can also make arrangements with local bars 
to reserve empty cans and bottles for your school. 
When using these flasks for titrations, 
students must practice swirling enough that the solution remains well mixed. 
Small water bottles may also be used.

Sometimes flasks are needed for dissolving salts 
to make solutions as their shape is particularly well suited 
for thorough mixing. 
But a half full plastic water bottle with a good cap 
can be shaken much more vigorously 
and will work as well if not better for most solutions. 
Plus, 
the solution is then already in a storage container.

If you need to prepare a solution that requires heating, 
be creative. 
Starch solution, 
for example, 
can be prepared in an aluminum pot without trouble. 
Liquor bottles also have caps for shaking and heat well 
(with the cap off!) in a water bath, 
especially if heated slowly.

\section{Funnel}
Plastic funnels are available at the market. 
Metal funnels are usually less expensive 
but need to be varnished for use with more reactive chemicals like acids. 
Glass funnels are entirely unnecessary. 
If you order funnels from a lab supply company 
you should buy plastic -- 
plastic funnels both from the supply company and the markets 
are suitable for concentrated acids.

\section{Glass blocks}
Glass blocks from a lab supply company are 
generally 15~mm thick rectangular pieces of glass with beveled edges, 
so students do not cut themselves. 
They can be expensive, 
especially if you need many. 
Fortunately, 
it is possible to buy your own glass 
and find a craftsman to make blocks for you, 
especially if you insist on the importance of clean, 
parallel cuts. 
 
8~mm glass is relatively common in towns 
and 10~mm glass can be found in industrial areas of the most major cities. 
12~mm and thicker glass exists though is even more difficult to find. 
However, 
for most optics practicals, 
several pieces of thinner glass can simply be stacked together and 
turned on their edge. 
This is a powerful way of showing refraction, 
and the necessary material (ordinary glass) is cheap and widely available.

\section{Gloves}

\subsection{Latex gloves}
These are worthless to the chemist, 
detrimental in fact because they make the hands less agile 
and give the user a false sense of security. 
Concentrated acids will burn through latex. 
Organic chemicals will pass straight through (and then through your skin) -- 
without any obvious signs. 
One \textit{Shika} author learned this when the skin on his hand started peeling off, 
under the latex glove. 
The only reasons to wear latex gloves is if one has open cuts on the hands 
and has no choice but to perform the practical (e.g. 
national exams), 
or if one needs to perform first aid. 
If for these reasons you want some of these gloves, 
pharmacies sell boxes of one hundred. 
Do not waste money on the individually wrapped sterile gloves.

The biology teacher may want to wear gloves for handling specimens. 
Latex are appropriate for this. 
Human skin is also relatively impervious if it is free of cuts. 
Just wash well with soap and water after handling specimens.

\subsection{Thick gloves}
Thick rubber gloves that withstand exposure more corrosive chemicals 
are sold by village industry supply companies, 
and some hardware stores. 
These gloves will withstand concentrated acids 
for long enough to protect your hands. 
They will, 
however, 
inevitably make you more clumsy, 
and more likely to splash acid or drop the bottle. 
Given that splashed acid and especially a broken bottle 
are much worse than some burned skin on the hand, 
using these gloves for concentrated acids is not recommended. 
Have weak base solution available for treating burns immediately 
and work carefully.

Thick gloves are recommended for when working with organic solvents. 
Remember that the most dangerous organic solvents (benzene, 
carbon tetrachloride) should never be used in a school, 
with or without gloves. 
Also remember that students will probably not have these gloves, 
so do not give them any chemicals 
that you would not use without the gloves. 
If you need to measure out one hundred samples of ether, 
for example, 
wear gloves because this task presents 
a much more significant exposure risk 
than an individual student handling her sample. 
If you are demonstrating technique for the students, 
however, 
do not wear gloves, 
unless you expect all of your students to wear them for the same process.

In general, 
avoid using chemicals that would make you want to wear gloves.

\section{Goggles}
These are essential. 
There are all sorts of ways to make goggles. 
For example, 
students can tie a strip of clean plastic around their heads 
so that it protects their eyes. 
You can make goggles from clear plastic and stiff paper and cardboard. 
Sunglasses work great. 
Be creative! For science labs, 
goggles do not need to be impact resistant -- 
they just need to stand between hazardous chemicals and your eyes. 
If you think there is any risk of students getting chemicals in their eyes, 
they should wear goggles. 
Anyone handling concentrated acid (or battery acid) should wear goggles. 
We only have two eyes, 
and they are very vulnerable to permanent damage. 
Skin heals after an acid burn -- eyes may not.

\section{Heat Sources}
\label{sec:heatsources}
There are many different fuels that may be burned: gas in Bunsen burners, 
butane in gas lighters, 
alcohol in spirit burners, 
kerosene in common cook stoves, 
refined heavy oil (used by caterers) in metal cups, 
wax in candles, 
and charcoal in clay or metal stoves. 
Fire has three uses in the school laboratory -- heating solutions, 
heating solid samples, 
and flame tests -- so it is good to know which you are trying to do.

\subsection{Heating solutions}
The ideal heat source has a high heat rate (Joules transferred per second), 
little smoke, 
and cheap fuel. 
A charcoal stove satisfies all of these 
but takes time to light and requires relatively frequent re-fuelling. 
Kerosene stoves have excellent heat rates but are smoky. 
Alcohol infused refined heavy oil burns smokeless 
and the heat rate scales with the size of the container -- 
filling the bottom half of a cut off aluminum can 
will produce significant heat per unit time. 
In some countries this fuel is advertised for home and commercial use (e.g. 
Motopoa brand in Tanzania) and use is easy to find. 
In other countries, 
it might be much less common.

\subsection{Heating solids}
The ideal heat source has a high temperature and no smoke. 
Ideal would be be a Bunsen burner. 
For heating small objects for a short time (no more than 10-20 seconds), 
a butane lighter provides a very high temperature. 
Refined heavy oil will provide a flame of satisfactory temperature 
for as long as necessary.

\subsection{Flame tests}
The ideal heat source has a high temperature 
and produces a non-luminous flame. 
The Bunsen burner is ideal. 
The next best flame is again refined heavy oil – hot and non-luminous. 
Spirit burners produce a non-luminous flame at much greater cost, 
unless methylated spirits are used as fuel 
in which case the flame is much cooler. 
A butane lighter produces a very hot flame of sufficient size 
and time for flame tests although the non-luminous region is small. 
Kerosene stoves will work for some salts, 
especially if you pull the wicks longer 
or remove the outer protective shell (usually green) 
to give students access to the hotter blue flame in between the inner shells.

As can be seen by the above discussion, 
alcohol infused heavy oil burners provide the best compromise heat source. 
They are also the easiest of these heat sources to use -- 
pour the fluid into an open-topped metal container and set it on fire. 
They are also the safest heat source -- 
they produce no smoke (unlike kerosene and candles), 
do not have fuel that spreads when it spills (kerosene and ethanol), 
nor can explode like gas. 
Students do not have to hold the burning apparatus (as with a lighter) 
and the flame may be extinguished by simply blowing it out 
or smothering it with a lid. 
Finally, 
the burners themselves are free -- soda bottle tops, 
medicine and liquor bottle caps, 
cut aluminum cans, 
and metal tins; different sizes for different size flames. 
If you have access to this fuel, 
we strongly recommend using it.

No matter which heat sources you use, 
always have available fire-fighting equipment that you know how to use. 
See item \ref{list:fire} in \nameref{sub:specguide} for more about fires. 
Remember that to put out a Bunsen burner safely, 
you need to turn off the gas.

\section{Lightbulbs}
Small shops, 
hardware stores, 
LEDs from broken phone chargers or flashlights, 
electronic shops. 
The external phone battery chargers that clamp the battery 
under a plastic jaw have a row of four LEDs with unusually long leads, 
perfect for wiring into circuits.

\section{Meter Rule}
Buy one, 
take it and a permanent pen to a carpenter, 
and leave with twenty. 
Measure each new one to the original rule to prevent compounding errors. 
Ceiling board is a cheap source of flat wood, 
although it is not very stiff.

\section{Microscope}
See \nameref{cha:microscopy}.

\section{Mirrors}
Large sheets of mirror glass is available in most towns, 
and scrap is usually available for sale. 
Glass vendors generally have the tools 
to make small squares of mirror glass 
and you can super glue these to small wooden blocks so they stand upright. 
These will also work for the hand mirrors sometimes required in biology.

\section {Mortar and Pestle}
To powder chemicals, 
place them between two nested metal spoons and grind down. 
Alternatively, 
you can crush chemicals on a sheet of paper on a table 
by pressing on them with the bottom of a glass bottle.

\section{Nichrome Wire}
For flame tests in chemistry, 
you can use a steel wire thoroughly scraped clean with iron or steel wool. 
For physics experiments, 
see \nameref{sec:wire}.

\section{Optical Pins}
Ordinary pins are cheap and perfectly effective. 
To make them easier to see, 
buy some brightly colored nail polish and paint the head. 
If you happen to purchase many plastic syringes, 
the needles make excellent optical pins. 
Pinch a point in the shaft with pliers so no one can take the needle 
and use it for injecting anything.

\section{Pipettes}
Do not buy and do not use. 
Use disposable plastic syringes. 
They come in 1, 
2, 
5, 
10, 
20, 
25, 
30, 
and 50~mL sizes and are available at both pharmacies and veterinary shops. 
These are easier to use, 
much more safe (no danger of mouth pipetting), 
do not break and are much less expensive. 
They are also often more accurate 
as glass pipettes are often incorrectly calibrated. 
To use them, 
suck first 1~mL of air and then put the syringe into the solution to suck up the liquid. 
There should be a flat meniscus under the layer of air.

\section{Resistors}
\label{sec:resistors}
You can buy these from electronic stores in town or from a repairman. 
You can also find old radios or other trash circuit boards 
and take the resistors off them. 
This requires melting the solder, 
easy with a soldering iron, 
but also possible with a stiff wire thrust into a charcoal stove. 
If you need to know the ohms, 
the resistors tell you. 
Each has four strips (five if there is a quality band) 
and should be read with the silver or gold strip for tolerance on the right. 
Each color corresponds to a number: black = 0, 
brown = 1, 
red = 2, 
orange = 3, 
yellow = 4, 
green = 5, 
blue = 6, 
violet = 7, 
gray = 8, 
white = 9, 
and additionally for the third stripe, 
gold = -1 and silver = -2. 
The first two numbers should be taken as a two digit number, 
so green violet would be 57, 
red black 20, 
etc. 
The third number should be taken as the power of ten (a $ 10^{n} $ term), 
so red orange yellow would be $ 23 \times 10^{4} = 230000 $, 
red brown black would be $ 21 \times 10^{0} = 21 $ 
and blue gray silver would be $ 68 \times 10^{-2} = 0.68 $. 
The unit is always ohms. 
The fourth and possibly fifth bands may be ignored.

Capacitors, 
diodes, 
transistors, 
inductors and other useful circuit parts can also be bought 
at shops in town or liberated from old circuit boards, 
radios, 
phone chargers, 
etc. 
Capacitors tend to state their capacitance in microFarads on their bodies.

\section{Retort stand}

\subsection{To hold burettes}
Satisfactory retort stands may be produced 
by cutting a piece of cement reinforcing rod (re-rod, 
about 1 cm in diameter) and placing it in a metal tin full of wet cement. 
Once the cement hardens, 
you may attach a boss head and a clamp to have an equivalent stand. 
Stiff wire may be used in place of boss heads and clamps 
but they do not hold burettes as well. 
Especially if you have the misfortune of owning fragile glass burettes, 
the investment in good clamps is worthwhile.

Of course, 
if you use syringes as burettes, 
there is no need for a retort stand.

\subsection{To hold pendulums}
Hang the pendulum from any elevated point. 
You can place a chair on a desk and hang a pendulum form the legs, 
or for a smaller diameter rod anchor a welding stick under a large rock.

\subsection{To hold other apparatus}
Improvise! 
Both wire and strings of bicycle inner tube are versatile 
and effective binding agents.

\section{Scalpels}
Razor blades. 
You can find ways to modify them 
to remove one of the edges and add a handle, 
though students are pretty skilled with these blades. 
Dull blades should be discarded -- 
because students need to apply more pressure when using them, 
there is a greater risk of slipping and thus of cuts. 
Sharp tools are much safer. 
For dissection, 
find a way to attach a handle to the blade 
to increase the pressure 
the student is able to safely apply to the cutting point.

\section{Spatula}
Get stainless steel spoons from the market or a small shop. 
These work better than traditional spatula. 
For removing salts from containers, 
you can use the other end of the spoon. 
Make sure to clean all metal tools promptly after using with hydroxide, 
potassium manganate (VII), 
or manganese (IV) oxide. 
If you forget and the spoon corrodes, 
you can remove most of the corrosion 
by scraping with another spoon or steel wool.

\section{Springs}
Ask around. 
Springs may be found at hardware stores, 
bike stores, 
and especially junk merchants in markets. 
If you tell some of these people what you are looking for, 
they will probably find it for you by the next time you come to town. 
There are also much thinner springs 
readily available at stores that sell window blinds. 
There is a small white long (1 meter to 2 meters) spring 
used to hold up window shades in homes. 
Use a knife to remove the plastic coating. 
Cut the spring into small segments 
and you have 10 to 20 5~cm long springs with a useful spring constant. 

\section{Stoppers}
These can be made by the people who cut up old tires 
or you can make them yourself from old sandals. 
However, 
stoppers are rarely required. 
If you are using the stopper because you want to shake a flask, 
consider just using a water bottle with a screw cap. 
If you want a stopper with a hole for passing out a gas you are producing, 
again use a water bottle 
and super glue the tip of a clear plastic pen body into the cap. 
You can then mount rubber tubing onto the pen tip 
for a reliable connection.

\section{Stopwatches}
Stop watches with the look of athletic and laboratory stopwatches 
are often available in big city goods markets 
for much less than at laboratory supply stores. 
Many digital wristwatches also have a stopwatch feature 
and these are widely available.

\section{Test tubes}
\label{sec:testtubes}
\subsection{Plastic test tubes}
These will work for everything in biology, 
everything in physics, 
and everything in chemistry except thermal decomposition of salts. 
They have the obvious advantage of not breaking. 
To make these, 
remove the needle and plunger from 10~mL syringes. 
Heat the end of the shell where the needle joined in a flame until it melts. 
Press the molten end against a flat surface (like the end of the plunger) 
to fuse it closed. 
If the tube leaks, 
fuse it again. 
Test tubes made this way may be heated in a water bath up to boiling, 
hot enough for most experiments.

\subsection{For thermal decomposition}
See \nameref{sec:deflagratingspoon}.

\subsection{Glass test tubes}
You can purchase glass vacuum sample tubes 
in bulk from medical and veterinary supply shops. 
These may generally be heated in open flame, 
although they are not labelled as borosilicate (Pyrex), 
and will probably break sooner. 
We have not tested them in Bunsen burners.

\section{Test tube holder / tongs}
For prolonged heating, 
you can wrap stiff wire tightly around the lip of the test tube. 
For shorter heating, 
you can do the same with a strip of paper or clot. 
You can also find a carpenter to make large wooden holders. 
Clothespins work well if you can find them large enough, 
or if you use smaller tubes, 
or it you use tubes made from syringes with useful flanges at the top.

\section{Test tube racks}
If you have test tubes, 
it is nice to have something to keep them from falling over and breaking. 
You can get some styrofoam and punch holes in it, 
or make one from a plastic water bottle -- 
just put sand in the bottom to increase stability 
and prevent hot tubes from melting the bottle. 
You can make a fancier rack by cutting up a bottle: 
slice it in half along the vertical axis 
and rest the two cut edges on a flat surface 
so the bottle half bows up towards you. 
Cut holes into it for the test tubes. 
Or just use a rectangular bottle. 
The possibilities are endless. 
Local carpenters can also make them from wood 
and this is a good if small way 
to get the village more involved with the school. 
Ordering these from a supply company should only be a last resort. 

\section{Tripod stands}
A welder or metal worker in town can make these. 
Bring a sample to make sure the stand is not too short or too tall. 
You can also make your own from stiff wire.

\section{Volumetric ``Glass''ware}
It is often necessary to measure large volumes (100~mL -- 2~L) 
of solution rather accurately. 
The ultimate equipment for this job are flat bottomed volumetric flasks, 
the spherical globes with the long vertical neck. 
Such precision is usually not required in secondary school, 
especially because titration solutions should be standardized prior to use. 
Relatively accurate measurements may be made using devices 
that are found in every village: plastic water bottles.

The trick is that because water bottles are made in a factory 
by injection molding, 
they are essentially all identical in size. 
Most bottles have various markings molded into the bottle, 
and since the engineers that design these bottles tend 
to prefer round numbers, 
many bottles have very convenient marks.

Borrow a graduated cylinder and find samples of various water bottles. 
Identify the volume of every useful mark on the bottles 
and then share the information widely.

Volumes not immediately measurable with bottles 
may sometimes be measured by addition or subtraction of bottle measures. 
Remember those egg timer problems?

\section{Wash bottle}
Put a hole in the cap of a water bottle. 
Perhaps the best method is to use a syringe needle. 
Gently and firmly apply pressure and push the needle through the lid. 
This is gives a very small hole 
that responds well to the application of pressure. 
If more movement of liquid is needed, 
puncture more holes with the syringe needle. 
You can also heat a stiff wire in a flame and burn the hole, 
hammer a nail through the cap, 
or use a small knife like an awl.

\section{Water bath}
\label{sec:hotwaterbathes}
Take an aluminum pot and fill with water. 
Put this on top of any heat source and let the water heat up. 
Place the test tubes in the hot water to heat. 
If heating the liquids in the test tubes to a specific temperature, 
make sure students put the thermometer in the test tube, 
not the water. 
For smaller scale work, 
use the bottom half of an aluminum can.

Many times, 
the water bath will be much larger than the test tubes 
and they might fall over, 
into the water. 
Devise methods to prevent this. 
You might clamp the tubes to the side with clothespins, 
attach parallel wires to the container to rest the tubes in between, 
or punch holes in a flat piece of plastic to put over the top of the water.

\section{Weights}

\subsection{Crude weights}
Batteries, 
coins, 
glass marbles from town, 
etc. 
You do NOT need to know the mass of these objects; just make new units. 
For example, 
if using marbles, 
measure force as 2 marble-meters-per-second-squared. 
This is an excellent way to teach the meaning of units. 
Note that coins often have surprising variation depending on age, 
wear, 
etc.

\subsection{Adding weight in known intervals}
For practicals where specific weights must be added 
to a system of unknown mass, 
e.g. 
when weights are added to a weigh pan in spring practicals, 
water may be used. 
As the weight of the pan is both unknown and irrelevant, 
consider ``zero added mass'' the displacement of the pan with an empty water bottle. 
Then, 
added 50~g, 
100 g, 
etc. masses with water bottles with 50~mL, 
100~mL, 
etc. of water.

\subsection{Precise weights}
Find small (250~mL) water bottles. 
Get as many as you need weights. 
They must be all the same type.
Remove the labels and make sure the bottles are completely dry. 
This is readily accomplished 
by leaving them uncapped outside on a warm day.
Use an accurate balance to find the mass of one bottle, 
cap included. 
If you do not have an accurate balance, 
visit a school that does. 
You should only have to do this once.
Subtract the mass of the empty bottle (say, 
1.24~g) from the mass you want for your weight (say, 
50~g). 
This mass in grams will be provided 
by this volume of water in milliliters (so, 
for our example, 
$ 50 - 1.24 = 48.76 $~mL water). 
Use a plastic syringe to add exactly this mass of water to the dry bottle.
Cap the bottle firmly and label it with permanent pen: ``50~g weight'' 
If you want your masses to have hooks, 
attach some wire around the neck of the bottle 
and bend one end to make a hook. 
Of course, 
do this before step 3 so you add that much less water.

You could also make weights by using a balance 
to fill small plastic bags with sand. 
This makes smaller weights (good!) 
but requires a balance for making each one, 
and a balance to replace any one that rips open.

\section{White tiles}
White paper works just as well. 
If your students are using syringes as burettes, 
they can also hold their flask up against a white wall.

\section{Wire}
\label{sec:wire}
\subsection{All-purpose wire}
Use speaker wire, 
the pairs of colored wires brained together. 
The wire is easy to strip using a wide variety of tools, 
or just your teeth, 
specifically the space between your incisors and front molars.

\subsection{Specific gauge wire}
Copper wire is imported and sold in large quantities in port cities 
for use in industry. 
These wires, 
often used for motor winding and other electrical applications, 
are generally coating with an insulating varnish 
and come in a variety of diameters (gauges). 
A useful chart for converting diameter to gauge may be found  
\href{http://www.dave-cushman.net/elect/wiregauge.html}{here}. 
If the wire is sold by weight, 
you can find the length if you know the diameter - 
the density of copper metal at room temperature is 8.94~g/cm$^{3}$. 
For example, 
with 0.375~mm wire, 
250~g is about 63 meters.

