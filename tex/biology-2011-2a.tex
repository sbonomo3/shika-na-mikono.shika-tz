\section{2011 - BIOLOGY 2A (ACTUAL PRACTICAL A)}

\begin{enumerate}
\item[1.] The solution prepared contained various food substances.
\begin{enumerate}
\item[(a)] Use the chemicals and reagents provided to identify the food substances present in solution \textbf{S$_1$}. Tabulate your work as shown in the following Table:

\begin{center}
\begin{tabular}{|p{3cm}|p{3cm}|p{3cm}|p{3cm}|} \hline
\textbf{FOOD TESTED}&\textbf{PROCEDURE}&\textbf{OBSERVATION}&\textbf{INFERENCE} \\ \hline
&&& \\
&&& \\
&&& \\
&&& \\ \hline
\end{tabular} \\[10pt]
\end{center}
\item[(b)] State the function in the human body of each food identified in 1(a) above.
\item[(c)] Name two enzymes necessary for digestion of food substance(s) identified in (a) above.
\item[(d)] To each type of food identified above, name at least one source in which the food has been extracted.
\end{enumerate}

\item[2.] Study specimen \textbf{A}, \textbf{B} and \textbf{C} then:
\begin{enumerate}
\item[(a)] Write common names of specimen \textbf{A}, \textbf{B} and \textbf{C}.
\item[(b)] Classify specimen \textbf{A} and \textbf{B} to the phylum level.
\item[(c)] State the habitat and one economic importance of specimen \textbf{A}.
\item[(d)] Outline four economic importance of specimen \textbf{B}.
\item[(e)] Use the scalpel provided to cut specimen \textbf{C} longitudinally into two equal halves. Then, draw a neat, well labelled diagram of a specimen.
\item[(f)] Name the division of specimen \textbf{C}.
\item[(g)] State the observable features you can use to place the specimen into its respective phylum/division.
\end{enumerate}

\end{enumerate}