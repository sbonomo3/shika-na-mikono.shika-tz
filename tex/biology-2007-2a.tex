\section{2007 - BIOLOGY 2A (ALTERNATIVE A PRACTICAL)}

\begin{enumerate}
\item[1.]
You are provided with solution \textbf{S}.
\begin{enumerate}
\item[(a)] Carry out experiments to identify the food substance present in solution \textbf{S}.
\begin{enumerate}
\item[(i)] Record your experimental work as shown in Table 1 below. \hfill \textbf{(16 marks)}\\

Table 1

\begin{center}
\begin{tabular}{|p{3cm}|p{3cm}|p{3cm}|p{3cm}|} \hline
\multicolumn{1}{|c|}{\textbf{Test for}}&\multicolumn{1}{c|}{\textbf{Procedure}}&\multicolumn{1}{c|}{\textbf{Observation}}&\multicolumn{1}{c|}{\textbf{Inference}} \\ \hline
&&& \\
&&& \\
&&& \\
&&& \\ \hline
\end{tabular} \\[10pt]
\end{center}

\item[(ii)] Solution \textbf{S} contains ------. \hfill \textbf{(3 marks)}
\end{enumerate}
\item[(b)] Suggest one storage organ in a plant from which solution \textbf{S} might have been prepared. \flushright \hfill \textbf{(1 mark)}
\item[(c)] For each food substance identified in (a)(ii) above, name its end product(s) of digestion. \flushright \hfill \textbf{(4 marks)}
\item[(d)] Which of the identified food substance is mostly needed by small children? \hfill \textbf{(1 mark)}
\end{enumerate}

\item[2.] You are provided with a beaker, tea bag and hot water. Carry out the following experiment:\\

Pour about 100 cm$^3$ of hot water into the beaker.\\
Put the tea bag into the beaker containing hot water.\\
Observe carefully the experiment for a few minutes.
\begin{enumerate}
\item[(a)]
\begin{enumerate}
\item[(i)] What happened to the tea bag when it was put in hot water? \hfill \textbf{(3 marks)}
\item[(ii)] Explain why the changes you observed occurred? \hfill \textbf{(4 marks)}
\end{enumerate}
\item[(b)]
\begin{enumerate}
\item[(i)] What do you think was the aim of the experiment? \hfill \textbf{(3 marks)}
\item[(ii)] Draw a conclusion from the experiment. \hfill \textbf{(3 marks)}
\end{enumerate}
\item[(c)]
\begin{enumerate}
\item[(i)] Name the physiological process investigated in this experiment. \hfill \textbf{(3 marks)}
\item[(ii)] Define the process named in (c)(i) above. \hfill \textbf{(4 marks)}
\item[(iii)] What is the importance of this process in nature? \hfill \textbf{(5 marks)}
\end{enumerate}
\end{enumerate}

\item[3.] Study the specimens \textbf{J}, \textbf{K}, \textbf{L}, \textbf{M} and \textbf{N} provided.
\begin{enumerate}
\item[(a)] Identify specimens \textbf{J}, \textbf{K}, \textbf{L}, \textbf{M} and \textbf{N} by their common names. \hfill \textbf{(5 marks)}
\item[(b)] Name the kingdoms for each of specimens \textbf{J}, \textbf{K}, \textbf{L}, \textbf{M} and \textbf{N}. \hfill \textbf{(5 marks)}
\item[(c)] Suggest the possible habitats for specimens \textbf{J} and \textbf{K}. \hfill \textbf{(4 marks)}
\item[(d)] Draw and label specimen \textbf{N}. \hfill \textbf{(7 marks)}
\item[(e)] List \textbf{four (4)} observable differences between specimens \textbf{J} and \textbf{K}. \hfill \textbf{(4 marks)}
\end{enumerate}
\end{enumerate}