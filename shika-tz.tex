% SHIKA NA MIKONO v4.0

% This code is published with the GNU General Public License v2.

% The remaining comments explain the code.
% -------------------------------------------

% This file - shika-tz.tex - is the backbone file for the manual.
% It's job is to set the global parameters and order the chapters.
% Each chapter lives in a different file (all in the tex folder)
% This keeps everything more organized.

% Before actually getting to the document itself, we need some PREAMBLE info - state the class of document and declare any LaTeX packages we are going to use.
\documentclass[pdftex,10pt,a4paper]{report}

% We want a variety of style packages.
% We have them in a separate file - shika.sty - to keep this file tidy.
% This command calls all of them at once.
\usepackage{shika}

% Now the document can begin:
\begin{document}

% The input command inserts the content of another file into this master document.
% Here it calls for the title page, title.tex
\input{./tex/title.tex}

\begin{center}
{\Huge Questions or Comments?}\\[12pt]
Thanks for using the \textit{Shika na Mikono} manual! If you have any questions, comments, or would like to request a copy of this manual, please use the contact information given below.\\[20pt]
shika-tz@gmail.com
\end{center}
\vfill
To download a digital version of this manual, please visit the \href{http://code.google.com/p/shika-na-mikono/downloads/list}{digital download page} at the following website: \url{http://code.google.com/p/shika-na-mikono/downloads/list}

% To make page numbers consistent w/ the numbering in the PDF viewer.
% (e.g. page 35 is actually page 35 in the .pdf)
\setcounter{page}{2} 

% Some front matter...
\input{./tex/about-this-book.tex}
\input{./tex/teaching-philosophy.tex}

% Now a table of contents:
\tableofcontents

% ...but we do not want a page number to appear on the table of contents
\thispagestyle{empty}

% The rest of the book follows, divided into different parts

% For these early Parts, we only want to show up to the Chapter level in Table of Contents
\settocdepth{chapter} 

% Part 1 - Lab Development
\input{./tex/part-lab-development.tex} % This is a page dedicated to annoucing Part 1
\input{./tex/starting-school-labs.tex}
\input{./tex/technical-lab-needs.tex}
\chapter{Improving an Existing School Laboratory} \index{Laboratory! improving}

If there is already a laboratory at your school, 
the immediate tasks are to see what it has, 
make it safe, 
get it organized, 
make repairs, 
and ensure smart use with sound management.

\section{Inventory}

\begin{itemize}

\item Making a list of what and how much of everything is in your lab is easy, 
if time consuming. 
Difficulties arise when you find apparatus you have never seen before, 
or containers of chemicals without labels.

\item There is no harm in unknown apparatus, 
they just are not useful until you know what they do. 
Ask around.

\item Unknown chemicals, 
however, 
pose a hazard, 
because it is unclear how to properly store them or how to clean up spills. 
If a chemical is unknown, 
there is no safe way to responsibly dispose of it. 
Therefore, 
it is best to attempt to identify unknown chemicals. 
For assistance in identifying unknown chemicals, 
please see \nameref{cha:unknownchemicals} (p.~\pageref{cha:unknownchemicals}).

\item Burettes and apparatus concerning electricity, 
for example voltmeters and ammeters, 
should be tested to ensure that they work. 
Please consult \nameref{cha:volanatech} (p.~\pageref{cha:volanatech})
to learn how to use burettes 
and \nameref{cha:voltamm} (p.~\pageref{cha:voltamm}) to do just that.

\end{itemize}

\section{Organize}

\subsection{Have enough space}
The key to organization is having enough space. 
Usually, 
this means building shelves. 
In the long term, 
find a carpenter to build good shelves. 
In the short term, 
boards and bricks, 
scrap materials, 
chairs, 
anything to provide sturdy and horizontal storage space. 
It should be possible to read the label of every chemical, 
and to see each piece of equipment

\subsection{Apparatus}
\begin{itemize}
\item{Arrange apparatus neatly so it is easy to find each piece.}
\item{Put similar things together.}
\item{Beakers can be nested like Russian dolls.}
\end{itemize}

\subsection{Chemicals}
\begin{itemize}
\item{Organize chemicals alphabetically. 
There are more complicated schemes involving the function 
or the properties of the chemical but what is most important 
is a scheme that everyone working in the lab can follow. 
ABC is the easiest, 
and has the best chance of being used. A good alternative is to organize by chemical makeup (e.g. sodium, etc.)}
\item{Glass bottles of liquid chemicals should be kept on the floor, 
unless the laboratory is prone to flooding, 
in which case they should be on a sufficiently elevated, 
broad and stable surface. 
What you do not want are these bottles falling and breaking open.}
\item{Million's Reagent, 
benzene, 
and other chemicals that should never be used should be kept in a special place, 
ideally locked away, 
and labelled to prevent use. 
See \nameref{cha:dangerchem} (p.~\pageref{cha:dangerchem}) for a list of chemicals that should never be used.}
\item{Label plastic containers directly with a permanent pen, 
especially if the printed label is starting to come off.} 
\item{Replace broken or cracked containers with new ones.}
\end{itemize}

\subsection{Make a map and ledger}

\begin{itemize}
\item Once you have labeled and organized everything in a lab, 
draw a map. 
\item Sketch the layout of your laboratory 
and label the benches and shelves. 
\item In a ledger or notebook, 
write down what you have and the quantity. 
For example, 
Bench 6 contains 20 test tubes, 
3 test tube holders, 
and 4 aluminum pots. 
\end{itemize}
This way, 
when you need something specific, 
you can find it easily. 
Further, 
this helps other teachers -- especially new ones -- better use the lab. 
Finally, 
having a continuously updated inventory will let you know what 
materials need to be replaced or are in short supply. 
Proper inventories are a critical part of maintaining a laboratory, 
and they really simplify things around exam time.


\section{Repair/Improve}

Once the lab is organized, 
it is easy to find small improvements. 
Here are some ideas:

\subsection{Build more shelves}
You really cannot have too many.

\subsection{Fix broken burettes}
Burettes are useful, 
expensive and -- if glass -- fragile. 
Broken burettes can often be made functional again. 
If you have broken burettes, 
see \nameref{cha:burettes} (p.~\pageref{cha:burettes}).

\subsection{Check voltage and current meters}
Voltmeters, ammeters and galvanometers often get discarded or unused despite still being able to function sufficiently for use during demonstrations and practicals. Before getting rid of these meters, see the section on \nameref{cha:voltamm} (p.~\pageref{cha:voltamm}).

\subsection{Identify key apparatus needs}
Sometimes a few pieces of apparatus can be very enabling, 
like enough measuring cylinders, 
for example. 
Buy plastic!

\section{What next?}

Once the lab is safe and organized, 
develop a system for keeping it that way. 
Consider the advice in \nameref{cha:routineup} (p.~\pageref{cha:routineup}). 
Make sure students and other teachers in involved.

Then, 
start using the lab! Every class can be a lab class. 
That is the whole point.

\input{./tex/testing-voltmeters.tex}
\input{./tex/burette-repair.tex}
\chapter{Identifying Unknown Chemicals} \index{Chemicals! unknown}
\label{cha:unknownchemicals}
Unlabelled chemicals are dangerous. 
If you do not know what the chemical is, 
then you do not know what to do if it spills, 
or how to safely get it out of your school.

%==============================================================================
\section{Identifying Bottles of Unknown Liquids}

Usually, 
these are:
\begin{itemize}
\item Concentrated acids (sulfuric, 
hydrochloric, 
nitric, 
ethanoic)
\item Concentrated ammonia solution
\item Organic solvents including methanol, 
ethanol, 
isobutanol, 
propanone (acetone), 
diethyl ether, 
ethyl ethanoate (ethyl acetate), 
dichloromethane, 
trichloromethane (chloroform), 
tetrachloromethane (carbon tetrachloride), 
trichloroethene, 
benzene, 
chlorobenzene, 
toluene, 
xylene, 
and petroleum spirits
\end{itemize}

\noindent Distinguishing these chemicals is important, 
and relatively possible. 
Here is a procedure:
\begin{enumerate}
\item \textbf{Protect yourself against whatever it might be.}\\ 
Concentrated acids burn skin on contact and blind if they get in the eyes. 
Concentrated hydrochloric acid and concentrated ammonia solution 
release fumes that corrode the throat and lungs. 
Diethyl ether and propanone rapidly evaporate at room temperature 
and pose a significant flash fire hazard if opened near flame. 
Ingesting even a small amount of toxic carbon tetrachloride can be fatal, 
and benzene is a proven and serious carcinogen.

\emph{Why, 
you might say, 
should I even attempt this? Because sooner or later, 
someone will, 
and better it be someone with these instructions than without. 
But if you do not feel comfortable, 
call a friend who is more excited about this process.}

Many precautions are available. 
\begin{itemize*}
\item Tie a cloth over your mouth and nose to mitigate inhalation. 
\item Find a pair of goggles or sunglasses to protect you eyes 
from any splash when opening the stopper in the bottle. 
\item Wear gloves or at least plastic bags on your hands. 
Neither will protect your hands for more than a second 
or a few against concentrated acids or some organics, 
but that second can be useful in this case. 
\item Thick rubber gloves are available (see \nameref{cha:labequip}, p.~\pageref{cha:labequip}) 
and offer greater protection. 
\item Regardless, 
have at the ready a bucket of water and a box of baking soda 
(bicarbonate of soda) to neutralize acid burns. 
\item Move the container outside and remain upwind. 
\item Have a small, 
dry, 
clean beaker ready to hold a sample.
\end{itemize*}

\item \textbf{Open the bottle.}\\
This may be as simple as unscrewing the top 
or there may be an internal stopper that requires prying off. 
\begin{itemize*}
\item Find a suitable tool, 
one that can pry under the cap but cut neither the cap nor you. 
A butter knife works well. 
\emph{Do not use your fingers.}

\item When the bottle opens, 
look at the top. 
Are there white fumes? 
Is there an obvious smell that you can perceive 
from where you are standing? 
White fumes suggest hydrochloric acid 
and an intense smell could be ammonia (smells like stale urine), 
hydrochloric or ethanoic acid (both smell like vinegar), 
or an organic solvent (various odors).

\item If the contents smell obviously like ammonia, 
there is no need to further experimentation. 
Nothing else in schools smells even remotely like ammonia. 
Stopper that bottle and give it a good label.

\item Otherwise, 
carefully, 
pour a few cubic centimeters 
of the liquid into your sample beaker. 
As you pour the liquid, 
observe the viscosity. 
Concentrated acids are all noticeably more viscous than water, 
especially concentrated sulfuric acid. 
Propanone, 
on the other hand, 
is noticeably more fluid than water. 
Close the bottle and take the beaker 
to a safe place for experimentation.

\item Color is surprisingly useless in identifying unknown liquids 
because most readily take on color 
from even small amounts of contamination.

\item Rest the beaker on a sturdy surface. 
If you have already noticed an intense smell, 
leave the cloth on your face. 
If you have not yet noticed a smell, 
remove it.
\end{itemize*}

\end{enumerate}

%==============================================================================
\section{Test one: Add to water}

\begin{itemize}
\item Fill a large, 
clean test tube half way with ordinary water. 
Alternatively, 
find the smallest beaker you have (probably 50~mL), 
and fill it about a quarter of the way with water. 
Carefully pour in a few drops of your unknown 
and observe what happens. 


\item If it does not mix with the water, 
instead forming a new (possibly quite small) layer on top, 
you have an \emph{organic solvent less dense than water}, 
probably one of: isobutanol, 
diethyl ether, 
ethyl ethanoate, 
benzene, 
chlorobenzene, 
toluene, 
xylene, 
or petroleum spirits. 

\item If it does not mix with the water, 
instead sinking to form a distinct layer on the bottom, 
you have an \emph{organic solvent more dense than water}, 
probably dichloromethane, 
chloroform, 
or carbon tetrachloride.

\item If your unknown does not mix with water, 
jump down to \nameref{sec:testorganic} on what to do with organics.

\item If the unknown seems to sink into the water but not mix completely, 
you probably have a \emph{concentrated acid}. 
The test tube might even get a little warmer. 
You might also have a very concentrated solution of some other solute, 
left over from a previous experiment.

\item If the unknown seems to mix into the water like, 
well, 
water, 
you probably have an \emph{aqueous solution} that is not very concentrated. 
It might be dilute acid, 
dilute hydroxide, 
hydrogen peroxide solution, 
etc. -- more work lies ahead.
\end{itemize}

%==============================================================================
\section{Test two: Is it an acid?}

\emph{This only applies to solutions that mix completely into water.}

\begin{enumerate}
\item This is easy with a piece of blue litmus paper. 
Dip a corner down into the test tube or beaker. 

\begin{itemize*}
\item If it turns bright red, 
you probably have an \emph{acid}, 
and if your liquid was noticeably viscous, 
a concentrated acid. 
\item If there is no change, 
move on to \nameref{sec:whatelse}.
\end{itemize*}

\item Another option is universal indicator or universal pH paper. 
\begin{itemize*}
\item Prepare a 100-fold dilution of the original acid 
and test with the indicator. 
\item If the color is bright red, 
you must have a \emph{strong acid}, 
like hydrochloric, 
sulfuric, 
or nitric acid. 
\item If the color is instead orange or yellow, 
you must have a \emph{weak acid}, 
like ethanoic acid. 
\end{itemize*}


\item If there is no universal indicator, 
you can show that something definitely is an acid 
if it causes methyl orange to turn from orange to red. 
However, 
if there is no color change, 
you might still have a weak acid, 
so you cannot use methyl orange to eliminate the possibility of an acid. 

\item You also cannot use POP to show that there is an acid, 
as both concentrated acid and tap water have the same effect on POP: 
none whatsoever.

\item If you do not have any litmus paper or other indicator, 
find another beaker and add 10-20~mL of ordinary water 
and dissolve a bit of baking soda (bicarbonate of soda). 
Carefully, 
with eye protection, 
add a few drops of your DILUTED unknown (from test one). 
If there are bubbles, 
you have an \emph{acid}. 
Adding a concentrated acid directly to baking powder 
can cause such vigorous effervescence as to eject acid from the test tube.
\end{enumerate}

%==============================================================================
\section{Test three: What kind of acid?}

%------------------------------------------------------------------------------
\subsection{Sulphuric acid} \index{Sulphuric acid! identification of}

\begin{itemize}

\item{Hints: obviously viscous, 
significantly denser than water, 
noticeable heat released on dilution, 
no smell}

\item{Confirmatory test: dip the wooden end of a match into the original solution. 
If the end appears to char, 
you have concentrated sulfuric acid. 
Another variant of this test is take some concentrated sulfuric acid 
and pour over some sugar in a beaker. 
After some time, 
a black color from carbon produced 
from the dehydration of sugar confirms sulfuric acid. 
Yes, 
the same thing happens to skin. 
The downside of this second test is that the beaker 
is almost impossible to clean.}

\item{Alternative test: Find or prepare a 0.1~M barium nitrate, 
barium chloride, 
or lead nitrate solution. 
In a test tube, 
add about one centimeter of your diluted sample 
and then a few drops of one of the above solutions. 
An instant, 
white, 
cloudy precipitate demonstrates that sulfate is present. 
To confirm that this is from sulfuric acid and not, 
say, 
your tap water, 
test in the same way the water you used for the dilution. 
Not much should happen. 
If your tap water contains sulfates, 
find some distilled (e.g. 
rain) water and remake the dilution.}
\end{itemize}

%------------------------------------------------------------------------------
\subsection{Hydrochloric acid} \index{Hydrochloric acid! identification of}

\begin{itemize}

\item{Hints: white fumes, 
intense acidic smell similar to vinegar, 
more dense than water}

\item{Confirmatory test: 
prepare a dilute potassium permanganate solution. 
This should be pink in color, 
which might require significant dilution. 
Fill a test tube with a couple centimeters of your dilute solution 
and add the potassium permanganate solution drop wise. 
If the pink color is rendered colorless 
after mixing with your diluted sample, 
you probably have hydrochloric acid. 
This reaction makes small amounts of chlorine gas, 
but that poses much less risk than the hydrochloric acid fumes.}

\item{Alternative test: 
Find or prepare a 0.05~M or 0.1~M silver nitrate solution. 
Remember that this chemical is very expensive, 
so only make a small quantity. 
In a test tube, 
add about one centimeter of the water you used 
for diluting your sample 
and then a few drops of one of silver nitrate solution. 
An instant, 
white to gray, 
cloudy precipitate demonstrates that chloride is present. 
If this happens, 
your tap water contains chlorine 
and you will have to prepare another dilution 
using rain or distilled water. 
If the water you used for dilution lacks chlorine, 
add a centimeter of the diluted sample 
to a clean test tube and add a few drops of silver nitrate solution. 
The precipitate confirms that you have hydrochloric acid. 
Note that for this test to be effective, 
the hydrochloric acid must be diluted. 
Concentrated hydrochloride acid reacts with aqueous silver 
to form the [\ce{AgCl2}]-complex, 
which is soluble.}

\end{itemize}

%------------------------------------------------------------------------------
\subsection{Ethanoic (acetic) acid}

\begin{itemize}

\item{Confirmatory Test: This acid smells strongly of vinegar. 
If you have a definite vingar smell, 
it is probably ethanoic acid, 
but beware that concentrated HCl can have a similar smell. 
To confirm ethanoic acid, 
use some diluted acid from test one 
and add a small amount of baking soda until it is just neutralized. 
Do not add excess baking soda - neutralization is the goal. 
After neutralizing, 
add a small amount of iron (III) chloride or nitrate. 
A blood red solution of iron (III) acetate 
proves that the acid is ethanoic. 
Boiling the solution should form a red brown precipitate. 
If you do not have iron (III) salts but do have universal indicator, 
use the indicator method above 
for confirming that your unknown is a weak acid -- 
ethanoic is the only common weak acid that smells like vinegar.} 
\end{itemize}

%------------------------------------------------------------------------------
\subsection{Nitric acid}
A concentrated acid in a school that does not smell like vinegar 
and is not hydrochloric or sulfuric acid is very likely to be nitric acid. 

\begin{itemize}

\item{Confirmatory Tests: Take a wooden splinter 
or match stick and dip it in the concentrated acid. 
If the splinter turns yellow, 
the acid is nitric. 
A second confirmatory test is adding copper wire 
or turnings to the concentrated acid. 
A brown gas of nitrogen dioxide is formed. 
Do this confirmatory test in a well ventilated area.}

\item{Special note: if you suspect nitric acid, 
dip a piece of copper wire into the solution. 
If it comes back with a silvery coating, 
you have Million’s Reagent, 
mercury metal dissolved in nitric acid. 
This is highly toxic, 
very dangerous, 
and should never be used in a school. 
Label the bottle “Million’s Reagent, 
Contains \ce{Hg2+}, 
TOXIC, 
CORROSIVE, 
do not use, 
do not dump” along with similar warnings 
in any local language(s) and find a safe place to store it.}

\end{itemize}

%==============================================================================
\section{Test four: What kind of organic?}
\label{sec:testorganic}
Let us be honest. 
Distinguishing between different kinds of organic solvents 
is hard with the resources that are probably available. 

\begin{itemize}
\item If the chemical is more dense than water 
and no one at the school claims that it is chloroform 
(trichloromethane) for the biology lab, 
there is no way to show that it is not carbon tetrachloride 
(tetrachloromethane), 
a toxic organic solvent responsible 
for the death students in several countries. 
Label the bottle ``Unknown organic solvent more dense than water, 
possibly carbon tetrachloride, 
TOXIC, 
never use, 
never dump,'' with similar warnings in any local language(s) 
and find a safe place to store it.

\item If the chemical is less dense than water 
and you are familiar with organic solvents, 
you might try a careful smell test.

\item If the unknown smells like strong booze and is soluble in water, 
it is probably ethanol or methanol. 
Do not drink it! -- methanol blinds. 
\item If it is bright red, 
it is probably Sudan III solution, 
for biology. 
Label and use it. 
\item If it is yellow or brown it might be iodine solution, 
see below in test five. 
\item If it is light purple or green or whatever 
the popular color in your country, 
it is probably methylated spirits, 
a mixture of about 70\% ethanol and 30\% water
with some impurities to make it undrinkable. 
Confirm this by showing that paper soaked in the chemical 
will burn with a blue flame but that paper soaked in a 50/50 
mixture of the chemical and water will not burn. 
\item If it is clear and someone at the school can assure 
that the contents are ethanol and not methanol, 
label the bottle ``ethanol'' and use it. 
\item If the bottle might be methanol, 
a poison, 
pour the contents into a large bucket 
and leave it in a place where no one will disturb it 
and where the fumes will not accumulate. 
Let it evaporate.

\item If the unknown smells like nail polish remover 
and is soluble in water, 
it is probably propanone (acetone). 
\item If you put a drop in a spoon it should evaporate relatively quickly. 
Label it ``Propanone, 
EXTREMELY FLAMMABLE'' and keep it around. 
\item If it is not soluble in water and smells like magic markers, 
it is probably diethyl ether or ethyl ethanoate (ethyl acetate). 
If you are familiar with organics, 
perhaps you can pick between these. 
Otherwise just label the bottle ``volatile organic solvent, 
insoluble in water, 
EXTREMELY FLAMMABLE'' and keep it around.

\item If the unknown has a sweet sickly smell it might be toluene. 
It also might be benzene. 
\item If you cannot further identify it and no one else can, 
label the bottle ``unknown non-volatile organic solvent 
less dense that water, 
possibly benzene, 
TOXIC, 
never use, 
never dump'' with similar warnings in any local language(s), 
and find a safe place to store it.
\end{itemize}

%==============================================================================
\section{Test five: What else?}
\label{sec:whatelse}
If your unknown is not an acid, 
not ammonia, 
and soluble in water, 
see what it smells like. 
\begin{itemize}
\item If it smells like booze or nail polish remover, 
it could be methanol, 
ethanol, 
or acetone. 
See the above section on organics. 
\item If it does not have a smell, 
it is probably a solution left over from an earlier lab. 
These are not nearly as dangerous as concentrated acids 
or some volatile organics. 
However, 
be sure to use proper handing methods. 

Here are some possibilities:
\end{itemize}

%------------------------------------------------------------------------------
\subsection{Sodium hydroxide solution} \index{Sodium hydroxide! identification of}

\begin{itemize}

\item{Hints: cloudy and a jammed stopper, 
but not always}

\item{Test: red litmus turns blue or POP pink.}

\item{What to do: sodium hydroxide is cheap when bought as caustic soda. 
Keep it around just for neutralizing acid wastes. 
If its presence disturbs you, 
add some indicator and then cheap acid until neutralization. 
After complete neutralization, 
dump.}

\end{itemize}

%------------------------------------------------------------------------------
\subsection{Hydrogen peroxide} \index{Hydrogen peroxide! identification of}

\begin{itemize}

\item{Hints: colorless liquid, 
in an opaque or dark bottle}

\item{Test: add a bit of acidified potassium permanganate solution. 
The potassium permanganate should turn colorless 
on mixing and bubbles of gas should be observed.}

\item{What to do: label and use. 
If you want to dispose of it for some reason, 
leave it in a bucket in the sun.}

\end{itemize}

%------------------------------------------------------------------------------
\subsection{Potassium permanganate solution}

\begin{itemize}

\item{Hints: intensely purple, 
pink after significant dilution}

\item{Test: to a very dilute solution, 
add crushed vitamin C (ascorbic acid) from a pharmacy. 
The solution should turn colorless.}

\item{What to do: test the pH with litmus paper 
or methyl orange to see if acid has been added. 
Then label ``(acidified) potassium permanganate'' and use. 
If you want to dispose of it, 
add crushed vitamin C until the color disappears 
and then pour into a pit latrine.}

\end{itemize}

%------------------------------------------------------------------------------
\subsection{Iodine solution} \index{Iodine solution! identification of}

\begin{itemize}

\item{Hints: brown color, 
smells like iodine tincture, 
and possibly also like ethanol}

\item{Test: to a dilute solution, 
add crushed vitamin C (ascorbic acid) from a pharmacy. 
The solution should turn colorless.}

\item{What to do: Put a centimeter of water in a test tube 
followed by a smaller quantity of cooking oil. 
Add a few drops of the solution, 
cap with your thumb and mix thoroughly for one minute. 
If two layers quickly separate, 
the iodine solution has been prepared without ethanol. 
If a cloudy mixture (an emulsion) forms, 
the iodine solution has been prepared with ethanol. 
Label the solution ``iodine solution (with ethanol)'' and use it.}

\end{itemize}

%------------------------------------------------------------------------------
\subsection{Potassium ferrocyanide solution}

\begin{itemize}

\item{Hints: light neon green or yellow color}

\item{Test: make a dilute solution of copper sulfate and add a few drops of the unknown. 
An instant, 
massive brown precipitate confirms potassium ferrocyanide solution.}

\item{What to do: Label and use. 
Do not dump while it remains useful.}

\end{itemize}

%==============================================================================
\section{Unidentifiable Liquids}

\ldots are worthless. 
In order to safely dump a liquid chemical, 
ensure the following are true:
\begin{itemize}
\item The liquid is water soluble (otherwise see the organic section above)
\item The liquid is neutral pH (if acid, 
neutralize with bicarbonate of soda, 
if base neutralize with acid waste, 
citric acid, 
or, 
carefully, 
battery acid)
\item The liquid does not contain heavy metals (to a small sample, 
add dilute sulfuric acid drop-wise. 
A precipitate indicated lead or barium. 
Continue adding until additional precipitation stops. 
Then neutralize with bicarbonate of soda. 
The solids are safe for disposal in a pit latrine, 
but may clog sink pipes).
\item The liquid does not contain mercury (to a small sample, 
add sodium hydroxide solution until POP turns the solution pink. 
A yellow precipitate indicates mercury. 
Label the solution ``Contains \ce{Hg2+}, 
TOXIC, 
do not use, 
do not dump'' and store it in a safe place.)
\end{itemize}
Then, 
dilute the chemical in a large amount of water 
and dispose of it in a lab sink or pit latrine.

%==============================================================================
\section{Deliquescent Salts}

If you have a chemical in a container that seems meant 
for holding solids but the chemical looks like a thick liquid, 
you probably have a deliquescent salt that fully deliquesced. 
These solutions can be quite dangerous 
because they are maximally concentrated. 
Make sure that no unknown chemicals touch your skin, 
and wear goggles for this work. 
Then, 
do the following:

%------------------------------------------------------------------------------
\subsection{Test for a base}
The most common deliquescent salt is sodium hydroxide. 
\begin{itemize}
\item Fill a test tube half way with water 
and a few drops of the unknown syrup 
followed by a few drops of POP indicator. 
If the solution turns pink, 
you almost certainly have either 
\emph{sodium hydroxide} or \emph{potassium hydroxide}. 
\item Dilute the liquid in at least 10 times its 
rough volume of water and titrate a sample against 1 M acid. 
\item Find a plastic water bottle with a screw cap for your dilution and 
label it ``sodium or potassium hydroxide, 
$n$~M'', 
where $n$ is the molarity you measured in your titration.
\end{itemize}

%------------------------------------------------------------------------------
\subsection{Color}
If the liquid is not a base, 
it is probably a chloride or nitrate salt of one element or another. 
\begin{itemize}
\item If it is colorless, 
the cation is probably in Group IIA (Ca, 
Sr, 
or Ba) or Group IIB (Zn, 
etc). 
Group IIA compounds have distinct flame test colors:
\begin{itemize*}
\item Ca = orange red,
\item Sr = bright red,
\item Ba = apple green
\end{itemize*}
while Zn has no flame test color.
\item If it is red or brown, 
it is probably iron (III) nitrate or iron (III) chloride. 
\item If it is intensely pink, 
it might be cobalt. 
\end{itemize}
To identify the compound completely, 
you will have to perform qualitative inorganic analysis. 
An introduction to the art is 
in the \nameref{cha:qualana} (p.~\pageref{cha:qualana}) section of this manual, 
and more advanced methods are available on the internet 
and in some advanced chemistry books.

%------------------------------------------------------------------------------
\subsection{Check for mercury}
\begin{itemize}
\item If the liquid is not a base, 
dilute a small sample in water 
and add sodium hydroxide solution until POP turns pink. 
\item If a bright yellow precipitate forms, 
you probably have a mercury salt. 
Transfer all of the compound to a sturdy container 
with a well-sealing lid, 
wash the original container with minimal water once 
and add the washings to the storage container. 
Then wash the original container 
and anything the liquid touched thoroughly. 
Label the new storage container ``Solution of unknown mercury salt, 
CONTAINS Hg!!, 
TOXIC! Do not use, 
Do not dump,'' along with appropriate warnings in any local language(s), 
and find a safe and secure place for long term storage.
\end{itemize}

%==============================================================================
\section{Identifying Unknown Solid Chemicals}

This is not nearly is important as identifying unknown liquids 
for two reasons:
\begin{enumerate}
\item These chemicals are generally (though not always!) less dangerous.
\item Accidental spills are less dramatic. 
\end{enumerate}
The smallest containers are the most likely to hold dangerous chemicals, 
like mercury salts. 
It is best to just leave these ones alone.

What you can do is look at the solid 
and see if it matches any of the descriptions below.
Color is much more useful for identifying solids.

\begin{itemize}

\item{\emph{Bright orange crystals} are likely a chromate 
or dichromate salt (toxic) or a ferricyanide salt (much less toxic). 
The later will form an intensely blue precipitate 
with a small amount of \ce{Fe2+}, 
perhaps from iron (II) sulfate. 
Chromates form a yellow solution that turns orange 
on addition of acid while dichromates for an orange solutions 
that turns yellow on addition of base.}

\item{\emph{Bright yellow, 
orange, 
or red powders} might be lead or mercury compounds. 
These are poisonous, 
the latter very. 
It also might be methyl orange powder. }
\begin{itemize*}
\item Try to dissolve a small amount in water. 
Methyl orange will dissolve readily to give a bright orange solution, 
one that turns red in acid and yellow in base. 
Label the powder and keep it around. 
\item If the salt dissolves but does not seem to be methyl orange, 
add sodium hydroxide until POP changes color. 
\item A yellow precipitate suggests mercury. 
Label as with mercury compounds encountered above. 
\item Most lead compounds are not soluble, 
and will not form a color in solution. Other mercury compounds are also insoluble. 
Label a container that might be lead or mercury as 
``possible lead or mercury compound, 
POISON,'' and store it for the long haul.
\end{itemize*}

\item{\emph{A yellow powder insoluble in water} may also be sulfur. 
It should smell like sulfur. 
A small amount will dissolve in kerosene, 
and the dry powder will melt when heated in a spoon 
over a flame and then burn with a blue flame -- 
producing sulfur dioxide, 
a poisonous gas. 
Do not heat an unknown yellow compound 
unless you are fairly sure it is sulfur.}

\item{\emph{Blue compounds} are often copper salts. 
These should have a green flame test.}

\item{\emph{Purple crystals or flakes} insoluble in water are probably iodine. 
Iodine will dissolve in kerosene to form a red solution.}

\item{One of the few \emph{green powders} is nickel carbonate.}

\item{\emph{Pink wet looking crystals} might be a cobalt compound. 
Heat them gently in a spoon and they should dehydrate to turn blue. 
The blue crystals should turn pink when dissolved in water. 
Cobalt is poisonous.}

\item{\emph{Crystals so purple they look brown or yellow} 
are probably potassium permanganate. 
They should form an intensely purple solution in water. 
Confirm as with potassium permanganate solution above.}

\item{\emph{White crystals and powders} are really hard to identify. 
Label them ``unknown white powder/crystals'' 
and move them to a safe and secure place.}

\item{Flat dull \emph{gray metallic ribbon} about 5~mm wide 
and 1~mm thick is probably magnesium metal. 
It should turn shiny if polished with steel wool. 
It will also burn with a very intense white light 
if lit in either a Bunsen burner or gas cigarette lighter. 
Hold it with tongs, 
and do not stare at the light.}

\item{A \emph{metal stored under oil} is probably sodium or potassium. 
If you are feeling adventurous, 
remove a sample and cut off a VERY small piece, 
perhaps 5~mm on a side. 
Both metals may be easily cut with a knife. 
Return the rest to the original container and seal it again. 
Then, 
add the piece of metal to an open container of water and stand back. 
Both react violently and generally send the piece of metal 
spinning around on a cushion of hydrogen gas. 
Potassium generally gets hot enough to ignite this gas 
which then burns with a lilac flame. 
If the hydrogen under sodium burns, 
it will be yellow. 
The water will become a solution of sodium or potassium hydroxide.}

\end{itemize}


% Part 2 - Lab Safety
\part{Laboratory Safety}
\label{cha:labsafety}
\input{./tex/laboratory-safety.tex}
\chapter{First Aid} \index{First aid}
\label{cha:firstaid}

In spite of taking all necessary precautions to avoid dangerous situations in the laboratory, emergencies may still arise which require the immediate use of First Aid techniques. Listed below are various types of possible emergencies, as well as some immediate treatment guidelines to follow until professional medical attention may be given to the victim.

For treatment information relating to specific chemicals, refer to the section on \nameref{cha:dangerchem} (p.~\pageref{cha:dangerchem}).

%\item{First Aid}
%\begin{enumerate}

\section{Cuts}

\begin{enumerate}
\item{Immediately wash cuts with lots of water 
to minimize chemicals entering the blood stream.}
\item{Then wash with soap to kill any bacteria that may have entered the wound.}
\item{To stop bleeding, apply pressure to the cut and raise it above the heart. 
If the victim is unable to apply pressure him/herself, 
remember to put something (gloves, a plastic bag, etc.) 
between your skin and their blood.}
\item{If the cut is deep (might require stitches) seek medical attention. 
Make sure that the doctor sees how deep the wound really is -- 
you might do such a good job cleaning the cut 
that the doctor will not understand how serious it is.}
\end{enumerate}

\section{Eyes}

\begin{enumerate}
\item{If chemicals get in the eye, immediately wash with lots of water.}
\item{Keep washing for fifteen minutes.}
\item{Remind the victim that fifteen minutes is a short time 
compared to blindness for the rest of life. 
Even in the middle of a national exam.}
\end{enumerate}

\section{First and Second Degree Burns}

\begin{enumerate}
\item{Skin red or blistered but no black char.}
\item{Immediately apply water.}
\item{Continue to keep the damaged skin in contact with water for 5-15 minutes, 
depending on the severity of the burn.}
\end{enumerate}

\section{Third Degree Burns}

\begin{enumerate}
\item{Skin is charred; there may be no pain.}
\item{Do not apply water.}
\item{Do not apply oil.}
\item{Do not removed fused clothing.}
\item{Cover the burn with a clean cloth and go to a hospital.}
\item{Ensure that the victim drinks plenty of water (one or more liters) 
to prevent dehydration.}
\end{enumerate}

\section{Chemical Burns}

\begin{enumerate}
\item{Treat chemical burns by neutralizing the chemical.}
\item{For acid burns, immediately apply a dilute solution of a weak base 
(e.g. sodium hydrogen carbonate).}
\item{For base burns, immediately apply a dilute solution of a weak acid 
(e.g. citric acid, ethanoic acid). 
Have these solutions prepared and waiting in bottles in the lab.}
\end{enumerate}

\section{Ingestion}
\begin{enumerate}

\item{If a student ingests (eats or drinks) the following, induce vomiting.}
\begin{enumerate}
\item{Barium (chloride, hydroxide, or nitrate)}
\item{Lead (carbonate, chloride, nitrate, oxide)}
\item{Silver (nitrate)}
\item{Potassium hexacyanoferrate (ferr[i/o]cyanide)}
\item{Ammonium ethandioate (oxylate)}
\item{Anything with mercury (see list above), 
but mercury compounds should just never be used.}
\end{enumerate}

\item{To induce vomiting:}
\begin{enumerate}
\item{Have the student put fingers into his/her throat}
\item{Have the student drink a strong solution of salt water 
(use food salt, not lab chemicals)}
\end{enumerate}

\item{Do not induce vomiting if a student ingests any organic chemical, 
acid, base, or strong oxidizing agent.}
\begin{enumerate}
\item{These chemicals do most of their damage to the esophagus 
and the only thing worse than passing once is passing twice.}
\item{Organic chemicals may be aspirated into the lungs if vomited, 
causing a sometimes fatal pneumonia-like condition.}
\end{enumerate}

\end{enumerate}

\section{Fainting}
\begin{enumerate}
\item{If a student passes out (faints), feels dizzy, has a headache, etc., 
move him/her outside until fully recovered.}
\item{Check unconscious students for breath and a pulse.}
\item{Perform CPR if necessary and you know how.}
\item{Generally, these ailments suggest 
that harmful gases are present in the lab -- 
find out what is producing them and stop it. 
Kerosene stoves, for example, may emit enough fumes to have this effect.}
\item{See Sources of Heat in the Materials section for alternatives.}
\item{Chemicals reacting in drain pipes can also emit harmful gases. 
See Waste Disposal.}
\end{enumerate}

\section{Electrocution} -- If someone is being electrocuted 
(their body is in contact with a live wire)
\begin{enumerate}
\item{First disconnect the power source. 
Turn off the switch or disconnect the batteries.}
\item{If that is not possible, use a non-conducting object, 
like a wood stick or branch, to move them away from the source of electricity.}
\item{Unless there is a lot of water around, 
the sole of your shoe is non-conducting.}
\end{enumerate}

\section{Seizure}
\begin{enumerate}
\item{If a student experiences a seizure, 
move everything away from him/her 
and then let the body finish moving on its own.}
\end{enumerate}
%\end{enumerate}
\input{./tex/dangerous-chemicals.tex}
\input{./tex/dangerous-techniques.tex}

% Part 3 - Lab Management
\part{Laboratory Management}
\label{prt:lab-mgmt}

\input{./tex/lab-classroom-management.tex}
\input{./tex/routine-upkeep.tex}
\input{./tex/waste-disposal.tex}
\input{./tex/recycling-silver-nitrate.tex}
\input{./tex/recycling-organic-waste.tex}
\input{./tex/industrial-ecology.tex}

% Part 4 - Lab Techniques
\part{Laboratory Techniques} \index{Laboratory techniques}
\label{prt:lab-techniques}

\input{./tex/using-a-beam-balance.tex}
\input{./tex/using-a-plastic-syringe.tex}
\input{./tex/measures-of-concentration.tex}
\chapter{Calculating the Molarity of Bottled Liquids} \index{Molarity! calculating for bottled liquids}
\label{cha:mol-bottle-liq}

You need three pieces of information to perform this calculation:

\begin{enumerate}

\item{The molecular mass of the acid. This is usually written on the bottle and can be easily calculated if it is not. For concentrated acids: sulfuric acid is 98 g/mol, hydrochloric acid is 36.5 g/mol, ethanoic acid is 60 g/mol, and nitric acid is 63 g/mol}

\item{The percent purity of the compound. This might be expressed as a percent (e.g. 31\% HCl), with the symbol $ ^m/_m $ (e.g. $ ^m/_m = 68\% $), or with the word purity ("98\% pure"). If you cannot find this information, see the note at the end.}

\item{The density ($\rho$) or specific gravity (s.g.) of the acid. This should be in grams per cubic centimeter (cc or $\mathrm{cm}^3$).}

\end{enumerate}

Then, you can calculate the molarity of your concentrated acid with this formula:

\[ \mathrm{molarity} = M = \frac{ ( \mathrm{ \text{percent purity} } )( \mathrm{density} )( 1000 \frac{ \mathrm{cm^3} }{ \text{L} } )}{ \mathrm{\text{molecular mass}} } \]

For example, the molarity of an acid bottle labeled “H2SO4, 98\%, s.g. 1.84” we would calculate:

\[ \mathrm{molarity} = M = \frac{ (0.98)(1.84 \frac{ \mathrm{g} }{ \mathrm{cm^3} })( 1000 \frac{ \mathrm{cm^3} }{ \mathrm{L} } ) }{ 98 \frac{ \mathrm{g} }{ \mathrm{mol} } } \]

Note that we used 0.98 for 98\%. Convert all percents to decimals.

Once you do this work, take out a permanent pen and label your stock bottle with its molarity. Then no one needs to do this calculation again.

Note: since you will correct the concentration of your solutions with relative standardization, you really just need to know the approximate molarity of your liquid stock reagent. For new bottles of concentrated acid, you may assume that sulfuric acid is about 18~M, hydrochloric acid is about 12~M, and that both nitric acid and ethanoic (acetic) acid are about 16~M. Battery acid should be 4.5~M $ \mathrm{H}_2 \mathrm{SO}_4 $.

\input{./tex/preservation-of-specimens.tex}
\input{./tex/dissection.tex}
\input{./tex/culture-media.tex}
\input{./tex/using-a-microscope.tex}
\input{./tex/low-power-microscopy.tex}

% Part 5 - NECTA Practicals
\settocdepth{subsection} % Now we want to show up to the subsection level in the ToC
\part{NECTA Practicals} \index{Practicals} \index{NECTA Practicals|see{Practicals}}

\input{./tex/biology-practicals.tex}
\chapter{Chemistry Practicals}

\section{Introduction to Chemistry Practicals}

\subsection{Format}
The format of the Chemistry practical exam was revised in 2011 to keep up with the 2007 updated syllabus. As such, there will be no further Alternative to Practical exams, pending approval from the Ministry of Education.

The Chemistry practical has 3 questions and students must answer all of them. Question 1 is on Volumetric Analysis and Laboratory Techniques and Safety. Question 2 is taken from Ionic Theory and Electrolysis\slash Chemical Kinetics, Equilibrium and Energy. Question 3 is on Qualitative Analysis. Question 1 is worth 20 marks, while Questions 2 and 3 carry 15 marks each. Students have 2$\frac{1}{2}$ hours to complete the exam.

Students are allowed to use Qualitative Analysis guidesheet pamphlets in the examination room.

\subsubsection{Chemistry 1 Theory Format}
The theory portion of the Chemistry exam comprises 100 marks, while the practical carries 50 marks. A student's final grade for Chemistry is thus found by taking her total marks from both exams out of 150.

The theory exam for Chemistry contains 3 sections. Section A has 2 questions and is worth 20 marks - Question 1 is 10 multiple choice and Question 2 is 10 matching. Section B has 9 short answer questions, each having two items, for a total of 54 marks. Section C has 2 essay questions without items for a total of 26 marks. Students are required to answer all questions.

\paragraph{Note} This information is current as of the time of publication of this manual. Updated information may be obtained by contacting the Ministry of Education.

\subsection{Notes for Teachers}

\subsubsection{NECTA Advance Instructions}
Advance instructions are given to teachers at least one month before the date of the exam, as well as 24 hour advance instructions, to enable them to prepare apparatus, equipment and materials required for the examination.

%[tips for identifying practicals from advance instructions, example advance instructions?]

%\subsubsection{Words of Advice}

%\subsection{NECTA Marking Information}
%[how NECTA marks the practicals, highlight most important parts]

\subsection{Common Practicals}
\begin{description}
\item[Volumetric Analysis]{determine the concentration of a solution of a known chemical by reacting it with a known concentration of another solution}
\item[Qualitative Analysis]{systematically identify an unknown salt through a series of chemical tests}
\item[Chemical Kinetics and Equilibrium]{observe changes in chemical reaction rates by varying conditions such as temperature and concentration}
\end{description}

\paragraph{Note} These are the most common practicals, but they are not necessarily the only practicals that can occur on a NECTA exam. Although the updated exam format lists Questions 1 and 3 as Volumetric Analysis and Qualitative Analysis respectively, Question 2 can come from a variety of topics which may not yet have been used in older past papers. Be sure to regularly check the most recent past NECTA papers to get a good idea of the types of questions to expect. 

%==============================================================================

\section{Volumetric Analysis}
%[brief 1 paragraph explanation of practical]\\

This section contains the following:
\begin{itemize}
\item Volumetric Analysis Theory
\item Preparation of Solutions
\item Relative Standardization
\item Preparation of Solutions without a Balance
\item Substituting Chemicals in Volumetric Analysis
\item Properties of Indicators
\item Traditional Volumetric Analysis Technique
\item Volumetric Analysis without Burettes
%\item Sample Practical with Solutions
\end{itemize}

\subsection{Volumetric Analysis Theory}

Most examples of volumetric analysis involve acid-base reactions, so first is a bit of acid-base theory.

\subsubsection{Acids, Bases, and pH}

The Bronsted-Lowery definition of an acid is a substance that provides $\mathrm{H}^{+}$ to a solution while a base is a substance that removes $\mathrm{H}^{+}$ from a solution.

It is important to remember that in a water solution, $\mathrm{H}^{+}$ does not exist. Rather, $\mathrm{H}^{+}$ binds with water to form the hydronium ion, $ \mathrm{H}_3 \mathrm{O}^{+} $ .

\[ \mathrm{H}^{+} + \mathrm{H}_2 \mathrm{O} \longrightarrow \mathrm{H}_3 \mathrm{O}^{+} \]

pH is defined as the power of the hydronium ion concentration. To find the pH of a solution:

\[ \mathrm{pH} = \log{[\mathrm{H}^{+}]_{aq}} \]

Pure water has $ 10^{7} $ moles of $\mathrm{H}_3 \mathrm{O}^{+}$ per liter, or $ \mathrm{pH} = 7 $. This is because some water molecules are always reversibly reacting with each other to form hydronium and hydroxide:

\[ 2\mathrm{H}_2\mathrm{O} \longleftrightarrow \mathrm{H}_3 \mathrm{O}^{+} + \mathrm{OH}^{-} \]

Acids increase the amount of $\mathrm{H}_3 \mathrm{O}^{+}$. By increasing the concentration of hydronium ion, the power of the concentration increases to a less negative number, and thus the solution will have a smaller pH. Bases decrease the amount of H3O+ and thus basic (alkaline) solutions have pH greater than 7.

\subsubsection{Types of Acids and Bases}

\paragraph{Strong Acids}

Strong acids are acids that dissociate completely to provide $\mathrm{H}^{+}$. One can approximate the molarity of $\mathrm{H}^{+}$ (or $\mathrm{H}_3 \mathrm{O}^{+}$) as the molarity of the acid. For example, a solution of 1~M HCl has one mole of $\mathrm{H}_3 \mathrm{O}^{+}$ per liter of solution (pH 0); most of the molecules of HCl have dissociated and the $\mathrm{H}^{+}$ has reacted with water to form $\mathrm{H}_3\mathrm{O}^{+}$.

\[ \mathrm{HCl} + \mathrm{H}_2\mathrm{O} \longrightarrow \mathrm{H}_3 \mathrm{O}^{+} + \mathrm{Cl}^{-} \]

The most common strong acids are sulfuric acid ($\mathrm{H}_2\mathrm{SO}_4$), hydrochloric acid (HCl), and nitic acid ($\mathrm{HNO}_3$).

\paragraph{Weak Acids}

Weak acids, however, are reticent to contribute $\mathrm{H}^{+}$ to solution. For example, in a solution of ethanoic acid, an equilibrium forms where only one in 250 ethanoic acid molecules dissociates to form $\mathrm{H}_3 \mathrm{O}^{+}$.

\[ \mathrm{CH}_3\mathrm{COOH} + \mathrm{H}_2\mathrm{O} \longleftrightarrow \mathrm{H}_3 \mathrm{O}^{+} + \mathrm{CH}_3\mathrm{COO}^{-} \]

The most common weak acids are ethanoic acid or acetic acid ($\mathrm{CH}_3\mathrm{COOH}$), ethandioic acid or oxalic acid ($\mathrm{C}_2\mathrm{H}_2\mathrm{O}_4$), and citric acid ($\mathrm{COOHCH}_2\mathrm{COH(COOH)CH}_2\mathrm{COOH}$).

One mole of hydrochloric acid and one mole of ethanoic acid both require the same amount of base for neutralization. The difference is how the pH of the solution changes during the titration. When hydrochloric acid is titrated, the pH remains very low until right before the endpoint when it jumps to alkaline. When ethanoic acid is titrated, the pH gradually rises through a range of acidic pH's and then jumps at the endpoint. This is why methyl orange cannot be used for titrations with weak acids – see Properties and Preparation of Indicators.

\paragraph{Strong Bases}

Strong bases are bases that either dissociate completely in solution to form $\mathrm{OH}^{-}$ which reacts to remove $\mathrm{H}_3\mathrm{O}^{+}$. The common strong are sodium hydroxide, NaOH, and potassium hydroxide, KOH.

\paragraph{Weak Bases}

Weak bases form an equilibrium with water where only a few of the molecules react to remove $\mathrm{H}_3\mathrm{O}^{+}$. Common weak bases include ammonia (ammonium hydroxide), soluble carbonates, $\mathrm{CO}_3^{2-}$ and all hydrogen carbonates, $\mathrm{HCO}_3^{-}$.

Much like strong and weak acids, both strong and weak bases readily react with acids to neutralize them. As with acids, weak bases will form a buffered solution that changes pH gradually whereas strong bases will change pH abruptly when the base is neutralized fully.

\subsubsection{Volumetric Analysis}

Volumetric Analysis is a method to find the concentration (molarity) of a solution of a known chemical by comparing it with the known concentration of a solution of another chemical known to react with the first.

For example, to find the concentration of a solution of citric acid, one might use a 0.1~M solution of sodium hydroxide because sodium hydroxide is known to react with citric acid.

The most common kinds of volumetric analysis are for acid-base reactions and oxidation-reduction reactions. Acid-base reactions require use of an indicator, a chemical that changes color at a known pH. Some oxidation-reduction reactions require an indicator, often starch solution, although many are self-indicating, that is one of the chemicals itself has a color. For more about indicators, read Properties and Preparation of Indicators. For more on the specific technique of volumetric analysis, read Traditional Volumetric Analysis Technique if you have burettes and Volumetric Analysis Without Burettes if you do not.

The process of volumetric analysis is often called \textit{titration.}


\subsection{Preparation of Solutions}

For many exercises, solutions do not need to be prepared accurately. Even a 50\% error in the preparation will still allow an effective experiment. For other activities, the solutions should be prepared with a great deal of accuracy. This is especially true for volumetric analysis and conductivity experiments. This section deals with the preparation of solutions when accuracy counts.

%==============================================================================
\subsubsection{Measure the Water}

\begin{itemize}

\item{Calculate the total volume of solution you need to prepare. For example, if you are doing a practical with 100 students and each requires 150~mL of solution you should make at least 15~L of solution. Making 20~L is probably wise, to have some extra.}

\item{Find a container large enough for the total volume. Plan ahead to ensure you have a large enough container.}

\item{Add the required volume of ordinary water.}

\item{If your syllabus encourages you to often practice acid-base titrations, designate a pair of suitably large buckets as your permanent ACID and BASE buckets and label them as such with a permanent pen. Then, use a 1 liter container to add water to these buckets, one liter at a time. Use the permanent pen to mark the water height after each liter. Use these marks when adding water to make solutions. Round up the volume you need to the nearest liter (e.g. 71 students * 200 mL per student = 14.2 L, so make 15 L). As long as you use relative standardization when you finish preparing the solutions, any errors you make when measuring the volume will not affect your students' results.}

\item{Distilled water is rarely necessary. If you are preparing solutions for volumetric analysis, read the section on Relative Standardization to learn how to correct small errors caused by the composition of the tap or river water. If the water forms a precipitate when making solutions of hydroxide or carbonate, allow the precipitate to settle and decant the solution. If you are making a dilute solution, you might add hydroxide or carbonate gradually with mixing until precipitation stops and then add the amount you actually need to the liquid after decantation. If the only water supply if muddy, let the dirt settle and decant or use a cloth filter. If the particles are very fine, add a chemical like potassium aluminum sulfate (alum) or iron sulfate to precipitate the dirt. If you think that you do need distilled water, rain water is almost always sufficient.}
\end{itemize}

What comes next depends on the nature of your stock chemical. In general, there are two kinds of solutions:
\begin{itemize}
\item{Solutions prepared from solid stock chemicals, e.g. sodium hydroxide, citric acid}
\item{Solutions prepared from liquid stock chemicals, e.g. sulfuric acid}
\end{itemize}

%==============================================================================
\subsubsection{Preparing solutions from solid stock chemicals}

\begin{itemize}

\item{Calculate the amount of solid chemical required. If the instructions give the required concentration in grams per liter (e.g. $ 4 ^g/_L $ NaOH solution), multiple the total volume by the required concentration (e.g. $ 4 ^g/_L \times 10 L = 40 g $). If the instructions give the required concentration in molarity or moles per liter (e.g. 0.1 M~NaOH solution), multiple the required molarity by the molecular mass of the compound to find the required concentration in grams per liter (e.g. $ 0.1 ^{mol}/_L \times 40 ^g/_{mol} = 4 ^g/_L $). Then, multiple the required concentration by the total volume ($ 4 ^g/_L \times 10 L = 40g $).}

\item{Use a balance to weigh the solid chemical. Remember to weigh the chemical in a plastic container or on a sheet of paper and not on the scale pan directly. Some chemicals (e.g. sodium hydroxide) react with the metal pan. If you are unfamiliar with how to use a balance, read How to Use a Beam Balance. If you do not have a balance, read the section on Preparation Solutions without a Balance.}

\item{Carefully add the solid chemical to the water and stir with something unreactive (e.g. glass rod, broken burette, thick copper wire) until it has completely dissolved.}

\end{itemize}

%==============================================================================
\subsubsection{Preparing solutions from liquid stock solutions}

\begin{itemize}

\item{Calculate the amount of liquid chemical required. To do this, you need to know the molarity of your stock chemical. See the section on Calculating the Molarity of Bottled Liquids. If the instructions give the required concentration in molarity or moles per liter, use the dilution equation to calculate the amount of concentrated required:

\[ (M_{concentrated})(V_{concentrated}) = (M_{dilute})(V_{dilute}) \]

rearranging

\[ V_{dilute} = \frac{(M_{concentrated})(V_{concentrated})}{M_{dilute}} \]

For example, if you need 10~L of 0.1~M HCl and you have 12~M stock solution, the required volume of concentrated acid is

\[ V_{dilute} = \frac{(12 M)(10 L)}{0.1 M} \]

}

\item{If the instructions give the required concentration in grams per liter, divide this concentration by the molecular mass to get molarity (e.g. $ \frac {3.65 ^g/_L}{36.5 ^g/_{mol}} = 0.1 ^{mol}/_L $) and then use the dilution equation as above.}

\item{Use a DRY measuring cylinder the measure the required amount of liquid chemical. Concentrated acids may be measured in standard lab grade plastic measuring cylinders – there is no need for glass. If you do not have a measuring cylinder, you can use a plastic syringe. Be sure to use the Air Cushion Method for measuring volumes with syringes (see the section on How to Use a Plastic Syringe) – concentrated acids will rapidly corrode the rubber in the syringe on contact, causing the syringe to jam and become dangerous. Also, please read the description of Concentrated Acids in Dangerous Chemicals.}

\item{Carefully pour the liquid chemical into the container of water. Stir with something non reactive (glass rod, broken burette, thick copper wire) for about one minute.}

\end{itemize}

Then, for all volumetric analysis solutions, use the instructions in the Relative Standardization section to perfect the mole ratio of your solutions.

\subsection{Relative Standardization}

Preparing large volumes of solution is difficult with great accuracy. Relative standardization is a technique to correct the concentration of solutions so that they give the correct results for practical exercises. Note that this technique is only useful in educational situations where the purpose is to prepare a pair of solutions for titration that give an answer known by the teacher. In scientific research, the aforementioned technique -- absolute standardization -- is used because the concentration of one of the solutions is truly unknown.

All schools should use relative standardization to check the concentration of the solutions they prepare for the national examinations. This ensures that the tests measure the ability of the students to perform the practical, and not the quality of the school's balance, water supply, glassware, etc. While useful for all schools, relative standardization is particularly helpful for schools with few resources, as it allows the preparation of high quality solutions with extremely low cost apparatus and chemicals.

\subsubsection{General Theory}

The principle of a titration is that the chemical in the burette is added until it exactly neutralizes the chemical in the flask. If the two chemicals react 1:1, e.g. 

\[ \mathrm{HCl}_{(aq)} + \mathrm{NaOH}_{(aq)} \longleftarrow \mathrm{NaCl}_{(aq)} + \mathrm{H}_{2}\mathrm{O}_{(l)} \]

then exactly one mole of the burette chemical is required to neutralize one mole of the chemical in the flask. If the two chemicals react 2:1, e.g. 

\[ 2\mathrm{HCl}_{(aq)} + \mathrm{Na}_{2}\mathrm{CO}_{3(aq)} \longleftarrow 2\mathrm{NaCl}_{(aq)} + \mathrm{H}_{2}\mathrm{O}_{(l)} + \mathrm{CO}_{2(g)} \]

then exactly two moles of the burette chemical is required to neutralize one mole of the chemical in the flask. Let us think of this reaction as a mole ratio.

\[ \frac{\mbox{moles of }A}{\mbox{moles of }B} = \frac{n_{A}}{n_{B}} \]

Where $ n_{A} $ and $ n_{B} $ are the stoichiometric coefficients of A and B respectively.

\[ \mathrm{moles} = \mathrm{molarity} \times \mathrm{volume} = M \times V \mbox{ (so long as V is measured in liters)} \]

By substitution,

\[ \frac{(M_{A})(V_{A})}{(M_{B})(V_{B})} = \frac{n_{A}}{n_{B}} \]

A student performing a titration might rearrange this equation to get

\[ M_{A} = \frac{(n_{a})(M_{B})(V_{B})}{(n_{B})(V_{A})} \]

or

\[ M_{B} = \frac{(n_{B})(M_{A})(V_{A})}{(n_{A})(V_{B})} \]

As teachers, however, we care with something else: making sure that our students find the required volume in the burette. Solving the equation for $ V_{A} $ we find that

\[ V_{A} = \frac{(n_{a})(M_{B})(V_{B})}{(n_{B})(M_{A})} \]

As $ n_{A} $ and $ n_{B} $ are both set by the reaction, as long as we use the correct chemicals there is no problem here.

$ V_{B} $ is measured by the students -- it is the volume they transfer into the flask. As long as the students know how to use plastic syringes accurately, they should get this value almost perfectly correct.

The remaining term, $ \frac{M_{B}}{M_{A}} $ is for the teacher, not the student, to make correct. If we prepare the solutions poorly, our students can do everything right but still get the wrong value for $ V_{A} $. It is very important that we ensure that our solutions have the correct ratio of $ \frac{M_{B}}{M_{A}} $ so that the exercise properly assesses the ability of our students.

Many people look at this ratio and decide that they therefore need to prepare both solutions perfectly, so that $ M_{B} $ and $ M_{A} $ are exactly what is required. This not true. The actual values for $ M_{B} $ and $ M_{A} $ are not important; what matters is the ratio $ M_{B} $ to $ M_{A} $!

For example, if the titration requires 0.10~M HCl and 0.10~M NaOH, our expected mole ratio is:

\[ \frac{M_{HCl}}{M_{NaOH}} = \frac{0.10}{0.10} = 1 \]

Preparing 0.11 M HCl and 0.09 M NaOH will cause the students to get the wrong answer:

\[ \frac{M_{HCl}}{M_{NaOH}} = \frac{0.11}{0.09} = 1.22 \]

However, preparing exactly 0.05 M HCl and 0.05 M NaOH results in the same molar ratio:

\[ \frac{M_{HCl}}{M_{NaOH}} = \frac{0.05}{0.05} = 1 \]

Thus the students can get exactly the right answer if they use the right technique even though neither solution was actually the correct concentration.

How can we ensure that we have the correct molar ratio between our solutions? Titrate your solutions against each other. If the volume is not the expected value, one of your solutions is too concentrated relative to the other. You can calculate exactly how much too concentrated and add the exact amount of water necessary to perfect the ratio. This process is called relative standardization, because you are standardizing one solution relative to the other.

\subsubsection{Procedure for Relative Standardization}

In some titrations the acid is in the burette and in some it is the base is in the burette. So let us not use ``acid'' and ``base'' to refer to the solutions, but rather ``solution 1'' and ``solution 2'' where solution 1 is the solution measured in the burette and solution 2 is measured by pipette (syringe).

You should have prepared a bucket or so of each. The volume you have prepared is $ V_{1} $ liters of solution 1 and $ V_{2} $ liters of solution 2.

Titrate the solutions against each other. Call the volume you measure in the burette ``actual titration volume'' You know the desired molarity of each solution, so from the above student equations you can calculate the burette volume you expect, which you might call ``theoretical titration volume.''

After the titration, there are three possibilities. If the actual titration volume equals the theoretical titration volume, your solutions are perfect. Well done.

If the actual titration volume is smaller than the theoretical titration volume, solution 1 is too concentrated and must be diluted. Use the ratio:

\[ \frac{V_{1} \mbox{ (before dilution)}}{V_{1} \mbox{ (after dilution)}} = \frac{\mbox{actual titration volume}}{\mbox{theoretical titration volume}} \]

If the actual titration volume is larger than the theoretical titration volume, solution 2 is too concentrated and must be diluted. Use the ratio:

\[ \frac{V_{2} \mbox{ (before dilution)}}{V_{2} \mbox{ (after dilution)}} = \frac{\mbox{theoretical titration volume}}{\mbox{actual titration volume}} \]

After diluting one of your solutions, repeat the process. After a few cycles, the solutions should be perfect. Remember that the volume ``before dilution'' is the volume actually in the bucket, so the amount you made less the amount used for these test titrations.

\subsection{Preparation of Solutions without a Balance}

The procedure in the section on Relative Standardization allows us to do something seemingly impossible – prepare solutions for volumetric analysis that allow students to get perfect results without using either a balance or volumetric glassware in the preparation. All that you have to do is make two solutions that are close, and then use several cycles of relative standardization to prefect the molarity ratio. 

To measure volume, we can use marks on plastic water bottles as described in the entry for volumetric glassware in the Sources of Equipment section. What follows is an example of how rough solutions can be prepared in Tanzania based on the water bottles available in that country. We encourage people in other countries to calibrate their water bottles and then to customize these instructions for the resources available to them.

\subsubsection{To make 0.05~M sulfuric acid (equivalent to 0.1~M HCl) for fifty students}
\begin{enumerate}
\item{Put 9.9 liters of water into a bucket. On the new 1.5~L Kilimanjaro water bottle, the bottom points of the crown embossed on the side correspond to 300~ml and the top of the mountain corresponds to 1.5~L. Therefore one can measure 9.9 liters by filling the bottle to the mountain top six times and then to the bottom points of the crown three times.}
\item{Add 110~mL of battery acid. This may be accomplished easily by filling a 10~mL plastic syringe eleven times. Please read the safety note in Dangerous Chemicals.}
\end{enumerate}

\subsubsection{To make 0.033~M citric acid (equivalent to 0.1~M HCl) for fifty students}
\begin{enumerate}
\item{Put 10 liters of water into a bucket. One the new 500~mL Kilimanjaro water bottle, the second straight line corresponds to 300~mL and the highest straight line corresponds to 400~mL. Therefore one can use the 1.5~L bottle six times to add nine liters and then use the 500~mL bottle to add one more liter, 400~mL + 300~mL + 300~mL.}
\item{Add 64~g of citric acid. In the absence of a balance, one can often have $^1/_8$ of a kilogram (125~g) measured in the market. Dissolved this in 20~L of water to produce a 0.033~M solution. Alternately, use a plastic syringe to find the volume of a plastic spoon. Fill the spoon with citric acid and push off any extra acid until there is a flat surface (like the water). Then use that spoon to add a total $38 \mathrm{cm}^3$ or mL of citric acid soda knowing the volume of each spoonful.}
\end{enumerate}

\subsubsection{To make 0.1~M sodium hydroxide for fifty students}
\begin{enumerate}
\item{Put 10 liters of water into a bucket. See the instructions above.}
\item{Add 40~g of caustic soda. In the absence of a balance, measure the volume of a spoon as above and add $19 \mathrm{cm}^3 \mathrm{or mL}$ of caustic soda. Please read the safety note in Dangerous Chemicals.}
\end{enumerate}

\subsubsection{To make 0.1~M sodium hydrogen carbonate for fifty students}
\begin{enumerate}
\item{Put 10 liters of water into a bucket. See the instructions above.}
\item{Add 84~g of bicarbonate of soda. In the absence of a balance, find the volume of a spoon as above and add $39 \mathrm{cm}^3 \mathrm{or mL}$ of bicarbonate of soda. Alternately, if 8.33 liters of solution is sufficient, measure this volume of water and then add one whole box of bicarbonate of soda. A box is 70~g.}
\end{enumerate}

\subsection{Substituting Chemicals in Volumetric Analysis}
\label{cha:subchemvolana}
\subsubsection{Theory}

The volumetric analysis practical exercises sometimes call for expensive chemicals, for example potassium hydroxide or oxalic acid. As the purpose of exercises and exams is to train or test the ability of the students and not the resources of the school, it is possible to use different chemicals as long as the solutions are calibrated to give equivalent results. For example, if the instructions call for a potassium hydroxide solution, you can use sodium hydroxide to prepare this solution. It will not affect the results of the practical -- if you make the correct calibration. How to calibrate solutions when substituting chemicals is the subject of this section.

Technically, only two chemicals are required to perform any volumetric analysis practical: one strong acid and one strong base. The least expensive options are sulfuric acid, as battery acid, and sodium hydroxide, as caustic soda. To substitute one chemical for another in volumetric analysis, the resulting solution must have the same normality (N).

\begin{itemize}

\item{For all monoprotic acids (HCl, ethanoic acid), the normality is the molarity.\\
\textit{Example: 0.1~M ethanoic acid = 0.1~N ethanoic acid}}
\item{For diprotic acids (sulfuric acid, ethandiotic acid), the normality is twice the molarity, because each molecule of diprotic acid brings two molecules of $\mathrm{H}^{+}$.\\
\textit{Example: 0.5~M sulfuric acid = 1.0~N sulfuric acid}}
\item{For the hydroxides and hydrogen carbonates used in ordinary level (NaOH, KOH, NaHCO$_{3}$), the normality is the molarity.\\
\textit{Example: 0.08~M KOH = 0.08~N KOH}}
\item{For the carbonates most commonly used ($\mathrm{Na}_2\mathrm{CO}_3$, $\mathrm{Na}_2\mathrm{CO}_3 /dot 10\mathrm{H}_2\mathrm{O}$, $\mathrm{K}_2\mathrm{CO}_3$), the normality is twice the molarity.\\
\textit{Example: $0.4 M \mathrm{Na}_2\mathrm{CO}_3 = 0.8 N \mathrm{Na}_2\mathrm{CO}_3$}}

\end{itemize}

\subsubsection{Substitution Calculations}

When instructions describe solutions in terms of molarity, calculating the molarity of the substitution is relatively simple. For example, suppose we want to use sulfuric acid to make a 0.2~M solution of ethanoic acid. 0.2~M ethanoic acid is 0.2~N ethanoic acid which will titrate the same as 0.2~N sulfuric acid. 0.2 N sulfuric acid is 0.1~M sulfuric acid, and thus we need to prepare 0.1~M sulfuric acid.

When instructions describe solutions in terms of concentration ($^g/_L$), we just need to add an extra conversion step. For example, suppose we want to use sodium hydroxide to make a $14.3 ^g/_L$ solution of sodium carbonate decahydrate. $14.3 ^g/_L$ sodium carboante decahydrate is 0.05~M sodium carbonate decahydrate which is 0.1~N sodium carbonate decahydrate. This will titrate the same as 0.1~N sodium hydroxide, which is 0.1~M sodium hydroxide or $4 ^g/_L$ sodium hydroxide, and thus we need to prepare $4 ^g/_L$ sodium hydroxide to have a solution that will titrate identically to $14.3 ^g/_L$ sodium carbonate decahydrate.

\subsubsection{Common Substitutions}
\label{sec:commonsubs}
To simplify future calculations, we have prepared general conversions for the most common chemicals used in volumetric analysis. Remember to check all final solutions with relative standardization to ensure that they indeed give the correct results.

\newgeometry{margin=1cm}
\begin{landscape}
\thispagestyle{empty}

\begin{table}
\centering

%\begin{center}
\begin{tabular}{| p{3.5cm} | p{4cm} | p{8cm} | p{4cm} | p{4cm} |}
\hline

\multicolumn{1}{|c}{\textbf{Required Chemical}} & 
\multicolumn{1}{|c}{\textbf{Low Cost Alternative}} & 
\multicolumn{1}{|c}{\textbf{Substiution Method}} & 
\multicolumn{1}{|c}{\textbf{Molarity Example}} & 
\multicolumn{1}{|c|}{\textbf{Concentration Example}} \\ \hline

Hydrochloric Acid & 
Sulfuric Acid (Battery Acid) & 
If you are required to prepare an X molarity solution of HCl, prepane a X$\times 0.5$ molarity solution of battery acid & 
The instructions call for 0.12~M HCl. Instead, prepare 0.06~M sulfuric acid & 
 \\ \hline

Ethanoic (Acetic) Acid & 
Sulfuric Acid (Battery Acid) & 
If you are required to prepare an M molarity solution of ethanoic acid, prepare a M$\times 0.5$ molarity solution of sulfuric acid & 
The instructions call for 0.10~M ethanoic acid. Prepare 0.05~M sulfuric acid. & 
 \\ \hline

Ethandioic (Oxalic) Acid dihydrate (C$_{2}$H$_{2}$O$_{4} \cdot$2H$_{2}$O) & 
Sulfuric Acid (Battery Acid) & 
If you are required to prepare an M molarity solution of ethandioic acid, prepare an M molarity solution of sulfuric acid. If you are required to prepare a C concentration solution of ethandioic acid, prepare a $^\text{C}/_{126}$ molarity solution of sulfuric acid. & 
The instructions call for 0.075~M ethandioic acid. Prepare 0.075~M sulfuric acid. & 
The instructions call for 6.3 $^\text{g}/_\text{L}$ ethandioic acid. Prepare 0.05~M sulfuric acid. \\ \hline

Potassium Hydroxide & 
Sodium Hydroxide (Caustic Soda) & 
For M molarity potassium hydroxide, make M molarity sodium hydroxide. For C concentration potassium hydroxide, make C$\times ^{40}/_{56}$ concentration sodium hydroxide. & 
The instructions call for 0.1~M potassium hydroxide. Just prepare 0.1~M sodium hydroxide. &
The instructions call for 2.8 $^\text{g}/_\text{L}$ potassium hydroxide. Prepare 2 $^\text{g}/_\text{L}$ sodium hydroxide. \\ \hline

Anhydrous Sodium Carbonate & 
Sodium Carbonate Decahydrate (Soda Ash) &  
For M molarity anhydrous sodium carbonate, make M molarity sodium carbonate decahydrate. For C concentration anhydrous sodium carbonate, make C$\times ^{286}/_{106}$ sodium carbonate decahydrate. & 
The instructions call for 0.09~M anhydrous sodium carbonate. Make 0.09~M sodium carbonate decahyrate. & 
The instructions call for 5.3 $^\text{g}/_\text{L}$ anhydrous sodium carbonate. Make 14.3 $^\text{g}/_\text{L}$ sodium carbonate decahydrate. \\ \hline

Anhydrous Sodium Carbonate & 
Sodium Hydroxide (caustic soda) & 
For M molarity anhydrous sodium carbonate, make M$\times 2$ molarity sodium hydroxide. For C concentration anhydrous sodium carbonate, make C$\times 2 \times ^{40}/_{106}$ sodium hydroxide. & 
The instructions call for 0.09~M anhydrous sodium carbonate. Make 0.18~M sodium hydroxide. & 
The instructions call for 5.3 $^\text{g}/_\text{L}$ anhydrous sodium carbonate. 4.0 $^\text{g}/_\text{L}$ sodium hydroxide. \\ \hline

Sodium Carbonate Decahydrate (Na$_2$CO$_3\cdot$10H$_2$O) &
sodium hydroxide (caustic soda) &
For M molarity sodium carbonate ecahydrate, make M$\times 2$ molarity sodium hydroxide. For C concentration sodium carbonate decahydrate, make C$\times 2 \times ^{40}/_{286}$ sodium hydroxide. &
The instructions call for 0.09~M sodium carbonate decahydrate. Make 0.18~M sodium hydroxide. &
The instructions call for 14.3 $^\text{g}/_\text{L}$ sodium carbonate decahydrate. Make 4.0 $^\text{g}/_\text{L}$ sodium hydroxide. \\ \hline

Anhydrous Potassium Carbonate & 
Sodium Carbonate decahydrate (Soda Ash) & 
For M molarity potassium carbonate, make M molarity sodium carbonate decahydrate. For C concentration potassium carbonate, make C$\times ^{286}/_{122}$ concentration sodium carbonate. & 
The instructions call for 0.08~M anhydrous potassium carbonate. Prepare 0.08~M sodium carbonate decahydrate. & 
The instructions call for 6.1 $^\text{g}/_\text{L}$ anhydrous potassium carbonate. Prepare 14.3 $^\text{g}/_\text{L}$ sodium carbonate decahydrate. \\ \hline

Anhydrous Potassium Carbonate & 
Sodium Hydroxide (caustic soda) & 
For M molarity potassium carbonate, make M$\times 2$ molarity sodium hydroxide. For C concentration potassium carbonate, make C$\times 2 \times ^{40}/_{122}$ concentration sodium hydroxide. & 
The instructions call for 0.08~M anhydrous potassium carbonate. Prepare 0.16~M sodium hydroxide. & 
The instructions call for 6.1 $^\text{g}/_\text{L}$ anhydrous potassium carbonate. Prepare 4.0 $^\text{g}/_\text{L}$ sodium hydroxide. \\ \hline

\end{tabular}
%\end{center}

\end{table}
\end{landscape}
\restoregeometry

\subsubsection{Additional Notes}

\begin{itemize}

\item{In volumetric analysis experiments with two indicators, it is not possible to substitute one chemical for another as the acid/base dissociation constant is critical and specific for each chemical. It is still possible to substitute sodium carbonate decahydrate for anhydrous sodium carbonate with the above conversion.}

\item{These substitutions only work for volumetric analysis. In qualitative analysis, the nature of the chemical matters. If the instructions call for sodium carbonate, you cannot provide sodium hydroxide and expect the students to get the right answer!}

\end{itemize}

\subsection{Properties of Indicators}

\subsubsection{Acid-base Indicators}\label{sss:acid-baseind}
These indicators are chemicals that change colors in a specific pH range, which makes them suited to use in acid-base reactions. When the pH of changes from low pH to high pH or from high to low, the color of the solution changes. 

Four common acid-base indicators are methyl orange (MO), phenolphthalein (POP), bromothymol blue (BB), and universal indicator (U)

\begin{itemize}

\item{Methyl Orange, MO, is always used when titrating a strong acid against a weak base. The pH range of MO is 4.0-6.0 and thus no color change is observed until the base is completely neutralized. If you use MO with a weak acid, the color might start to change before completely neutralizing the acid.}

\item{Phenolphthalein, POP, is always used when titrating a weak acid against a strong base. The pH range of POP is 8.3-10.0, and thus no color change is observed until the weak acid is completely neutralized. If you use POP with a weak base, the color might start to change before completely neutralizing the base.}

\item{Bromothymol Blue, BB, is used in the same manner as methyl orange.}

\item{Universal indicator, U, is not suitable for volumetric analysis involving either weak acids or bases as it changes color continuously rather than in a limited pH range. It is very useful for tracking the pH continuously over a titration, perhaps by performing two titrations side by side, one with a standard indicator and another with universal indicator.}

\end{itemize}

Any indicator can be used when titrating a strong acid against a strong base. Universal indicator, however, will not produce very accurate results.

No indicator is suitable for titrating a weak acid against a weak base.

In some experiments, more than one indicator may be used in the same flask, for example when titrating a mixture of strong and weak acids or bases.

\paragraph{Colors of Indicators}
The colors of the above indicators in acid and base are:

\begin{center}
\begin{tabular}{| c | c | c | c |}
\hline
Indicator & Acid & Neutral & Base \\ \hline
Methyl Orange & Red & Orange & Yellow \\ \hline
Phenolphthalein & Colorless & Colorless & Pink \\ \hline
Bromothymol Blue & Yellow & Blue & Blue \\ \hline
Universal Indicator & Red, Orange, Yellow & Yellow/Green & Green, Blue, Indigo  \\
\hline
\end{tabular}
\end{center}

Titration is finished when the indicator starts a permanent color change. For example, when methyl orange turns orange, the titration is finished. If students wait until methyl orange turns pink (or yellow) they have overshot the endpoint of the titration, and their volume will be incorrect. Likewise, POP indicates that the titration is finished when it turns light pink. If students wait until they have an intensely pink solution, they will use too much base and get the wrong answer. 

Note that light pink POP solutions may turn colorless if left for a few minutes. This is due to carbon dioxide in the air reacting to neutralize bases in solution.

\paragraph{Note on technique}
Students should use as little acid-base indicator as possible. This is because some acid or base is required to react with the indicator so that it changes color. If a lot of indicator is used, students will add more acid or base than they need.

\subsubsection{Other Indicators}
Starch indicator is used in oxidation-reduction titrations involving iodine. This is because iodine forms an intense blue to black colored complex in the presence of starch. Thus starch allows a very sensitive assessment of the presence of iodine in a solution.

It is important to add the starch indicator close to the end point when there is an acid present. The acid will cleave the starch and that will prevent the starch from working properly. Students using starch should use a pilot run to get an idea when to add the starch indicator.

\subsubsection{Preparation of Indicators}
\begin{itemize}

\item{Methyl orange (MO): if you have a balance, weigh out about 1~g of methyl orange powder and dissolve it in about 1~L of water. Store the solution in a 
plastic water bottle with a screw on cap and it will keep for years. If it gets thick and cloudy, add a bit more water and shake. If you do not have a balance, add half of a small tea spoon to a liter of water.}

\item{Phenolphthalein (POP): Dissolve about 0.2~g of phenolphthalein powder in 100~mL of pure ethanol; then add 100~mL water with constant stirring. If you use 
much more water than ethanol, solid phenolphthalein will precipitate. Store POP in a plastic water bottle with a screw on cap. We recommend making POP in smaller quantities than MO as it does not keep as well, mostly due to the evaporation of ethanol. If the solution develops a precipitate, add a bit of ethanol and shake. We do not recommend using purple methylated spirits as a source of ethanol for making POP. You can distill purple spirits to make clear spirits. For clear methylated spirits, use 140ml of spirit and 60ml of water, as spirits generally are already 30\% water.}

\item{Starch: place about 1~g of starch in 10~mL of water in a test tube. Mix well. Pour this suspension into 100~mL of boiling water and continue to boil for 
one minute or so. Alternatively, use the water leftover after boiling pasta or potatoes. If this is too concentrated, dilute it with regular water.}

\item{The authors have never prepared bromothymol blue or universal indicator from powder, but suspect their preparation is similar to methyl orange.}

\end{itemize}

Note that the exact mass of indicator used is not very important. You just need to use enough so that the color is clearly visible. Students use very little indicator in each titration, and a liter of indicator solution should last you a long time.

\subsection{Traditional Volumetric Analysis Technique}
\label{cha:volanatech}

\subsubsection{Burettes}

In most acid-base titrations, the acid comes from the burette, although sometimes the burette holds the base. Prior to use, the student should thoroughly wash the burette to remove any residue from previous use. Then, the student should close the stopcock and add about 5ml of the solution that they will use in the burette. With their thumb over the open end of the burette they should make sure the solution covers every surface of the burette. They should then run this solution out into a waste container. This step is to replace the residue of water from the first washing with a layer of the titration solution. If students do not perform this step, the water reside will dilute their titration solution.

Most burettes have a volume of 50 mL. The 0 mL mark is at the top, and the 50 mL mark is at the bottom. This is because the burette tells you the volume of solution used, not the volume of solution present. If you start at 0 mL, and finish at 20 mL, then you have used 20 mL of acid.

Many burettes do not have stopcocks. Instead, they have a piece of rubber tubing at the bottom, which has a glass tip inserted into it. Either a metal clip is used to hold the rubber tubing closed or there is a small bead in the tubing around which fluid may pass when the tube is squeezed at that point. Broken burettes can often be repaired; see the section on Repairing Burettes.

\subsubsection{Reading Measurements}

\begin{itemize}

\item{Always read burettes at eye-level. If the burette is clamped to a stand, remove it from the stand so you can hold it at eye-level. Or move the stand.}

\item{Always read from the bottom of the meniscus. Students often forget this; it helps to remind them at the beginning of a practical. In plastic apparatus, there is often no meniscus.}

\item{Burettes are accurate to 2 decimal places. Many times, students are taught that the last number should be either 5 or 0, like 15.55 or 15.50. This is incorrect – students should estimate the fluid level in the burette to the nearest 0.01 mL.}

\end{itemize}

\subsubsection{Titration Procedure}

\begin{itemize}

\item{Clean the burette with water. Then rinse it with the solution you will be using for titration.}

\item{Fill the burette with the solution. Allow a little solution to run out of the tip until the top of the fluid is at either 0.00 mL exactly or any value below. An initial volume of 1.32 mL is completely acceptable, at least from a scientific point of view. Your country may have specific expectations for marking exams.}

\item{Record the initial burette reading.}

\item{Use a syringe to transfer the other solution into a conical flask. Record the volume moved by the syringe.}

\item{If you are using indicator, add a few drops to the conical flask. For acid-base indicators, the less indicator used the better. In order to change color the indicator itself must react with some of the fluid from the burette. This consumes more chemical than is technically needed for neutralization; the additional chemicals required for titrating the indicator is called indicator error. One or two drops is really all you need. For starch indicator, use about 1 mL. The starch is not titrated, unlike acid-base indicators, so you can use more and often must to get a good color.}

\item{Slowly add solution from the burette to the conical flask. As you titrate, swirl the flask to mix. Do not shake it back and forth, because the solution in the flask will splatter onto the sides of the flask and thus will not be part of the neutralization reaction. Much the same, be careful to add the drops from the burette so they fall into the solution and are not stuck on the side of the flask. Stop titration when the indicator starts a permanent, slight color change. This is the endpoint. Again, the slightest change in color to the appropriate color indicates the endpoint, as long as the color remains after a few swirls.}

\item{Record the final burette reading.}

\end{itemize}

Titration is often done four times: a pilot followed by three trials. The purpose of the pilot is to find the approximate volume from the burette. The pilot is done quickly, and often overshoots the endpoint. In subsequent titration, use the results of the pilot to avoid overshooting while speeding up the work. For example, if the pilot gave an endpoint of 26 mL, add your volume rapidly from the burette until about 20 mL. Then add drop by drop until you find the endpoint.

The result from the pilot is not considered in calculations, as it is not expected to be accurate. Do not include it when finding the average volume or the variance.

\subsection{Volumetric Analysis without Burettes}

\subsubsection{Theory}

Burettes are not necessary to perform volumetric analysis with reasonable precision. Students may use plastic syringes in place of burettes. These should be the most precise syringes available, which as of late 2010 were the 10~mL NeoJect brand plastic syringes. These syringes are more accurate than the low cost glass pipettes that many schools purchase. As the accuracy of the titration is no better than its least accurate instrument, a titration with two plastic syringes is more accurate than a titration with a burette and a cheap glass pipette.

If use of these syringes is new to you, please read Use of Plastic Syringes before proceeding.

To get maximum precision from plastic syringes, students should learn how to estimate values between the lines on the syringe body. The NeoJect syringes are marked with lines every 0.2~mL. Students should observe the top of the fluid and decide if it is on the line exactly, half way in between, or in between half way and one of the lines. This allows them to divide the space between lines into four parts, giving them a precision of 0.05~mL. Estimation between gradations is standard practice with scientific instruments; even students using burettes should estimate the fluid height between the lines to at least 0.05~mL. Syringes have the capacity to deliver the precision required by most if not all  national exams.

If students are using syringes in place of burettes, they require two syringes for the practical, one as a burette and a different one as the pipette. We recommend that you label the syringes, for example, on one syringe writing ‘Burette’ with a permanent pen to help students remember which is which.

\subsubsection{Titration Procedure without Burettes}

\begin{enumerate}

\item{Clean the ‘pipette’ syringe with water. Then rinse it with the acid or base solution you will be putting in the flask.}

\item{Use a syringe to transfer the required amount of acid or base to the flask. To do this transfer accurately, add first 1 mL of air to the syringe and then suck up the fluid to beyond the desired amount. Push back the plunger until the top of the fluid is exactly the volume required. Delivering the required volume to the flask may take multiple transfers with the single syringe. Record the total volume transferred to the flask as the ‘volume of pipette used’}

\item{Add one or two drops of indicator to the flask.}

\item{Clean the ‘burette’ syringe with water. Then rinse it with the acid or base solution you will be using to titrate.}

\item{Add 1~mL of air to the syringe and then suck up the acid or base solution to beyond the 10~mL mark. Slowly push back the plunger until the top of the fluid is exactly at the 10 mL line.}

\item{Slowly add the solution from the syringe to the flask. As you titrate, swirl the flask to mix. As described above, swirl instead of shaking to keep all of the liquid together. Make sure that each drop from the syringe hits the liquid rather than getting suck on the edge of the container. Stop titration when the indicator starts a permanent color change. Just as with a burette, this is the endpoint.}

\item{Often the volume required from the ‘burette’ is greater than 10~mL. This is no problem – after finishing the syringe students should simply fill it again as they did the first time and continue. On their rough paper (scratch paper), they should note that they have already consumed 10~mL.}

\end{enumerate}

\subsubsection{Table of Results when using syringes in place of burettes}

At present, many national exam marking boards expect students to use burettes. The obvious problem is that while the top line on a burette is 0~mL, the top of the syringe reads 10~mL. For students to get the marks their careful technique deserves, they must record their results in a manner consistent with traditional reporting. On rough paper, students should calculate the volume of solution used in their titration. This is easy -- if the syringe started at 10.00 mL and ended at 2.55~mL, the student used $10.00 \mathrm{mL} - 2.55 \mathrm{mL} = 7.45 \mathrm{mL}$ of solution. If the student used two full syringes and the third finished at 4.65~mL, then the student used $10.00 \mathrm{mL} - 4.65 \mathrm{mL} = 5.35 \mathrm{mL}$ in the last syringe plus 10~mL in each of the first two syringes, so $5.35 \mathrm{mL} + 10 \mathrm{mL} + 10 \mathrm{mL} = 25.35 \mathrm{mL}$ total.

In the Table of Results, the student should then write 25.35~mL for the Volume Used. If this volume had been used in a burette, the student would have found an initial volume of 0.00~mL and a final volume of 25.35~mL. The rest of the table should be filled in this manner. When using a syringe as a burette, the student should always write 0.00~mL for the Initial Volume and then for Final Volume they should write the total number they calculated for Volume Used. This method will ensure that the students gets the marks he or she deserves for careful titration – and likewise ensure that he or she loses the appropriate marks for mistakes.

\subsection{Sample Practical Question}
The following is a sample practical question from 2012.\\


\noindent You are provided with the following solution:\\

\noindent \textbf{TZ}: Containing 3.5 g of impure sulphuric acid in 500 cm$^3$ of solution;\\
\textbf{LO}: Containing 4 g of sodium hydroxide in 1000 cm$^3$ of solution;\\
Phenolphthalein and Methyl indicators.\\

\textbf{Questions}:\\
\begin{enumerate}
\item[(a)]
\begin{enumerate}
\item[(i)] What is the suitable indicator for the titration of the given solutions?\\
Give a reason for your answer.
\item[(ii)] Write a balanced chemical equation for the reaction between \textbf{TZ} and \textbf{LO}.
\item[(iii)] Why is it important to swirl or shake the contents of the flask during the addition of the acid?\\
\end{enumerate}

\item[(b)] Titrate the acid (in a burette) against the base (in a conical flask) using two drops of your indicator and obtain three titre values.\\

\item[(c)] 
\begin{enumerate}
\item[(i)] \_\_\_\_ cm$^3$ of acid required \_\_\_\_ cm$^3$ of base for complete reaction.
\item[(ii)] Showing your procedures clearly, calculate the percentage purity of \textbf{TZ}.
\end{enumerate}
\end{enumerate}


\raggedleft \textbf{(20 marks)}

\raggedright

\subsubsection{Discussion}
This practical question requires students to know and understand how to use volumetric analysis apparatus and technique. Since this question involves the titration of sulphuric acid (strong acid) and sodium hydroxide (strong base), either phenolphthalein or methyl orange are acceptable indicators to use. An explanation of suitable indicators can be found in \nameref{sss:acid-baseind}.

Make sure that students create a table for the first pilot titration and three titre values, for a total of four titrations. Only the titre values (not the pilot) that are within $\pm$0.02 ml of each other will be used to calculate the average titrated volume. Students should also be swirling the contents of the volumetric flask in order to thoroughly mix the acid and base together. The titration is complete only when there is permanent color change in the indicator.

Note that although this procedure states the number of drops of indicator and how many number of titre values, it does not indicate what volume to use in the flask. The typical volume is 25 ml, but students can use any volume as long as they are consistent for each trial.

The practical question for volumetric analysis will always ask students to either determine percentage purity, molecules of crystallization of water, unknown concentration of one of the solutions, or molar mass of one of the solutions. %See ----- for more explanation on volumetric analysis calculations.


%\subsection{Sample Practical Solutions}


%==============================================================================
%==============================================================================

\section{Qualitative Analysis}
\label{cha:qualana}
%[brief 1 paragraph explanation of practical]\\

This section contains the following:
\begin{itemize}
\item Overview of Qualitative Analysis
\item Teaching Qualitative Analysis with Local and Low Cost Materials
\item The Steps of Qualitative Analysis
\item Hazards and Cleanliness
\item Sample Practical: Preparation of Copper Carbonate for Qualitative Analysis
\end{itemize}

%==============================================================================

\subsection{Overview of Qualitative Analysis}

The salts requiring identification have one cation and one anion. Generally, these are identified separately although often knowing one helps interpret the results of tests for the other. For ordinary level in Tanzania, students are confronted with binary salts made from the following ions:

\begin{itemize}
\item{Cations: NH$_{4}^{+}$, 
Ca$^{2+}$, 
Fe$^{2+}$, 
Fe$^{3+}$, 
Cu$^{2+}$, 
Zn$^{2+}$, 
Pb$^{2+}$, 
Na$^{+}$}
\item{Anions: CO$_{3}^{2-}$, 
HCO$_{3}^{-}$, 
NO$_{3}^{-}$, 
SO$_{4}^{2-}$, 
Cl$^{-}$}
\end{itemize}
At present, 
ordinary level students receive only one salt at a time. The teacher may also make use of qualitative analysis to identify unlabeled salts.

The ions are identified by following a series of ten steps, divided into three stages. These are:
\begin{itemize}
\item{Preliminary tests:
These tests use the solid salt. They are: appearance, action of heat, action of dilute H$_{2}$SO$_{4}$, action of concentrated H$_{2}$SO$_{4}$, flame test, and solubility.}
\item{Tests in solution: The compound should be dissolved in water before carrying out these tests. If it is not soluble in water, use dilute acid (ideally \ce{HNO3}) to dissolve the compound. The tests in solution involve addition of NaOH and NH$_{3}$.}
\item{Confirmatory tests: These tests confirm the conclusions students draw from the previous steps. By the time your students start the confirmatory tests, they should have a good idea which cation and which anion are present. Have students do one confirmatory test for the cation they believe is present, and one for the anion you believe is present. Even if several confirmatory tests are listed, students only need to do one. When identifying an unlabelled container, however, you might be moved to try several, especially if you are new to this process.}
\end{itemize}

%==============================================================================

\subsection{Teaching Qualitative Analysis with Local and Low Cost Materials}

\subsubsection{General Suggestions}
\begin{itemize}
\item{Heat sources: Motopoa burners cost nothing to make (soda bottle caps) and consume only a small amount of fuel. They give a non-luminous flame ideal for flame tests and still produce enough heat for the other tests.}
\item{Test tubes: Most of the tests do not involve heating, so students may perform these experiments in plastic tubes made from disposable plastic syringes. Many of the tests requiring heating use the salt in solution and thus can be performed by holding the plastic test tube in a hot water bath. For the Action of Heat test, salts may be heating in metal spoons to observe residue products, although it is difficult to test the gases produced. Do not wait for test tubes to start teaching qualitative analysis. Do try to find at least one borosilicate (Pyrex, 
Borosil) test tube for each student before the national exams.}
\item{Litmus paper: Make your own. See the instructions in chapter on Acids and Bases. Rosella flowers give very good results.}
\item{Low cost sources of chemicals. Many chemicals have low cost alternatives, for example table salt (sodium chloride), gypsum powder (calcium sulphate), ammonium sulphate (sulphate of ammonia fertilizer), cautic soda (sodium hydroxide), soda ash (sodium carbonate), battery acid (sulphuric acid), and copper (II) sulphate (the local medicine mlutulutu). Other chemicals can be manufactured locally in small quantities, for example iron (II) sulphate, iron (III) sulphate, calcium carbonate, copper carbonate, zinc sulphate, and zinc carbonate. Indeed the preparations several of these compounds are described in the chapter on Compounds of Metals. For more information about any specific chemical, read its entry in the Sources of Chemicals section.}
\item{Share expensive chemicals among many schools. A single container of potassium ferrocyanide, for example, can supply ten or even twenty schools for several years. Schools should consider bartering 10~g of one chemical for 10~g of another. Schools without any expensive chemicals could produce Benedict's solution from local materials, for example, and exchange this for 10~g samples of expensive salts. Another alternative is for all of the schools in a district or town to pool money to buy one container of each required imported reagent, and then divide the chemicals evenly. Remember to keep chemicals in air-tight containers and out of sunlight. Also remember to label containers very clearly.}
\end{itemize} 

%===================
\subsubsection{Timeline of Lessons}

To teach qualitative analysis, first make sure the students know how to use the apparatus: test tubes, droppers, and motopoa burners. All of these apparatus are used in other experiments throughout this book.

Once the students are familiar with the apparatus, teach them each step separately, using the activities outlined below. Give adequate time to each step; each one can be used to review chemistry learned in previous topics.

Once the students are proficient at the individual steps, practice the whole process with example unknown salts. Sodium carbonate and locally manufactured copper (II) carbonate are good options for practice.

Finally, as the exam approaches, get some lead nitrate, a small amount of fully concentrated sulphuric acid, and some borosilicate test tubes. Use these materials to teach:
\begin{description}
\item[Conformation of sulphates by addition of lead nitrate solution]{Prepare this solution by dissolving about one level tea spoon of lead nitrate in about 100~mL of distilled (rain) water. Use only 2-3 \textit{drops} to confirm sulphates.}
\item[Thermal decomposition of nitrates to form nitrogen dioxide]{Nitrogen dioxide is a poisonous brown gas. Add a very small amount of lead nitrate to a borosilicate test tube and head strongly over a motopoa burner.}
\item[Flame test for lead]{See the instructions for flame tests below. Only a very small amount of lead nitrate is required for the test.}
\item[Conformation of nitrates by brown ring test]{Add a very small amount of lead nitrate to a test tube and dissolve in 2~mL of distilled (rain) water. In a separate test tube prepare about 1~mL of iron (II) sulphate solution from locally manufactured iron (II) sulphate (make sure it is still light green and not yellow!) in distilled water. Mix the solutions and note that lead sulphate will precipitate. Use this to teach the conformation of lead by precipitation with sulphate. Note that the nitrate remains in solution. Decant the liquid into a borosilicate test tube. Hold the test tube at an angle and carefully add about 1~mL of fully concentrated sulphuric acid down the inside. The acid will form a separate layer at the bottom. If nitrates are present, in a few minutes a brown ring should form where the two layers meet. Remember to neutralize this waste before disposal.}
\end{description}

Note that lead nitrate is poisonous. Add some dilute sulphuric acid* to all waste containing lead nitrate to precipitate any soluble lead. Note also that fully concentrated sulphuric acid is very dangerous. Only use it for the brown ring test. Dissolve one box of bicarbonate of soda in 500~mL of water and have this solution available wherever concentrated acid is being used. In the advent of acid spills, use this solution to neutralize the acid to stop burns.

%==============================================================================

\subsection{The Steps of Qualitative Analysis}
\setcounter{secnumdepth}{3}
\subsubsection{Appearance}

Three properties of the salt may be observed directly: colour, texture, and smell.
\begin{description}

\item[Colour]{While most salts are white, salts of transition metals are often colored. Thus colour is an easy way to identify iron and copper cations in salts.}

\item[Texture]{Carbonates and hydrogen carbonates generally form powders although sometimes they can form crystals. Sulphate, nitrates, and chlorides are almost always founds as crystals.}

\item[Smell]{Some ammonium salts smell distinctly like ammonia. Some, however, have no smell. Therefore the smell of ammonia can confirm the presence of ammonium cations, but its absent can not be used to prove the absence of ammonium.}

\end{description}

\paragraph{Materials}
soda bottle caps, table salt, bicarbonate of soda, soda ash (sodium carbonate), copper (II) sulphate*, ammonium sulphate*, locally manufactured iron (II) sulphate*, locally manufactured iron (III) sulphate*, locally manufactured copper (II) carbonate

\paragraph{Preparation}
\begin{enumerate}
\item{Place a small amount of each sample in a different soda bottle cap for observation.}
\end{enumerate}

\paragraph{Activity Steps}
\begin{enumerate}
\item{Look at the samples. Describe their colour, texture, and smell. Do not touch or inhale the salts.}
\end{enumerate}

\paragraph{Results and Conclusion}
\begin{itemize}
\item{Colour}
\begin{description}
\item[White]{Copper and iron absent}
\item[Blue]{Copper cation present}
\item[Green]{Iron (II) or copper present}
\item[Light green]{Iron (II) present}
\item[Yellow or red-brown]{Iron (III) present}
\end{description}
\item{Texture}
\begin{description}
\item[Powder]{Carbonate or hydrogen carbonate anion present}
\item[Crystals]{Sulphate, chloride, or nitrate anion probably present}
\item[Wet crystals]{Chloride or nitrate anion present}
\end{description}
\item{Smell}
\begin{description}
\item[Smell of ammonia]{Ammonium cation present}
\item[No smell of ammonia]{Inconclusive -- some ammonium compounds have no smell}
\end{description}
\end{itemize}

\paragraph{Clean Up}
\begin{enumerate}
\item{Collect salts for use another day. Do not mix.}
\item{Wash and return soda bottle caps.}
\end{enumerate}

\paragraph{Notes}
%Note that ions by themselves have no colour. The colour in all colored salts are transition metal complex compounds. For example, copper (II) salts are blue because

Wet crystals are the result of the salt absorbing water from the atmosphere. Qualitative analysis salts with this property are not locally available. However, caustic soda (sodium hydroxide) has this property, so samples of caustic soda can be used to show the absorption of water from the air and how this changes the appearance of the salt. Note that caustic soda burns skin, blinds in eyes and corrodes metal, so care is required.
%======================================
\subsubsection{Action of heat}

Many salts thermally decompose when heated. When these salts decompose, they produce gases that may be identified to identify the anion of the salt. After decomposition, many salts also leave a residue that may identify the cation.

\paragraph{Materials}
soda bottle caps, motopoa, matches, long handled metal spoons, steel wool, sand, beaker*, water, table salt, copper (II) sulphate*, bicarbonate of soda, locally prepared copper (II) carbonate*, soda ash (sodium carbonate), locally prepared zinc carbonate*

%locally prepared iron (II) sulphate* OR locally prepared iron (III) sulphate* (a mixture will also suffice), -- ?? use carbonate?? does iron (II) carbonate exist??

\paragraph{Hazards and Safety}
\begin{itemize}
\item{Ammonium nitrate explodes when heated. For this reason, ammonium nitrate should never be used in qualitative analysis when the Action of Heat test is used.}
\end{itemize}

\paragraph{Preparation}
\begin{enumerate}
\item{Fill a beaker with water.}
\item{Make a small pile of sand on the table for resting the hot spoon.}
\item{Place a small amount of each sample in a different soda bottle cap.}
\item{Add motopoa to another soda bottle cap to use as a burner.}
\end{enumerate}

\paragraph{Activity Steps}
\begin{enumerate}
\item{Light the motopoa. Note that the flame will be invisible.}
\item{Place a very small amount of a sample on the spoon. Generally, the smallest amounts of sample give the best results because they are easier to heat to a hotter temperature.}
\item{Heat the sample strongly, observing all changes.}
\item{Place the hot spoon on the sand to cool.}
\item{Once the spoon has mostly cooled, dip it in the beaker of water to remove the rest of the heat.}
\item{Use the steel wool to remove all residue from the spoon.}
\item{Repeat these steps with each sample.}
\end{enumerate}

\paragraph{Results and Conclusion}
\begin{itemize}
\item{Gas released}
\begin{description}
\item[Brown gas]{Nitrogen dioxide, nitrates present, confirmed}
\item[Colourless gas with smell of ammonia]{Ammonia, ammonium present, confirmed}
\item[Colourless gas with no smell]{Very likely carbon dioxide, especially if the compound decomposes near the start of heating, carbonate or hydrogen carbonate present}
\item[No change]{Salt probably a chloride, sulphate (very high temperatures are required to decompose many sulphates), or sodium carbonate}
\end{description}
\item{Residue}
\begin{description}
\item[No residue]{Ammonium cation present}
\item[Black residue]{Copper cation probably present}
\item[Red residue when hot, dark when cool]{Iron cation present}
\item[Yellow residue when hot, white when cool]{Zinc cation present}
\item[Red residue when hot, yellow when cool]{Lead cation present}
\end{description}
\item{Sound}
\begin{description}
\item[Cracking sound]{Sodium chloride or lead nitrate present}
\end{description}
\end{itemize}

\paragraph{Clean Up}
\begin{enumerate}
\item{Thoroughly remove all residues from the spoons.}
\end{enumerate}

\paragraph{Notes}
Sodium carbonate is the only carbonate used in qualitative analysis that does not thermally decompose. Therefore a white powder that does not decompose when heated is probably sodium carbonate.

%======================================
\subsubsection{Action of dilute \texorpdfstring{\ce{H2SO4}}{H2SO4}}

Carbonates and hydrogen carbonates react with dilute acid. Sulphates, chlorides and nitrates do not. Therefore reaction with dilute acid is useful test to help identify the anion. Sulphuric acid is used because it is the least expensive.

\paragraph{Materials}
dilute sulphuric acid*, droppers*, bicarbonate of soda, table salt

\paragraph{Hazards and Safety}
\begin{itemize}
\item{Use only a few drops of acid. These are all that are necessary and using more can be dangerous.}
\end{itemize}

\paragraph{Preparation}
\begin{enumerate}
\item{Place a small amount of each sample in a different soda bottle cap.}
\item{Fill droppers with 1-2~mL dilute acid.}
\end{enumerate}

\paragraph{Activity Steps}
\begin{enumerate}
\item{Add a few drops of acid to each sample. Observe the results.}
\end{enumerate}

\paragraph{Results and Conclusion}
\begin{description}
\item[Bubbles of gas]{Carbon dioxide produces; carbonate or hydrogen carbonate anion present}
\item[No bubbles of gas]{Carbonate and hydrogen carbonate absent}
\end{description}

\paragraph{Clean Up}
\begin{enumerate}
\item{Neutralize spills of dilute sulphuric acid with bicarbonate of soda.}
\item{Mix the remains from the reactions together so the extra bicarbonate of soda can neutralize the acid used to test table salt. Dilute the resulting mixture with a large amount of water and dispose down a sink, into a waste storage tank, or into a pit latrine.}
\end{enumerate}

\paragraph{Notes}
You can confirm that the gas produced is carbon dioxide by testing to see if it extinguishes a glowing splint. To do this, light a match, use about 0.5~mL of acid (rather than a few drops), and see if the gas released will extinguish the match.

%======================================
\subsubsection{Action of concentrated \texorpdfstring{\ce{H2SO4}}{H2SO4}}

Concentrated sulphuric acid can convert chloride anions to hydrogen chloride gas and some nitrates to nitrogen dioxide. Because both of these gases are easy to detect, the addition of concentrated acid is used to distinguish between nitrates, chlorides, and sulphates. The concentrated acid used in this experiment should be about 5~M, similar to battery acid.

\paragraph{Materials}
battery acid, droppers*, spoons, test tubes*, test tube rack*, test tube holder*, heat source*, hot water bath*, table salt (sodium chloride), gypsum (calcium sulphate)*, ammonium sulphate*, blue litmus paper*, beaker*, water

\paragraph{Hazards and Safety}
\begin{itemize}
\item{Use battery acid or another source of 5~M sulphuric acid for this experiment. Do not use fully concentrated 18~M sulphuric acid directly from either industry or laboratory supply. 18~M is too concentrated and very dangerous to use.}
\item{Concentrate acid reacts violently with carbonates and hydrogen carbonates. The previous test -- the addition of dilute acid -- will detect carbonates and hydrogen carbonates. If that test is positive, do not test the sample with concentrated sulphuric acid.}
\end{itemize}

\paragraph{Preparation}
\begin{enumerate}
\item{Place a small amount of each sample in a different soda bottle cap.}
\item{Add about 1~mL of air to each dropper syringe (no needle!) and then 2~mL of battery acid. Distribute the dropper syringes in the test tube racks so they stand with the outlet pointing down. The goal is to prevent the battery acid from reacting with the rubber plunger.}
\end{enumerate}

\paragraph{Activity Steps}
\begin{enumerate}
\item{Light the heat source and start heating the hot water bath. The water in the hot water bath should boil.}
\item{Use the spoon to add a small amount of a sample to a test tube.}
\item{Add two drops of battery acid to the sample to make sure there is no violent reaction.}
\item{Add just enough battery acid to cover the sample. Avoid spilling drops of acid on the inside walls of the test tube.}
\item{If a brown gas is released, stop at this step.}
\item{Moisten the blue litmus paper by quickly dipping it in the water of the hot water bath.}
\item{Place the litmus paper over the mouth of the test tube to receive any gases produces. If the litmus paper changes colour, stop at this step.}
\item{Hold the test tube in the hot water bath and heat for a while. Stop heating before the acid in the test tube boils. If the litmus paper changes colour before the acid boils, this is a useful result. If the acid boils, fumes from the acid itself will change the colour of the litmus paper -- this result is not useful, and acid fumes are dangerous.}
\end{enumerate}

\paragraph{Results and Conclusion}
\begin{description}
\item[Bubbles with a few drops of acid]{Carbonate or hydrogen carbonate anion  present}
\item[Brown gas produced]{Nitrate anion present}
\item[Litmus changes to red]{Hydrogen chloride gas produced; chloride anion present}
\item[No effect observed]{Sulphate anion probably present}
\end{description}

\paragraph{Clean Up}
\begin{enumerate}
\item{Fill a large beaker half way with room temperature water. This will be the waste beaker.}
\item{Pour waste from the test tubes into the waste beaker.}
\item{Fill each test tube half way with water and add this water to the waste beaker.}
\item{Return unused battery acid from the droppers to a well-labelled storage container for future use. Immediately fill each dropper (syringe) with water and transfer this water to the waste beaker.}
\item{Slowly add bicarbonate of soda to the waste beaker until addition no longer causes bubbling. This is to neutralize the acid in the waste.}
\item{Dilute the resulting mixture with a large amount of water and dispose down a sink, into a waste storage tank, or into a pit latrine.}
\item{Thoroughly wash all apparatus, including the test tubes and droppers, and return them to the proper places.}
\end{enumerate}

%======================================
\subsubsection{Flame test}

Some metal ions produce a characteristically coloured flame when added to fire.

\paragraph{Materials}
soda bottle caps, motopoa, metal spoons, beaker*, steel wool, water, table salt (sodium chloride), gypsum (calcium sulphate)*, copper (II) sulphate*, ammonium sulphate*

\paragraph{Preparation}
\begin{enumerate}
\item{Fill a beaker with water.}
\item{Place a small amount of each sample in a different soda bottle cap.}
\item{Add motopoa to another soda bottle cap to use as a burner.}
\end{enumerate}

\paragraph{Activity Steps}
\begin{enumerate}
\item{Light the motopoa. Note that the flame will be invisible.}
\item{Place a small amount of sample on the edge of the spoon. For some spoons, it is better to hold the spoon by the wide part and to place the sample on the end of the handle.}
\item{Hold the sample into the hottest part of the flame, 1-2~cm above the motopoa. If necessary, tilt the spoon so that the sample touches the flame directly. Do not spill the sample into the flame.}
\item{Dip the hot end of the spoon into the beaker of water to cool it and remove the sample. If necessary, clean the spoon with steel wool.}
\item{Repeat these steps with each sample.}
\end{enumerate}

\paragraph{Results and Conclusion}
\begin{description}
\item[Blue or green flame]{Copper present, confirmed}
\item[Golden yellow flame]{Sodium present, confirmed}
\item[Brick red flame]{Calcium present}
\item[Bluish white flame]{Lead present}
\item[No flame colour]{Copper and sodium absent; calcium and lead probably absent; cation is probably ammonia, iron, or zinc}
\end{description}

\paragraph{Clean Up}
\begin{enumerate}
\item{Collect unused samples for use another day.}
\item{Wash and return all apparatus.}
\end{enumerate}

%======================================
\subsubsection{Solubility}

\paragraph{Materials}
soda bottle caps, two spoons, test tubes*, test tube rack*, hot water bath*, heat source*, distilled (rain) water*, table salt (sodium chloride), soda ash (sodium carbonate)*, gypsum (calcium sulphate)*, powdered coral rock (calcium carbonate)* or locally manufactured calcium carbonate* or locally manufactured copper (II) carbonate*

\paragraph{Preparation}
\begin{enumerate}
\item{Fill a beaker with water.}
\item{Place a small amount of each sample in a different soda bottle cap.}
\end{enumerate}

\paragraph{Activity Steps}
\begin{enumerate}
\item{Light the heat source and start heating the hot water bath. The water in the hot water bath should boil.}
\item{Decide which spoon will be used for transferring samples and which will be used for stirring.}
\item{Use the transfer spoon to transfer a very small amount of a sample to a test tube.}
\item{Add 3-5~mL of distilled water to the test tube.}
\item{Use the handle of the stirring spoon to thoroughly mix the contents of the test tube.}
\item{If the sample does not dissolve, heat the test tube in the water bath until the contents of the test tube are almost boiling (small bubbles rise from the bottom). Mix.}
\item{Repeat these steps with each sample.}

\end{enumerate}

\paragraph{Results and Conclusion}
\begin{description}
\item[Sample dissolves in room temperature water]{Soluble salt present}
\item[Sample dissolves only in hot water]{Calcium sulphate or lead chloride present}
\item[Sample does not dissolve in even hot water]{Insoluble salt present}
\end{description}

Solubility Rules
\begin{itemize}
\item{All Group I (sodium, potassium, etc) and ammonium salts are soluble (sodium borate is an exception but not relevant to qualitative analysis)}
\item{All nitrates and hydrogen carbonates are soluble}
\item{Most chlorides are soluble (silver and lead chlorides are exceptions, 
although the latter is soluble in hot water)}
\item{Carbonates of metals outside of Group I are generally insoluble (note that aluminum and iron (III) carbonate do not exist)}
\item{Lead sulphate is insoluble and calcium sulphate is soluble only in hot water. Magnesium sulphate is completely soluble while sulphates of the Group II metals heavier than calcium 
(strontium and barium) are insoluble. All other sulphates used in qualitative analysis are soluble}]
\end{itemize}

Table of Solubility for Qualitative Analysis
\begin{center}
\begin{tabular}{ | c | c | c | c | c | c | c | c | } \hline
& ammonium & sodium & copper & iron & zinc & calcium & lead \\ \hline
nitrate & O & O & O & O & O & O & O \\ \hline
chloride & O & O & O & O & O & O & $\Delta$ \\ \hline
sulphate & O & O & O & O & O & $\Delta$ & X \\ \hline
carbonate & O & O & X & X & X & X & X \\ \hline
hydrogen carbonate & O & O & -- & -- & -- & -- & -- \\ \hline
\end{tabular}
\end{center}

KEY:
\begin{itemize}
\item{O = soluble at room temperature}
\item{$\Delta$ = soluble only when heated}
\item{X = insoluble in water}
\item{-- = salt does not exist}
\end{itemize}

\paragraph{Clean Up}
\begin{enumerate}
\item{Collect all unused (dry) samples for use another day.}
\item{Unless copper carbonate is used, none of the salts listed in the materials section of this activity are harmful to the environment.}
\item{Dispose of solutions in a sink, waste tank, or pit latrine.}
\item{Dispose of solids and liquid wastes with precipitates in a waste tank or pit latrine -- never dispose of solids in sinks.}
\item{If using copper carbonate, collect all waste containing copper carbonate and filter to recover the copper carbonate. Save for use another day.}
\item{If you do this activity with a lead nitrate or lead chloride, collect these wastes in a separate container. Add dilute sulphuric acid dropwise until no further precipitation is observed. Neutralize with bicarbonate of soda. Dispose this mixture in a waste tank or a pit latrine. The lead sulphate precipitate is highly insoluble will not enter the environment.}
\item{Wash and return all apparatus.}
\end{enumerate}

\paragraph{Notes}
Calcium carbonate or copper carbonate are recommended qualitative analysis salts to use as examples of insoluble salts. If these are difficult to get, other insoluble compounds may be used for teaching this specific step (but not for other parts of qualitative analysis). Examples of other insoluble compounds include sulphur power, manganese (IV) oxide from batteries, and chokaa (calcium hydroxide, which is only slightly soluble so a significant precipitate will remain).
%========================================
\subsubsection{Addition of NaOH solution}

\paragraph{Materials}
soda bottle caps, two spoons, test tubes*, test tube rack*, beakers*, medium droppers (5~mL syringes without needles)*, large droppers (10~mL syringes without needles), caustic soda (sodium hydroxide)*, table salt (sodium chloride), ammonium sulphate*, copper (II) sulphate*, locally manufactured iron (II) sulphate*, locally manufactured iron (III) sulphate*, locally manufactured zinc sulphate*, distilled (rain) water

\paragraph{Preparation}
\begin{enumerate}
\item{Fill a 500~mL water bottle about half way with distilled (rain) water.}
\item{Add one level tea spoon of caustic soda and then wash the spoon.}
\item{Label the bottle ``1~M sodium hydroxide -- corrosive''}
\item{Place a small amount of each sample in a different soda bottle cap.}
\item{Pour some of the sodium hydroxide solution into a clean beaker.}
\item{For each small dropper syringe, suck in about 1~mL of air and then add about 4~mL of sodium hydroxide solution. Distribute the dropper syringes in the test tube racks so they stand with the outlet pointing down. The goal is to prevent the sodium hydroxide from reacting with the rubber plunger.}
\end{enumerate}

\paragraph{Activity Steps}
\begin{enumerate}
\item{Decide which spoon will be used for transferring samples and which will be used for stirring.}
\item{Use the transfer spoon to transfer a very small amount of a sample to a test tube.}
\item{Use the large dropper syringe to add 3-5~mL of distilled water to the test tube.}
\item{Use the handle of the stirring spoon to thoroughly mix the contents of the test tube.}
\item{Use the small dropper to add a few drops of sodium hydroxide solution to the test tube.}
\item{Observe the colour of any precipitate formed. Also waft the air from the top of the test tube towards your nose to test for smell.}
\item{If a white precipitate forms, use the stir spoon to transfer a very small quantity of the precipitate to a clean test tube. Add 1-2~mL of sodium hydroxide directly to this sample to see if the precipitate is soluble in excess sodium hydroxide solution.}
\end{enumerate}

\paragraph{Results and Conclusion}
\begin{description}
\item[No precipitate and smell of ammonia]{Ammonium cation present, confirmed}
\item[No precipitate and no smell]{Sodium cation probably present}
\item[Blue precipitate]{Copper (II) cation present}
\item[Green precipitate]{Iron (II) cation present}
\item[Red-brown precipitate]{Iron (III) cation present}
\item[White precipitate not soluble in excess NaOH]{Calcium cation present}
\item[White precipitate soluble in excess NaOH]{Lead or zinc cation present}
\end{description}

\paragraph{Clean Up}
\begin{enumerate}
\item{Save all waste from this experiment, labeling it ``basic qualitative analysis waste, no heavy metals'' and leave it in an open container. Over time atmospheric carbon dioxide will react with the sodium hydroxide to make less harmful carbonates. After 2-3 days, dispose of the waste in a waste tank or a pit latrine.}
\end{enumerate}

%======================================
\subsubsection{Addition of \texorpdfstring{\ce{NH3}}{NH3} solution}

This test is very similar to the addition of sodium hydroxide solution. The useful difference is that zinc forms a precipitate in ammonia that is soluble in excess ammonia whereas lead forms a precipitate in ammonia that is not soluble in excess ammonia. Therefore, this test is mainly used to separate lead and zinc. Neither lead salts nor ammonia are locally available in Tanzania. Because the process of this test is the same as the addition of NaOH and the results so similar, students can adequately learn about the Addition of \ce{NH3} test by practicing the Addition of NaOH. For the national exam, a small amount of ammonia solution can be obtained.

Note also that the addition of ammonia to a solution of copper (II) will produce a blue precipitate that dissolves in excess ammonia to form a deep blue solution. This is a useful conformation of the presence of copper, but such conformation is generally unnecessary because the flame test for copper is so reliable.

If you have ammonia solution, store it in a well-sealed container to prevent the ammonia from escaping. A good container for this is a well labeled plastic water bottle with a screw on cap.

%======================================
\subsubsection{Confirmatory Tests}
\setcounter{secnumdepth}{4}

Every cation and anion has at least one specific test that can be used to prove its presence. Not all of these tests are possible with local materials, but many of them are. The following list shows how to confirm each possible cation and anion.

\paragraph{Confirmatory Tests for the Cation}

\subparagraph{Ammonium}
\begin{itemize}
\item{Example salt: ammonium sulphate*}
\item{Procedure: add sodium hydroxide solution and heat in a water bath}
\item{Confirming result: smell of ammonia} 
\item{Reagents: NaOH solution as used above}
\end{itemize}

\subparagraph{Calcium}
\begin{itemize}
\item{Example salt: calcium sulphate}

\item{Procedure: Two options
\begin{enumerate}
\item{flame test}
\item{addition of NaOH solution}
\end{enumerate}
} % Procedure

\item{Confirming results:
\begin{enumerate}
\item{flame test: brick red flame}
\item{addition of NaOH: white precipitate insoluble in excess}
\end{enumerate}
} % Confirming results

\item{Reagents:
\begin{enumerate}
\item{none}
\item{NaOH solution}
\end{enumerate}
} % Reagents

\end{itemize} % Calcium

\subparagraph{Copper}
\begin{itemize}
\item{Example salt: copper sulphate}
\item{Procedure: flame test}
\item{Confirming result: blue/green flame}
\item{Reagents: none}
\end{itemize}

\subparagraph{Iron (II)}
\begin{itemize}
\item{Example salt: locally manufactured iron sulphate 
(keep away from water and air)}
\item{Procedure: addition of sodium hydroxide solution 
and then transfer of precipitate to the table surface}
\item{Confirming result: green precipitate 
that oxidizes to brown when exposed to air}
\item{Reagent: sodium hydroxide solution from above}
\end{itemize}

\subparagraph{Iron (III)}
\begin{itemize}
\item{Example salt: locally manufactured iron sulphate 
(oxidized by water and air)}
\item{Procedure: addition of sodium ethanoate solution}
\item{Confirming result: yellow to red solution}
\item{Reagent: slowly add bicarbonate of soda to vinegar; stop adding when further addition does not cause bubbles; label the solution ``sodium ethanoate for detection of iron (III)''}
\end{itemize}

\subparagraph{Lead}
\begin{itemize}
\item{Example salt: no local sources for safe manufacture, consider purchasing lead nitrate}

\item{Procedure: Three options
\begin{enumerate}
\item{flame test} 
\item{addition of dilute sulphuric acid}
\item{addition of potassium iodide solution}
\end{enumerate}
} % Procedure

\item{Confirming results:
\begin{enumerate}
\item{flame test: blue/white flame}
\item{addition of dilute sulphuric acid: white precipitate}
\item{addition of KI solution: yellow precipitate that dissolves when heated and reforms when cold}
\end{enumerate}
} % Confirming results

\item{Reagents:
\begin{enumerate}
\item{none but a very hot flame, e.g. Bunsen burner, is required} 
\item{dilute sulphuric acid used in Step 5 above}
\item{obtain pure potassium iodide by evaporating iodine tincture until only white crystals remain; do this outside and do not breathe the fumes; it might also be possible to use the KI solution prepared for electrolysis in the chapter on ionic theory}
\end{enumerate}
} % Reagents

\end{itemize} % Lead

\subparagraph{Sodium}
\begin{itemize}
\item{Example salts: sodium chloride, sodium carbonate, sodium hydrogen carbonate}
\item{Procedure: flame test}
\item{Confirming result: golden yellow flame}
\item{Reagents: none}
\end{itemize}

\subparagraph{Zinc}
\begin{itemize}
\item{Example salt: locally manufactured zinc carbonate 
or zinc sulphate}
\item{Procedure: addition of 0.1~M potassium ferrocyanide solution}
\item{Confirming result: gelatinous gray precipitate}
\item{Reagents: no local source of potassium ferrocyanide -- consider collaborating with many schools to share a container; only a very small quantity is required}
\end{itemize}

\paragraph{Confirmatory Tests for the Anion} 

\subparagraph{Hydrogen carbonate}
\begin{itemize}
\item{Example salt: sodium hydrogen carbonate}
\item{Procedure: add magnesium sulphate solution and then boil in a water bath}
\item{Confirming result: white precipitate forms only after boiling}
\item{Reagent: dissolve Epsom salts (magnesium sulphate)* in distilled (rain) water*}
\end{itemize}

\subparagraph{Carbonate}
\begin{itemize}
\item{Example salt: sodium carbonate}
\item{Procedure for soluble salts: addition of magnesium sulphate solution}
\item{Confirming result: white precipitate forming in cold solution}
\item{Reagent: dissolve Epsom salts (magnesium sulphate)* in distilled (rain) water*}

\item{Note that insoluble salts that effervesce with dilute acid are likely carbonates. 
None of the other anions described here produce gas with dilute acid. Note also that all hydrogen carbonates are soluble.}

\end{itemize}

\subparagraph{Chloride}
\begin{itemize}
\item{Example salt: sodium chloride}
\item{Procedure: Three Options
\begin{enumerate}
\item{addition of silver nitrate solution}
\item{addition of manganese (IV) oxide and concentrated sulphuric acid followed by heating in a water bath}
\item{addition of weak acidified potassium permanganate solution followed by heating in a water bath}
\end{enumerate}
} % Procedure
\item{Confirming results:
\begin{enumerate}
\item{silver nitrate: white precipitate of silver chloride} 
\item{manganese (IV) oxide: production of chlorine gas that bleaches litmus} 
\item{acidified permanganate: decolourization of permanganate}
\end{enumerate}
} % Confirming results
\item{Reagents:
\begin{enumerate}
\item{Silver nitrate has no local source but may be shared among many schools as only a very small amount is required.}
\item{Manganese dioxide may be purified from used batteries and battery acid is concentrated sulphuric acid. Note that careful purification is required to remove all chlorides from the battery powder. This method is useful because of its low cost, but remember that chlorine gas is poisonous! Students should use very little sample salt in this test.}
\item{Prepare a solution of potassium permanganate, dilute with distilled water until the colour is light pink, and then add about 1 percent of the solution's volume in battery acid. Note that this solution will cause lead to precipitate, and will also be decolourized by iron II, so it is not a perfect substitute for silver nitrate. This final option is also not yet recognized by examination boards, i.e. NECTA}
\end{enumerate}
} % Reagents
\end{itemize}

\subparagraph{Sulphate}
\begin{itemize}
\item{Example salt: copper sulphate, calcium sulphate, iron sulphate}
\item{Procedure: addition of a few drops of a solution of lead nitrate, 
barium nitrate, or barium chloride}
\item{Confirming result: white precipitate}
\item{Reagents: none of these chemicals have local sources. Because lead nitrate is also an example salt, it is the most useful and the best to buy. The ideal strategy is to share one of these chemicals among many schools. Remember that all are quite toxic.}
\end{itemize}

\setcounter{secnumdepth}{2}
\paragraph{Notes}

Emphasize to students that they need to carry out only one confirmatory test for the cation, 
and one for the anion. If the test gives the expected result, then they can be sure that the ion they have identified is present. If the test does not give the expected result, they have probably made a mistake, and they should revisit the results of their previous tests 
and choose a different possibility to confirm.

%==============================================================================

\subsection{Hazards and Cleanliness}

Qualitative analysis practicals are full of hazards, 
from open flames to concentrated acids. 
To reduce the risk of accidents, 
teach students how to use their flame source 
before the day of the practical, 
especially if you are using Bunsen burners. 
Most students have never used gas before, 
and do not know the basic safety precautions involved in using gas. 
If you have choice about what salts are offered, 
do away with those requiring concentrated acid, and poisonous reagents like lead and barium.

Teach students to hold their test tubes at an angle when they heat them or perform reactions in them. Test tubes should be pointed away from the student holding them and from other students. 
This will prevent injuries due to splashing chemicals, and will also minimize inhalation of any gases produced.

Teach students never to fill test tubes or any other container more than half. That way, 
they minimize spills and boiling over of chemicals during heating. In addition, this also prevents bumping in the test tubes (when a gas bubble forms suddenly), which can cause dangerous spray.

Teach students that if they get chemicals on their hands, they should wash them off immediately, without asking for permission first. Some students have been taught to wait for a teacher's permission before doing anything in the lab, even if concentrated acid is burning their hands. On the first day, give them permission to wash their hands if they ever spill chemicals on them. Also, teach students to tell you immediately when chemicals are spilled. 
Sometimes they hide chemical spills for fear of punishment. Do not punish them for spills -- legitimate accidents happen. Do punish them for unsafe behavior of any kind, even if it does not result in an accident. 

Practicals involving nitrates, chlorides, ammonium compounds, and some sulphates produce harmful gases. Open the lab windows to maximize airflow. Kerosene stoves also produce noxious fumes -- it is much better to use motopoa. If students feel dizzy or sick from the fumes, let them go outside to recover.

Make absolutely sure that students clean their tables and glassware before they leave. Leaving chemicals lying around is dangerous, especially when they are not labelled. Qualitative analysis experiments can leave residues in glass test tubes that are difficult to clean with brushes alone. If possible, heat samples in metal spoons. To remove stubborn residues in glass test tubes, pour a little dilute nitric acid into the test tube. The acid should dissolve the precipitate and leave a clean test tube behind. Remember that the utility of nitric acid -- 
that it will dissolve almost anything -- is also a serious hazard.

%==============================================================================

%\subsection{Sample Practical: Preparation of Copper Carbonate for Qualitative Analysis}
%
%After teaching students all of the individual steps of qualitative analysis, it is good to allow them to practice all of the step on one practical session, as they will have to do for NECTA. The following activity describes how to prepare copper carbonate and how to perform qualitative analysis on it. The teacher should try this activity alone first to review qualitative analysis and then students should perform this activity in groups.
%
%\paragraph{Objective}
%\begin{itemize}
%\item{To perform all steps of qualitative analysis to identify an unknown salt}
%\end{itemize}
%
%\subsubsection{Materials}
%plastic water bottles, filter funnel*, heat source* and take-away tray (optional), plastic spoon, copper (II) sulphate*, soda ash (sodium carbonate)*
%
%\subsubsection{Preparation}
%\begin{enumerate}
%\item{Combine 5 spoons of copper (II) sulphate and 200 mL water in a plastic bottle. Cap and shake until the copper (II) sulphate has fully dissolved.}
%\item{Combine 10 spoons of sodium carbonte and 200 mL water in a separate plastic bottle. Cap and shake until the sodium carbonate has fully dissolved.}
%\item{Combine the two solutions. A green/blue precipitate should form.}
%\item{Alternatively, if another form is practising precipitation reactions using copper (II) sulphte and sodium carbonate, save their solid waste.}
%\item{Pour the mixture into a filter funnel. Let sit until all liquid has passed through.}
%\item{Transfer the solid to a plastic bottle and add about half a litre of water. Shake thoroughly. This is to remove any sodium or sulphate from the original chemicals.}
%\item{Pour the mixture into a filter funnel. Let sit until all liquid as passed through.}
%\item{Dry the precipitate, either in the sun or by heating very gentle in a take-away container over a heat source. If heating, stir often, and remove from the heat before all the water has evaporated or else the copper carbonate will start to thermally decompose.}
%\item{Transfer the dry blue powder to a clean container and label it ``copper (II) carbonate.''}
%\end{enumerate}
%
%\subsubsection{Activity Procedure}
%\begin{enumerate}
%\item{Perform the qualitative analysis steps described above on the sample.}
%\end{enumerate}
%
%\subsubsection{Notes}
%Because copper carbonate is insoluble in water, add a little dilute sulphuric acid solution to bring the ion into solution for the NaOH test. Add the acid drop by drop and avoid adding excess.

\subsection{Sample Practical Question}
The following is a sample practical question from 2012.\\[6pt]


Substance \textbf{V} is a simple salt which contains one cation and one anion. Carry our the experiments described below. Record carefully your observations and make appropriate inferences and hence identify the anion and cation present in sample \textbf{V}.\\

\begin{center}
\begin{tabular}{|l|p{8cm}|l|l|}
\hline
\textbf{S/n}&\textbf{Experiment}&\textbf{Observation}&\textbf{Inference}\\ \hline
1&Observe the appearance of sample \textbf{V}.&&\\ \hline
2&Put a little amount of sample \textbf{V} in a test tube then add water and shake.&&\\ \hline
3&Heat a little amount of \textbf{V} in a dry test tube.&&\\ \hline
{\multirow{4}{*}{4}}&To a little sample \textbf{V} in a test tube add dilute Hydrochloric acid. Add more of the acid until the test tube is half full. Divide the resulting solution into three portions and add the following:&&\\ \cline{2-4}
&\begin{enumerate}
\item[a)] To the one portion add NaOH solution drop wise then excess.
\end{enumerate}&&\\ \cline{2-4}
&\begin{enumerate}
\item[b)] To the second portion add ammonia solution drop wise then in excess.
\end{enumerate}&&\\ \cline{2-4}
&\begin{enumerate}
\item[c)] To the third portion add ammonium oxalate solution.
\end{enumerate}&&\\ \hline
5&Perform flame test.&&\\ \hline
\end{tabular}\\

\end{center}

Conclusion\\

\begin{enumerate}
\item[(i)] The cation in sample \textbf{V} is \_\_\_\_.\\
\item[(ii)] The anion in sample \textbf{V} is \_\_\_\_.\\
\item[(iii)] The chemical formula of \textbf{V} is \_\_\_\_.\\
\item[(iv)] The name of compound \textbf{V} is \_\_\_\_.\\
\end{enumerate}


\raggedleft \textbf{(15 marks)}

\raggedright

\subsubsection{Discussion}
This particular example was to identify calcium carbonate; however, the above procedure follows the same format commonly used for other unknown salts. The only differences may be the specific solutions used in some of the steps.

At times, the procedure may not be explicit or indicated whatsoever and the student is required to write the detailed procedure in addition to the observations and inferences. %The proper way to write the procedure, observations, and inferences can be found in ------ 

Emphasize to students that in addition to the qualitative analysis procedure they have to do only one confirmatory test for cations and one for anions.


%==============================================================================
%==============================================================================

\section{Chemical Kinetics and Equilibrium}
%[brief 1 paragraph explanation of practical]\\

%This section contains the following:
%\begin{itemize}
%\item 
%\end{itemize}

\subsection{Theory}
Compared to the other two NECTA chemistry practicals - Acid/Base Titration (i.e. Volumetric Analysis) and Qualitative Analysis - Chemical Kinetics has few alternative chemicals that can be used.\\

The chemical reaction in the NECTA exam is a precipitation of sulphur. 

\[ \mathrm{Na_2S_2O_3}_{(aq)} + \mathrm{2HCl}_{(aq)} \longrightarrow \mathrm{2NaCl}_{(aq)} + \mathrm{H}_{2}\mathrm{O}_{(l)} + \mathrm{SO_2}_{(g)} + \mathrm{S}_{(s)} \]

This reaction is consistent and easy to practice, but it requires sodium thiosulphate which can be expensive and hard to get a hold of. The hydrochloric acid can be replaced with sulphuric (battery) acid. An alternative reaction to demonstrate chemical kinetics (that can be performed for less than 5000 shillings) is the iodization of acetone, seen in the reaction below.

\[ \mathrm{CH_3COCH_3}_{(aq)} + \mathrm{I_2} \longrightarrow \mathrm{CH_3COCH_2I}_{(aq)} + \mathrm{H}^{+}_{(aq)} + \mathrm{I}^{-}_{(aq)} \]

This reaction starts as a dark opaque solution and eventually proceeds to a colorless, transparent solution. It requires an acidic environment to occur (either hydrochloric or sulphuric acid suffice). The amount of time it takes to reach the point of colorlessness varies depending on the concentration of the acetone. It is easy to demonstrate the relationship between concentration and rate of reaction. (By varying the temperature or amount of acid catalyst, the reaction visibly proceeds at differing rates.)

\subsection{Materials}
7 beakers, 3 syringes, stopwatch

\subsection{Chemicals}
Nail polish remover, iodine tincture, sulphuric (battery) acid, water

\subsection{Preparation}
\begin{enumerate}
\item Prepare an iodine solution using iodine tincture. Solutions purchased in the local drugstores are often 0.2 M Iodine (with several other chemicals as well). Create a 0.02 M solution by adding 9 parts water to one part tincture. Put this solution in the first beaker.
\item Prepare an acidic acetone solution by mixing one part nail polish remover to one part 1 M Sulphuric acid solution. Put this solution in the second beaker. This is roughly a 5 M solution of acetone.
\item In the third beaker place clean water.
\end{enumerate}

\subsection{Procedure}
\begin{enumerate}
\item Pour 8 mL of the acetone solution in a clean beaker. 
\item Clear and start a stopwatch and add 8 mL of iodine to the beaker. Swirl the solution and stop the watch when the solution becomes colorless. Record the time taken in a table as shown below.
\item  Repeat the experiment, but this time start with 6 mL of acetone solution and 2 mL of clean water in a clean beaker. Clear and start a stopwatch and add 8 mL of iodine to the beaker, swirl the solution, and stop the watch when the solution becomes colorless. Record the time taken.
\item Repeat the experiment for the remaining volumes of acetone solution and clean water as shown in the table below.
\end{enumerate}

\begin{center}
\begin{tabular}{|c|p{2cm}|p{2cm}|p{2cm}|p{2cm}|p{2cm}|p{1.5cm}|} \hline
Test&Volume of Iodine Solution (mL)&Volume of Acetone Solution (mL)&Volume of Water (mL)&Molarity of Acetone Solution ($^{\text{mol}}/_\text{L}$)&Time for color to disappear (s)&Reciprocal of time ($^1/_\text{s}$) \\ \hline
1&8&8&0&5 M&& \\ \hline
2&8&6&2&3.75 M&& \\ \hline
3&8&4&4&2.5 M&& \\ \hline
4&8&2&6&1.25 M&& \\ \hline
\end{tabular} \\[10pt]
\end{center}

This data can then be used to plot a graph of concentration of acetone against the rate of reaction.

\subsection{Notes}
\begin{itemize}
\item Patience is required for the tests using lower concentrations as they can take over 4 minute to complete.
\item Concentrations may vary depending on where the tincture and remover are purchased.
\item The endpoint of this reaction can be somewhat ambiguous depending on the color of the nail polish remover. Criteria for determining the endpoint may vary.
\item It should be noted that if the nail polish remover is already a specific color it will affect the final color of the solution. Some solutions may never become fully colorless.
\item The reaction used in the NECTA exams goes from transparent to opaque while this alternative goes from opaque to transparent. Make sure students understand this difference.
\item The reaction used in NECTA exams is a neutralization reaction so there is little that needs to be done to process the waste. This alternative reaction is very acidic when finished so be prepared to 	neutralize it before disposal.
\end{itemize}

\subsection{Sample Practical Question}
The following is a sample practical question from 2012.\\[6pt]



Your are provided with the following materials:\\
\begin{enumerate}
\item[ ] \textbf{ZO}:  A solution of 0.13 M Na$_2$S$_2$O$_3$ (sodium thiosulphate);
\item[ ] \textbf{UU}:  A solution of 2 M HCl;
\item[ ] Thermometer;
\item[ ] Heat source/burner;
\item[ ] Stopwatch.\\
\end{enumerate}

Procedure:\\
\begin{enumerate}
\item[(i)] Place 500 cm$^3$ beaker, which is half-filled with water, on the heat source as a water bath.
\item[(ii)] Measure 10 cm$^3$ of \textbf{ZO} and 10 cm$^3$ of \textbf{UU} into two separate test tubes.
\item[(iii)] Put the two test tubes containing \textbf{ZO} and \textbf{UU} solutions into a water bath.
\item[(iv)] When the solutions attain a temperature of 60$^o$C, remove the test tubes from the water bath and pour both solutions into 100 cm$^3$ empty beaker and immediately start the stop watch.
\item[(v)] Place the beaker with the contents on top of a piece of paper marked \textbf{X}.
\item[(vi)] Note the time taken for the mark \textbf{X} to disappear.
\item[(vii)] Repeat step (i) to (vi) at temperature 70$^o$C, 80$^o$C and 90$^o$C.
\item[(viii)] Record your results as in Table 1.
\end{enumerate}

\begin{center}
\begin{tabular}{|p{5cm}|p{5cm}|p{3cm}|}
\multicolumn{1}{l}{Table 1}&\multicolumn{1}{l}{ }&\multicolumn{1}{l}{ }\\ \hline
\textbf{Experiment}&\textbf{Temperature}&\textbf{Time (s)}\\ \hline
1&60$^o$C&\\ \hline
2&70$^o$C&\\ \hline
3&80$^o$C&\\ \hline
4&90$^o$C&\\ \hline
\end{tabular}
\end{center}

\textbf{Questions:}\\
\begin{enumerate}
\item Write a balanced chemical equation for reaction between \textbf{UU} and \textbf{ZO}.
\item What is the product which causes the solution to cloud the letter \textbf{X}?
\item Plot a graph of temperature against time (s).
\item What conclusion can you draw from you graph?\\
\end{enumerate}


\raggedleft \textbf{(15 marks)}

\raggedright

\subsubsection{Discussion}
This particular example was to investigate how temperature affects the rate of chemical reaction. %add more for surface area, concentration, and catalyst variations of chemical kinetics...
\chapter{Physics Practicals} \index{Practicals! Physics}

\section{Introduction to Physics Practicals}

\subsection{Format}
The format of the Physics practical exam was revised in 2011 to keep up with the 2007 updated syllabus. As such, there will be no further Alternative to Practical exams, pending approval from the Ministry of Education.

The Physics practical has 2 questions and students must answer both. Question 1 comes from Mechanics, and Question 2 can come from Heat, Light, Waves or Electricity. Each question is worth 25 marks, and students have 2$\frac{1}{2}$ hours to complete the exam.

\subsubsection{Physics 1 Theory Format}
The theory portion of the Physics exam comprises 100 marks, while the practical carries 50 marks. A student's final grade for Physics is thus found by taking her total marks from both exams out of 150.

The theory exam for Physics contains 3 sections. Section A has 3 questions and is worth 30 marks - Question 1 is 10 multiple choice, Question 2 is 10 matching, and Question 3 is 10 fill-in-the-blank. Section B has 6 long answer questions and is worth 60 marks. Section C has 2 questions regarding the use of apparatus and simple technological appliances in everyday life, though students must answer \textbf{only 1} of these questions. It is worth 10 marks.

\paragraph{Note} This information is current as of the time of publication of this manual. Updated information may be obtained by contacting the Ministry of Education. 

\subsection{Notes for Teachers}

\subsubsection{NECTA Advance Instructions}
Advance instructions are given to teachers at least one month before the date of the exam. However, unlike Biology and Chemistry, \textbf{there are no 24 hour advance instructions given for Physics}.

%[tips for identifying practicals from advance instructions, example advance instructions?]

\subsubsection{Words of Advice}
%Even for the experienced teacher, the physics practical can seem a daunting task.
%It has multiple sections spanning four years of a dense, unrelenting syllabus, combining
%physics and math topics alike that might or might not have been taught by previous
%teachers. Furthermore, the typical student in a secondary school will have little to no
%experience with common apparatus.

The Physics practical is different from the Chemistry and Biology practicals in that
the exams feature a greater variety of questions. That means we need to teach it all, even if the teacher before you never found his or her way into the classroom and you realize at the end of Form Four that the students still have not studied the Form Two syllabus. If you are teaching all forms, do not wait to start practicals until later forms; always do a practical when the corresponding topic comes up. In addition, train the students well in the general principles of collecting data, graphing data, and writing up experimental points. These skills are required in every physics practical, and carry most of the points.

The practical section of the exam is a third of a student’s total score, and fully half
of that is graphing and labeling experimental data. 
%There are typically three questions on
%the same general topics: mechanics, light and electricity. The students must answer the
%mechanics question, but they can choose between light and electricity. Most students tend
%to choose the light experiment as it is easier to understand and is generally taught more
%than the electricity topics. It is also easier to prepare as a teacher. However, this does not
%mean that the electricity practical is too difficult; if prepared well, it can actually be the
%easiest to perform provided the students have a clear understanding of the apparatus, and you have prepared and tested it well. This is where you come in.
Though the practical is varied, a student does not necessarily need a deep
understanding of the concept in question. If they are familiar with the apparatus and the
process of drawing and interpreting a graph, the practical should be quite simple.
Whenever possible, allow the students to play and experiment with the apparatus,
whether it is a metre bridge, mirror, pendulum, etc. If they have done each of these experiments several times, they will be confident in their ability.

The more familiar your students are with these techniques, the better they will do.
Perform these practicals as often as possible: when the topic comes up, when preparing
for the mock and NECTA exams, and any time you can get them to come in for an
evening session or a weekend. They will make many, many mistakes the first couple of
times through but that is exactly what you want as they will learn from their mistakes and
remember them.

%\subsection{NECTA Marking Information}
%[how NECTA marks the practicals, highlight most important parts]

\subsection{Common Practicals}
\begin{description} \itemsep1pt \parskip0pt \parsep0pt
\item[Mathematics]{a brief overview of some of the mathematical and graphing skills required to perform many of the common physics practicals}
\item[Mechanics] \hfill 
\begin{description} 
\item[Hooke's Law (Form 1)]{use a spring and various masses to determine either the spring constant or the value of an unknown mass graphically}
\item[Simple Pendulum (Form 2)]{find the acceleration due to gravity using a pendulum and stopwatch}
\item[Principle of Moments (Form 2)]{verify the Principle of Moments using masses and a ruler on a knife edge, or calculate the mass of a metre rule}
%\item[Archimedes' Principle (Form 1)]{verify Archimedes' Principle by measuring the volume of water displaced by an object submerged in a Eureka can}
\end{description}
\item[Light] \hfill 
\begin{description}
\item[Plane Mirror (Reflection) (Form 1)]{generally 3 varieties: find image distance, verify the Laws of Reflection, or find the number of images produced by two plane mirrors placed at different relative angles}
\item[Rectangular Prism (Refraction) (Form 3)]{find the refractive index or critical angle for a glass block by varying the angle of incidence and measuring the corresponding angles of refraction}
\end{description}
\item[Electricity] \hfill 
\begin{description} 
\item[Potentiometers]{find the drop in potential along a length of resistance wire}
\item[Metre Bridges]{determine the value of an unknown resistor using a known resistor and a galvanometer to find a point of equal potential along a resistance wire}
\item[Ohm's Law (Form 2)]{verify Ohm's Law or determine the internal resistance of a cell}
\end{description}
\end{description}

\paragraph{Note} These are the most common practicals, but they are not necessarily the only practicals that can occur on a NECTA exam. Physics practical questions can come from a variety of topics which may not yet have been used in older past papers. Be sure to regularly check the most recent \nameref{cha:past-papers-phys} to get a good idea of the types of questions to expect. 

\subsection{Recent Practicals}
Given below is an attempt to characterize the Physics practical questions from recent years according to category and topic. Each question is given with its corresponding topic on top and the objective, or what is to be solved for, on bottom. This information can be used to try and find trends in what exam writers like to test students on, and what topics are most likely to occur on an exam. However, nothing should be assumed to be a guarantee, and students should be well-prepared in all practicals so that they can take on any question they may face on the NECTA.

Note a few things from the table below:
\begin{itemize*}
\item Beginning in 2011, the format of the exam changed to consist of only 2 problems, one of which must come from Mechanics.
\item The exam committee has a tendency to repeat some problems. For example, the Mechanics questions from 2004 and 2011 are nearly identical, as are the Light questions from 2004 and 2008.
\end{itemize*}

\begin{center}
\begin{tabular}{|c|c|c|c|} \hline
\textbf{Year} & \textbf{Mechanics} & \textbf{Light} & \textbf{Electricity} \\ \hline
\multirow{2}{*}{2013} & Principle of Moments & Plane Mirror & \multirow{2}{*}{-----} \\
& \emph{verify} & \emph{object, image dist.} & \\ \hline
\multirow{2}{*}{2012} & Density & Plane Mirror & \multirow{2}{*}{-----} \\
& \emph{verify} $\rho = \frac{m}{V}$ & \emph{number of images} & \\ \hline
\multirow{2}{*}{2011} & Principle of Moments & \multirow{2}{*}{-----} & Ohm's Law \\
& \emph{mass of ``AA'' battery} & & \emph{e.m.f., int. resistance} \\ \hline
\multirow{2}{*}{2010} & Elasticity & Refraction & Ohm's Law \\
& \emph{unknown mass, $k$} & \emph{refr. index of water} & \emph{$\rho$ of a wire} \\ \hline
\multirow{2}{*}{2009} & Simple Pendulum & Refraction & Ohm's Law \\
& \emph{relate $l \rightarrow T$} & \emph{refr. index of glass} & \emph{verify $V = IR$} \\ \hline
\multirow{2}{*}{2008} & Elasticity & Refraction & Ohm's Law \\
& \emph{verify $F=kx$, find $k$} & \emph{$\cfrac{d\cos{r}}{\sin{(i-r)}}$}  & \emph{verify $V=IR$} \\ \hline
\multirow{2}{*}{2007} & Elasticity & Refraction & Potentiometer \\
& \emph{unknown mass, $k$} & \emph{refr. index of glass} & \emph{$\Delta V$ along wire} \\ \hline
\multirow{2}{*}{2006} & Elasticity & Plane Mirror & Metre Bridge \\
& \emph{unknown mass, $F = kx$} & \emph{object, image dist.} & \emph{unknown $R$} \\ \hline
\multirow{2}{*}{2005} & Elasticity & Refraction & Ohm's Law \\
& \emph{unknown mass, $k$} & \emph{critical angle} & \emph{e.m.f., int. resistance} \\ \hline
\multirow{2}{*}{2004} & Principle of Moments & Refraction & Potentiometer \\
& \emph{mass of ``AA'' battery} & \emph{$\cfrac{d\cos{r}}{\sin{(i-r)}}$} & \emph{$\rho$ of wire, int. res.} \\ \hline
\end{tabular}
\end{center}

%For a collection of NECTA past papers from the Physics Practical exam, see []

%==============================================================================

\section{Mathematics} \index{Practicals! Physics! mathematics of}

No physics experiment is complete without a healthy dose of graphing and
formulas. As math is typically the worst subject for most students, it is often upon the
physics teacher to drive home the understanding of how to draw and interpret graphs, as
well as how to apply formulas to those graphs. It comes down to a few simple things:
correctly setting up a graph (scales, units, labels, etc.), plotting points from a table of
data, and fitting a best-fit line. After this, the students need to find the slope of this line
and its $y$-intercept.

\subsection{Graphing}
Most of the graphs will be linear, meaning the slope is constant, so we apply the standard
equation for a line
$$y=mx + c$$
where $y$ represents the vertical axis, $x$ represents the horizontal axis, $m$ is the slope of the line and $c$ is the point on the vertical axis where the line crosses. Almost every practical will make use of this equation, so be sure that your students understand it inside and out. It often helps to do repetitive practice using just the mathematical symbols before introducing physics concepts. Note that very rarely non-linear graphs appear, e.g. cooling curves in heat practicals. In this case students will not have to find a mathematical relationship, just describe and explain the trend in the data.

\subsection{Formulas}
Now comes the physics; all the practicals will involve an equation that can be
rewritten in this linear form. The exam question will dictate which variable is
independent ($x$) and which is dependent ($y$). It is up to the student to simply rewrite the
formula with each variable on its respective side and then infer what $m$, the slope, and $c$,
the $y$-intercept, must be.

The most common formulas used for mechanics, light and electricity are as follows:

\begin{center}
\begin{tabular}{ | c | c | }
\hline
Hooke's Law & $ F = ke $ or $ F = ke - B $ \\ \hline
Period of a Pendulum & $T = 2\pi\sqrt{\frac{l}{g}}$ \\ \hline
Principle of Moments & $(F \times d)_{\mathrm{clockwise}} = (F \times d)_{\mathrm{anticlockwise}}$ \\ \hline
Snell's Law & $n_1 \times \sin{i} = n_2 \times \sin{r}$ \\ \hline
Ohm's Law & $V=IR$ or $V = I(R + r)$\\ \hline
Resistance of a Wire & $R = \frac{\rho l}{A}$ \\ \hline
Wheatstone Bridge & $\cfrac{R_1}{L_1} = \cfrac{R_2}{L_2}$ \\ \hline
\end{tabular}
\end{center}

In each case, one quantity will be changed (independent) and another will be
measured (dependent) over the course of the experiment. The student will therefore need
to rearrange the equation so that the dependent variable is the subject in the form
$$y = mx + c$$

\subsubsection{Example Problem - Snell's Law}
As an example, in an experiment to measure the index of refraction of a glass block, a
student will be measuring angles of incidence and refraction. This means we need to use
Snell’s Law $$n_1 \times \sin{i} = n_2 \times \sin{r}$$
The question will typically ask students to plot a graph of their measurements, with $\sin{i}$
on the $y$-axis and $\sin{r}$ on the $x$-axis, or vice-versa. To rewrite Snell’s law in the form of
$y = mx + c$ is simple; we get $$\sin{i} = \sin{r} \times \frac{n_2}{n_1}$$
and we can see that the value corresponding to $m$ is the ratio $\frac{n_1}{n_2}$, and $c$ must be zero.

Since we are trying to find $n_2$ (the refractive index of glass), and we know $n_1$ is 1.0, we
simply measure the slope and solve to find $n_2$.

The approach itself is relatively simple, but students will need lots of practice
with graphing, rewriting equations in linear form, and determining what corresponds to $m$
and $c$ in each case. The same approach is used to find the quantities in each of the
equations above. 

A complete list of each equation in its most commonly found $y = mx + c$ form, along with its corresponding dependent variable $y$, independent variable $x$, slope
$m$ and $y$-intercept $c$, is given below. Note that these equations are not always used in the given form on practicals. It is up to the student to determine how each equation must be analyzed during an exam. The variables used and methods of graphing change from year to year, and so the following table should by no means be memorized or assumed to be applicable for a given problem.

%TABLE OF EQUATIONS
\begin{center}
\begin{tabular}{|c|c|c|c|c|c|c|} \hline
\textbf{Name of Law} & \textbf{Equation} & \textit{\textbf{y = mx + c}} & \textit{\textbf{y}} & \textit{\textbf{x}} & \textit{\textbf{m}} & \textit{\textbf{c}} \\ \hline \hline
Hooke's Law & $ F = ke $ & $ F = ke - B $ & $F$ & $e$ & $k$ & $-B$\\ \hline
Period of a Pendulum & $T = 2\pi\sqrt{\frac{l}{g}}$ & $T^2 = 4\pi \frac{l}{g}$ & $T^2$ & $l$ & $\frac{4\pi}{g}$ & 0\\ \hline
Principle of Moments & $(F \times d)_{\mathrm{cw}} = (F \times d)_{\mathrm{acw}}$ & $a = \left(\frac{m_2 + x}{m_1}\right)b$ & $a$ & $b$ & $\left(\frac{m_2 + x}{m_1}\right)$ & 0\\ \hline
Snell's Law & $n_1 \times \sin{i} = n_2 \times \sin{r}$ & $\sin{i} = \frac{n_2}{n_1}\sin{r}$ & $\sin{i}$ & $\sin{r}$ & $\frac{n_2}{n_1}$ & 0\\ \hline
\multirow{2}{*}{Ohm's Law} & $V=IR$ & $V=IR$ & $V$ & $I$ & $R$ & 0 \\ 
& $V = I(R + r)$ & $R = \frac{E}{I} - r$ & $R$ & $\frac{1}{I}$ & $E$ & $-r$ \\ \hline
\multirow{2}{*}{Resistance of a Wire} & \multirow{2}{*}{$R = \cfrac{\rho l}{A}$} & \multirow{2}{*}{$V = \cfrac{I\rho l}{A}$} & $V$ & $l$ & $\frac{I\rho}{A}$ & 0\\ 
& & & $V$ & $I$ & $\frac{\rho l}{A}$ & 0 \\ \hline
Wheatstone Bridge & $\cfrac{R_1}{L_1} = \cfrac{R_2}{L_2}$ & $R_2 = \frac{100R_1}{L_1} - R_1$ & $R_2$ & $\frac{100}{L_1}$ & $R_1$ & $-R_1$ \\ \hline
\end{tabular}
\end{center}

\subsection{Units}
Be sure to always stress the importance of units when performing practicals, especially in graphing. Using the wrong units can lead to inaccurate interpretations of graphical data. For example, when using Hooke's Law, a spring constant is typically given in units of $^{\text{N}}/_{\text{m}}$. But if a student is graphing mass (in g) against extension (in cm), then the slope will be in units of $^{\text{g}}/_{\text{cm}}$. In order to get units of $^N/_{\text{m}}$, one would have to first convert the slope into units of $^{\text{kg}}/_{\text{m}}$ and then multiply by the acceleration of gravity ($g = 10$ $^N/_{\text{kg}}$). A problem may or may not ask for specific units in its answers, but regardless, students should always be conscious of what units they are using when making calculations.

Thinking of units can also help students to understand a problem they are struggling with. If they can remember that slope is \emph{change in y over change in x}, then they may be able to deduce the meaning of the slope of a graph by looking at the units of the $y$ and $x$ axes.

The most important part of any experiment, though, is following directions. If a
student can follow directions, which usually are clearly provided by the exam, and can
graph data, they can easily perform any experiment. If anything, the practical exam is a
test in a student’s ability to follow instructions.

%==============================================================================
\settocdepth{subsection}
\section{Mechanics} 

The mechanics section is mandatory on every exam and typically falls into three
categories: Hooke’s Law, Simple Pendulum, and Principle of Moments. However, other topics are possible, as evidenced by a question on Archimedes' Principle on the 2012 exam. These experiments use the following materials:
\begin{description}
\item[Metre Rules]{If unavailable, go to a local fundi to mass produce them.}
\item[Masses]{See \nameref{cha:labequip} for local varieties.}
\item[Springs]{Can be bought at lab stores, or can use substitutes such as rubber bands or strips of elastic from a tailor.}
\item[Retort Stands]{May be available or can be made using a filled 1.5 L water bottle with a bamboo stick taped at the top and extending to one side.}
\item[Eureka Can]{Cut off the bottom of a 500 mL water bottle and cut a slit at the top that can be folded downward to make a curved spout.}
\end{description}

\subsection{Hooke’s Law} \index{Practicals! Physics! Hooke's law} \index{Hooke's law}

This is the most common practical, usually involving a spring but sometimes a
rubber band or piece of string. This experiment can be tricky simply because NECTA
likes to switch it up every year; try to give your students as much practice with different
variations. It is likely that NECTA will require a spring of known spring constant, and
you will need known masses. Either can be bought at a laboratory supply store in town,
but it is possible to make your own. The practical is simple to perform, but there are
some common mistakes: be sure the students understand that the extension is the change
in length, not the ultimate length shown on the ruler. Also, do not confuse mass and
weight, as is common.

An example question from the 2007 NECTA is shown below. After reviewing
the topic with your students, let them try this on their own. You will need to repeat it
several times before they are comfortable, using different springs and masses each time.

%\subsubsection{Sample Practical - 2010 (1)}
%
%\begin{enumerate}
%\item The aim of this experiment is to find the mass of the unknown load labeled ``$W$'' and the spring constant $K$. Proceed as follows:
%
%\begin{center}
%\includegraphics[width=14cm]{./img/2010-1.jpg}
%\end{center}
%
%Set up the apparatus as shown in Figure 1. Put a mass of 50 g on the scale pan and record the equilibrium position $X_0$ of the pointer. Put on the scale pan the unknown weight marked $W$. Without removing $W$ and the 50 g mass in the scale pan, add a load $L$ of 50 g and record the new position of the pointer $X$. Calculate the extension $E = (X - X_0)$. Repeat this process for $L$ = 100 g, 150 g, 200 g and 250 g.
%\begin{enumerate}
%\item Record you conclusions as shown in Table 1.\\[10pt]
%
%Equilibrium position $X_0$..................\\[10pt]
%
%Table 1\\[10pt]
%
%%\begin{center}
%\begin{tabular}{|p{3cm}|p{3cm}|p{3cm}|} \hline
%\multicolumn{1}{|c|}{Load (g)} & \multicolumn{1}{c|}{$X$ (cm)} & \multicolumn{1}{c|}{$E = X - X_0$ (cm)} \\ \hline
%\multicolumn{1}{|c|}{50} & & \\ \hline
%\multicolumn{1}{|c|}{100} & & \\ \hline
%\multicolumn{1}{|c|}{150} & & \\ \hline
%\multicolumn{1}{|c|}{200} & & \\ \hline
%\multicolumn{1}{|c|}{250} & & \\ \hline
%\end{tabular}\\[10pt]
%%\end{center}
%
%\item Plot the graph of load L against absolute value of extension E. The scale of the vertical axis should be chosen to range from 200 g to 300 g.
%\item From the graph, determine the unknown weight marked W, given that L = KE + W where K is a constant.
%\item What does the gradient of the graph represent?
%\item State the sources of errors and precautions that should be taken in the experiment.
%\end{enumerate}
%\end{enumerate}
%\flushright \textbf{(25 marks)}
%\flushleft

\subsubsection{Sample Practical Question}

The aim of this experiment is to determine the mass of a given object ``B'', and the constant of the spring provided.

\begin{center}
\includegraphics{./img/spring-practical.png}
\end{center}

\begin{itemize}
\item[]
\begin{itemize}
\item[(i)] Set up the apparatus as shown in the figure with zero mark of the metre-rule at the top of the rule and record the scale reading by the pointer, $S_0$.
\item[(ii)] Place the object ``B'' and standard weight (mass) W equal to 20 g in the pan and record the new pointer reading $S_1$. Calculate the extension, $e = S_1 - S_0$ in cm.
\item[(iii)] Repeat the procedure in (ii) above with W = 40 g, 60 g, 80 g and 100 g.
\end{itemize}
\item[(a)] Record your results in tabular form as shown below:\\
Table of Results:

\begin{tabular}{|p{2cm}|p{3cm}|p{3cm}|p{3cm}|}\cline{1-1}
\multicolumn{1}{|p{2cm}|}{$S_0 = $}&\multicolumn{2}{c}{} & \multicolumn{1}{p{2.5cm}}{} \\ \hline
\multicolumn{1}{|c|}{Mass} & \multicolumn{1}{c|}{Force, F (N)} & \multicolumn{1}{c|}{Pointer reading $S_1$} & \multicolumn{1}{c|}{Extension}\\
\multicolumn{1}{|c|}{(kg)} & \multicolumn{1}{c|}{} & \multicolumn{1}{c|}{(cm)} & \multicolumn{1}{c|}{$= S_1 - S_0$ (cm)}\\ \hline
\multicolumn{1}{|c|}{0} & \multicolumn{1}{c|}{} & \multicolumn{1}{c|}{} & \multicolumn{1}{c|}{}\\ 
\multicolumn{1}{|c|}{0.02} & \multicolumn{1}{c|}{} & \multicolumn{1}{c|}{} & \multicolumn{1}{c|}{}\\ 
\multicolumn{1}{|c|}{0.04} & \multicolumn{1}{c|}{} & \multicolumn{1}{c|}{} & \multicolumn{1}{c|}{}\\ 
\multicolumn{1}{|c|}{0.06} & \multicolumn{1}{c|}{} & \multicolumn{1}{c|}{} & \multicolumn{1}{c|}{}\\ 
\multicolumn{1}{|c|}{0.08} & \multicolumn{1}{c|}{} & \multicolumn{1}{c|}{} & \multicolumn{1}{c|}{}\\ 
\multicolumn{1}{|c|}{0.10} & \multicolumn{1}{c|}{} & \multicolumn{1}{c|}{} & \multicolumn{1}{c|}{}\\ \hline
\end{tabular}
\item[(b)] Plot graph of Force F (vertical axis) against extension $e$ (horizontal axis).
\item[(c)] Use your graph to evaluate
\begin{itemize}
\item[(i)] mass of B
\item[(ii)] spring constant, K, given that force, extension, constant and weight of B are related as follows:\\
F = K$e$ - B
\end{itemize}
\end{itemize}

%The aim of this experiment is to determine the mass of a given object $B$, and the
%constant of the spring provided.
%
%\begin{center}
%\includegraphics{./img/spring-practical.png}
%\end{center}
%
%\begin{enumerate}
%\item{Set up the apparatus as shown with the zero mark of the meter-rule at the top
%of the rule and record the scale reading, as shown by the pointer, $S_0$.}
%\item{Place the object $B$ and standard weight (mass) $W$ equal to 20~g in the pan
%and record the new pointer reading, $S_1$. Calculate the extension, $e = S_1 - S_0$ in
%cm.}
%\item{Repeat the procedure above with $W$ = 40~g, 60~g, 80~g and 100~g.}
%\item{Record your results in tabular form as shown below:
%
%\begin{center}
%\begin{tabular}{ | c | c | c | c | c | }
%\hline
%$S_0$ & Mass (kg) & Force, $F$ (N) & Pointer Reading $S_1$ (cm) & Extension, $e = S_1 - S_0$ (cm) \\ \hline
%& 0 & & & \\ \hline
%& 0.02 & & & \\ \hline
%& 0.04 & & & \\ \hline
%& 0.06 & & & \\ \hline
%& 0.08 & & & \\ \hline
%& 0.1 & & & \\ \hline
%\end{tabular}
%\end{center}
%
%}%Record your results...
%\item{Plot a graph of force $F$ (vertical axis) against extension $e$ (horizontal axis).}
%\item{Use your graph to evaluate
%\begin{enumerate}
%\item{mass of $B$}
%\item{spring constant, $K$, given that the force, extension, constant and
%weight of $B$ are related as follows: $$F = Ke - B$$}
%\end{enumerate}
%}%Use your graph...
%\end{enumerate}

\subsubsection{Discussion}

This practical has two parts: the first is to find the spring constant $k$, the second is
to find the mass of an unknown object $B$. By looking at the equation above, we
can see that $F$ is the dependent variable, $e$ is the independent variable, $K$ is the slope and
$-B$ is the intercept. When the graph is drawn, $K$ and $B$ can be found easily. Note that the
intercept on the graph will be negative.

The procedure is simply to start from a certain point on the metre rule (it does not
need to be a specific number) and to add masses one at a time, measuring the distance
from your starting point to the new position. This distance is called the extension, $e$. Be
sure that you are not simply reading the metre rule, but are measuring the distance from
the starting point.

\subsection{Simple Pendulum} \index{Practicals! Physics! simple pendulum} \index{Simple pendulum}

With some practice, this experiment should be simple for anyone to perform. The
trick comes with the math and graphing (again, an example is shown below). The
materials can all be local (string, stones, ruler) except for the stopwatches (for which you
should consult the materials section).

The practical usually has one objective: to find the acceleration due to gravity, $g$.
We know that the mass of a pendulum and its angle of deflection (for small angles) do
not affect its period. Therefore we vary only the length $L$ of the pendulum and measure
its period, as shown in the following example question.

\subsubsection{Sample Practical Question}

The aim of this experiment is to determine the magnitude of the acceleration due to
gravity, $g$. Proceed as follows:

\begin{center}
\includegraphics{./img/pendulum.png}
\end{center}


\begin{enumerate}
\item{Make a simple pendulum by suspending a weight on a string 10~cm long from a retort
stand.}
\item{Allow the pendulum to swing for twenty oscillations, using a stopwatch to record the
time. Repeat this procedure for pendulum lengths of 20~cm, 30~cm, 40~cm, and 50~cm.}
\item{Record your results in tabular form as shown below}

\begin{center}
\begin{tabular}{ | c | c | c | c | }
\hline
Pendulum Length l (m) & Time for 20 oscillations (s) & Period $T = \frac{t}{20}$ (s) & $T^2$ ($s^2$) \\ \hline
0.1 & & & \\ \hline
0.2 & & & \\ \hline
0.3 & & & \\ \hline
0.4 & & & \\ \hline
0.5 & & & \\ \hline
\end{tabular}
\end{center}

\item{Plot a graph of $T^2$ (vertical axis) against Pendulum Length (horizontal axis).}
\item{Calculate the slope of the graph.}
\item{Use the slope to calculate the value of $g$.}
\item{What are possible sources of error in this experiment?}

\end{enumerate}

\subsubsection{Discussion}

The period of a pendulum can be calculated using $$T = 2\pi\sqrt{\frac{l}{g}}$$
where $l$ is the length of the pendulum, $T$ is the period and $g$ is the acceleration due to
gravity. By squaring both sides, we get a much easier equation to graph: $$T^2 = 4\pi\frac{l}{g}$$ In this equation we see that $T^2$ is the dependent variable (y-axis) and $l$ is the
independent variable (x-axis), so the slope must be $$\mathrm{slope} = \frac{4\pi}{g}$$ 
When the graph is complete, the value of $g$ can be calculated easily.

Many students are confused by the difference between the time for many oscillations and
the period, which is the time for one oscillation. Be sure that they can change between
the two easily.

Note that pendulum practicals do not always require students to find $g$. Sometimes they are just required to find the relationship between $l$ and $t$. Again, it is essential that students read and understand the examination question, rather than memorize past solutions, and that they have lots of practice in collecting, organizing, and graphing data from a variety of experiments.

\subsection{Principle of Moments} \index{Practicals! Physics! principle of moments} \index{Principle of moments}

This experiment is used to verify the Principle of Moments, or equilibrium, by
balancing a meter rule on a knife-edge with masses at various distances. For this experiment, students need
a solid understanding of the Center of Gravity, the Moment of a force, and equilibrium.
Questions can range from finding the mass of an object to asking for
the mass of the metre rule. They are all
variations on the same practical: using the condition of equilibrium to find mass.

The following example is from the 2011 NECTA exam and asks students to find the mass of a battery using the Principle of Moments. Following is a brief explanation of the alternative practical of finding the mass of a metre rule.

\subsubsection{Finding the Mass of an Object}

\noindent \textbf{Sample Practical Question}\\

\noindent The aim of this experiment is to determine the mass of a given dry cell size ``AA''. Proceed as follows:
\begin{itemize}
\item[(a)] Locate and note the centre of gravity $C$ of the metre rule by balancing it on the knife edge.
\item[(b)] Suspend the 50 g mass at length `$a$' cm on one side of the metre rule and the 20 g mass together with the dry cell at length `$b$' cm on the other side of the metre rule. Fix the 50 g mass at length 30 cm from the fulcrum and adjust the position of the 20 g mass together with the dry cell until the metre rule balances horizontally. Read and record the values of $a$ and $b$ as $a_0$ and $b_0$ respectively.
\item[(c)] Draw the diagram for this experiment.
\item[(d)] By fixing $a = 5$ cm from fulcrum $C$, find its corresponding length $b$.
\item[(e)] Repeat the procedure in (d) above for $a = 10$ cm, 15 cm, 20 cm and 25 cm. Tabulate your results.
\item[(f)] Draw a graph of `$a$' against `$b$' and calculate its slope $G$.
\item[(g)] Calculate $X$ from the equation $50 = \cfrac{b_0}{a_0}(20 + X)$.
\item[(h)] Comment on the value of $\cfrac{b_0}{a_0}$.
\item[(i)] Sate the principle governing this experiment.
\end{itemize}

\noindent \textbf{Discussion}\\
This practical utilizes the Principle of Moments to find the mass of a ``AA'' battery. Initially, a known mass of 50 g is balanced with the (battery $+$ 20 g mass) system. Note that `$a$' and `$b$' are measured \emph{from the fulcrum} and so students should be careful not to just read the cm mark on the ruler where each object is suspended.

Also note that students are required to actually find the centre of gravity $C$ of the ruler rather than assuming it to be the 50 cm mark. This measured value of $C$ is to be used as the starting point for all future measurements of $a$ and $b$.

In part (g), students should recognize the equation $50 = \cfrac{b_0}{a_0}(20 + X)$ as coming from the Principle of Moments. Starting with
$$F_{\mathrm{clockwise}} \times d_{\mathrm{clockwise}} = F_{\mathrm{anticlockwise}} \times d_{\mathrm{anticlockwise}}$$
we get $$(mg)_{\mathrm{clockwise}} \times d_{\mathrm{clockwise}} = (mg)_{\mathrm{anticlockwise}} \times d_{\mathrm{anticlockwise}}$$
or $$(50 \text{g})(g) \times a_0 \text{ cm} = (20 \text{g} + X \text{g})(g) \times b_0 \text{ cm}$$
Canceling $g$ and dividing by $a_0$ reveals
$$50 = \cfrac{b_0}{a_0}(20 + X)$$
where $\cfrac{b_0}{a_0}$ is the ratio of the lever arm distances for the two weights being used. If the mass of the battery $X$ is less than 30 g, this ratio should be greater than 1, but if the mass is greater than 30 g, the ratio should be less than 1.

\subsubsection{Finding the Mass of a Metre Rule}

This question is less frequently seen on NECTA exams as compared to finding an unknown mass. However, it utilizes the same principles of equilibrium and balancing moments, and therefore is a useful alternative practical to ensure that students understand the concept rather than memorizing solutions to one version of the problem.

The mass of a uniform solid object, like a metre rule, is assumed to be at the
center of the object. In the case of the metre rule, we can say that the center of mass is at
the 50~cm mark, directly in the center. If we want it to be in equilibrium, the moments on
either side of a pivot must be equal, or $$ \mathrm{\text{Clockwise moment}} = \mathrm{\text{Anticlockwise moment}}$$
To find the mass of the metre rule itself, we begin by placing a known mass at one
point on the metre rule. We then move the pivot to one side or another until the metre
rule is perfectly balanced in equilibrium. As shown in the diagram below, the pivot will
not be at the 50~cm mark.

\begin{center}
\includegraphics{./img/meter-rule.png}
\end{center}

If the metre rule is in equilibrium, we know that the moments must be equal, or
that $$F_{\mathrm{clockwise}} \times d_{\mathrm{clockwise}} = F_{\mathrm{anticlockwise}} \times d_{\mathrm{anticlockwise}}$$
In this case, the anticlockwise force is the weight of the object, and the distance is that
from the pivot to the object. The clockwise force is the weight of the metre rule, and the
distance is that from the 50~cm mark (center of mass) to the pivot. Therefore our
equation is: $$ W_{\mathrm{rule}} \times d_{\mathrm{rule}} = W_{\mathrm{object}} \times d_{\mathrm{object}} $$
Because the weight of the object is known, and the two distances can be measured, we
can easily calculate the mass and therefore the weight of the metre rule:
$$W_{\mathrm{rule}} = \frac{W_{\mathrm{object}} \times d_{\mathrm{object}}}{d_{\mathrm{rule}}}$$
From this we can calculate mass of the metre rule using $F = mg$.

%==============================================================================
\section{Light}
The light practical typically involves plane mirrors or glass blocks (rectangular
prisms). Presumably you will have already done these practicals with the students in Form
three (refraction) and Form 1 (plane mirrors), but a little practice will make the theory and
execution clear, especially if they can work in groups. The materials you will need are as
follows:
\begin{description}
\item[Cork Board]{Use cardboard for this, about 0.5 to 0.75~cm thick.}
\item[Optical pins]{Use sewing pins or syringe needles. If using syringe needs, be
sure to crimp the ends so students do not prick themselves.}
\item[Protractors]{These are cheap and students are supposed to have them anyway.
Small ones come in local mathematical sets.}
\item[Glass Block / Rectangular Prism]{A simple rectangular piece of 6~mm glass,
about 8~cm by 12~cm, will work.}
\item[Plane Mirror]{You can buy mirror glass in town in small sections for 200/= or
less; it should be available in villages through the local craftsmen if they work on
windows. Alternately, you can smoke one side of a piece of glass to make the
other side like a mirror.}
\end{description}

\subsection{Plane Mirror (Reflection)} \index{Practicals! Physics! plane mirror} \index{Mirrors! plane! practical of}

These are not as common as the rectangular prism, but they come in a variety of
questions:
\begin{itemize}
\item{Placing pins in front of a mirror at different distances and finding the distance of
the image.}
\item{Verifying the Law of Reflection at plane mirrors.}
\item{Placing two mirrors at different relative angles to find the number of images
produced.}
\end{itemize}
These are not overly complicated, but you should definitely practice with your students creating images in
mirrors -- they are not as accustomed to playing with mirrors as you might be.

Given below is an example practical from the 2006 NECTA exam which asks students to find a relationship between object distance and image distance in a plane mirror.

\subsubsection{Sample Practical Question}

Set up the experiment as shown in the diagram below using plane mirror, soft board, three pins and a white sheet of paper.

\begin{center}
\includegraphics[width=8cm]{./img/2006-2-alt.png}
\end{center}

Fix a white sheet of paper on the soft board. Draw a line across the width at about the middle of the white sheet (MP). Draw line ONI perpendicular to MP.

Fix optical pin O to make ON = U = 3 cm. By using plasticine or otherwise, fix plane mirror along portion of MP with O in front of the mirror. With convenient position of eye, E, look into the mirror and fix optical pins A and B to be in line with image, I, of pin O.

Measure and record NI = V. Repeat procedure for U = 6 cm, 9 cm and 12 cm.

\begin{enumerate}
\item[]
\begin{enumerate}
\item[(a)] Tabulate your results as follows:

\quad \begin{tabular}{|c|c|c|c|c|} \hline
U (cm) & 3 & 6 & 9 & 12 \\ \hline
V (cm) & & & & \\ \hline
\end{tabular}

\item[(b)] Plot graph of U against V.
\item[(c)] Calculate slope, $m$ of the graph to the nearest whole number.
\item[(d)] State relationship between U and V.
\item[(e)] Write equation connecting U and V using numerical value of $m$ with symbols U and V.
\item[(f)] From your equation give position of the image when object is touching the face of the mirror.
\end{enumerate}
\end{enumerate}

\subsubsection{Discussion}
For a plane mirror, object distance and image distance are equal. That is, U and V should be approximately equal values for this practical. Note that students should extend line ONI far behind the mirror since they don't know where exactly the image I is going to be. The location of I is found at the intersection of the extended line ONI and the extended line AB connecting the two optical pins.

From the graph, the slope should be found to be 1 after rounding to the nearest whole number. From this, we can see that U = V. When the object is touching the face of the mirror, the object distance U is 0, and so the image distance V will also be 0.

\subsection{Rectangular Prism (Refraction)} \index{Practicals! Physics! refraction} \index{Refraction}

Students will be asked to find the refractive index and/or critical angle of the glass block by varying
the angle of incidence $i$ and measuring the corresponding angles of refraction $r$ as
described in the Mathematics section earlier. They will do this by placing two pins in
front of the prism, which together form a `ray' (the light ray), and then placing two more
pins on the other side of the prism so that, when observed through the prism from either
side, the four pins line up exactly. By drawing the lines that the pins make on the paper,
the refracted ray inside the prism can be easily traced, and the refracted angle measured.
An example question from the 2007 NECTA is given below.

\subsubsection{Sample Practical Question}

The aim of this experiment is to find the refractive index of a glass block. Proceed
as follows:\\

Place the given glass block in the middle of the drawing paper on the drawing
board. Draw lines along the upper and lower edges of the glass block. Remove the
glass block and extend the lines you have drawn. Represent the ends of these line
segments as $SS_1$ and $TT_1$. Draw the normal $NN_1$ to the parallel lines $SS_1$ and
$TT_1$ as shown in the figure below:

\begin{center}
\includegraphics{./img/light-block-1.png}
\end{center}

Draw five evenly spaced lines from O to represent incident rays at different
angles of incidence (10$^\circ$, 20$^\circ$, 30$^\circ$, 40$^\circ$, and 50$^\circ$ from the normal). Replace the
glass block carefully between $SS_1$ and $TT_1$. Stick two pins $P_1$ and $P_2$ as shown in
the figure as far apart as possible along one of the lines drawn to represent an
incident ray. Locate an emergent ray by looking through the block and stick pins
$P_3$ and $P_4$ exactly in line with images $I_1$ and $I_2$ of pins $P_1$ and $P_2$. Draw the
emergent ray and repeat the procedure for all the incident rays you have drawn. Finally draw in the corresponding refracted rays.

\begin{center}
\includegraphics{./img/light-block-2.png}
\end{center}

\begin{enumerate}
\item[]
\begin{enumerate}
\item[(a)]{Record the angles of incidence $i$ and the measured corresponding angles of
refraction $r$ in a table. Your table of results should include the values of
$\sin{i}$ and $\sin{r}$.}
\item[(b)]{Plot the graph of $\sin{i}$ (vertical axis) against $\sin{r}$ (horizontal axis).}
\item[(c)]{Determine the slope of the graph.}
\item[(d)]{What is the refractive index of the glass block used?}
\item[(e)]{Mention any sources of errors in this experiment.}
\end{enumerate}
\end{enumerate}

\subsubsection{Discussion}

In this experiment, pins are used to simulate a ray of light. If all of the pins are
aligned as you look through the block, they act as a single ray. It takes practice to be able
to align the pins while looking through the block, so practice often with your students.

Light slows down as it enters a denser medium, so in order to minimize the time
required to pass through that medium, it changes direction until it moves back into its
original medium. In this case, light is moving from air into glass and then back into air,
so its direction changes while inside the glass, then returns to its original direction when
passing back into air. This effect is called refraction and it depends on the nature of the
media, in this case air and glass. Snell’s law gives us the relationship between the nature
of the media and the resulting angles of incidence and refraction:
$$n_1 \times \sin{i} = n_2 \times \sin{r}$$

In this experiment, the incident angle $i$ is being changed and the refracted angle $r$
is being measured. The refractive index of medium 1 (air) is known as 1.0, so we can use
these three to find the refractive index of medium 2 (glass). On the graph, $\sin{i}$ is the
dependent variable and $\sin{r}$ is the independent variable, so the equation becomes
$$\sin{i} = \sin{r} \frac{n_2}{n_1}$$
In this case the slope must be $\frac{n_2}{n_1}$.

The refractive index of medium air is simply 1.0, so the slope is the refractive index of
medium 2.

This practical is one of the easiest to perform with students because it does not
require much preparation. Syringe needles should be readily available and glass blocks are
cheap, so it is possible to have every students try this themselves many times before
taking the exam.

\subsubsection{Finding the Critical Angle} \index{Critical angle}
Some questions may ask students to find the critical angle of a glass block in addition to its refractive index. The relationship between critical angle, $C$, and refractive index, $n$, for a particular medium is given by 
$$n = \frac{1}{\sin{C}}$$ or $$\frac{\sin{i}}{\sin{r}} = \frac{1}{\sin{C}}$$.

Thus a graph of $\sin{i}$ against $\sin{r}$ can be used to find the critical angle. However, take care to note that we must first take the reciprocal of the slope, i.e.
$$\sin{C} = \frac{1}{n}$$.

This gives us $\sin{C}$, so to get $C$ by itself, we need to use mathematical tables. Turn to the page for Natural Sines and search the table for the 4-figure value you obtained above by taking $\frac{1}{\text{slope}}$. The corresponding row gives the angle in degrees and the column gives the additional minutes of the angle.

For example, say we plot our graph of $\sin{i}$ ($y$-axis) against $\sin{r}$ ($x$-axis) and we calculate the slope to be 1.43. This is the refractive index of the glass block (since we can remember that glass has a refractive index of 1.5, we can do a quick mental check to make sure this makes sense). Then $\sin{C} = \frac{1}{1.43} = 0.6993$. From the mathematical tables, we get $C = 44^\circ 22'$ [6993 falls between 6984 ($18'$) and 6997 ($24'$), so we use the Mean Differences table to add on $4'$ giving a total of $22'$].

Note that the method of finding $n$ and $C$ changes if we are instead told to graph $\sin{r}$ ($y$-axis) against $\sin{i}$ ($x$-axis). Be sure to practice both versions with students to ensure their understanding.

%==============================================================================
\section{Electricity}

This is by far the least attempted practical on the exam, but not because it is
difficult. The electricity practical, if properly set up, is one of the easiest to perform. It
can appear in many different forms but will typically involve a simple circuit and some
kind of variable resistor in order to measure current or EMF for different resistances. The
materials you will probably need are as follows:

\begin{description}
\item[Connecting Wires]{Use speaker wire; it is cheap and available in most villages
and towns.}
\item[Voltmeters, Ammeters, Galvanometers]{This is unavoidable; you can get full
digital multimeters in town for about 10,000/=, galvanometers can be found in
any lab store or can be made using a compass and insulated copper wire.}
\item[Batteries]{Two to four D-size batteries should easily be enough for these experiments. Try to avoid Tiger brand if possible. Panasonic is highly superior in quality for roughly the same price.}
\item[Resistance Wire]{These are used to make small resistors for
the metre bridge or potentiometer. The most common type of wire to use is
nichrome, which can be found in a hardware or lab store. Steel will also work, though it
is less resistant and therefore harder to measure.}
\item[Metre Bridges]{See the activity that describes the construction of a metre bridge
and potentiometer. It is best to make both together as the construction is almost
identical and both are used frequently.}
\item[Variable Resistor (Rheostat)]{This is optional as it is typically only used to set a
level that can be easily read by the voltmeter. However, if you are using a
multimeter, you can simply change the magnitude setting on the multimeter to
account for unusually low or high resistances.}
%See How to Use a Rheostat
\item[Soldering Iron]{Not required, but may be a good investment for making reliable battery connections. Using electrical tape can lead to inconsistencies. Check large towns.}
\end{description}

\subsection{Potentiometers} \index{Practicals! Physics! potentiometers} \index{Potentiometers}

This experiment is very simple but requires the correct materials, namely the
meter bridge/potentiometer described above. A complete circuit is created with a switch
(optional), power source, variable resistor and 1~m of bare resistance wire, all in series.

The potentiometer itself is simple to construct; all preparation is done by the teacher, so
the student simply follows the instructions as shown in the following example from the 2007 NECTA.

\subsubsection{Sample Practical Question}

The aim of this experiment is to determine the potential fall along a uniform
resistance wire carrying a steady current. Proceed as follows:

\begin{center}
\includegraphics[width=10cm]{./img/meter-bridge-1.png}
\end{center}


Connect up the circuit as shown in the figure. Adjust the rheostat so that when the sliding
contact J is near B and the key is closed the voltmeter V indicates an almost full scale
of deflection. Do not alter the rheostat again.

Close key K and make contact with J, so that AJ = 10~cm. Record the potential
difference V volts between A and J as registered on the voltmeter.

Repeat this procedure for AJ = 20~cm, 30~cm, 50~cm, and 70~cm.

\begin{enumerate}
\item[]
\begin{enumerate}
\item[(a)]{Tabulate your results for the values of AJ and V.}
\item[(b)]{Plot a graph of V (vertical axis) against AJ (horizontal axis).}
\item[(c)]{Calculate the slope of the graph.}
\item[(d)]{What is your comment on the slope?}
\item[(e)]{State any precautions on the experiment.}
\end{enumerate}
\end{enumerate}

\subsubsection{Discussion}

This is a simple test of the relationship between the length of a wire and its
resistance, which we know is $$R=\frac{\rho l}{A} $$

Where $l$ is the length of the wire, $\rho$ is the resistivity of the wire, and $A$ is the cross-sectional
area of the wire. We expect that as the length of wire increases, its potential difference
will also increase. This is because the resistance (and therefore potential difference) of a wire is
directly related to its length. The voltmeter in this experiment is measuring just the
potential difference over the length of wire (10~cm, 20~cm, etc.), so if we use Ohm’s Law to say that $V = IR$, we can write:\\
$$ V = \frac{I \rho l}{A} $$
In this experiment, $I$, $\rho$ and $A$ are all constant, so the slope is
$$\mathrm{slope} = \frac{I \rho}{A} $$
Though it is not asked for directly in the question, we can find the resistivity, $\rho$, by measuring $I$ with an ammeter/galvanometer and $A$ with vernier calipers or a micrometer screw gauge.

\subsection{Metre Bridges} \index{Practicals! Physics! metre bridge} \index{Metre bridge}

A metre bridge resembles a potentiometer, except that it uses a galvanometer to
measure the difference in current between two points on the circuit, hence the name
``bridge.'' The same materials can be used as with the potentiometer, though it is best to
use small coils of resistance wire for the small resistors (between 3$\Omega$ and 20$\Omega$ is a good resistance). A galvanometer can be made easily if one is not available.

\begin{center}
\includegraphics[width=10cm]{./img/meter-bridge-2.png}
\end{center}

Resistors $R_1$ and $R_2$ have different resistances, but they should be somehow similar so
that one resistor does not take all of the current (this will make it difficult to measure the
length to the galvanometer). About 5$\Omega$ and 10$\Omega$, for example, would work well.

However, for the sake of the practical, one resistor should not be known; the objective of
the practical is to find the unknown resistance. The long wire along the bottom edge is a
metre of nichrome wire or other resistance wire. One terminal of the galvanometer is
connected between the two resistors, and the other terminal is connected to a flying wire (or jockey)
that is free to move along the length of the nichrome wire.

The practical instructs you to move the galvanometer's flying wire back and forth
along the nichrome wire until it reads zero. At this point, we know that no current is
passing through the galvanometer, so the potential difference across it is zero. This
means that the current flowing through $R_1$ is the same as that current flowing through $R_2$,
and the current flowing through the nichrome wire is constant. From this we can
conclude that
$$ \frac{R_1}{L_1} = \frac{R_2}{L_2} $$
or that the ratio of the two resistors is equal to the ratio of distances from the flying wire
to either end of the nichrome wire. The resistance of one resistor (say, $R_1$) is known and
the lengths $L_1$ and $L_2$ can be measured from the flying wire to either side of the
nichrome wire. Using the ratio above, we can easily calculate the unknown resistance $R_2$.

An example is given below from the 2006 NECTA exam.

\subsubsection{Sample Practical Question}
You are required to determine the unknown resistance labeled X using a metre bridge circuit. Connect your circuit as shown below, where $R$ is a resistance box, G is a galvanometer, J is a jockey and others are common circuit components.

\begin{center}
\includegraphics[width=12cm]{./img/2006-3-alt.png}
\end{center}

Procedure:\\

With $R$ = 1 $\Omega$, obtain a balance point on a metre bridge wire AB using a jockey J. Note the length $l$ in centimetres. Repeat the experiment with R equal to 2 $\Omega$, 4 $\Omega$, 7 $\Omega$ and 10 $\Omega$.\\

Tabulate your results for $R$, $l$ and $^1/_l$.

\begin{itemize}
\item[(a)]
\begin{itemize}
\item[(i)] Plot a graph of $R$ (vertical axis) against $^1/_l$ (horizontal axis).
\item[(ii)] Determine the slope S of your graph.
\item[(iii)] Using your graph, find the value of $R$ for which $^1/_l = 0.02$.
\end{itemize}
\item[(b)] Read and record the intercept $R_0$ on the vertical axis.
\item[(c)] Given that,\\
\quad \quad $R = \cfrac{100 \text{X}}{l} - \text{X}$\\
Use the equation and your graph to determine the value of X.
\item[(d)] Comment on your results in (a)(iii), (b) and (c) above.
\end{itemize}

\subsubsection{Discussion}
The procedure for this question is similar to most other wheatstone bridge problems: vary a known resistor and see how it affects the relative lengths in resistance wire required to balance the potential difference and give no current through the galvanometer. It may not be obvious at first, however, where the equation $R = \frac{100 \text{X}}{l}$ - X comes from.

Starting with the balancing ratio for a wheatstone bridge, $$ \frac{R_1}{L_1} = \frac{R_2}{L_2} $$ we can solve for the unknown resistor $$R_2 = R_1\left(\frac{L_2}{L_1}\right)$$. 

Recall that $L_1$ and $L_2$ are the corresponding lengths from either end of the metre rule to the jockey (in cm), and so taking them together, we get $$L_1 + L_2 = 100$$. Dividing both sides by $L_1$ gives $$1 + \frac{L_2}{L_1} = \frac{100}{L_1}$$. Then solving for $\frac{L_2}{L_1}$, $$\frac{L_2}{L_1} = \frac{100}{L_1} - 1$$. Now substitute this into our previous equation for $R_2$: $$R_2 = R_1\left(\frac{100}{L_1} - 1\right)$$. Replacing $R_2$ with $R$, $L_1$ with $l$ and $R_1$ with X for this problem and distributing gives, $$R = \frac{100 \text{X}}{l} - \text{X}$$.

From this equation, we have dependent variable $R$, independent variable $\frac{1}{l}$, slope $100 \text{X}$ and $y$-intercept $-\text{X}$. So we can obtain the value of the unknown resistor X either by using the $y$-intercept (note the resistance is a positive value) or by taking the slope divided by 100.

\subsection{Ohm’s Law} \index{Practicals! Physics! Ohm's law} \index{Ohm's law}

The practical may give any kind of experiment to use or verify Ohm’s Law in a
simple circuit. Finding the e.m.f. and internal resistance of cells appears frequently. Students should be very familiar with the law, as well as the factors that determine resistance in a wire and the effect of internal resistance of a cell on a circuit. Given below is an example problem taken from the 2011 NECTA exam.

\subsubsection{Sample Practical Question}
You are provided with an ammeter, A, resistance box, R, dry cell, D, a key, K and connecting wires. Proceed as follows:
\begin{enumerate}
\item[]
\begin{enumerate}
\item[(a)] Connect the circuit in series.
\item[(b)] Put $R$ = 1 $\Omega$ and quickly read the value of current $I$ on the ammeter.
\item[(c)] Repeat procedure (b) above for $R$ = 2 $\Omega$, 3 $\Omega$, 4 $\Omega$ and 5 $\Omega$. Record your results in a tabular form.
\item[(d)] Draw the circuit diagram for this experiment.
\item[(e)] Plot the graph of $R$ against $\cfrac{1}{I}$.
\item[(f)] Determine the slope of the graph.
\item[(g)] If the graph obeys the equation $R=\cfrac{E}{I}-r$, then
\begin{enumerate}
\item[(i)] suggest how $E$ and $r$ may be evaluated from your graph.
\item[(ii)] compute $E$.
\item[(iii)] compute $r$.
\end{enumerate}
\item[(h)] State one source of error and suggest one way of minimizing it.
\item[(i)] Suggest the aim of this experiment.
\end{enumerate}
\end{enumerate}

\subsubsection{Discussion}
To see where the equation $R=\cfrac{E}{I}-r$ comes from, first start with Ohm's Law, $V = IR$. Accounting for the internal resistance of the cell, $r$, this becomes $$V = I(R + r)$$. To solve for resistance, we divide both sides by $I$, which gives $$(R + r) = \frac{V}{I}$$. From this we can see that, using $E$ as e.m.f. for this problem, $$R=\cfrac{E}{I}-r $$. 

In this form, the equation resembles the classic $y = mx + c$, where $R$ is the dependent variable, $\frac{1}{I}$ the independent variable, $E$ is the slope and $-r$ is the y-intercept (note the internal resistance is a positive value).

%\pagebreak
%
%\section{NECTA Past Papers}
%\setcounter{secnumdepth}{1}
%\subsection{2011 - PHYSICS 2A ACTUAL PRACTICAL A}

\begin{enumerate}
\item The aim of this experiment is to determine the mass of a given dry cell size ``AA''. Proceed as follows:
\begin{itemize}
\item[(a)] Locate and note the centre of gravity $C$ of the metre rule by balancing it on the knife edge.
\item[(b)] Suspend the 50 g mass at length `$a$' cm on one side of the metre rule and the 20 g mass together with the dry cell at length `$b$' cm on the other side of the metre rule. Fix the 50 g mass at length 30 cm from the fulcrum and adjust the position of the 20 g mass together with the dry cell until the metre rule balances horizontally. Read and record the values of $a$ and $b$ as $a_0$ and $b_0$ respectively.
\item[(c)] Draw the diagram for this experiment.
\item[(d)] By fixing $a = 5$ cm from fulcrum $C$, find its corresponding length $b$.
\item[(e)] Repeat the procedure in (d) above for $a = 10$ cm, 15 cm, 20 cm and 25 cm. Tabulate your results.
\item[(f)] Draw a graph of `$a$' against `$b$' and calculate its slope $G$.
\item[(g)] Calculate $X$ from the equation $50 = \cfrac{b_0}{a_0}(20 + X)$.
\item[(h)] Comment on the value of $\cfrac{b_0}{a_0}$.
\item[(i)] Sate the principle governing this experiment.
\end{itemize}
\end{enumerate}

\flushright \textbf{(25 marks)}
\begin{enumerate}
\item[2.] You are provided with an ammeter, A, resistance box, R, dry cell, D, a key, K and connecting wires. Proceed as follows:
\begin{enumerate}
\item[(a)] Connect the circuit in series.
\item[(b)] Put $R$ = 1 $\Omega$ and quickly read the value of current $I$ on the ammeter.
\item[(c)] Repeat procedure (b) above for $R$ = 2 $\Omega$, 3 $\Omega$, 4 $\Omega$ and 5 $\Omega$. Record your results in a tabular form.
\item[(d)] Draw the circuit diagram for this experiment.
\item[(e)] Plot the graph of $R$ against $\cfrac{1}{I}$.
\item[(f)] Determine the slope of the graph.
\item[(g)] If the graph obeys the equation $R=\cfrac{E}{I}-r$, then
\begin{enumerate}
\item[(i)] suggest how $E$ and $r$ may be evaluated from your graph.
\item[(ii)] compute $E$.
\item[(iii)] compute $r$.
\end{enumerate}
\item[(h)] State one source of error and suggest one way of minimizing it.
\item[(i)] Suggest the aim of this experiment.
\end{enumerate}

\end{enumerate}

\flushright \textbf{(25 marks)}
\flushleft
%\pagebreak
%\section{2010 - PHYSICS 2A ALTERNATIVE A PRACTICAL}

\begin{enumerate}
\item[1.] The aim of this experiment is to find the mass of the unknown load labeled ``$W$'' and the spring constant $K$. Proceed as follows:

\begin{center}
\includegraphics[width=14cm]{./img/2010-1-alt.png}
\end{center}

Set up the apparatus as shown in Figure 1. Put a mass of 50 g on the scale pan and record the equilibrium position $X_0$ of the pointer. Put on the scale pan the unknown weight marked $W$. Without removing $W$ and the 50 g mass in the scale pan, add a load $L$ of 50 g and record the new position of the pointer $X$. Calculate the extension $E = (X - X_0)$. Repeat this process for $L$ = 100 g, 150 g, 200 g and 250 g.
\begin{enumerate}
\item[(a)] Record you conclusions as shown in Table 1.\\[10pt]

Equilibrium position $X_0$..................\\[10pt]

Table 1\\[10pt]

%\begin{center}
\begin{tabular}{|p{3cm}|p{3cm}|p{3cm}|} \hline
\multicolumn{1}{|c|}{Load (g)} & \multicolumn{1}{c|}{$X$ (cm)} & \multicolumn{1}{c|}{$E = X - X_0$ (cm)} \\ \hline
\multicolumn{1}{|c|}{50} & & \\ \hline
\multicolumn{1}{|c|}{100} & & \\ \hline
\multicolumn{1}{|c|}{150} & & \\ \hline
\multicolumn{1}{|c|}{200} & & \\ \hline
\multicolumn{1}{|c|}{250} & & \\ \hline
\end{tabular}\\[10pt]
%\end{center}

\item[(b)] Plot the graph of load L against absolute value of extension E. The scale of the vertical axis should be chosen to range from 200 g to 300 g.
\item[(c)] From the graph, determine the unknown weight marked W, given that L = KE + W where K is a constant.
\item[(d)] What does the gradient of the graph represent?
\item[(e)] State the sources of errors and precautions that should be taken in the experiment.
\end{enumerate}
\end{enumerate}
\flushright \textbf{(25 marks)}

\pagebreak

\begin{enumerate}
\item[2.] The aim of this experiment is to determine the refractive index of water. Proceed as follows:

\begin{center}
\includegraphics[width=10cm]{./img/2010-2-alt.png}
\end{center}

\begin{enumerate}
\item[(a)] Arrange your apparatus as in Figure 2. Put about 150 cm$^3$ of clear water in the measuring cylinder. Drop an office pin at the bottom so that it rests touching the wall of the cylinder.
\item[(b)] Look in the cylinder from Figure 2. Use another office pin as a search pin, move it up and down outside the cylinder, and locate the image position by no parallax method. Locate the image position of the ruler. Measure and record the depth ($H_1$) of the image. Measure and record the depth ($H_2$) of water. Repeat the experiment with 175 cm$^3$, 200 cm$^3$, 225 cm$^3$ and 250 cm$^3$ of water in the measuring cylinder.
\item[(c)] 
\begin{enumerate}
\item[(i)] Record in Table 2 your values of $H_1$ and $H_2$ corresponding to the volumes of water in the measuring cylinder.\\[10pt]

Table 2\\[10pt]

%\begin{center}
\begin{tabular}{|p{3cm}|p{3cm}|p{3cm}|} \hline
\multicolumn{1}{|c|}{Volume of water V (cm)} & \multicolumn{1}{c|}{$H_1$} & \multicolumn{1}{c|}{$H_2$} \\ \hline
\multicolumn{1}{|c|}{150} & & \\ \hline
\multicolumn{1}{|c|}{175} & & \\ \hline
\multicolumn{1}{|c|}{200} & & \\ \hline
\multicolumn{1}{|c|}{225} & & \\ \hline
\multicolumn{1}{|c|}{250} & & \\ \hline
\end{tabular}\\[10pt]
%\end{center}

\item[(ii)] Plot the graph of $H_2$ versus $H_1$.
\item[(iii)] Determine the slope of the graph.
\item[(iv)] What is the physical meaning of the slope?
\item[(v)] State sources of error in this experiment.
\end{enumerate}
\end{enumerate}
\end{enumerate}
\flushright \textbf{(25 marks)}

\pagebreak

\begin{enumerate}
\item[3.] The aim of this experiment is to determine the resistivity of an electrical conductor $P$.

\begin{center}
\includegraphics[width=10cm]{./img/2010-3-alt.png}
\end{center}

With $P$ having a length $l = 50$ cm, connect up the circuit as shown in Figure 3. Close one key S and adjust the rheostat R so that the current in $P$ is 0.20 A. Record the current $I$ and the potential difference $V$ between its ends.\\[10pt]

Repeat the procedure with current $I = $0.30 A, 0.40 A, 0.50 A and 0.60 A.
\begin{enumerate}
\item[(a)] Record your results in Table 3.\\[10pt]

Table 3\\[10pt]

%\begin{center}
\begin{tabular}{|p{3cm}|p{3cm}|} \hline
\multicolumn{1}{|c|}{Current $I$ (A)} & \multicolumn{1}{c|}{P.d. (volts)} \\ \hline
\multicolumn{1}{|c|}{0.20} &  \\ \hline
\multicolumn{1}{|c|}{0.30} &  \\ \hline
\multicolumn{1}{|c|}{0.40} &  \\ \hline
\multicolumn{1}{|c|}{0.50} &  \\ \hline
\multicolumn{1}{|c|}{0.60} &  \\ \hline
\end{tabular}\\[10pt]
%\end{center}

\item[(b)] Plot a graph of $V$ against $I$ and calculate the slope $G$.
\item[(c)] Deduce the resistivity of the conductor $P$ given that; $\rho = \cfrac{G\pi d^2}{4l}$.\\[10pt]
Where $\rho$ = resistivity\\
$d$ = diameter of $P$ (measured using the micrometer screw gauge provided).
\end{enumerate}
\end{enumerate}

\flushright \textbf{(25 marks)}
\flushleft
%\pagebreak
%\section{2009 - PHYSICS 2A ALTERNATIVE A PRACTICAL}

\begin{enumerate}
\item[1.] In this experiment you are required to find the relationship between the length of a simple pendulum and its period. Proceed as follows:
\begin{itemize}
\item[(a)] Suspend a simple pendulum of length L = 100 cm. Displace the pendulum through a small angle so that it swings parallel to the edge of the bench or table, determine the time for 20 oscillations. Continue reducing the length of the pendulum by 10 cm each time and obtain a total of six readings.
\item[(b)] Record your readings in a table as shown below.


\begin{tabular}{|p{2.5cm}|p{2.5cm}|p{2.5cm}|p{2.5cm}|p{2.5cm}|} \hline
Length of pendulum L (cm) & Log$_{10}$L & Time for 20 oscillations & Period T & Log$_{10}$T \\ \hline
&&&& \\ 
&&&& \\ 
&&&& \\ 
&&&& \\ 
&&&& \\ 
&&&& \\ 
&&&& \\ 
&&&& \\ \hline
\end{tabular}\\[10pt]

\noindent Assuming that T $ \propto $ L$^a$, we have T = $k$L$^a$ and taking logarithms to base ten on both sides we get $\log_{10}$T = $a\log_{10}$L + $\log_{10}k$.

\begin{itemize}
\item[(i)] Plot a graph of $\log_{10}$T (vertical axis) against $\log_{10}$L (horizontal axis) hence determine the values of $a$ and $k$ each correct to one decimal place.
\item[(ii)] From your answer in (i) above write down the values of $a$ and $k$ each in the form of $\cfrac{b}{c}$ where $b$ and $c$ are integers (i.e. whole numbers).
\item[(iii)] From the assumption and your answer in (ii) deduce the form of the equation governing the motion of the simple pendulum.
\end{itemize}

\end{itemize}
\end{enumerate}
\flushright \textbf{(25 marks)}


\begin{enumerate}
\item[2.] The aim of this experiment is to determine the refractive index $\eta$ of a given glass block.

\begin{center}
\includegraphics[width=8cm]{./img/2009-2-alt.png}
\end{center}

Place the rectangular glass block on the white paper on a drawing board. Using a pencil trace the outline of the block. Remove the glass block and draw a normal NOM near the left end of the block (Figure 1).\\[10pt]

\noindent Using a protractor and a pencil measure $\theta = 20^\circ$, draw a line making the angle $20^\circ$ with the surface RR of the block. Erect two pins T$_1$ and T$_2$ on this line and at a suitable distance from one another. Return the block and erect the pins T$_3$ and T$_4$ at positions such that they lie in a straight line with pins T$_1$ and T$_2$ as seen through the block. Now remove the block and draw a complete path of the ray (Figure 1).\\[10pt]

\noindent Measure the length MN$'$ and ON$'$: Repeat the procedure for values of $\theta = 30^\circ, 40^\circ$ and $60^\circ$ respectively. In each case make a drawing on a fresh part of the drawing paper.

\begin{itemize}
\item[(a)] Record the values of $\theta$, MN$'$, ON$'$, $\cfrac{\text{MN}'}{\text{ON}'}$ and $\cos \theta$ in a tabular form.
\item[(b)] Plot a graph of $\cfrac{\text{MN}'}{\text{ON}'}$ against $\cos \theta$.
\item[(c)] Find the slope $G$ of the graph.
\item[(d)] Calculate the value of the refractive index $\eta$; given that $G = \cfrac{1}{\eta}$.
\item[(e)] State two sources of errors. \hfill \textbf{(25 marks)}
\end{itemize}

\end{enumerate}


\begin{enumerate}
\item[3.] The aim of this experiment is to verify Ohm's Law.

\begin{center}
\includegraphics[width=7cm]{./img/2009-3-alt.png}
\end{center}

\begin{itemize}
\item[(a)] Set up the apparatus as shown on Figure 2, close switch S. Adjust the Rheostat Rh by sliding slowly from one end, read and record the value V of the voltmeter and current I of the ammeter.
\item[(b)] Repeat the experiment by changing the Rheostat slider to obtain about five pair of readings.\\[10pt]

\textbf{NB:} Adjust the Rheostat until when the pointer is exactly on the division of the metre scale.\\[10pt]

Table of results\\[10pt]
\begin{tabular}{|c|c|c|c|c|c|c|c|}\hline
V (V)&&&&&&&\\ \hline
I (A)&&&&&&& \\ \hline
\end{tabular}

\item[(c)] Plot a graph of V (vertical axis) against I (horizontal axis).
\item[(d)] 
\begin{itemize}
\item[(i)] Find the slope of the graph.
\item[(ii)] What is the relation between V and I?
\item[(iii)] Find the resistance R. \hfill \textbf{(25 marks)}
\end{itemize}
\end{itemize}

\end{enumerate}
\flushleft

%\pagebreak
%\subsection{2007 - PHYSICS 2A ALTERNATIVE A PRACTICAL}

\begin{enumerate}
\item[1.] The aim of this experiment is to determine the mass of a given object ``B'', and the constant of the spring provided.

\begin{center}
\includegraphics[width=7cm]{./img/2007-1-alt.png}
\end{center}

\begin{itemize}
\item[]
\begin{itemize}
\item[(i)] Set up the apparatus as shown in Fig. 1 with zero mark of the metre-rule at the top of the rule and record the scale reading by the pointer, $S_0$.
\item[(ii)] Place the object ``B'' and standard weight (mass) W equal to 20 g in the pan and record the new pointer reading $S_1$. Calculate the extension, $e = S_1 - S_0$ in cm.
\item[(iii)] Repeat the procedure in (ii) above with W = 40 g, 60 g, 80 g and 100 g.
\end{itemize}
\item[(a)] Record your results in tabular form as shown below:\\
Table of Results:

\begin{tabular}{|p{2cm}|p{3cm}|p{3cm}|p{3cm}|}\cline{1-1}
\multicolumn{1}{|p{2cm}|}{$S_0 = $}&\multicolumn{2}{c}{} & \multicolumn{1}{p{2.5cm}}{} \\ \hline
\multicolumn{1}{|c|}{Mass} & \multicolumn{1}{c|}{Force, F (N)} & \multicolumn{1}{c|}{Pointer reading $S_1$} & \multicolumn{1}{c|}{Extension}\\
\multicolumn{1}{|c|}{(kg)} & \multicolumn{1}{c|}{} & \multicolumn{1}{c|}{(cm)} & \multicolumn{1}{c|}{$= S_1 - S_0$ (cm)}\\ \hline
\multicolumn{1}{|c|}{0} & \multicolumn{1}{c|}{} & \multicolumn{1}{c|}{} & \multicolumn{1}{c|}{}\\ 
\multicolumn{1}{|c|}{0.02} & \multicolumn{1}{c|}{} & \multicolumn{1}{c|}{} & \multicolumn{1}{c|}{}\\ 
\multicolumn{1}{|c|}{0.04} & \multicolumn{1}{c|}{} & \multicolumn{1}{c|}{} & \multicolumn{1}{c|}{}\\ 
\multicolumn{1}{|c|}{0.06} & \multicolumn{1}{c|}{} & \multicolumn{1}{c|}{} & \multicolumn{1}{c|}{}\\ 
\multicolumn{1}{|c|}{0.08} & \multicolumn{1}{c|}{} & \multicolumn{1}{c|}{} & \multicolumn{1}{c|}{}\\ 
\multicolumn{1}{|c|}{0.10} & \multicolumn{1}{c|}{} & \multicolumn{1}{c|}{} & \multicolumn{1}{c|}{}\\ \hline
\end{tabular}
\item[(b)] Plot graph of Force F (vertical axis) against extension $e$ (horizontal axis).
\item[(c)] Use your graph to evaluate
\begin{itemize}
\item[(i)] mass of B
\item[(ii)] spring constant, K, given that force, extension, constant and weight of B are related as follows:\\
F = K$e$ - B
\end{itemize}
\end{itemize}

\end{enumerate}
\flushright \textbf{(25 marks)}



\begin{enumerate}
\item[2.] The aim of this experiment is to find the refractive index of a glass block. Proceed as following:\\[10pt]

Place the given glass block in the middle of the drawing paper on the drawing board. Draw lines along the upper and lower edge of the glass block. Remove the glass block and extend the line you have drawn. Represent the ends of those line segments as SS$^1$ and TT$^1$. Draw the normal NN$^1$ to the parallel lines SS$^1$ and TT$^1$ as shown in Fig. 2(a).

\begin{center}
\includegraphics[width=8cm]{./img/2007-2a-alt.png}
\end{center}

Draw five evenly spaced lines from O to represent incident rays at different angles of incidence (10$^\circ$, 20$^\circ$, 30$^\circ$, 40$^\circ$ and 50$^\circ$ from the normal). Replace the glass block carefully between SS$^1$ and TT$^1$. Stick two pins P$_1$ and P$_2$ as shown in Fig. 2(b) as far apart as possible along one of the lines drawn to represent an incident ray. Locate an emergent ray by looking through the block and stick pins P$_3$ and P$_4$ exactly in line with images I$_1$ and I$_2$ of pins P$_1$ and P$_2$. Draw the emergent ray and repeat the procedure for all the incident rays you have drawn. Finally draw in the corresponding refracted rays.\\[10pt]

NOTE: The drawing paper should be handed in together with other answer sheets.

\begin{center}
\includegraphics[width=10cm]{./img/2007-2b-alt.png}
\end{center}

\begin{itemize}
\item[(a)] Record the angles of incidence I and the measured corresponding angles of refraction ``r'' in a table. Your table of results should include the values of $\sin$ I and $\sin$ r.
\item[(b)] Plot the graph of $\sin$ I (vertical - axis) against $\sin$ r (horizontal - axis).
\item[(c)] Determine the slope of the graph.
\item[(d)] What is the refractive index of the glass block used?
\item[(e)] Mention any sources of errors in this experiment.
\end{itemize}

\end{enumerate}
\flushright \textbf{(25 marks)}


\begin{enumerate}
\item[3.] The aim of this experiment is to determine the potential fall along a uniform resistance wire carrying a steady current.\\[10pt]

Proceed as follows:

\begin{center}
\includegraphics[width=12cm]{./img/2007-3-alt.png}
\end{center}

Connect up the circuit as shown in Fig. 3. Adjust the rheostat so that when the sliding contact J is near B, and the key is closed the voltmeter V indicates an almost full scale deflection. Do not alter the rheostat again.\\

Close key K and make contact with J, so that AJ = 10 cm. Record the potential different V volts between A and J as registered on the voltmeter.\\

Repeat this procedure for AJ = 20 cm, 30 cm, 50 cm and 70 cm.

\begin{itemize}
\item[(a)] Tabulate your results for the values of AJ and V.
\item[(b)] Plot a graph of V (vertical axis) against AJ (horizontal axis).
\item[(c)] Calculate the slope of the graph.
\item[(d)] What is your comment on the slope?
\item[(e)] State any precautions on the experiment.
\end{itemize}

\end{enumerate}
\flushright \textbf{(25 marks)}
\flushleft
%\pagebreak
%\subsection{2006 - PHYSICS 2A ALTERNATIVE A PRACTICAL}

\begin{enumerate}
\item[1.] In this experiment you are required to determine the mass of unknown object ``X''.

\begin{center}
\includegraphics[width=10cm]{./img/2006-1-alt.png}
\end{center}

Assemble the pieces of apparatus as shown in Figure 1, with zero mark scale of the rule at the lower most end.\\

Record the reading of the position of pointer on the scale of metre-rule when the pan is empty as $S_0$.\\

Put 20 g to the pan and record pointer reading $S$.\\

Find extension $e = S - S_0$ cm.\\

Repeat the procedure for mass of 40 g, 60 g, 80 g and 100 g. Put object X on the pan and record its pointer reading.

\begin{itemize}
\item[(a)] Summarize your results in a table as follows:\\[10pt]
\begin{center}
\begin{tabular}{|l|c|c|c|c|c|c|} \hline
Mass on pan (g) &20&40&60&80&100&X \\ \hline
Pointer reading (cm) &&&&&& \\ \hline
Extension, $e = S - S_0$ (cm) &&&&&& \\ \hline
\end{tabular} \\[10pt]
\end{center}
\item[(b)] Plot graph of mas against extension (m Vs. $e$).
\item[(c)] Find slope, P, of your graph.
\item[(d)] Find mass X.
\item[(e)] Find Q, given that Q = P $\times$ $e_\text{x}$, where $e_\text{x}$ is extension of X.
\item[(f)] Comment on Q and X.
\end{itemize}


\end{enumerate}

\begin{enumerate}
\item[2.] Set up the experiment as shown in the diagram below using plane mirror, soft board, three pins and a white sheet of paper.

\begin{center}
\includegraphics[width=6cm]{./img/2006-2-alt.png}
\end{center}

Fix a white sheet of paper on the soft board. Draw a line across the width at about the middle of the white sheep (MP). Draw line ONI perpendicular to MP.\\

Fix optical pin O to make ON = U = 3 cm. By using plasticine or otherwise, fix plane mirror along portion of MP with O in front of the mirror. With convenient position of eye, E, look into the mirror and fix optical pins A and B to be in line with image, I, of pin O.\\

Measure and record NI = V. Repeat procedure for U = 6 cm, 9 cm and 12 cm.

\begin{itemize}
\item[(a)] Tabulate your results as follows:\\[10pt]

\quad \quad \begin{tabular}{|l|c|c|c|c|} \hline
U (cm) &3&6&9&12 \\ \hline
V (cm) &&&& \\ \hline
\end{tabular} \\[10pt]

\item[(b)] Plot graph of U against V.
\item[(c)] Calculate slope, m, of the graph to the nearest whole number.
\item[(d)] State relationship between U and V.
\item[(e)] Write equation connecting U and V using numerical value of m with symbols U and V.
\item[(f)] From your equation give position of the image when object is touching the face of the mirror.
\end{itemize}

\end{enumerate}


\begin{enumerate}
\item[3.] You are required to determine the unknown resistance labeled X using a metre bridge circuit. Connect your circuit as shown below, where R is a resistance box, G is a galvanometer, J is a jockey and others are common circuit components.

\begin{center}
\includegraphics[width=14cm]{./img/2006-3-alt.png}
\end{center}

Procedure:\\

With R = 1 $\Omega$, obtain a balance point on a metre bridge wire AB using a jockey J. Note the length $l$ in centimetres. Repeat the experiment with R equal to 2 $\Omega$, 4 $\Omega$, 7 $\Omega$ and 10 $\Omega$.\\

Tabulate your results for R, $l$ and $^1/_l$.

\begin{itemize}
\item[(a)]
\begin{itemize}
\item[(i)] Plot a graph of R (vertical axis) against $^1/_l$ (horizontal axis).
\item[(ii)] Determine the slope S of your graph.
\item[(iii)] Using your graph, find the value of R for which $^1/_l = 0.02$.
\end{itemize}
\item[(b)] Read and record the intercept R$_0$ on the vertical axis.
\item[(c)] Given that,\\
\quad \quad R = $\cfrac{100\text{X}}{l}$ - X\\
Use the equation and your graph to determine the value of X.
\item[(d)] Comment on your results in (a)(iii), (b) and (c) above.
\end{itemize}

\end{enumerate}
%\pagebreak
%\section{2005 - PHYSICS 2A ALTERNATIVE A PRACTICAL}

\begin{enumerate}
\item[1.] The aim of this experiment is to determine the mass of unknown weight labelled \textbf{X} and the force constant of the spring \textbf{k}.

\begin{center}
\includegraphics[width=12cm]{./img/2005-1-alt.png}
\end{center}

Set up the apparatus provided as shown in figure 1 above. Add 50 g mass on to the weight pan so that any ``kinks'' in the spring are removed. Leave this weight for the whole experiment but ignore it in all readings. Record the scale reading $S_0$. Add 50 g on to the weight pan and record the new scale reading $S$. Calculate the extension ($e = S - S_0$) caused by the weight. Repeat with different weights ($W$) to obtain at least five readings. Tabulate your results. Replace the weights ($W$) by the weight \textbf{X} provided and find the corresponding extension.\\[10pt]

Record this extension as $S_{\text{X}}$ ............. cm

\begin{enumerate}
\item[(a)] Plot a graph of load against extension.
\item[(b)]
\begin{enumerate}
\item[(i)] Find the gradient (G) of your graph.
\item[(ii)] What is the physical meaning of the gradient?
\end{enumerate}
\item[(c)] From the graph, what is the mass of the weight labelled \textbf{X}?
\end{enumerate}

\item[2.] The aim of this experiment is to find the critical angle \textbf{C} of the given glass block.

\begin{center}
\includegraphics[width=10cm]{./img/2005-2-alt.png}
\end{center}

\textbf{Proceed as follows:}\\[10pt]

Place a white sheet of paper on the drawing board. Place the glass block, with one of its largest surfaces top most on top of the white paper. Mark the outline of the glass block on the paper with a pencil. Then remove the glass block and draw a line which cuts its largest sides normally at E and F as shown in figure 2 above. \\[10pt]

Using a protractor draw an angle $\alpha = 30^\circ$ with the glass block. Replace the glass block in its original position and stick the first pin P$_1$ and second pin P$_2$ along the line of angle $\alpha = 30^\circ$. Stick the third and fourth pins P$_3$ and P$_4$ respectively on the opposite side of the glass block such that P$_3$ and P$_4$ fall on a straight line with P$_1$ and P$_2$ when viewed through side CD of the glass block. \\[10pt]

Remove the glass block and trace the straight path taken by the ray G P$_3$ P$_4$. Using a ruler, join G and E. \\[10pt]

Measure the angle of refraction $r^\circ$, then calculate the values of $\cos{\alpha}$ and $\sin{r^\circ}$. Repeat the same procedure for values $\alpha = 40^\circ$, $50^\circ$, $60^\circ$, $70^\circ$ and $80^\circ$. Record your results in tabular form for the values of $\alpha$, $r^\circ$, $\sin{r^\circ}$ and $\cos{\alpha}$.

\begin{enumerate}
\item[(a)] Plot a graph of $\sin{r^\circ}$ (vertical axis) against $\cos{\alpha}$ (horizontal axis).
\item[(b)] Find the slope of the graph.
\item[(c)] Calculate the value of $C$ where slope = $\sin{C}$.
\item[(d)] State the possible sources of error and precautions you have taken during the experiment.
\end{enumerate}

\item[3.] The aim of this experiment is to determine the \textbf{e.m.f. E} and internal resistance \textbf{r} of a cell.

\begin{center}
\includegraphics[width=7cm]{./img/2005-3-alt.png}
\end{center}

\begin{enumerate}
\item[(a)] Connect the circuit as shown in figure 3 above. Put $R = 1 \Omega$ and quickly read the value of $i$ on the ammeter.
\item[(b)] Repeat the procedure in 3 (a) above, for values of $R = 2 \Omega$, $3 \Omega$, $4 \Omega$ and $5 \Omega$ respectively.
\item[(c)] Tabulate your results and complete the following table.
\begin{center}
\begin{tabular}{|c|c|c|} \hline
\textbf{Resistance $R$ ($\Omega$)} & \textbf{Current $i$ (A)} & \textbf{$\cfrac{1}{i} \left(\text{A}^{-1}\right)$ } \\ \hline
1&& \\
2&& \\
3&& \\
4&& \\
5&& \\ \hline
\end{tabular} \\[10pt]
\end{center}
\item[(d)] Plot the graph of $R$ against $\cfrac{1}{i}$.
\item[(e)] The graph uses the equation $R = \cfrac{E}{i} - r$.
\begin{enumerate}
\item[(i)] Suggest how $E$ and $r$ may be evaluated from your graph.
\item[(ii)] Evaluate $E$ for one cell.
\item[(iii)] Evaluate $r$ for one cell.
\end{enumerate}
\item[(f)] State one source of error and suggest one way of minimizing it.
\end{enumerate}

\end{enumerate}
%\pagebreak
%\subsection{2004 - PHYSICS 2A ALTERNATIVE A PRACTICAL}

\begin{enumerate}
\item[1.] The aim of this experiment is to determine the mass of a given dry cell, size ``AA''.\\[5pt]

You are provided with a dry cell, a knife edge, two weights 50 g and 20 g, and a metre rule.\\[5pt]

Proceed as follows:
\begin{enumerate}
\item[(a)] Locate and note the centre of gravity C of the metre rule by balancing on the knife edge.
\item[(b)] Suspend the 50 g mass on one side of the metre rule, and 20 g together with the dry cell on the other side of the metre rule adjusting their position until the metre rule balances horizontally, as shown in Figure 1 below.

\begin{center}
\includegraphics[width=10cm]{./img/2004-1-alt.png}
\end{center}

\item[(c)] By fixing a = 5 cm from C find its corresponding length, b, from C.
\item[(d)] Repeat and tabulate your results using a = 10 cm, 15 cm, 20 cm and 25 cm.
\item[(e)] Draw a graph of ``a'' against ``b'' and calculate its slope G.
\item[(f)] Calculate X from the equation $\text{G} = \cfrac{20 + \text{X}}{50}$. \hfill \textbf{(25 marks)}
\end{enumerate}


\item[2.] You are provided with a glass block, drawing board, optical pins and plane papers.\\

Place a white piece of paper on the drawing board. Place the glass block with one of its largest surface top most on top of the white paper. Mark the outline of the glass block on the paper with a pencil. Remove the glass block and draw a normal as shown in Figure 2 below.

\begin{center}
\includegraphics[width=10cm]{./img/2004-2-alt.png}
\end{center}

\begin{enumerate}
\item[(a)] Draw a line making an angle of incidence, $i$ of $30^\circ$. Erect two pins P$_1$ and P$_2$ on this line at a suitable distance apart. Replace the glass block and erect two more pins P$_3$ and P$_4$ at positions which appear to be in a straight line with the other two pins as seen through the glass block from the other side.\\[10pt]

Remove the glass block and draw the complete path of the ray (see Fig. 2). Measure the angle of refraction, $r$.
\item[(b)]
\begin{enumerate}
\item[(i)] Extend the direction of the incident ray as shown by the dotted line.
\item[(ii)] Measure the perpendicular distance `$d$' between extended incident ray and the emergent ray.
\end{enumerate}
\item[(c)] Repeat the procedure in (a) and (b) above for angles of incidence of $30^\circ$, $40^\circ$, $50^\circ$, $60^\circ$ and $70^\circ$. (In each case make your drawings on a fresh part of the drawing paper).
\item[(d)] Tabulate your results as shown in Table 1 below.
\begin{center}
\begin{tabular}{|p{0.15\textwidth}|p{0.15\textwidth}|p{0.15\textwidth}|p{0.15\textwidth}|p{0.15\textwidth}|} \hline
\multicolumn{1}{|c|}{$i$ (deg)} & \multicolumn{1}{c|}{$r$ (deg)} & \multicolumn{1}{c|}{$d$ (cm)} & \multicolumn{1}{c|}{$d\cos{r}$} & \multicolumn{1}{c|}{$\sin{(i-r)}$} \\ \hline
\multicolumn{1}{|c|}{30}&&&& \\
\multicolumn{1}{|c|}{40}&&&& \\
\multicolumn{1}{|c|}{50}&&&& \\
\multicolumn{1}{|c|}{60}&&&& \\
\multicolumn{1}{|c|}{70}&&&& \\ \hline
\end{tabular}\\[10pt]
\end{center}
\begin{enumerate}
\item[(i)] Plot a graph of $d\cos{r}$ against $\sin{(i-r)}$.
\item[(ii)] Find the gradient of the graph.
\item[(iii)] Measure the width of the glass block.
\item[(iv)] How is the gradient of the graph in 2 (a)(ii) and the width of the glass block in 2 (a)(iii) related?\\[5pt]
\end{enumerate}
\item[] NB: \quad Hand in your diagrams (drawings) together with your answer booklet. 
\item[] \flushright \textbf{(25 marks)}

\end{enumerate}

\item[3.] Determine the resistivity $\rho$ of the wire labelled W and the internal resistance of the battery provided.\\

Proceed as follows:

\begin{center}
\includegraphics[width=10cm]{./img/2004-3-alt.png}
\end{center}

Connect the circuit as shown in fig. 3 above. With the plug key open adjust the length of wire W to a value of 20 cm. Note the ammeter reading.\\[10pt]

\noindent NB: The plug key should remain open throughout the experiment.

\begin{enumerate}
\item[(a)] Repeat the procedure above for $L_W$ = 40 cm, 60 cm, 80 cm and 100 cm each time recording the ammeter reading.
\item[(b)] Tabulate your results as shown in Table 2 below.

\begin{tabular}{|p{2.5cm}|c|c|} \hline
Length $L_W$ of wire (cm)& Current $I$ (A)& $\cfrac{1}{I} \left(\text{A}^{-1}\right)$ \\ \hline
&& \\
&& \\
&& \\
&& \\ \hline
\end{tabular}\\[10pt]

\item[(c)]
\begin{enumerate}
\item[(i)] Plot a graph of $\cfrac{1}{I}$ (vertical) against $L_W$ (horizontal).
\item[(ii)] Determine the slope G.
\item[(iii)] Determine the intercept $Y$ on the vertical axis.
\end{enumerate}
\item[(d)] Measure and record the diameter at four different places on the wire. Hence find the mean value of diameter $d$.
\item[(e)] Given that $G = \cfrac{4\rho}{\pi d^2E}$ and $Y = \cfrac{R + r}{E}$\\[10pt]
Where $E$ is the emf of the battery, and $R = 2 \Omega$, Find the
\begin{enumerate}
\item[(i)] Resistivity $\rho$ of the wire.
\item[(ii)]	Internal resistance $r$ of the battery. \hfill \textbf{(25 marks)}
\end{enumerate}
\end{enumerate}

\end{enumerate}
%\pagebreak
%\setcounter{secnumdepth}{2}
\settocdepth{section} % Return to default

% Appendix Chapters
% Any chapters coming after the \appendix command are just counted as chapters in the appendix
\appendix 

% NECTA Past Papers
\part*{NECTA Practical Past Papers}
\addcontentsline{toc}{part}{NECTA Practical Past Papers}
\phantomsection
\addcontentsline{toc}{part}{Biology Past Papers}
\part*{Biology Past Papers} \index{Past Papers!Biology}
\label{cha:past-papers-bio}


%\titlespacing*{\chapter}{0pt}{-50pt}{20pt}
%\titleformat{\chapter}{\normalfont\huge\bfseries}{}{0pt}{\Huge}

\setcounter{secnumdepth}{0}
\index{Past Papers!Biology!2013}
\section{2013 - BIOLOGY 2A (ACTUAL PRACTICAL A)}

\begin{enumerate}
\item[1.] You have been provided with solution \textbf{B}.
\begin{enumerate}
\item[(a)] Identify the food substances present in solution \textbf{B} by using the reagents provided. Tabulate your work as shown in the following Table:

\begin{center}
\begin{tabular}{|p{3cm}|p{3cm}|p{3cm}|p{3cm}|} \hline
\multicolumn{1}{|c|}{\textbf{Food Tested}}&\multicolumn{1}{c|}{\textbf{Procedure}}&\multicolumn{1}{c|}{\textbf{Observation}}&\multicolumn{1}{c|}{\textbf{Inference}} \\ \hline
&&& \\
&&& \\
&&& \\
&&& \\ \hline
\end{tabular} \\[10pt]
\end{center}

\item[(b)] For each food substance identified in 1(a);
\begin{enumerate}
\item[(i)] Name two common sources.
\item[(ii)] State their role in the body of human being.
\end{enumerate}
\item[(c)] The digestion of one of the identified food substance in 1(a) starts in the mouth.
\begin{enumerate}
\item[(i)] Name this food substance.
\item[(ii)] Identify the enzyme responsible for its digestion in the mouth.
\end{enumerate}
\item[(d)] The digestive system of human being has several parts.
\begin{enumerate}
\item[(i)] Name the part of digestive system in which most of digestion and absorption of food takes place.
\item[(ii)] Explain how the named part in (d) (i) is adapted for absorption of digested food substances.
\end{enumerate}
\end{enumerate}

\item[2.] You have been provided with specimens \textbf{S$_1$}, \textbf{S$_2$}, \textbf{S$_3$} and \textbf{S$_4$}.
\begin{enumerate}
\item[(a)] Use the hand lens to observe these specimens then:
\begin{enumerate} 
\item[(i)] Identify specimen \textbf{S$_1$}, \textbf{S$_2$}, \textbf{S$_3$} and \textbf{S$_4$} by their common names.
\item[(ii)] Classify specimen \textbf{S$_1$}, \textbf{S$_2$} and \textbf{S$_3$} to Class level.
\end{enumerate}
\item[(b)] Study specimen \textbf{S$_1$} carefully then answer the following questions:
\begin{enumerate}
\item[(i)] Draw a neat, large and well labeled diagram of specimen \textbf{S$_3$}.
\item[(ii)] State the habitat of specimen \textbf{S$_3$}.
\item[(iii)] In what ways is specimen \textbf{S$_3$} important to a farmer?
\end{enumerate}
\item[(c)] State two advantages of specimen \textbf{S$_1$}.
\item[(d)] State four advantages of specimen \textbf{S$_4$}.
\item[(e)] Give reason why specimen \textbf{S$_4$} was formally placed in the Kingdom Plantae?
\end{enumerate}
\end{enumerate}
\pagebreak
\index{Past Papers!Biology!2012}
\section{2012 - BIOLOGY 2A (ACTUAL PRACTICAL A)}

\begin{enumerate}
\item[1.] You have been provided with specimens \textbf{F} and \textbf{G}.
\begin{enumerate}
\item[(a)] Study specimens \textbf{F} and \textbf{G} carefully, then:
\begin{enumerate}
\item[(i)] Identify specimens \textbf{F} and \textbf{G} using their common names.
\item[(ii)] Compare specimens \textbf{F} and \textbf{G}, then state their observable differences.
\item[(iii)] Briefly explain the types of germination which occurs in specimens \textbf{F} and \textbf{G}.
\end{enumerate}
\item[(b)] Using a scalpel, remove the outer coat from specimen \textbf{F}. Split the two parts with the inner sides facing upwards. Then:
\begin{enumerate}
\item[(i)] Draw a well labelled diagram to show the structures of one part of the split specimen \textbf{F} as would be seen from above.
\item[(ii)] For each structure labelled in specimen \textbf{F}, state the role they play in seed germination.
\end{enumerate}
\item[(c)] Using a scalpel, prepare a longitudinal section of specimen \textbf{G}.
\begin{enumerate}
\item[(i)] Draw a well labelled diagram of the cut surface of specimen \textbf{G}.
\item[(ii)] Identify the part used by specimen \textbf{G} to absorb water during seed germination.
\end{enumerate}
\end{enumerate}

\item[2.] You have been provided with specimens \textbf{H}, \textbf{I}, \textbf{J} and\textbf{K}.
\begin{enumerate}
\item[(a)] Study carefully specimens \textbf{H} and \textbf{I} then:
\begin{enumerate}
\item[(i)] Identify specimens \textbf{H} and \textbf{I} using their common names.
\item[(ii)] Suggest the mode of locomotion of specimens \textbf{H} and \textbf{I}. Give reason to support your answer.
\item[(iii)] State the features used to place specimen \textbf{H} in the Kingdom Animalia.
\end{enumerate}
\item[(b)] Use the hand lens to observe specimens \textbf{J} and \textbf{K} then:
\begin{enumerate}
\item[(i)] Identify specimens \textbf{J} and \textbf{K} by their common names.
\item[(ii)] Name the habitats for each of specimens \textbf{J} and \textbf{K}.
\item[(iii)] Briefly explain the features which enable specimen \textbf{H} to survive in its habitat.
\item[(iv)] Classify specimens \textbf{J} and \textbf{K} to the phylum level.
\item[(v)] Write down one advantage and one disadvantage for each specimen \textbf{J} and \textbf{K}.
\end{enumerate}
\end{enumerate}

\end{enumerate}
\pagebreak
\index{Past Papers!Biology!2011}
\section{2011 - BIOLOGY 2A (ACTUAL PRACTICAL A)}

\begin{enumerate}
\item[1.] The solution prepared contained various food substances.
\begin{enumerate}
\item[(a)] Use the chemicals and reagents provided to identify the food substances present in solution \textbf{S$_1$}. Tabulate your work as shown in the following Table:

\begin{center}
\begin{tabular}{|p{3cm}|p{3cm}|p{3cm}|p{3cm}|} \hline
\textbf{FOOD TESTED}&\textbf{PROCEDURE}&\textbf{OBSERVATION}&\textbf{INFERENCE} \\ \hline
&&& \\
&&& \\
&&& \\
&&& \\ \hline
\end{tabular} \\[10pt]
\end{center}
\item[(b)] State the function in the human body of each food identified in 1(a) above.
\item[(c)] Name two enzymes necessary for digestion of food substance(s) identified in (a) above.
\item[(d)] To each type of food identified above, name at least one source in which the food has been extracted.
\end{enumerate}

\item[2.] Study specimen \textbf{A}, \textbf{B} and \textbf{C} then:
\begin{enumerate}
\item[(a)] Write common names of specimen \textbf{A}, \textbf{B} and \textbf{C}.
\item[(b)] Classify specimen \textbf{A} and \textbf{B} to the phylum level.
\item[(c)] State the habitat and one economic importance of specimen \textbf{A}.
\item[(d)] Outline four economic importance of specimen \textbf{B}.
\item[(e)] Use the scalpel provided to cut specimen \textbf{C} longitudinally into two equal halves. Then, draw a neat, well labelled diagram of a specimen.
\item[(f)] Name the division of specimen \textbf{C}.
\item[(g)] State the observable features you can use to place the specimen into its respective phylum/division.
\end{enumerate}

\end{enumerate}
\pagebreak
\index{Past Papers!Biology!2010}
\section{2010 - BIOLOGY 2A (ALTERNATIVE A PRACTICAL)}

\begin{enumerate}
\item[1.] You have been provided with solution \textbf{T$_1$}.
\begin{enumerate}
\item[(a)] Carry out food tests to identify the substances present in solution \textbf{T$_1$}. Record your work in a table as shown below.

\begin{center}
\begin{tabular}{|p{3cm}|p{3cm}|p{3cm}|p{3cm}|} \hline
\multicolumn{1}{|c|}{\textbf{Test for}}&\multicolumn{1}{c|}{\textbf{Procedure}}&\multicolumn{1}{c|}{\textbf{Observation}}&\multicolumn{1}{c|}{\textbf{Inference}} \\ \hline
&&& \\
&&& \\
&&& \\
&&& \\ \hline
\end{tabular} \\[10pt]
\end{center}

\item[(b)] What are the functions of the food substances identified in \textbf{T$_1$} in the human body?
\item[(c)] 
\begin{enumerate}
\item[(i)] State the favourable/suitable pH condition at which the enzymes which digest the food substances present in \textbf{T$_1$} work best.
\item[(ii)] Which of the food substances present in \textbf{T$_1$} is not stored in the human body?
\item[(iii)] What happens when the levels of this substance mentioned in (c) (ii) above, rises in the body?
\end{enumerate}
\end{enumerate}

\item[2.] You are provided with specimens \textbf{A}, \textbf{B}, \textbf{C}, \textbf{D} and \textbf{E}. Observe them carefully and answer the questions that follow:
\begin{enumerate}
\item[(a)]
\begin{enumerate}
\item[(i)] Write down the common names of specimens \textbf{A}, \textbf{B}, \textbf{C}, \textbf{D} and \textbf{E}.
\item[(ii)] To which kingdom do specimens \textbf{C} and \textbf{D} belong?
\item[(iii)] Name one (1) common epidemic disease transmitted by specimen \textbf{A}.
\end{enumerate}
\item[(b)]
\begin{enumerate}
\item[(i)] Draw a large well labelled diagram of specimen \textbf{C}.
\item[(ii)] State the economic importance of specimen \textbf{C}.
\end{enumerate}
\item[(c)]
\begin{enumerate}
\item[(i)] What are the distinguishing characteristics of the Phylum/Division to which specimen \textbf{E} belongs?
\item[(ii)] Where can specimen \textbf{E} be found?
\end{enumerate}

\end{enumerate}
\end{enumerate}
\pagebreak
\index{Past Papers!Biology!2009}
\section{2009 - BIOLOGY 2A (ALTERNATIVE A PRACTICAL)}

\begin{enumerate}
\item[1.] 
\begin{enumerate}
\item[(a)] You are provided with solution \textbf{S$_1$}. Carry out experiments to identify the food substances present in it. Record your procedure, observation and inferences as shown in the table below.

\begin{center}
\begin{tabular}{|p{3cm}|p{3cm}|p{3cm}|p{3cm}|} \hline
\multicolumn{1}{|c|}{\textbf{Test for}}&\multicolumn{1}{c|}{\textbf{Procedure}}&\multicolumn{1}{c|}{\textbf{Observation}}&\multicolumn{1}{c|}{\textbf{Inference}} \\ \hline
&&& \\
&&& \\
&&& \\
&&& \\ \hline
\end{tabular} \\[10pt]
\end{center}

\item[(b)]
\begin{enumerate}
\item[(i)] Name the food substances you have identified.
\item[(ii)] State \textbf{two (2)} sources of each food substance named in 1(b) (i) above.
\item[(iii)] Mention \textbf{one (1)} role of each food substance you have identified.
\end{enumerate}
\item[(c)] In which parts of the digestive system are the above mentioned food substances digested? In each case mention the enzyme and the products.
\end{enumerate}

\item[2.]
\begin{enumerate}
\item[(a)] Using a hand lens examine specimen A$_1$.
\begin{enumerate}
\item[(i)] Identify specimen A$_1$ by its common name.
\item[(ii)] Name the phylum and class to which specimen A$_1$ belongs.
\item[(iii)] Give an example of another organism which belongs to the same phylum as specimen A$_1$.
\end{enumerate}
\item[(b)] Draw a well labelled diagram of specimen A$_1$.
\item[(c)] How is specimen A$_1$ adapted to its mode of nutrition?
\item[(d)] What is the economic importance of specimen A$_1$?
\item[(e)] Where can specimen A$_1$ be found?
\end{enumerate}
\end{enumerate}

\pagebreak
\index{Past Papers!Biology!2008}
\section{2008 - BIOLOGY 2A (ALTERNATIVE A PRACTICAL)}

\begin{enumerate}
\item[1.] You have been provided with specimens S$_1$, S$_2$, S$_3$ and S$_4$. Observe the specimens carefully and answer the following questions:
\begin{enumerate}
\item[(a)]
\begin{enumerate}
\item[(i)] What characteristics are common among specimens S$_1$, S$_2$, S$_3$ and S$_4$? \hfill \textbf{(3 marks)}
\item[(ii)] Name the kingdom and phylum/division to which specimens S$_1$, S$_2$, S$_3$ and S$_4$ belong.\flushright \textbf{(4 marks)}\\
\item[(iii)] Why are S$_3$ and S$_4$ placed in different classes? \hfill \textbf{(2 marks)}
\end{enumerate}
\item[(b)]
\begin{enumerate}
\item[(i)] What distinctive features place specimen S$_2$ in its respective kingdom? \hfill \textbf{(2 marks)}
\item[(ii)] Why are specimens S$_3$ and S$_4$ classified under the same phylum? \hfill \textbf{(4 marks)}
\end{enumerate}
\item[(c)]
\begin{enumerate}
\item[(i)] Suggest how the specimen labelled S$_1$ is adapted to its mode of life. \hfill \textbf{(4 marks)}
\item[(ii)] Give reasons why specimen S$_1$ can not grow taller? \hfill \textbf{(2 marks)}
\end{enumerate}
\item[(d)] Describe the advantages and disadvantages of the organisms which belong to the class into which S$_3$ is found. \hfill \textbf{(4 marks)} \\
\end{enumerate}

\item[2.] You have been provided with a variegated leaf and iodine solution. Carefully follow the instructions given below and answer the questions that follow.
\begin{enumerate}
\item[]
\begin{enumerate}
\item[(i)] Heat some water to boiling point in a beaker and then turn off the source of heat.
\item[(ii)] Use forceps to dip the leaf in the hot water for about 30 seconds.
\item[(iii)] Remove the leaf from the beaker.
\item[(iv)] Push the leaf into the bottom of the test-tube and cover it with alcohol (ethanol).
\item[(v)] Place the tube in hot water until the alcohol boils together with the leaf.
\item[(vi)] Remove the leaf from the test-tube containing ethanol and dip it into hot water.
\item[(vii)] Spread the decolourized leaf on a white tile and drop iodine solution on to it.
\end{enumerate}
\item[(a)] What was the aim of the experiment?
\item[(b)] Why was the leaf dipped in hot water for 30 seconds?
\item[(c)]
\begin{enumerate}
\item[(i)] Give reason, why the leaf was boiled in ethanol?
\item[(ii)] Why was the leaf dipped once again in hot water?
\end{enumerate}
\item[(d)] Give the interpretation of the results observed when a few drops of iodine solution were poured onto the decolourized leaf.
\flushright \textbf{(25 marks)}
\end{enumerate}

\end{enumerate}
\pagebreak
\index{Past Papers!Biology!2007}
\section{2007 - BIOLOGY 2A (ALTERNATIVE A PRACTICAL)}

\begin{enumerate}
\item[1.]
You are provided with solution \textbf{S}.
\begin{enumerate}
\item[(a)] Carry out experiments to identify the food substance present in solution \textbf{S}.
\begin{enumerate}
\item[(i)] Record your experimental work as shown in Table 1 below. \hfill \textbf{(16 marks)}\\

Table 1

\begin{center}
\begin{tabular}{|p{3cm}|p{3cm}|p{3cm}|p{3cm}|} \hline
\multicolumn{1}{|c|}{\textbf{Test for}}&\multicolumn{1}{c|}{\textbf{Procedure}}&\multicolumn{1}{c|}{\textbf{Observation}}&\multicolumn{1}{c|}{\textbf{Inference}} \\ \hline
&&& \\
&&& \\
&&& \\
&&& \\ \hline
\end{tabular} \\[10pt]
\end{center}

\item[(ii)] Solution \textbf{S} contains ------. \hfill \textbf{(3 marks)}
\end{enumerate}
\item[(b)] Suggest one storage organ in a plant from which solution \textbf{S} might have been prepared. \flushright \hfill \textbf{(1 mark)}
\item[(c)] For each food substance identified in (a)(ii) above, name its end product(s) of digestion. \flushright \hfill \textbf{(4 marks)}
\item[(d)] Which of the identified food substance is mostly needed by small children? \hfill \textbf{(1 mark)}
\end{enumerate}

\item[2.] You are provided with a beaker, tea bag and hot water. Carry out the following experiment:\\

Pour about 100 cm$^3$ of hot water into the beaker.\\
Put the tea bag into the beaker containing hot water.\\
Observe carefully the experiment for a few minutes.
\begin{enumerate}
\item[(a)]
\begin{enumerate}
\item[(i)] What happened to the tea bag when it was put in hot water? \hfill \textbf{(3 marks)}
\item[(ii)] Explain why the changes you observed occurred? \hfill \textbf{(4 marks)}
\end{enumerate}
\item[(b)]
\begin{enumerate}
\item[(i)] What do you think was the aim of the experiment? \hfill \textbf{(3 marks)}
\item[(ii)] Draw a conclusion from the experiment. \hfill \textbf{(3 marks)}
\end{enumerate}
\item[(c)]
\begin{enumerate}
\item[(i)] Name the physiological process investigated in this experiment. \hfill \textbf{(3 marks)}
\item[(ii)] Define the process named in (c)(i) above. \hfill \textbf{(4 marks)}
\item[(iii)] What is the importance of this process in nature? \hfill \textbf{(5 marks)}
\end{enumerate}
\end{enumerate}

\item[3.] Study the specimens \textbf{J}, \textbf{K}, \textbf{L}, \textbf{M} and \textbf{N} provided.
\begin{enumerate}
\item[(a)] Identify specimens \textbf{J}, \textbf{K}, \textbf{L}, \textbf{M} and \textbf{N} by their common names. \hfill \textbf{(5 marks)}
\item[(b)] Name the kingdoms for each of specimens \textbf{J}, \textbf{K}, \textbf{L}, \textbf{M} and \textbf{N}. \hfill \textbf{(5 marks)}
\item[(c)] Suggest the possible habitats for specimens \textbf{J} and \textbf{K}. \hfill \textbf{(4 marks)}
\item[(d)] Draw and label specimen \textbf{N}. \hfill \textbf{(7 marks)}
\item[(e)] List \textbf{four (4)} observable differences between specimens \textbf{J} and \textbf{K}. \hfill \textbf{(4 marks)}
\end{enumerate}
\end{enumerate}
\pagebreak
\setcounter{secnumdepth}{2}
\index{Past Papers!Chemistry}
\part*{Chemistry Past Papers}
\label{cha:past-papers-chem}
\addcontentsline{toc}{part}{Chemistry Past Papers}

%\titleformat{\chapter}[display]{\normalfont\huge\bfseries}{\chaptertitlename\
%\thechapter}{20pt}{\Large}

%\chapter{NECTA Past Papers - Chemistry}

%\pagebreak

\setcounter{secnumdepth}{0}
\section{2013 - CHEMISTRY 2A ACTUAL PRACTICAL A} \index{Past Papers!Chemistry! 2013}

\begin{enumerate}
\item[1.] You are provided with the following solutions:\\
\begin{enumerate}
\item[ ] \textbf{JJ}: Containing 3.0 g of acetic acid in 0.50 dm$^3$ of solution;\\
\item[ ] \textbf{KK}: Containing 1.5 g of impure potassium hydroxide in 250 dm$^3$ of solution;\\
\item[ ] Phenolphthalein indicator.\\
\end{enumerate}

\textbf{Questions}:\\
\begin{enumerate}
\item[(a)] Is the use of methyl orange indicator in this experiment as suitable as phenolphthalein?\\
Give a reason for your answer.
\item[(b)] Titrate the acid (in a burette) against the base (in a conical flask) using two drops of your indicator and obtain three titre values.\\
\item[(c)] 
\begin{enumerate}
\item[(i)] \_\_\_\_ cm$^3$ of \textbf{JJ} required \_\_\_\_ cm$^3$ of \textbf{KK} for complete reaction.
\item[(ii)] Write a balanced chemical equation for the reaction between \textbf{JJ} and \textbf{KK}.
\end{enumerate}
\item[(d)] Showing your procedures clearly, calculate the percentage purity of potassium hydroxide.
\end{enumerate}

\item[2.] Your are provided with the following:\\
\begin{enumerate}
\item[ ] \textbf{L$_1$}:  0.50 M sodium thiosulphate;\\
\item[ ] \textbf{L$_2$}:  0.10 M hydrochloric acid;\\
\item[ ] Distilled water;\\
\item[ ] Stop watch;\\
\item[ ] Plain paper.\\
\end{enumerate}

\textbf{Theory}\\
When a solution of sodium thiosulphate is mixed with hydrochloric acid, they react quantitatively and gradually the solution becomes opaque.\\[10pt]

\textbf{Procedure}\\
\begin{enumerate}
\item[(i)] Write a clear letter X on a white piece of paper.
\item[(ii)] Place a 100 cm$^3$ beaker on top of letter X, such that the letter X is visible when viewed from above.
\item[(iii)] Using a measuring cylinder, measure 25 cm$^3$ of \textbf{L$_1$} and pour into the 100 cm$^3$ beaker in (ii) above.
\item[(iv)] Measure 25 cm$^3$ of \textbf{L$_2$} and pour it into the beaker containing solution \textbf{L$_1$} in (iii) above and immediately start the stop watch/clock.
\item[(v)] Shake the reaction mixture only once and record the time taken for the letter X to disappear completely.
\item[(vi)] Repeat steps (ii) to (v) by varying the volume of \textbf{L$_1$} and distilled water as indicated in Table 1.
\end{enumerate}

\newpage

\begin{center}
\begin{tabular}{|p{2.5cm}|p{2.5cm}|p{2.5cm}|p{2cm}|p{4cm}|}
\multicolumn{1}{l}{Table 1}&\multicolumn{1}{l}{ }&\multicolumn{1}{l}{ }\\ \hline
\textbf{Volume of \textbf{L$_1$} in cm$^3$}&\textbf{Volume of\newline water in cm$^3$}&\textbf{Volume of \textbf{L$_2$} in cm$^3$}&\textbf{Time (t)/s}&\textbf{Rate of reaction $\frac{1}{t}$(s$^-1$)}\\ \hline
25&0&25&&\\ \hline
20&5&25&&\\ \hline
15&10&25&&\\ \hline
10&15&25&&\\ \hline
5&20&25&&\\ \hline
\end{tabular}
\end{center}

\textbf{Questions:}\\
\begin{enumerate}
\item[(a)] What is the aim of this experiment?
\item[(b)] Complete Table 1.
\item[(c)] Write the electronic configuration of the product which causes the solution to cloud letter X.
\item[(d)] With state symbols, write the ionic equation for the reaction between \textbf{L$_1$} and \textbf{L$_2$}.
\item[(e)] Plot a graph of volume of \textbf{L$_1$} against rate of reaction.
\item[(f)] What can you conclude from the graph?\\
\end{enumerate}

\item[3.] Sample \textbf{U} contains one cation and one anion. Using systematic qualitative analysis procedures, record carefully your experiments, observations, inferences and finally identify the anion and cation present in sample \textbf{U}. Record your work in a tabular form as Table 2 shows.\\

\begin{center}
\begin{tabular}{|p{1cm}|p{5cm}|p{3cm}|p{3cm}|}
\hline
\textbf{S/n}&\textbf{Experiment}&\textbf{Observation}&\textbf{Inference}\\ \hline
&&&\\
&&&\\
&&&\\
&&&\\
&&&\\
&&&\\
\hline
\end{tabular}\\
\end{center}

\textbf{Conclusion}\\
\begin{enumerate}
\item[(i)] The cation in sample \textbf{U} is \_\_\_\_.\\
\item[(ii)] The anion in sample \textbf{U} is \_\_\_\_.\\
\end{enumerate}

\end{enumerate}
\pagebreak
\section{2012 - CHEMISTRY 2A ACTUAL PRACTICAL A}

\begin{enumerate}
\item[1.] You are provided with the following solution:\\

\textbf{TZ}: Containing 3.5 g of impure sulphuric acid in 500 cm$^3$ of solution;\\
\textbf{LO}: Containing 4 g of sodium hydroxide in 1000 cm$^3$ of solution;\\
Phenolphthalein and Methyl indicators.\\

\textbf{Questions}:\\
\vspace{-4pt}
\begin{enumerate}
\item[(a)] 
\begin{enumerate}
\item[(i)] What is the suitable indicator for the titration of the given solutions?\\
Give a reason for your answer.
\item[(ii)] Write a balanced chemical equation for the reaction between \textbf{TZ} and \textbf{LO}.
\item[(iii)] Why is it important to swirl or shake the contents of the flask during the addition of the acid?\\
\end{enumerate}
\vspace{-4pt}
\item[(b)] Titrate the acid (in a burette) against the base (in a conical flask) using two drops of your indicator and obtain three titre values.\\
\vspace{-4pt}
\item[(c)] 
\begin{enumerate}
\item[(i)] \_\_\_\_ cm$^3$ of acid required \_\_\_\_ cm$^3$ of base for complete reaction.
\item[(ii)] Showing your procedures clearly, calculate the percentage purity of \textbf{TZ}.
\end{enumerate}

\end{enumerate}
\raggedleft \textbf{(20 marks)}

\raggedright

\item[2.] Your are provided with the following materials:\\
\begin{enumerate}
\item[ ] \textbf{ZO}:  A solution of 0.13 M Na$_2$S$_2$O$_3$ (sodium thiosulphate);
\item[ ] \textbf{UU}:  A solution of 2 M HCl;
\item[ ] Thermometer;
\item[ ] Heat source/burner;
\item[ ] Stopwatch.\\
\end{enumerate}

Procedure:\\
\begin{enumerate}
\item[(i)] Place 500 cm$^3$ beaker, which is half-filled with water, on the heat source as a water bath.
\item[(ii)] Measure 10 cm$^3$ of \textbf{ZO} and 10 cm$^3$ of \textbf{UU} into two separate test tubes.
\item[(iii)] Put the two test tubes containing \textbf{ZO} and \textbf{UU} solutions into a water bath.
\item[(iv)] When the solutions attain a temperature of 60$^o$C, remove the test tubes from the water bath and pour both solutions into 100 cm$^3$ empty beaker and immediately start the stop watch.
\item[(v)] Place the beaker with the contents on top of a piece of paper marked \textbf{X}.
\item[(vi)] Note the time taken for the mark \textbf{X} to disappear.
\item[(vii)] Repeat step (i) to (vi) at temperature 70$^o$C, 80$^o$C and 90$^o$C.
\item[(viii)] Record your results as in Table 1.
\end{enumerate}

\newpage

\begin{center}
\begin{tabular}{|p{5cm}|p{5cm}|p{3cm}|}
\multicolumn{1}{l}{Table 1}&\multicolumn{1}{l}{ }&\multicolumn{1}{l}{ }\\ \hline
\textbf{Experiment}&\textbf{Temperature}&\textbf{Time (s)}\\ \hline
1&60$^o$C&\\ \hline
2&70$^o$C&\\ \hline
3&80$^o$C&\\ \hline
4&90$^o$C&\\ \hline
\end{tabular}
\end{center}

\textbf{Questions:}\\
\begin{enumerate}
\item[(a)] Write a balanced chemical equation for reaction between \textbf{UU} and \textbf{ZO}.
\item[(b)] What is the product which causes the solution to cloud the letter \textbf{X}?
\item[(c)] Plot a graph of temperature against time (s).
\item[(d)] What conclusion can you draw from you graph?\\
\end{enumerate}

\raggedleft \textbf{(15 marks)}

\raggedright


\item[3.] Substance \textbf{V} is a simple salt which contains one cation and one anion. Carry our the experiments described below. Record carefully your observations and make appropriate inferences and hence identify the anion and cation present in sample \textbf{V}.\\

\begin{center}
\begin{tabular}{|l|p{8cm}|l|l|}
\hline
\textbf{S/n}&\textbf{Experiment}&\textbf{Observation}&\textbf{Inference}\\ \hline
1&Observe the appearance of sample \textbf{V}.&&\\ \hline
2&Put a little amount of sample \textbf{V} in a test tube then add water and shake.&&\\ \hline
3&Heat a little amount of \textbf{V} in a dry test tube.&&\\ \hline
{\multirow{4}{*}{4}}&To a little sample \textbf{V} in a test tube add dilute Hydrochloric acid. Add more of the acid until the test tube is half full. Divide the resulting solution into three portions and add the following:&&\\ \cline{2-4}
&\begin{enumerate}
\item[a)] To the one portion add NaOH solution drop wise then excess.
\end{enumerate}&&\\ \cline{2-4}
&\begin{enumerate}
\item[b)] To the second portion add ammonia solution drop wise then in excess.
\end{enumerate}&&\\ \cline{2-4}
&\begin{enumerate}
\item[c)] To the third portion add ammonium oxalate solution.
\end{enumerate}&&\\ \hline
5&Perform flame test.&&\\ \hline
\end{tabular}\\

\end{center}

Conclusion\\

\begin{enumerate}
\item[(i)] The cation in sample \textbf{V} is \_\_\_\_.\\
\item[(ii)] The anion in sample \textbf{V} is \_\_\_\_.\\
\item[(iii)] The chemical formula of \textbf{V} is \_\_\_\_.\\
\item[(iv)] The name of compound \textbf{V} is \_\_\_\_.\\
\end{enumerate}


\raggedleft \textbf{(15 marks)} \pagebreak

\raggedright

\end{enumerate}
\pagebreak
\section{2011 - CHEMISTRY 2A ACTUAL PRACTICAL A} \index{Past Papers!Chemistry! 2011}

\begin{enumerate}
\item[1.] You are provided with the following:\\
\textbf{AA}  A solution of 0.2 M nitric acid (HNO$_3$);\\
\textbf{BB}  A solution of 4.2 g NaxCO$_3$ per 0.5 dm$^3$ of solution;\\
\textbf{MO} is methyl orange indicator.\\[10pt]

\textbf{Procedure}\\

Put solution \textbf{AA} into the burette. Pipette 20 cm$^3$ or 25 cm$^3$ of solution \textbf{BB} in a titration flask. Add two drops of methyl orange indicator into the titration flask. Titrate solution \textbf{BB} against \textbf{AA} until the end point is reached. Record the burette reading. Repeatthe procedure to obtain three more readings and record your results in a tabular form.\\[10pt]

\textbf{Questions}:\\
\begin{enumerate}
\item[(a)] 
\begin{enumerate}
\item[(i)] Calculate the average titre volume.
\item[(ii)] Summary:  \_\_\_\_ cm$^3$ of solution \textbf{BB} required \_\_\_\_ cm$^3$ of solution \textbf{AA} for complete reaction.
\end{enumerate}

\item[(b)] If the mole ratio for the reaction is 1:1 find:\\
\begin{enumerate}
\item[(i)] Concentration of NaxCO$_3$ in mol/dm$^3$ and g/dm$^3$.
\item[(ii)] Molecular mass of NaxCO$_3$.
\item[(iii)] Atomic mass of x and replace it in the formula NaxCO$_3$.
\end{enumerate}

\item[(c)] Write a balanced chemical equation for the reaction in this experiment.
\item[(d)] What is the significance of the indicator in this experiment?
\item[(e)] Why is there a colour change when enough acid has been added to the base?

\end{enumerate}
\raggedleft \textbf{(20 marks)}

\raggedright

\item[2.] Your are provided with the following materials:\\
\textbf{TT}:  A solution of 0.13 M Na$_2$S$_2$O$_3$ (sodium thiosulphate);
\textbf{HH}:  A solution of 2 M HCl;
Distilled water;
Stopwatch.\\

\textbf{Procedure:}\\
\begin{enumerate}
\item[(i)] Using 10 cm$^3$ measuring cylinder, measure 20 cm$^3$ of solution \textbf{TT} and put into 100 cm$^3$ beaker.
\item[(ii)] Use different measuring cylinder to measure 10 cm$^3$ of \textbf{HH} and pour it into the beaker containing solution \textbf{TT}, immediately start the stop watch. Swirl the beaker twice.
\item[(iii)] Place the beaker with the contents on top of a piece of paper marked \textbf{X}.
\item[(iv)] Look down vertically through the mouth of the beaker so as to see the cross at the bottom of the beaker. Stop the clock when the letter \textbf{X} is invisible.
\item[(v)] Record the time taken for the letter \textbf{X} to disappear completely.
\item[(vi)] Repeat the experiment as shown in Table 1.
\item[(vii)] Record your results in tabular form as shown in Table 1.
\end{enumerate}

\indent Table 1: Table of results\\

\begin{center}
\begin{tabular}{|p{2cm}|p{2cm}|p{2cm}|p{2cm}|p{2.5cm}|p{2cm}|}
\hline
Exp. No.&Vol. of \textbf{HH} (cm$^3$)&Vol. of \textbf{TT} (cm$^3$)&Vol. of Distilled water (cm$^3$)&Time (sec)&$\frac{1}{t}$ (s$^-1$)\\ \hline
1&10&20&0&&\\ \hline
2&10&15&5&&\\ \hline
3&10&10&10&&\\ \hline
4&10&5&15&&\\ \hline
\end{tabular}
\end{center}

\newpage

\textbf{Questions:}\\
\begin{enumerate}
\item[(a)] Complete filling the table of results (Table 1).
\item[(b)] Write a balanced equation for reaction between \textbf{TT} and \textbf{HH}.
\item[(c)] What is the reaction product which causes the solution to cloud the letter \textbf{X}?
\item[(d)] How was the factor of concentration varied in this experiment?
\item[(e)] Plot a graph of 1/t against the volume of the thiosulphate.
\item[(f)] Use the graph to explain how variation of concentration affects the rate of chemical reaction.\\
\end{enumerate}

\raggedleft \textbf{(15 marks)}

\raggedright


\item[3.] Sample \textbf{S} is a simple salt containing one cation and one anion. Carry our the experiments described below. Record your observations and inferences as shown in Table 2.\\

Table 2: Experimental results\\

\begin{center}
\begin{tabular}{|l|p{8cm}|l|l|}
\hline
\textbf{S/n}&\textbf{Experiment}&\textbf{Observation}&\textbf{Inference}\\ \hline
(a)&Observe the appearance of sample \textbf{S}.&&\\ \hline
(b)&Place a spoonful of sample \textbf{S} in a test tube, add water and shake to dissolve.&&\\ \hline
(c)&Put a spatulaful of sample \textbf{S} in a test tube and heat.&&\\ \hline
(d)&Add three drops of sodium hydroxide solution to the solid sample in a test tube.&&\\ \hline
(e)&Put a spatulaful of sample \textbf{S} in a dry test tube and add concentrated sulphuric acid. Warm the mixture and test for any gas evolved.&&\\ \hline
(f)&Put a spatulaful of sample \textbf{S} in a dry test tube and add concentrated sulphuric acid and manganese dioxide. Warm the mixture and test for any gas evolved.&&\\ \hline
(g)&To a portion of the solution from (f) add aqueous silver nitrate followed by aqueous ammonia.&&\\ \hline
\end{tabular}\\
\end{center}

\textbf{Conclusion:}\\
\begin{enumerate}
\item[(a)] The cation present in \textbf{S} is \_\_\_\_ and the anion is \_\_\_\_.\\
\item[(b)] The name of sample \textbf{S} is \_\_\_\_.\\
\item[(c)] Write a balanced chemical equation for the reactions taking place in experiments (c) and (d).\\
\end{enumerate}


\raggedleft \textbf{(15 marks)} \pagebreak

\raggedright

\end{enumerate}
\pagebreak
\input{./tex/chemistry-2010-2a.tex}
\pagebreak
%\section{2009 - CHEMISTRY 2A ALTERNATIVE A PRACTICAL}
%2009 requires students to answer two (2) of the following questions, including number 1.

\begin{enumerate}

\item[1.] You are provided with the following:\\
Solution \textbf{D} containing 6.90 g of T$_2$CO$_3$ per 0.50 dm$^3$ of solution\\
Solution \textbf{N} containing 1.55 g of hydrochloric acid per 200 cm$^3$ of solution\\
Methyl orange indicator solution.\\

\textbf{Procedure:}\\
Put solution N in the burette. Pipette 20 cm$^3$ or (25 cm$^3$) of D into a titration flask. Add a few drops of methyl orange indicator. Titrate solution N from the burette against solution D in the titration flask to the end point. Note the burette reading. Repeat the procedure to obtain three more values and record the results as shown in the following table.\\

\begin{enumerate}
\item[(a)] Table of results\\
\begin{enumerate}
\item[(i)] Burette readings\\
\begin{center}
\begin{tabular}{|l|p{2cm}|p{2cm}|p{2cm}|p{2cm}|} \hline
\textbf{Titration}&\multicolumn{1}{|c|}{\textbf{Pilot}}&\multicolumn{1}{|c|}{\textbf{1}}&\multicolumn{1}{|c|}{\textbf{2}}&\multicolumn{1}{|c|}{\textbf{3}}\\ \hline
Final reading (cm$^3$)&&&&\\ \hline
Initial reading (cm$^3$)&&&&\\ \hline
Volume used (cm$^3$)&&&&\\ \hline
\end{tabular}\\
\end{center}
\item[(ii)] The volume of the pipette used was \_\_\_\_ cm$^3$.
\item[(iii)] The volume of the burette used was \_\_\_\_ cm$^3$.
\item[(iv)] \_\_\_\_ cm$^3$ of solution D required \_\_\_\_ cm$^3$ of solution N for complete reaction.\\
\item[(v)] The colour change at the end point was from \_\_\_\_ to \_\_\_\_.\\
\end{enumerate}

\item[(b)] Write a balanced equation for the above neutralization reaction.\\
\item[(c)] Calculate the following:\\
\begin{enumerate}
\item[(i)] molarity of acid solution N.
\item[(ii)] molarity of the base solution D.
\item[(iii)] molar weight of T$_2$CO$_3$.
\item[(iv)] atomic mass of element T.
\end{enumerate}
\item[(d)] Identify element T in T$_2$CO$_3$
\end{enumerate}

\raggedleft \textbf{(25 marks)}\newpage

\raggedright

\item[2.] Sample \textbf{B} is a simple salt containing \textbf{one} cation and \textbf{one} anion. Carry out the experiments described in the following table carefully and record all your observations and appropriate inferences. Identify the cation and anion present in sample \textbf{B}.\\

\begin{center}
\begin{tabular}{|l|p{8cm}|l|l|}
\hline
\multicolumn{2}{|c|}{\textbf{Experiment}}&\textbf{Observation}&\textbf{Inference}\\ \hline
(a)&Appearance of sample B.&&\\ \hline
(b)&Put a spatulaful of sample B in a test-tube. Add water until half test-tubeful. Stir and divide the solution into five portions in different test tubes and do the following:&&\\ \cline{2-4}
&\begin{enumerate}
\item[(i)] add fresh zinc metal granules to the first portion. Heat for a while. Decant the result. Pour the solid material onto a filter paper and observe. Let it dry, then observe again.
\end{enumerate}&&\\ \cline{2-4}
&\begin{enumerate}
\item[(ii)] add NaOH solution until excess to the second portion then heat and observe again.
\end{enumerate}&&\\ \cline{2-4}
&\begin{enumerate}
\item[(iii)] add ammonia solution dropwise to the third portion until excess.
\end{enumerate}&&\\ \cline{2-4}
&\begin{enumerate}
\item[(iv)] add AgNO$_3$ to the fourth portion followed by dil. HNO$_3$.
\end{enumerate}&&\\ \cline{2-4}
&\begin{enumerate}
\item[(v)] add AgNO$_3$ to the fifth portion followed by ammonia solution.
\end{enumerate}&&\\ \hline
\end{tabular}\\

\end{center}


\textbf{Conclusion}\\

\begin{enumerate}
\item[(a)] The cation present in sample B is \_\_\_\_ and the anion is \_\_\_\_.
\item[(b)] What has been happening in the experiments (b)(i) and (b)(ii)? Use reaction equations where possible.
\end{enumerate}

\raggedleft \textbf{(25 marks)}

\raggedright

\item[3.] Substance Z contains \textbf{one} basic radical and and \textbf{one} acidic radical. Using systematic qualitative analysis procedures carry out experiments on sample Z and make appropriate observations and inferences to identify the radicals.\\

\begin{center}
\begin{tabular}{|p{4cm}|p{4cm}|p{4cm}|}
\hline
\textbf{Experiment}&\textbf{Observation}&\textbf{Inference}\\ \hline
&&\\
\hline
\end{tabular}\\
\end{center}

\textbf{Conclusion}\\

The Basic radical in sample Z is \_\_\_\_ and the acidic radical is \_\_\_\_.\\


\end{enumerate}

\raggedleft \textbf{(25 marks)}\\

\raggedright

%\pagebreak
%\section{2008 - CHEMISTRY 2A ALTERNATIVE A PRACTICAL}
%2008 requires students to answer two (2) of the following questions, including Question 1.

\begin{enumerate}

\item[1.] You are provided with the following:\\
\vspace{4pt}
Solution M containing 9.0 g of H$_2$X per dm$^3$ of the solution.\\
\vspace{4pt}
Solution N containing 4.91 g of sodium hydroxide per dm$^3$ of the solution.\\
\vspace{4pt}
Solution P is phenolphthalein indicator.\\
\vspace{10pt}
\textbf{Procedure}\\
\vspace{4pt}
Put solution M into the burette. Pipette 25 cm$^3$ (or 20 cm$^3$) of solution N into the titration flask. Put two to three drops of P into the titration flask. Titrate solution M from the burette against solution N in the titration flask until a colour change is observed. Note the burette reading. Repeat the procedure to obtain three more readings. Record your results as shown in Table 1.\\
\vspace{10pt}
\textbf{Results}\\
\vspace{6pt}
\textbf{Table 1: Burette readings}\\

\begin{center}
\begin{tabular}{|l|p{2cm}|p{2cm}|p{2cm}|p{2cm}|} \hline
\textbf{Titration}&\multicolumn{1}{|c|}{\textbf{Pilot}}&\multicolumn{1}{|c|}{\textbf{1}}&\multicolumn{1}{|c|}{\textbf{2}}&\multicolumn{1}{|c|}{\textbf{3}}\\ \hline
Final reading (cm$^3$)&&&&\\ \hline
Initial reading (cm$^3$)&&&&\\ \hline
Volume used (cm$^3$)&&&&\\ \hline
\end{tabular}\\
\end{center}

\begin{enumerate}
\item[(a)] Give the volume of the pipette used.
\item[(b)] Give the volume of solution M needed for complete neutralization of solution N.
\item[(c)] Tell the colour change of the indicator at the end point of the titration.
\item[(d)] Write the balanced chemical equation for the reaction between solution M and N.
\item[(e)] Calculate the
\begin{enumerate}
\item[(i)] molarity of solution M
\item[(ii)] molar mass of H$_2$X
\item[(iii)] mass of X in H$_2$X.
\end{enumerate}
\end{enumerate}

\raggedleft \textbf{(25 marks)}\newpage

\raggedright

\item[2.] Sample D is a simple salt containing one cation and one anion. Carry out carefully the experiments described below recording all your observations and appropriate inferences as shown in Table 2 to identify the cation and anion present in D.\\
\vspace{10pt}
\textbf{Table 2}\\

\begin{center}
\begin{tabular}{|l|p{8cm}|l|l|}
\hline
\multicolumn{2}{|c|}{\textbf{Experiment}}&\textbf{Observation}&\textbf{Inference}\\ \hline
(a)&Observe the appearnce of salt D.&&\\ \hline
(b)&Put a little solid sample D in a clean and dry test tube and heat.&&\\ \hline
(c)&Put a spatulaful of sample D in a test tube, add distilled water, stir and divide the obtained solution into four portions in different test tubes. To the&&\\ \hline
&\begin{enumerate}
\item[(i)] first portion of the solution of sample D in a state tube add aqueous ammonia slowly till excess.
\end{enumerate}&&\\ \hline
&\begin{enumerate}
\item[(ii)] second portion of the solution of sample D in a test tube add aqueous ammonia slowly till excess.
\end{enumerate}&&\\ \hline
&\begin{enumerate}
\item[(iii)] third portion of the solution of sample D in a test tube add potassium hexacyanoferrate (II).
\end{enumerate}&&\\ \hline
&\begin{enumerate}
\item[(iv)] fourth portion of the solution of sample D in a test tube add dilute HCl followed by BaCl$_2$ solution.
\end{enumerate}&&\\ \hline
\end{tabular}\\

\end{center}


\textbf{Conclusion:}\\
\vspace{6pt}
The cation in sample D is \_\_\_\_ and the anion is \_\_\_\_.\\
The molecular formula of salt D is \_\_\_\_.\\

\raggedleft \textbf{(25 marks)}

\raggedright

\item[3.] Sample Y is a simple salt containing one anion and one cation. Using systematic qualitative analysis procedures carry out tests on sample Y and make appropriate observations and inferences to identify the cation and anion present in sample Y.\\

\begin{center}
\begin{tabular}{|p{4cm}|p{4cm}|p{4cm}|}
\hline
\textbf{Experiment}&\textbf{Observation}&\textbf{Inference}\\ \hline
&&\\
&&\\
&&\\
&&\\
\hline
\end{tabular}\\
\end{center}

\textbf{Conclusion:}\\
\vspace{6pt}
The cation in sample Y is \_\_\_\_ and the anion is \_\_\_\_.\\

\end{enumerate}

\raggedleft \textbf{(25 marks)}\\

\raggedright

%\pagebreak
\input{./tex/chemistry-2007-2a.tex}
\pagebreak
%\input{./tex/chemistry-2006-2a.tex}
%\pagebreak
%\input{./tex/chemistry-2005-2a.tex}
%\pagebreak
\setcounter{secnumdepth}{2}
\index{Past Papers!Physics}
\part*{Physics Past Papers}
\label{cha:past-papers-phys}
\addcontentsline{toc}{part}{Physics Past Papers}

%\titlespacing*{\chapter}{0pt}{-50pt}{20pt}
%\titleformat{\chapter}[display]{\normalfont\huge\bfseries}{\chaptertitlename\
%\thechapter}{20pt}{\Large}

%\chapter{NECTA Past Papers - Physics}

%\pagebreak

\setcounter{secnumdepth}{0}
\index{Past Papers!Physics!2013}
\chapter{2013 - PHYSICS 2A ACTUAL PRACTICAL A}

\begin{enumerate}
\item[1.] You are provided with a metre rule, a knife edge, two strings of length 100 cm each and two weights $W_1$ and $W_2$ of masses 50 g and 100 g respectively. Proceed as follows:
\begin{enumerate}
\item[(a)] Balance a metre rule on a knife edge, put a mark and write G at the balancing point using a piece of chalk or a pencil. Measure and record the length $l$, width $w$ and thickness $t$ of a metre rule using a vernier caliper.
\item[(b)] Place the metre rule on a knife edge so that the knife edge is at 60 cm of your metre rule (see Figure 1 (a)). Suspend weight $W_2$ of 100 g on the right hand side of the knife edge. Adjust $W_2$ until the metre rule balances horizontally. Read and record lengths `b' and `c' as seen in Figure 1 (a).

\begin{center}
\includegraphics[width=9cm]{./img/2013-1a-alt.png}
\end{center}

\begin{enumerate}
\item[(i)] Suspend weight $W_1$ of 50 g on the left hand side of the knife edge at the position 47 cm and adjust weight $W_2$ until the metre rule balances horizontally as seen in Figure 1 (b). Read and record the lengths `a' and `b'.

\begin{center}
\includegraphics[width=9cm]{./img/2013-1b-alt.png}
\end{center}

\item[(ii)] Repeat the procedures in (b) (i) by adjusting the position of $W_1$ to the left at the interval of 3 cm to obtain other four (4) readings.
\end{enumerate}
\item[(c)] Tabulate your results as shown in Table 1.\\

\quad Table 1\\
\quad \quad \begin{tabular}{|p{4cm}|p{4cm}|} \hline
\multicolumn{1}{|c|}{a (cm)} & \multicolumn{1}{c|}{b (cm)} \\ \hline
& \\ \hline
& \\ \hline
\end{tabular}\\[10pt]

\item[(d)] Plot a graph of ``b'' against ``a''.
\item[(e)] What is the nature of the graph?
\item[(f)] Calculate the slope $S$ of the graph.
\item[(g)]
\begin{enumerate}
\item[(i)] Read the $b$-intercept, given that $b = Sa + \cfrac{W}{W_2} \times c$
\item[(ii)] What does $\left(\cfrac{W}{W_2}\right)c$ represent in your graph?
\item[(iii)] Calculate the value of $W$ using the relation $W_2 = \cfrac{Wc}{9.5 \text{cm}}$. What does $W$ represent?
\end{enumerate}
\item[(h)]
\begin{enumerate}
\item[(i)] Find the value of the ratio $P = \cfrac{l \times w \times t}{m}$.
\item[] \textbf{Note:} The mass $m$ of a meter rule can be obtained by calculations.
\item[(ii)] What is the physical meaning of the value of $P$?
\end{enumerate}
\item[(i)] State a possible source of error in this experiment.
\item[(j)] How can you minimize error in 1 (i)?
\item[(k)] State the aim of this experiment.
\end{enumerate}
\end{enumerate}
\flushright \textbf{(25 marks)}
\flushleft

\begin{enumerate}
\item[2.] You are provided with a Plane mirror, a Ruler, Protract, Drawing board, Optical pins, Office pins and Plain papers. Proceed as follows:
\begin{enumerate}
\item[(a)] On the plain paper provided, draw a line 13 cm from the top of the paper and call it M$_1$M$_2$. Pin your paper on the board provided and place the reflecting surface of the mirror along the line M$_1$M$_2$ as seen in Figure 2.

\begin{center}
\includegraphics[width=10cm]{./img/2013-2-alt.png}
\end{center}

\item[(b)] Insert pin O as an object at 4.0 cm in front of the mirror. Place pins P$_1$ and P$_2$ so as to appear in one straight line with the image of object O seen in the plane mirror.
\item[(c)] Remove pins P$_1$ and P$_2$, using other pins, place pins P$_3$ and P$_4$ so as to appear in a straight line with the image of object O in the other side (see Figure 2).
\item[(d)] Remove the mirror and pins. Draw lines joining P$_1$ and P$_2$ on one side and the other joining P$_3$ and P$_4$ on the other side of object O, extend both lines to meet at I on the other side of line M$_1$M$_2$.
\item[(e)] Join OI, a line cutting the reflecting surface at N.
\item[(f)] Repeat this procedure for the distance of an object being 6, 8, 10 and 12 cm.
\item[(g)] On all the diagrams drawn:
\begin{enumerate}
\item[(i)] Measure the distance ON and NI.
\item[(ii)] Comment on the distances obtained in 2 (g) (i).
\item[(iii)] What is the nature of image? Give reasons for your answer.
\item[(iv)] State four characteristics of the image you obtained.
\item[(v)] What is the aim of this experiment?
\item[(vi)] Mention and state the law governing this experiment.
\item[(vii)] Explain a source of error in this experiment.
\item[(viii)] How can you minimize the error in (vii) above?\\

\item[] \textbf{Note:} The papers used for drawing should be attached and collected together with answer booklets.
\end{enumerate}
\end{enumerate}
\end{enumerate}
\flushright \textbf{(25 marks)}
\flushleft
\pagebreak
\index{Past Papers!Physics!2012}
\chapter{2012 - PHYSICS 2A ACTUAL PRACTICAL A}

\begin{enumerate}
\item[1.] You are provided with a measuring cylinder, eureka can, nylon thread, standard masses and water. Proceed as follows:
\begin{enumerate}
\item[(a)] Pour water into eureka can until it is just beginning to overflow.

\begin{center}
\includegraphics[width=13cm]{./img/2012-1-alt.png}
\end{center}

\item[(b)] Hold a suitable measuring cylinder under the spout and immerse a standard mass of 50 g into eureka can as shown in Figure 1. Water will pass through the spout and will be collected by the measuring cylinder. Wait for it to drop until it starts to cease and take long interval to drop. Record the reading of the water collected.
\item[(c)] Repeat the procedures in 1 (b) for standard masses of 100 g, 150 g, 200 g and 250 g.
\item[(d)] Tabulate your results showing the quantities as follows:\\

\begin{tabular}{|p{4cm}|p{4cm}|p{4cm}|}\hline
Mass (g) & Volume (cm$^3$) & Mass $\div$ Volume (g/cm$^3$) \\ \hline
50 & & \\ \hline
100 & & \\ \hline
150 & & \\ \hline
200 & & \\ \hline
250 & & \\ \hline
\end{tabular}\\[10pt]

\item[(e)] Plot a graph of mass against volume.
\item[(f)] State the nature of the graph.
\item[(g)] From the graph:
\begin{enumerate}
\item[(i)] Calculate the slope.
\item[(ii)] What does the slope of the graph show?
\item[(iii)] What is the relationship between mass and volume?
\item[(iv)] Establish the formula governing the experiment.
\end{enumerate}
\item[(h)] Identify with reasons the best to the least satisfactory method of finding the constant value of mass divide by volume.
\item[(i)] State two possible errors in this experiment.
\item[(j)] How can you minimize errors in 1 (i)?
\end{enumerate}

\pagebreak

\item[2.] You are provided with two plane mirrors, an optical pin, a sheet of plane drawing paper, mirror holder or office pins, a protractor, a ruler and a drawing table. Proceed as follows:
\begin{enumerate}
\item[(a)] Draw two lines at right angles.
\item[(b)] Place the two plane mirrors along the top two lines using the mirror holders or office pins as shown in Figure 2.

\begin{center}
\includegraphics[width=5cm]{./img/2012-2-alt.png}
\end{center}

\item[(c)] Put an optical pin at O when $\theta = 90^\circ$. Look onto one of the mirrors and count the number of images, $n$, you see.
\item[(d)] Repeat the procedures in 2 (c) for $\theta = 72^\circ$, $\theta = 60^\circ$, $\theta = 45^\circ$ and $\theta = 30^\circ$.
\item[(e)] Tabulate your results for the values of $\theta$, $n$ and $\cfrac{360^\circ}{\theta}$.
\item[(f)] Plot a graph of number of images, $n$, against $\cfrac{360^\circ}{\theta}$.
\item[(g)] From the graph:
\begin{enumerate}
\item[(i)] Determine the slope.
\item[(ii)] Find the number of images when $\cfrac{360^\circ}{\theta} = 9$
\item[(iii)] Find the value of the $y$-intercept.
\item[(iv)] Derive the equation relating the number of images and $\cfrac{360^\circ}{\theta}$
\end{enumerate}
\item[(h)] From your experiment:
\begin{enumerate}
\item[(i)] What happens to the number of images as the value angle $\theta$ is reduced?
\item[(ii)] What happens to the number of images when $\theta = 0^\circ$?
\end{enumerate}
\item[(i)] State a possible source of error and how you can minimize it.
\item[(j)] What is the aim of this experiment?
\end{enumerate}
\end{enumerate}
\pagebreak
\index{Past Papers!Physics!2011}
\subsection{2011 - PHYSICS 2A ACTUAL PRACTICAL A}

\begin{enumerate}
\item The aim of this experiment is to determine the mass of a given dry cell size ``AA''. Proceed as follows:
\begin{itemize}
\item[(a)] Locate and note the centre of gravity $C$ of the metre rule by balancing it on the knife edge.
\item[(b)] Suspend the 50 g mass at length `$a$' cm on one side of the metre rule and the 20 g mass together with the dry cell at length `$b$' cm on the other side of the metre rule. Fix the 50 g mass at length 30 cm from the fulcrum and adjust the position of the 20 g mass together with the dry cell until the metre rule balances horizontally. Read and record the values of $a$ and $b$ as $a_0$ and $b_0$ respectively.
\item[(c)] Draw the diagram for this experiment.
\item[(d)] By fixing $a = 5$ cm from fulcrum $C$, find its corresponding length $b$.
\item[(e)] Repeat the procedure in (d) above for $a = 10$ cm, 15 cm, 20 cm and 25 cm. Tabulate your results.
\item[(f)] Draw a graph of `$a$' against `$b$' and calculate its slope $G$.
\item[(g)] Calculate $X$ from the equation $50 = \cfrac{b_0}{a_0}(20 + X)$.
\item[(h)] Comment on the value of $\cfrac{b_0}{a_0}$.
\item[(i)] Sate the principle governing this experiment.
\end{itemize}
\end{enumerate}

\flushright \textbf{(25 marks)}
\begin{enumerate}
\item[2.] You are provided with an ammeter, A, resistance box, R, dry cell, D, a key, K and connecting wires. Proceed as follows:
\begin{enumerate}
\item[(a)] Connect the circuit in series.
\item[(b)] Put $R$ = 1 $\Omega$ and quickly read the value of current $I$ on the ammeter.
\item[(c)] Repeat procedure (b) above for $R$ = 2 $\Omega$, 3 $\Omega$, 4 $\Omega$ and 5 $\Omega$. Record your results in a tabular form.
\item[(d)] Draw the circuit diagram for this experiment.
\item[(e)] Plot the graph of $R$ against $\cfrac{1}{I}$.
\item[(f)] Determine the slope of the graph.
\item[(g)] If the graph obeys the equation $R=\cfrac{E}{I}-r$, then
\begin{enumerate}
\item[(i)] suggest how $E$ and $r$ may be evaluated from your graph.
\item[(ii)] compute $E$.
\item[(iii)] compute $r$.
\end{enumerate}
\item[(h)] State one source of error and suggest one way of minimizing it.
\item[(i)] Suggest the aim of this experiment.
\end{enumerate}

\end{enumerate}

\flushright \textbf{(25 marks)}
\flushleft
\pagebreak
\index{Past Papers!Physics!2010}
\section{2010 - PHYSICS 2A ALTERNATIVE A PRACTICAL}

\begin{enumerate}
\item[1.] The aim of this experiment is to find the mass of the unknown load labeled ``$W$'' and the spring constant $K$. Proceed as follows:

\begin{center}
\includegraphics[width=14cm]{./img/2010-1-alt.png}
\end{center}

Set up the apparatus as shown in Figure 1. Put a mass of 50 g on the scale pan and record the equilibrium position $X_0$ of the pointer. Put on the scale pan the unknown weight marked $W$. Without removing $W$ and the 50 g mass in the scale pan, add a load $L$ of 50 g and record the new position of the pointer $X$. Calculate the extension $E = (X - X_0)$. Repeat this process for $L$ = 100 g, 150 g, 200 g and 250 g.
\begin{enumerate}
\item[(a)] Record you conclusions as shown in Table 1.\\[10pt]

Equilibrium position $X_0$..................\\[10pt]

Table 1\\[10pt]

%\begin{center}
\begin{tabular}{|p{3cm}|p{3cm}|p{3cm}|} \hline
\multicolumn{1}{|c|}{Load (g)} & \multicolumn{1}{c|}{$X$ (cm)} & \multicolumn{1}{c|}{$E = X - X_0$ (cm)} \\ \hline
\multicolumn{1}{|c|}{50} & & \\ \hline
\multicolumn{1}{|c|}{100} & & \\ \hline
\multicolumn{1}{|c|}{150} & & \\ \hline
\multicolumn{1}{|c|}{200} & & \\ \hline
\multicolumn{1}{|c|}{250} & & \\ \hline
\end{tabular}\\[10pt]
%\end{center}

\item[(b)] Plot the graph of load L against absolute value of extension E. The scale of the vertical axis should be chosen to range from 200 g to 300 g.
\item[(c)] From the graph, determine the unknown weight marked W, given that L = KE + W where K is a constant.
\item[(d)] What does the gradient of the graph represent?
\item[(e)] State the sources of errors and precautions that should be taken in the experiment.
\end{enumerate}
\end{enumerate}
\flushright \textbf{(25 marks)}

\pagebreak

\begin{enumerate}
\item[2.] The aim of this experiment is to determine the refractive index of water. Proceed as follows:

\begin{center}
\includegraphics[width=10cm]{./img/2010-2-alt.png}
\end{center}

\begin{enumerate}
\item[(a)] Arrange your apparatus as in Figure 2. Put about 150 cm$^3$ of clear water in the measuring cylinder. Drop an office pin at the bottom so that it rests touching the wall of the cylinder.
\item[(b)] Look in the cylinder from Figure 2. Use another office pin as a search pin, move it up and down outside the cylinder, and locate the image position by no parallax method. Locate the image position of the ruler. Measure and record the depth ($H_1$) of the image. Measure and record the depth ($H_2$) of water. Repeat the experiment with 175 cm$^3$, 200 cm$^3$, 225 cm$^3$ and 250 cm$^3$ of water in the measuring cylinder.
\item[(c)] 
\begin{enumerate}
\item[(i)] Record in Table 2 your values of $H_1$ and $H_2$ corresponding to the volumes of water in the measuring cylinder.\\[10pt]

Table 2\\[10pt]

%\begin{center}
\begin{tabular}{|p{3cm}|p{3cm}|p{3cm}|} \hline
\multicolumn{1}{|c|}{Volume of water V (cm)} & \multicolumn{1}{c|}{$H_1$} & \multicolumn{1}{c|}{$H_2$} \\ \hline
\multicolumn{1}{|c|}{150} & & \\ \hline
\multicolumn{1}{|c|}{175} & & \\ \hline
\multicolumn{1}{|c|}{200} & & \\ \hline
\multicolumn{1}{|c|}{225} & & \\ \hline
\multicolumn{1}{|c|}{250} & & \\ \hline
\end{tabular}\\[10pt]
%\end{center}

\item[(ii)] Plot the graph of $H_2$ versus $H_1$.
\item[(iii)] Determine the slope of the graph.
\item[(iv)] What is the physical meaning of the slope?
\item[(v)] State sources of error in this experiment.
\end{enumerate}
\end{enumerate}
\end{enumerate}
\flushright \textbf{(25 marks)}

\pagebreak

\begin{enumerate}
\item[3.] The aim of this experiment is to determine the resistivity of an electrical conductor $P$.

\begin{center}
\includegraphics[width=10cm]{./img/2010-3-alt.png}
\end{center}

With $P$ having a length $l = 50$ cm, connect up the circuit as shown in Figure 3. Close one key S and adjust the rheostat R so that the current in $P$ is 0.20 A. Record the current $I$ and the potential difference $V$ between its ends.\\[10pt]

Repeat the procedure with current $I = $0.30 A, 0.40 A, 0.50 A and 0.60 A.
\begin{enumerate}
\item[(a)] Record your results in Table 3.\\[10pt]

Table 3\\[10pt]

%\begin{center}
\begin{tabular}{|p{3cm}|p{3cm}|} \hline
\multicolumn{1}{|c|}{Current $I$ (A)} & \multicolumn{1}{c|}{P.d. (volts)} \\ \hline
\multicolumn{1}{|c|}{0.20} &  \\ \hline
\multicolumn{1}{|c|}{0.30} &  \\ \hline
\multicolumn{1}{|c|}{0.40} &  \\ \hline
\multicolumn{1}{|c|}{0.50} &  \\ \hline
\multicolumn{1}{|c|}{0.60} &  \\ \hline
\end{tabular}\\[10pt]
%\end{center}

\item[(b)] Plot a graph of $V$ against $I$ and calculate the slope $G$.
\item[(c)] Deduce the resistivity of the conductor $P$ given that; $\rho = \cfrac{G\pi d^2}{4l}$.\\[10pt]
Where $\rho$ = resistivity\\
$d$ = diameter of $P$ (measured using the micrometer screw gauge provided).
\end{enumerate}
\end{enumerate}

\flushright \textbf{(25 marks)}
\flushleft
\pagebreak
\index{Past Papers!Physics!2009}
\section{2009 - PHYSICS 2A ALTERNATIVE A PRACTICAL}

\begin{enumerate}
\item[1.] In this experiment you are required to find the relationship between the length of a simple pendulum and its period. Proceed as follows:
\begin{itemize}
\item[(a)] Suspend a simple pendulum of length L = 100 cm. Displace the pendulum through a small angle so that it swings parallel to the edge of the bench or table, determine the time for 20 oscillations. Continue reducing the length of the pendulum by 10 cm each time and obtain a total of six readings.
\item[(b)] Record your readings in a table as shown below.


\begin{tabular}{|p{2.5cm}|p{2.5cm}|p{2.5cm}|p{2.5cm}|p{2.5cm}|} \hline
Length of pendulum L (cm) & Log$_{10}$L & Time for 20 oscillations & Period T & Log$_{10}$T \\ \hline
&&&& \\ 
&&&& \\ 
&&&& \\ 
&&&& \\ 
&&&& \\ 
&&&& \\ 
&&&& \\ 
&&&& \\ \hline
\end{tabular}\\[10pt]

\noindent Assuming that T $ \propto $ L$^a$, we have T = $k$L$^a$ and taking logarithms to base ten on both sides we get $\log_{10}$T = $a\log_{10}$L + $\log_{10}k$.

\begin{itemize}
\item[(i)] Plot a graph of $\log_{10}$T (vertical axis) against $\log_{10}$L (horizontal axis) hence determine the values of $a$ and $k$ each correct to one decimal place.
\item[(ii)] From your answer in (i) above write down the values of $a$ and $k$ each in the form of $\cfrac{b}{c}$ where $b$ and $c$ are integers (i.e. whole numbers).
\item[(iii)] From the assumption and your answer in (ii) deduce the form of the equation governing the motion of the simple pendulum.
\end{itemize}

\end{itemize}
\end{enumerate}
\flushright \textbf{(25 marks)}


\begin{enumerate}
\item[2.] The aim of this experiment is to determine the refractive index $\eta$ of a given glass block.

\begin{center}
\includegraphics[width=8cm]{./img/2009-2-alt.png}
\end{center}

Place the rectangular glass block on the white paper on a drawing board. Using a pencil trace the outline of the block. Remove the glass block and draw a normal NOM near the left end of the block (Figure 1).\\[10pt]

\noindent Using a protractor and a pencil measure $\theta = 20^\circ$, draw a line making the angle $20^\circ$ with the surface RR of the block. Erect two pins T$_1$ and T$_2$ on this line and at a suitable distance from one another. Return the block and erect the pins T$_3$ and T$_4$ at positions such that they lie in a straight line with pins T$_1$ and T$_2$ as seen through the block. Now remove the block and draw a complete path of the ray (Figure 1).\\[10pt]

\noindent Measure the length MN$'$ and ON$'$: Repeat the procedure for values of $\theta = 30^\circ, 40^\circ$ and $60^\circ$ respectively. In each case make a drawing on a fresh part of the drawing paper.

\begin{itemize}
\item[(a)] Record the values of $\theta$, MN$'$, ON$'$, $\cfrac{\text{MN}'}{\text{ON}'}$ and $\cos \theta$ in a tabular form.
\item[(b)] Plot a graph of $\cfrac{\text{MN}'}{\text{ON}'}$ against $\cos \theta$.
\item[(c)] Find the slope $G$ of the graph.
\item[(d)] Calculate the value of the refractive index $\eta$; given that $G = \cfrac{1}{\eta}$.
\item[(e)] State two sources of errors. \hfill \textbf{(25 marks)}
\end{itemize}

\end{enumerate}


\begin{enumerate}
\item[3.] The aim of this experiment is to verify Ohm's Law.

\begin{center}
\includegraphics[width=7cm]{./img/2009-3-alt.png}
\end{center}

\begin{itemize}
\item[(a)] Set up the apparatus as shown on Figure 2, close switch S. Adjust the Rheostat Rh by sliding slowly from one end, read and record the value V of the voltmeter and current I of the ammeter.
\item[(b)] Repeat the experiment by changing the Rheostat slider to obtain about five pair of readings.\\[10pt]

\textbf{NB:} Adjust the Rheostat until when the pointer is exactly on the division of the metre scale.\\[10pt]

Table of results\\[10pt]
\begin{tabular}{|c|c|c|c|c|c|c|c|}\hline
V (V)&&&&&&&\\ \hline
I (A)&&&&&&& \\ \hline
\end{tabular}

\item[(c)] Plot a graph of V (vertical axis) against I (horizontal axis).
\item[(d)] 
\begin{itemize}
\item[(i)] Find the slope of the graph.
\item[(ii)] What is the relation between V and I?
\item[(iii)] Find the resistance R. \hfill \textbf{(25 marks)}
\end{itemize}
\end{itemize}

\end{enumerate}
\flushleft

\pagebreak
\index{Past Papers!Physics!2008}
\chapter{2008 - PHYSICS 2A  ALTERNATIVE A PRACTICAL}

\begin{enumerate}
\item[1.] The aim of this experiment is to investigate whether string A obeys Hooke's law.

\begin{center}
\includegraphics[width=15cm]{./img/2008-1-alt.png}
\end{center}

Proceed as follows:\\[6pt]

Clamp string A at one end, attach a weighing pan at the other end and a pointer to give a reading on a scale as shown in figure 1 above.\\[6pt]

Measure the height, $h_0$ when the pan is empty.\\[6pt]

Place 50 g mass on the pan and record the new height $h$ indicated by the pointer.\\[6pt]

Add another 50 g mass each time up to 300 g, and record the corresponding values of $h$ for added mass.

\begin{enumerate}
\item[(a)] Tabulate your results as shown in the table below.
\item[] $h_0 = $ \begin{tabular}{p{1.5cm}}
\\ \hline
\end{tabular} cm.\\[10pt]

\begin{tabular}{|c|c|c|c|c} \hline
Mass, $m$ (g) & Height, $h$ (cm) & Extension ($h - h_0$) & Stretching \\
&&(cm)&force, $F$ (N) \\ \hline
50&&& \\ 
100&&& \\ 
150&&& \\ 
200&&& \\ 
250&&& \\ 
300&&& \\ \hline
\end{tabular}\\[10pt]
\item[(b)] Plot a graph of force $F$ (N) against extension (cm).
\item[(c)] From the graph find the
\begin{enumerate}
\item[(i)] slope, K of the graph.
\item[(ii)] extension caused by a mass of 180 g.
\end{enumerate}
\item[(d)] Deduce whether string A obeys Hooke's law.
\item[(e)] State the law. \hfill \textbf{(25 marks)}
\end{enumerate}
\end{enumerate}

\begin{enumerate}
\item[2.] You are provided with a glass block, four sheets of drawing paper, four optical pins (or office pins) and a drawing board.\\[6pt]

Proceed as follows:\\[6pt]

Place the glass block flat on the drawing paper fixed to the drawing board and with a sharp pencil, draw its outline.

\begin{center}
\includegraphics[width=9cm]{./img/2008-2-alt.png}
\end{center}

Remove the glass block and draw a normal NN' to the longer edge of the block (see fig. 2).\\
Draw a line making an angle of incidence ($i$) of 30$^\circ$. Stick two vertical pins P$_1$ and P$_2$ on this line. Replace the glass block. Stick two more pins P$_3$ and P$_4$ on the other side of the block so that they appear to be in the same straight line with the images of pins P$_1$ and P$_2$ as seen through the block.\\
Remove the block and draw the complete path of the ray entering and leaving the block.\\
Measure the angle of refraction ($r$).\\[6pt]

Produce the incident ray as shown in fig. 2 and measure the perpendicular distance ($d$) between the incident ray and the emergent ray.\\
Repeat this procedure for angles of incidence of $40^\circ$, $50^\circ$, $60^\circ$ and $70^\circ$. In each case draw the block again on a fresh part of the paper.
\begin{enumerate}
\item[(a)] Record your results in a table as follows:
\begin{tabular}{|p{0.15\textwidth}|p{0.15\textwidth}|p{0.15\textwidth}|p{0.15\textwidth}|p{0.15\textwidth}|} \hline
\multicolumn{1}{|c|}{$i^\circ$} & \multicolumn{1}{c|}{$r^\circ$} & \multicolumn{1}{c|}{$d$ (cm)} & \multicolumn{1}{c|}{$d\cos{r}$ (cm)} & \multicolumn{1}{c|}{$\sin{(i-r)}$} \\ \hline
\multicolumn{1}{|c|}{30}&&&& \\
\multicolumn{1}{|c|}{40}&&&& \\
\multicolumn{1}{|c|}{50}&&&& \\
\multicolumn{1}{|c|}{60}&&&& \\
\multicolumn{1}{|c|}{70}&&&& \\ \hline
\end{tabular}\\[10pt]
\item[(b)] Plot a graph of $d \cos{r}$ (vertical axis) against $\sin{(i - r)}$ (horizontal axis).
\item[(c)] Find the gradient of the graph.
\item[(d)] Measure the width of the glass block.
\item[(e)] How is the gradient related to the with of the glass block?
\item[] NB: Hand in your diagrams together with your answer booklet. \hfill \textbf{(25 marks)}
\end{enumerate}
\end{enumerate}

\begin{enumerate}
\item[3.] The aim of this experiment is to determine the resistance of a wire W.\\
Proceed as follows:
\begin{enumerate}
\item[(a)] Connect in series the full length of wire W of unknown resistance, battery B (3 V), a switch K, a rheostat Rh of a few ohms and an ammeter A of $0 - 1$ A.\\
Connect the voltmeter V of $0 - 3$ V across W. Check that the +ve side of the ammeter A and the +ve side of the voltmeter V are both on the +ve side of the battery B.
\item[(b)] Switch on the current. Adjust the rheostat to obtain five widely different values of $V$ and corresponding values of current $I$.
\item[(c)] Tabulate your results as follows:\\[10pt]
\begin{tabular}{|p{4cm}|p{4cm}|} \hline
\multicolumn{1}{|c|}{Potential difference} & \multicolumn{1}{c|}{Current $I$ (amperes)} \\
\multicolumn{1}{|c|}{$V$ (volts)} & \\ \hline
& \\ 
& \\ 
& \\ \hline
\end{tabular}
\item[(d)]
\begin{enumerate}
\item[(i)] Draw a circuit diagram.
\item[(ii)] Plot a graph of potential difference $V$ against current $I$.
\item[(iii)] Find the slope of the graph.
\item[(iv)] Determine the resistance of the wire W.
\item[(v)] Mention \textbf{two (2)} main precautions to be taken in this experiment.
\end{enumerate}
\end{enumerate}
\end{enumerate}
\flushright \textbf{(25 marks)}
\flushleft
\pagebreak
\index{Past Papers!Physics!2007}
\subsection{2007 - PHYSICS 2A ALTERNATIVE A PRACTICAL}

\begin{enumerate}
\item[1.] The aim of this experiment is to determine the mass of a given object ``B'', and the constant of the spring provided.

\begin{center}
\includegraphics[width=7cm]{./img/2007-1-alt.png}
\end{center}

\begin{itemize}
\item[]
\begin{itemize}
\item[(i)] Set up the apparatus as shown in Fig. 1 with zero mark of the metre-rule at the top of the rule and record the scale reading by the pointer, $S_0$.
\item[(ii)] Place the object ``B'' and standard weight (mass) W equal to 20 g in the pan and record the new pointer reading $S_1$. Calculate the extension, $e = S_1 - S_0$ in cm.
\item[(iii)] Repeat the procedure in (ii) above with W = 40 g, 60 g, 80 g and 100 g.
\end{itemize}
\item[(a)] Record your results in tabular form as shown below:\\
Table of Results:

\begin{tabular}{|p{2cm}|p{3cm}|p{3cm}|p{3cm}|}\cline{1-1}
\multicolumn{1}{|p{2cm}|}{$S_0 = $}&\multicolumn{2}{c}{} & \multicolumn{1}{p{2.5cm}}{} \\ \hline
\multicolumn{1}{|c|}{Mass} & \multicolumn{1}{c|}{Force, F (N)} & \multicolumn{1}{c|}{Pointer reading $S_1$} & \multicolumn{1}{c|}{Extension}\\
\multicolumn{1}{|c|}{(kg)} & \multicolumn{1}{c|}{} & \multicolumn{1}{c|}{(cm)} & \multicolumn{1}{c|}{$= S_1 - S_0$ (cm)}\\ \hline
\multicolumn{1}{|c|}{0} & \multicolumn{1}{c|}{} & \multicolumn{1}{c|}{} & \multicolumn{1}{c|}{}\\ 
\multicolumn{1}{|c|}{0.02} & \multicolumn{1}{c|}{} & \multicolumn{1}{c|}{} & \multicolumn{1}{c|}{}\\ 
\multicolumn{1}{|c|}{0.04} & \multicolumn{1}{c|}{} & \multicolumn{1}{c|}{} & \multicolumn{1}{c|}{}\\ 
\multicolumn{1}{|c|}{0.06} & \multicolumn{1}{c|}{} & \multicolumn{1}{c|}{} & \multicolumn{1}{c|}{}\\ 
\multicolumn{1}{|c|}{0.08} & \multicolumn{1}{c|}{} & \multicolumn{1}{c|}{} & \multicolumn{1}{c|}{}\\ 
\multicolumn{1}{|c|}{0.10} & \multicolumn{1}{c|}{} & \multicolumn{1}{c|}{} & \multicolumn{1}{c|}{}\\ \hline
\end{tabular}
\item[(b)] Plot graph of Force F (vertical axis) against extension $e$ (horizontal axis).
\item[(c)] Use your graph to evaluate
\begin{itemize}
\item[(i)] mass of B
\item[(ii)] spring constant, K, given that force, extension, constant and weight of B are related as follows:\\
F = K$e$ - B
\end{itemize}
\end{itemize}

\end{enumerate}
\flushright \textbf{(25 marks)}



\begin{enumerate}
\item[2.] The aim of this experiment is to find the refractive index of a glass block. Proceed as following:\\[10pt]

Place the given glass block in the middle of the drawing paper on the drawing board. Draw lines along the upper and lower edge of the glass block. Remove the glass block and extend the line you have drawn. Represent the ends of those line segments as SS$^1$ and TT$^1$. Draw the normal NN$^1$ to the parallel lines SS$^1$ and TT$^1$ as shown in Fig. 2(a).

\begin{center}
\includegraphics[width=8cm]{./img/2007-2a-alt.png}
\end{center}

Draw five evenly spaced lines from O to represent incident rays at different angles of incidence (10$^\circ$, 20$^\circ$, 30$^\circ$, 40$^\circ$ and 50$^\circ$ from the normal). Replace the glass block carefully between SS$^1$ and TT$^1$. Stick two pins P$_1$ and P$_2$ as shown in Fig. 2(b) as far apart as possible along one of the lines drawn to represent an incident ray. Locate an emergent ray by looking through the block and stick pins P$_3$ and P$_4$ exactly in line with images I$_1$ and I$_2$ of pins P$_1$ and P$_2$. Draw the emergent ray and repeat the procedure for all the incident rays you have drawn. Finally draw in the corresponding refracted rays.\\[10pt]

NOTE: The drawing paper should be handed in together with other answer sheets.

\begin{center}
\includegraphics[width=10cm]{./img/2007-2b-alt.png}
\end{center}

\begin{itemize}
\item[(a)] Record the angles of incidence I and the measured corresponding angles of refraction ``r'' in a table. Your table of results should include the values of $\sin$ I and $\sin$ r.
\item[(b)] Plot the graph of $\sin$ I (vertical - axis) against $\sin$ r (horizontal - axis).
\item[(c)] Determine the slope of the graph.
\item[(d)] What is the refractive index of the glass block used?
\item[(e)] Mention any sources of errors in this experiment.
\end{itemize}

\end{enumerate}
\flushright \textbf{(25 marks)}


\begin{enumerate}
\item[3.] The aim of this experiment is to determine the potential fall along a uniform resistance wire carrying a steady current.\\[10pt]

Proceed as follows:

\begin{center}
\includegraphics[width=12cm]{./img/2007-3-alt.png}
\end{center}

Connect up the circuit as shown in Fig. 3. Adjust the rheostat so that when the sliding contact J is near B, and the key is closed the voltmeter V indicates an almost full scale deflection. Do not alter the rheostat again.\\

Close key K and make contact with J, so that AJ = 10 cm. Record the potential different V volts between A and J as registered on the voltmeter.\\

Repeat this procedure for AJ = 20 cm, 30 cm, 50 cm and 70 cm.

\begin{itemize}
\item[(a)] Tabulate your results for the values of AJ and V.
\item[(b)] Plot a graph of V (vertical axis) against AJ (horizontal axis).
\item[(c)] Calculate the slope of the graph.
\item[(d)] What is your comment on the slope?
\item[(e)] State any precautions on the experiment.
\end{itemize}

\end{enumerate}
\flushright \textbf{(25 marks)}
\flushleft
\pagebreak
\index{Past Papers!Physics!2006}
\subsection{2006 - PHYSICS 2A ALTERNATIVE A PRACTICAL}

\begin{enumerate}
\item[1.] In this experiment you are required to determine the mass of unknown object ``X''.

\begin{center}
\includegraphics[width=10cm]{./img/2006-1-alt.png}
\end{center}

Assemble the pieces of apparatus as shown in Figure 1, with zero mark scale of the rule at the lower most end.\\

Record the reading of the position of pointer on the scale of metre-rule when the pan is empty as $S_0$.\\

Put 20 g to the pan and record pointer reading $S$.\\

Find extension $e = S - S_0$ cm.\\

Repeat the procedure for mass of 40 g, 60 g, 80 g and 100 g. Put object X on the pan and record its pointer reading.

\begin{itemize}
\item[(a)] Summarize your results in a table as follows:\\[10pt]
\begin{center}
\begin{tabular}{|l|c|c|c|c|c|c|} \hline
Mass on pan (g) &20&40&60&80&100&X \\ \hline
Pointer reading (cm) &&&&&& \\ \hline
Extension, $e = S - S_0$ (cm) &&&&&& \\ \hline
\end{tabular} \\[10pt]
\end{center}
\item[(b)] Plot graph of mas against extension (m Vs. $e$).
\item[(c)] Find slope, P, of your graph.
\item[(d)] Find mass X.
\item[(e)] Find Q, given that Q = P $\times$ $e_\text{x}$, where $e_\text{x}$ is extension of X.
\item[(f)] Comment on Q and X.
\end{itemize}


\end{enumerate}

\begin{enumerate}
\item[2.] Set up the experiment as shown in the diagram below using plane mirror, soft board, three pins and a white sheet of paper.

\begin{center}
\includegraphics[width=6cm]{./img/2006-2-alt.png}
\end{center}

Fix a white sheet of paper on the soft board. Draw a line across the width at about the middle of the white sheep (MP). Draw line ONI perpendicular to MP.\\

Fix optical pin O to make ON = U = 3 cm. By using plasticine or otherwise, fix plane mirror along portion of MP with O in front of the mirror. With convenient position of eye, E, look into the mirror and fix optical pins A and B to be in line with image, I, of pin O.\\

Measure and record NI = V. Repeat procedure for U = 6 cm, 9 cm and 12 cm.

\begin{itemize}
\item[(a)] Tabulate your results as follows:\\[10pt]

\quad \quad \begin{tabular}{|l|c|c|c|c|} \hline
U (cm) &3&6&9&12 \\ \hline
V (cm) &&&& \\ \hline
\end{tabular} \\[10pt]

\item[(b)] Plot graph of U against V.
\item[(c)] Calculate slope, m, of the graph to the nearest whole number.
\item[(d)] State relationship between U and V.
\item[(e)] Write equation connecting U and V using numerical value of m with symbols U and V.
\item[(f)] From your equation give position of the image when object is touching the face of the mirror.
\end{itemize}

\end{enumerate}


\begin{enumerate}
\item[3.] You are required to determine the unknown resistance labeled X using a metre bridge circuit. Connect your circuit as shown below, where R is a resistance box, G is a galvanometer, J is a jockey and others are common circuit components.

\begin{center}
\includegraphics[width=14cm]{./img/2006-3-alt.png}
\end{center}

Procedure:\\

With R = 1 $\Omega$, obtain a balance point on a metre bridge wire AB using a jockey J. Note the length $l$ in centimetres. Repeat the experiment with R equal to 2 $\Omega$, 4 $\Omega$, 7 $\Omega$ and 10 $\Omega$.\\

Tabulate your results for R, $l$ and $^1/_l$.

\begin{itemize}
\item[(a)]
\begin{itemize}
\item[(i)] Plot a graph of R (vertical axis) against $^1/_l$ (horizontal axis).
\item[(ii)] Determine the slope S of your graph.
\item[(iii)] Using your graph, find the value of R for which $^1/_l = 0.02$.
\end{itemize}
\item[(b)] Read and record the intercept R$_0$ on the vertical axis.
\item[(c)] Given that,\\
\quad \quad R = $\cfrac{100\text{X}}{l}$ - X\\
Use the equation and your graph to determine the value of X.
\item[(d)] Comment on your results in (a)(iii), (b) and (c) above.
\end{itemize}

\end{enumerate}
\pagebreak
\index{Past Papers!Physics!2005}
\section{2005 - PHYSICS 2A ALTERNATIVE A PRACTICAL}

\begin{enumerate}
\item[1.] The aim of this experiment is to determine the mass of unknown weight labelled \textbf{X} and the force constant of the spring \textbf{k}.

\begin{center}
\includegraphics[width=12cm]{./img/2005-1-alt.png}
\end{center}

Set up the apparatus provided as shown in figure 1 above. Add 50 g mass on to the weight pan so that any ``kinks'' in the spring are removed. Leave this weight for the whole experiment but ignore it in all readings. Record the scale reading $S_0$. Add 50 g on to the weight pan and record the new scale reading $S$. Calculate the extension ($e = S - S_0$) caused by the weight. Repeat with different weights ($W$) to obtain at least five readings. Tabulate your results. Replace the weights ($W$) by the weight \textbf{X} provided and find the corresponding extension.\\[10pt]

Record this extension as $S_{\text{X}}$ ............. cm

\begin{enumerate}
\item[(a)] Plot a graph of load against extension.
\item[(b)]
\begin{enumerate}
\item[(i)] Find the gradient (G) of your graph.
\item[(ii)] What is the physical meaning of the gradient?
\end{enumerate}
\item[(c)] From the graph, what is the mass of the weight labelled \textbf{X}?
\end{enumerate}

\item[2.] The aim of this experiment is to find the critical angle \textbf{C} of the given glass block.

\begin{center}
\includegraphics[width=10cm]{./img/2005-2-alt.png}
\end{center}

\textbf{Proceed as follows:}\\[10pt]

Place a white sheet of paper on the drawing board. Place the glass block, with one of its largest surfaces top most on top of the white paper. Mark the outline of the glass block on the paper with a pencil. Then remove the glass block and draw a line which cuts its largest sides normally at E and F as shown in figure 2 above. \\[10pt]

Using a protractor draw an angle $\alpha = 30^\circ$ with the glass block. Replace the glass block in its original position and stick the first pin P$_1$ and second pin P$_2$ along the line of angle $\alpha = 30^\circ$. Stick the third and fourth pins P$_3$ and P$_4$ respectively on the opposite side of the glass block such that P$_3$ and P$_4$ fall on a straight line with P$_1$ and P$_2$ when viewed through side CD of the glass block. \\[10pt]

Remove the glass block and trace the straight path taken by the ray G P$_3$ P$_4$. Using a ruler, join G and E. \\[10pt]

Measure the angle of refraction $r^\circ$, then calculate the values of $\cos{\alpha}$ and $\sin{r^\circ}$. Repeat the same procedure for values $\alpha = 40^\circ$, $50^\circ$, $60^\circ$, $70^\circ$ and $80^\circ$. Record your results in tabular form for the values of $\alpha$, $r^\circ$, $\sin{r^\circ}$ and $\cos{\alpha}$.

\begin{enumerate}
\item[(a)] Plot a graph of $\sin{r^\circ}$ (vertical axis) against $\cos{\alpha}$ (horizontal axis).
\item[(b)] Find the slope of the graph.
\item[(c)] Calculate the value of $C$ where slope = $\sin{C}$.
\item[(d)] State the possible sources of error and precautions you have taken during the experiment.
\end{enumerate}

\item[3.] The aim of this experiment is to determine the \textbf{e.m.f. E} and internal resistance \textbf{r} of a cell.

\begin{center}
\includegraphics[width=7cm]{./img/2005-3-alt.png}
\end{center}

\begin{enumerate}
\item[(a)] Connect the circuit as shown in figure 3 above. Put $R = 1 \Omega$ and quickly read the value of $i$ on the ammeter.
\item[(b)] Repeat the procedure in 3 (a) above, for values of $R = 2 \Omega$, $3 \Omega$, $4 \Omega$ and $5 \Omega$ respectively.
\item[(c)] Tabulate your results and complete the following table.
\begin{center}
\begin{tabular}{|c|c|c|} \hline
\textbf{Resistance $R$ ($\Omega$)} & \textbf{Current $i$ (A)} & \textbf{$\cfrac{1}{i} \left(\text{A}^{-1}\right)$ } \\ \hline
1&& \\
2&& \\
3&& \\
4&& \\
5&& \\ \hline
\end{tabular} \\[10pt]
\end{center}
\item[(d)] Plot the graph of $R$ against $\cfrac{1}{i}$.
\item[(e)] The graph uses the equation $R = \cfrac{E}{i} - r$.
\begin{enumerate}
\item[(i)] Suggest how $E$ and $r$ may be evaluated from your graph.
\item[(ii)] Evaluate $E$ for one cell.
\item[(iii)] Evaluate $r$ for one cell.
\end{enumerate}
\item[(f)] State one source of error and suggest one way of minimizing it.
\end{enumerate}

\end{enumerate}
\pagebreak
\index{Past Papers!Physics!2004}
\subsection{2004 - PHYSICS 2A ALTERNATIVE A PRACTICAL}

\begin{enumerate}
\item[1.] The aim of this experiment is to determine the mass of a given dry cell, size ``AA''.\\[5pt]

You are provided with a dry cell, a knife edge, two weights 50 g and 20 g, and a metre rule.\\[5pt]

Proceed as follows:
\begin{enumerate}
\item[(a)] Locate and note the centre of gravity C of the metre rule by balancing on the knife edge.
\item[(b)] Suspend the 50 g mass on one side of the metre rule, and 20 g together with the dry cell on the other side of the metre rule adjusting their position until the metre rule balances horizontally, as shown in Figure 1 below.

\begin{center}
\includegraphics[width=10cm]{./img/2004-1-alt.png}
\end{center}

\item[(c)] By fixing a = 5 cm from C find its corresponding length, b, from C.
\item[(d)] Repeat and tabulate your results using a = 10 cm, 15 cm, 20 cm and 25 cm.
\item[(e)] Draw a graph of ``a'' against ``b'' and calculate its slope G.
\item[(f)] Calculate X from the equation $\text{G} = \cfrac{20 + \text{X}}{50}$. \hfill \textbf{(25 marks)}
\end{enumerate}


\item[2.] You are provided with a glass block, drawing board, optical pins and plane papers.\\

Place a white piece of paper on the drawing board. Place the glass block with one of its largest surface top most on top of the white paper. Mark the outline of the glass block on the paper with a pencil. Remove the glass block and draw a normal as shown in Figure 2 below.

\begin{center}
\includegraphics[width=10cm]{./img/2004-2-alt.png}
\end{center}

\begin{enumerate}
\item[(a)] Draw a line making an angle of incidence, $i$ of $30^\circ$. Erect two pins P$_1$ and P$_2$ on this line at a suitable distance apart. Replace the glass block and erect two more pins P$_3$ and P$_4$ at positions which appear to be in a straight line with the other two pins as seen through the glass block from the other side.\\[10pt]

Remove the glass block and draw the complete path of the ray (see Fig. 2). Measure the angle of refraction, $r$.
\item[(b)]
\begin{enumerate}
\item[(i)] Extend the direction of the incident ray as shown by the dotted line.
\item[(ii)] Measure the perpendicular distance `$d$' between extended incident ray and the emergent ray.
\end{enumerate}
\item[(c)] Repeat the procedure in (a) and (b) above for angles of incidence of $30^\circ$, $40^\circ$, $50^\circ$, $60^\circ$ and $70^\circ$. (In each case make your drawings on a fresh part of the drawing paper).
\item[(d)] Tabulate your results as shown in Table 1 below.
\begin{center}
\begin{tabular}{|p{0.15\textwidth}|p{0.15\textwidth}|p{0.15\textwidth}|p{0.15\textwidth}|p{0.15\textwidth}|} \hline
\multicolumn{1}{|c|}{$i$ (deg)} & \multicolumn{1}{c|}{$r$ (deg)} & \multicolumn{1}{c|}{$d$ (cm)} & \multicolumn{1}{c|}{$d\cos{r}$} & \multicolumn{1}{c|}{$\sin{(i-r)}$} \\ \hline
\multicolumn{1}{|c|}{30}&&&& \\
\multicolumn{1}{|c|}{40}&&&& \\
\multicolumn{1}{|c|}{50}&&&& \\
\multicolumn{1}{|c|}{60}&&&& \\
\multicolumn{1}{|c|}{70}&&&& \\ \hline
\end{tabular}\\[10pt]
\end{center}
\begin{enumerate}
\item[(i)] Plot a graph of $d\cos{r}$ against $\sin{(i-r)}$.
\item[(ii)] Find the gradient of the graph.
\item[(iii)] Measure the width of the glass block.
\item[(iv)] How is the gradient of the graph in 2 (a)(ii) and the width of the glass block in 2 (a)(iii) related?\\[5pt]
\end{enumerate}
\item[] NB: \quad Hand in your diagrams (drawings) together with your answer booklet. 
\item[] \flushright \textbf{(25 marks)}

\end{enumerate}

\item[3.] Determine the resistivity $\rho$ of the wire labelled W and the internal resistance of the battery provided.\\

Proceed as follows:

\begin{center}
\includegraphics[width=10cm]{./img/2004-3-alt.png}
\end{center}

Connect the circuit as shown in fig. 3 above. With the plug key open adjust the length of wire W to a value of 20 cm. Note the ammeter reading.\\[10pt]

\noindent NB: The plug key should remain open throughout the experiment.

\begin{enumerate}
\item[(a)] Repeat the procedure above for $L_W$ = 40 cm, 60 cm, 80 cm and 100 cm each time recording the ammeter reading.
\item[(b)] Tabulate your results as shown in Table 2 below.

\begin{tabular}{|p{2.5cm}|c|c|} \hline
Length $L_W$ of wire (cm)& Current $I$ (A)& $\cfrac{1}{I} \left(\text{A}^{-1}\right)$ \\ \hline
&& \\
&& \\
&& \\
&& \\ \hline
\end{tabular}\\[10pt]

\item[(c)]
\begin{enumerate}
\item[(i)] Plot a graph of $\cfrac{1}{I}$ (vertical) against $L_W$ (horizontal).
\item[(ii)] Determine the slope G.
\item[(iii)] Determine the intercept $Y$ on the vertical axis.
\end{enumerate}
\item[(d)] Measure and record the diameter at four different places on the wire. Hence find the mean value of diameter $d$.
\item[(e)] Given that $G = \cfrac{4\rho}{\pi d^2E}$ and $Y = \cfrac{R + r}{E}$\\[10pt]
Where $E$ is the emf of the battery, and $R = 2 \Omega$, Find the
\begin{enumerate}
\item[(i)] Resistivity $\rho$ of the wire.
\item[(ii)]	Internal resistance $r$ of the battery. \hfill \textbf{(25 marks)}
\end{enumerate}
\end{enumerate}

\end{enumerate}
\pagebreak
\setcounter{secnumdepth}{2}

% Other appendix chapters
\input{./tex/sources-of-equipment.tex}
\input{./tex/sources-of-chemicals.tex}
\chapter{Kiswahili Laboratory Glossary}

While managing a lab filled with dangerous chemicals, breakable glassware, open flames, and inexperienced students, it is important to know how to communicate with students easily to keep the place safe and running smoothly. Below is a list of Kiswahili words and phrases that may be helpful in a lab setting.

\begin{center}
\begin{longtable}{|p{0.5\textwidth}|p{0.5\textwidth}|}\hline

\multicolumn{1}{|c|}{\textbf{English}}	&	\multicolumn{1}{c|}{\textbf{Kiswahili}}	\\	\hline

to absorb	&	-fyonza	\\	\hline
to be absorbed	&	-fyonzwa	\\	\hline
accident	&	ajali	\\	\hline
to affect, to influence	&	-athiri	\\	\hline
ashes	&	majivu	\\	\hline
battery acid	&	maji makali	\\	\hline
to boil 	&	-chemsha 	\\	
        Ex. I boil the water.	&	        Ninachemsha maji.	\\	\hline
To be boiling 	&	-chemka 	\\	
        Ex. The water is boiling.	&	        Maji yanachemka.	\\	\hline
to break 	&	-vunja 	\\	
        Ex. I broke the glass.	&	        Nimevunja kioo.	\\	\hline
to be broken 	&	-vunjika 	\\	
        Ex. The glass broke.	&	        Kioo kimevunjika.	\\	\hline
broken, bad, rotten	&	bovu	\\	\hline
to burn	&	-unguza 	\\	
        Ex. I am burning paper.  	&	        Ninaunguza karatasi.	\\	\hline
(to be) burnt	&	-ungua 	\\	
        Ex. The paper is burnt.	&	        Karatasi imeungua.	\\	\hline
calcium hydroxide solution (lime water)	&	maji chokaa	\\	\hline
carefully	&	taratibu	\\	\hline
to cause	&	-sababisha	\\	\hline
to be caused by	&	-sababishwa	\\	\hline
caution 	&	utaratibu 	\\	
        Ex. Heat with caution.	&	        Ex. Pasha na utaratibu.	\\	\hline
to change 	&	-badilisha 	\\	
        Ex. I am changing the color of this.	&	        Ninabadilisha rangi ya hii. 	\\	\hline
to be changed 	&	-badilika 	\\	
        Ex. The color has been changed. 	&	        Rangi imebadilika.	\\	\hline
changes	&	mabadiliko	\\	\hline
chemical	&	kemikali	\\	\hline
container, glassware	&	chombo	\\	\hline
to clean	&	-safisha	\\	\hline
to collide	&	-gongana	\\	\hline
color	&	rangi	\\	\hline
to cool	&	-poa	\\	\hline
danger	&	hatari	\\	\hline
to decrease (transitive)	&	-punguza	\\	
        Ex. Decrease the heat.	&	        Punguza moto.  	\\	\hline
to decrease in size, to grow smaller (intransitive)	&	-pungua	\\	
        Ex: The heat has decreased.	&	        Moto umepungua.	\\	\hline
to destroy, to damage, to contaminate	&	-haribu	\\	
        Ex. I contaminated/damaged the chemical.	&	        Nimeharibu kemikali.	\\	\hline
to be destroyed, damaged, contaminated, or expired.	&	-haribika	\\	
        Ex. The chemical has gone bad or expired.	&	        Kemikali imeharibika.	\\	\hline
distilled water (from a hardware or car repair shop) 	&	maji baridi	\\	\hline
to distribute	&	-gawa	\\	\hline
to draw	&	-chora	\\	\hline
to drink	&	-nywa	\\	
        Ex. Do not drink!	&	        Usinywe!	\\	\hline
drop (as in “a drop of water”)	&	tone 	\\	\hline
to dry	&	- kausha	\\	\hline
effect	&	athari	\\	\hline
to estimate	&	- kadiri	\\	\hline
to evaporate (causative)	&	-vukiza 	\\	
        Ex. I am evaporating water. 	&	        Ninavukiza maji.	\\	\hline
to explode	&	- lipuka	\\	\hline
to fill	&	-jaza	\\	
        Ex. I filled the container.	&	        Nimejaza chombo.	\\	\hline
to be full	&	-jaa	\\	
        Ex. The container is full.	&	        Chombo kimejaa. 	\\	\hline
to filter	&	-chuja	\\	\hline
fire	&	moto	\\	\hline
gas, air	&	hewa	\\	\hline
glass	&	kioo	\\	\hline
group	&	kikundi (vi-)	\\	\hline
to grow	&	kukua	\\	\hline
harm (harmful)	&	madhara (yenye madhara)	\\	\hline
to haul off and slap someone	&	-ezeka makofi	\\	\hline
to heat	&	-pasha	\\	\hline
height, length	&	urefu	\\	\hline
to hit, to knock	&	-gonga	\\	\hline
to increase in size or number, to swell (intransitive)	&	-ongezeka	\\	
        Ex. The amount of food present increased.	&	        Chakula kimeongezeka.	\\	\hline
to increase, to add (transitive)	&	-ongeza	\\	
        Ex. I added more food.	&	        Nimeongeza chakula.	\\	\hline
instrument/apparatus/	&	kifaa (vi-)	\\	\hline
tool	&		\\	\hline
to kick someone out     	&	-fukuza 	\\	
        Ex. If you break a rule, you will be kicked out     of the laboratory.	&	Kama unakiuka sheria, utafukuzwa maabara.	\\	\hline
lab activity/experiment	&	practical	\\	\hline
living	&	hai	\\	\hline
match / lighter	&	kiberiti (vi-) / kiberiti cha gasi	\\	\hline
to measure, to test	&	-pima	\\	\hline
to be melted, to be dissolved (Not exactly “to be melted.” It's just the intransitive form of to melt or dissolve.)	&	-yeyuka 	\\	
        Ex.The salt melted/dissolved.	&	        Chumvi imeyeyuka.	\\	\hline
mass	&	uzito	\\	\hline
methylated spirits	&	spiriti	\\	\hline
microscope	&	hadubini	\\	\hline
to mix	&	-changanya	\\	\hline
to be mixed	&	-changanyika	\\	\hline
mixture	&	mchanganyiko	\\	\hline
particle, atom, small piece	&	chembe ndogo ndogo	\\	\hline
permission	&	ruhusa	\\	\hline
poison	&	sumu	\\	\hline
to pour	&	-mimina	\\	\hline
to pour out	&	-mwaga	\\	\hline
to press on, to push on (used to described squeezing of eye droppers)	&	-bonyeza	\\	\hline
to put, to keep	&	-weka	\\	\hline
rubber, rubber tubing	&	mpira	\\	\hline
to rush, to hurry	&	-harakisha 	\\	
        Ex. Do not rush!	&	        usiharakishe	\\	\hline
scientist	&	mwanasayansi	\\	\hline
serious, attentive	&	makini	\\	\hline
to shake	&	-tikisa	\\	\hline
to share out, to divide	&	-gawana	\\	\hline
slowly	&	pole pole	\\	\hline
to solidify, to freeze	&	-ganda	\\	\hline
steam	&	mvuke	\\	\hline
to stir	&	-koroga	\\	\hline
to stop	&	-acha	\\	\hline
stove	&	jiko (ma-)	\\	\hline
strong, harsh, fierce, dangerous, concentrated (for an acid)	&	kali	\\	\hline
to suck, to pull	&	-vuta	\\	\hline
to throw away, to chuck	&	-tupa	\\	\hline
to touch 	&	-gusa	\\	
        Ex. Do not touch!	&	        Usiguse!	\\	\hline
to turn off	&	-zima	\\	\hline
to turn on, to light a flame	&	-washa 	\\	\hline
volume	&	ujazo	\\	\hline
to wash dishes or glassware	&	-osha	\\	\hline
to wash your hands	&	-nawa	\\	\hline
to watch	&	-angalia	\\	\hline
yeast	&	hamira	\\	\hline




\end{longtable}
\end{center}
%\newgeometry{margin=1cm}
%\begin{landscape}
\thispagestyle{empty}

\chapter{Qualitative Analysis Guidesheet}

The goal of qualitative analysis is to identify an unknown salt. The ions used in the O-level qualitative analysis are:\\
Cations: NH$_4^+$, Ca$^{2+}$, Fe$^{2+}$, Fe$^{3+}$, Cu$^{2+}$, Zn$^{2+}$, Pb$^{2+}$, Na$^+$\\
Anions: CO$_3^{2-}$, HCO$_3^-$, NO$_3^-$, SO$_4^{2-}$, Cl$^-$\\
The ions are identified by following a series of ten steps, divided into three stages. These are:
\begin{enumerate*}
\item \textbf{Preliminary tests}: Preliminary tests use the solid salt. They are: appearance, flame test, action of heat, action of dilute H$_2$SO$_4$, action of concentrated H$_2$SO$_4$, and solubility. \hfill
\item \textbf{Tests in solution}: The compound should be dissolved in water before carrying out these tests. If it is not soluble in water, it should be dissolved in dilute HNO$_3$. The tests in solution are addition of NaOH and addition of NH$_3$. \hfill
\item \textbf{Confirmatory tests}: These tests confirm that the conclusions of the previous eight steps are correct. One confirmatory test should be done for the cation and one confirmatory test should be done for the anion.
\end{enumerate*}

\noindent \textbf{Description of Each Step}

\begin{enumerate}
\item \textbf{Appearance}: Observe three properties: the color of the salt, whether the salt is powdery or crystalline, and the salt’s smell.
\item \textbf{Flame Test}: A solid sample of the salt is placed in a flame using a wire, spatula, or glass rod. Some salts produce a characteristic flame color.
\item \textbf{Action of Heat}: The purpose of this test is to decompose the salt. Observe two things: whether a gas is formed, and whether a residue is formed. Damp pieces of blue and red litmus paper should be held over the test tube to test the acidity or alkalinity of any gas formed. Gases can be problematic. The only easily-identified gases are nitrogen dioxide (which is brown) and ammonia (which is alkaline). Most sulfates do not decompose to sulphur dioxide when heated, so do not expect to identify a gas if sulfate is present. Carbonates decompose, but carbon dioxide does not usually alter the color of litmus paper. Chlorides do not decompose when heated. So a test that produces no identifiable gas is inconclusive: a carbonate, sulfate, or chloride may be present.
\item \textbf{Action of dilute H$_2$SO$_4$}: This step tests for carbonates. If effervescence (bubbles) is observed, then carbonates or hydrogen carbonates are present. If there is no effervescence, then they are absent.
\item \textbf{Action of concentrated H$_2$SO$_4$}: If no identifiable gas was formed due to the action of heat or dilute H$_2$SO$_4$, the addition of concentrated H$_2$SO$_4$ will help determine whether a chloride or sulfate is present. Gently heat the solution and observe if any gas is present using litmus paper. If no gas is formed, then a sulfate is present. If a colorless, acidic gas (hydrogen chloride) forms, then a chloride is present. Do not heat the solution until boiling, as this will created sulphuric acid fumes.
\item \textbf{Solubility}: Determine if the salt is soluble in water, and observe the color of the solution. If the salt is not soluble in cold water, then test its solubility in warm water. A solubility chart can be used to help identify possible salts.
\item \textbf{Addition of NaOH solution}: NaOH solution is added to a solution of the salt. The purpose of this test is to identify the cation present. A small amount of NaOH solution should be added first (to identify the color of the precipitate), then a large amount should be added (to test if the precipitate is soluble in excess).
\item \textbf{Addition of NH$_3$ solution}: NH$_3$ solution is added to a solution of the salt. The purpose of this test is to identify the cation. As with NaOH, a small amount of solution should be added first, followed by a large amount.
\end{enumerate}




\begin{table}
\centering

%\begin{center}
\begin{tabular}{| p{3.5cm} | p{4cm} | p{8cm} | p{4cm} | p{4cm} |}
\hline

\multicolumn{1}{|c}{\textbf{Required Chemical}} & 
\multicolumn{1}{|c}{\textbf{Low Cost Alternative}} & 
\multicolumn{1}{|c}{\textbf{Substiution Method}} & 
\multicolumn{1}{|c}{\textbf{Molarity Example}} & 
\multicolumn{1}{|c|}{\textbf{Concentration Example}} \\ \hline

Hydrochloric Acid & 
Sulfuric Acid (Battery Acid) & 
If you are required to prepare an X molarity solution of HCl, prepane a X$\times 0.5$ molarity solution of battery acid & 
The instructions call for 0.12~M HCl. Instead, prepare 0.06~M sulfuric acid & 
 \\ \hline

Ethanoic (Acetic) Acid & 
Sulfuric Acid (Battery Acid) & 
If you are required to prepare an M molarity solution of ethanoic acid, prepare a M$\times 0.5$ molarity solution of sulfuric acid & 
The instructions call for 0.10~M ethanoic acid. Prepare 0.05~M sulfuric acid. & 
 \\ \hline

Ethandioic (Oxalic) Acid dihydrate (C$_{2}$H$_{2}$O$_{4} \cdot$2H$_{2}$O) & 
Sulfuric Acid (Battery Acid) & 
If you are required to prepare an M molarity solution of ethandioic acid, prepare an M molarity solution of sulfuric acid. If you are required to prepare a C concentration solution of ethandioic acid, prepare a $^\text{C}/_{126}$ molarity solution of sulfuric acid. & 
The instructions call for 0.075~M ethandioic acid. Prepare 0.075~M sulfuric acid. & 
The instructions call for 6.3 $^\text{g}/_\text{L}$ ethandioic acid. Prepare 0.05~M sulfuric acid. \\ \hline

Potassium Hydroxide & 
Sodium Hydroxide (Caustic Soda) & 
For M molarity potassium hydroxide, make M molarity sodium hydroxide. For C concentration potassium hydroxide, make C$\times ^{40}/_{56}$ concentration sodium hydroxide. & 
The instructions call for 0.1~M potassium hydroxide. Just prepare 0.1~M sodium hydroxide. &
The instructions call for 2.8 $^\text{g}/_\text{L}$ potassium hydroxide. Prepare 2 $^\text{g}/_\text{L}$ sodium hydroxide. \\ \hline

Anhydrous Sodium Carbonate & 
Sodium Carbonate Decahydrate (Soda Ash) &  
For M molarity anhydrous sodium carbonate, make M molarity sodium carbonate decahydrate. For C concentration anhydrous sodium carbonate, make C$\times ^{286}/_{106}$ sodium carbonate decahydrate. & 
The instructions call for 0.09~M anhydrous sodium carbonate. Make 0.09~M sodium carbonate decahyrate. & 
The instructions call for 5.3 $^\text{g}/_\text{L}$ anhydrous sodium carbonate. Make 14.3 $^\text{g}/_\text{L}$ sodium carbonate decahydrate. \\ \hline

Anhydrous Sodium Carbonate & 
Sodium Hydroxide (caustic soda) & 
For M molarity anhydrous sodium carbonate, make M$\times 2$ molarity sodium hydroxide. For C concentration anhydrous sodium carbonate, make C$\times 2 \times ^{40}/_{106}$ sodium hydroxide. & 
The instructions call for 0.09~M anhydrous sodium carbonate. Make 0.18~M sodium hydroxide. & 
The instructions call for 5.3 $^\text{g}/_\text{L}$ anhydrous sodium carbonate. 4.0 $^\text{g}/_\text{L}$ sodium hydroxide. \\ \hline

Sodium Carbonate Decahydrate (Na$_2$CO$_3\cdot$10H$_2$O) &
sodium hydroxide (caustic soda) &
For M molarity sodium carbonate ecahydrate, make M$\times 2$ molarity sodium hydroxide. For C concentration sodium carbonate decahydrate, make C$\times 2 \times ^{40}/_{286}$ sodium hydroxide. &
The instructions call for 0.09~M sodium carbonate decahydrate. Make 0.18~M sodium hydroxide. &
The instructions call for 14.3 $^\text{g}/_\text{L}$ sodium carbonate decahydrate. Make 4.0 $^\text{g}/_\text{L}$ sodium hydroxide. \\ \hline

Anhydrous Potassium Carbonate & 
Sodium Carbonate decahydrate (Soda Ash) & 
For M molarity potassium carbonate, make M molarity sodium carbonate decahydrate. For C concentration potassium carbonate, make C$\times ^{286}/_{122}$ concentration sodium carbonate. & 
The instructions call for 0.08~M anhydrous potassium carbonate. Prepare 0.08~M sodium carbonate decahydrate. & 
The instructions call for 6.1 $^\text{g}/_\text{L}$ anhydrous potassium carbonate. Prepare 14.3 $^\text{g}/_\text{L}$ sodium carbonate decahydrate. \\ \hline

Anhydrous Potassium Carbonate & 
Sodium Hydroxide (caustic soda) & 
For M molarity potassium carbonate, make M$\times 2$ molarity sodium hydroxide. For C concentration potassium carbonate, make C$\times 2 \times ^{40}/_{122}$ concentration sodium hydroxide. & 
The instructions call for 0.08~M anhydrous potassium carbonate. Prepare 0.16~M sodium hydroxide. & 
The instructions call for 6.1 $^\text{g}/_\text{L}$ anhydrous potassium carbonate. Prepare 4.0 $^\text{g}/_\text{L}$ sodium hydroxide. \\ \hline

\end{tabular}
%\end{center}

\end{table}
%\end{landscape}
\restoregeometry

% That's it!
\end{document}
